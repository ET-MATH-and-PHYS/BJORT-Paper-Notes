\subsection{September 27 Notes}


\begin{itemize}
    \item Began looking through BJORT\cite{BJORT} section 2, and in particular the notion of a cross-effect 
    \item Went over preliminary definitions, such as that of an abelian category
    \item Analyzed the inductive definition of the cross-effect functor, and determined how it is explicitly constructed.
\end{itemize}


\subsection{October 4 Notes}


\begin{itemize}
    \item Went through the proofs of Lemma 2.4 and Proposition 2.5 ourselves and argued for the naturality of the counit.
\end{itemize}



\subsection{October 18 Notes}


\begin{itemize}
    \item In the paper we begin with a functor $F:\mathcal{B}\rightarrow \mathcal{A}$, and produce a functor $D_1F:\mathcal{B}\rightarrow \cat{Ch}\mathcal{A}$, but this results in issues of composition and functoriality if we have another functor $G:\mathcal{C}\rightarrow \mathcal{B}$. Although we can consider $D_1(F\circ G)$, $D_1(F)\circ D_1(G)$ is not well typed as $D_1(F):\mathcal{B}\rightarrow \cat{Ch}\mathcal{A}$ and $D_1(G):\mathcal{C}\rightarrow \cat{Ch}\mathcal{B}$.
    \item Question:
    \begin{quotation}
        \noindent To what degree is $\cat{Ch}:\cat{AbCat}\rightarrow \cat{AbCat}$ a monad on $\cat{AbCat}$? (In fact it is a psuedo-monad, and we must be careful on how maps on 1-cells and 2-cells is defined)
    \end{quotation}
    \item We are not going to work with $\cat{AbCat}$ directly, but rather a quotient of $\cat{AbCat}$. 
    \item In this context we can ask if chain homotopy equivalences are pointwise, or can be promoted to being natural?
    \item 
\end{itemize}