\section{Linear Approximations}


In this section we define the linear approximation of a functor $F:\mathcal{B}\rightarrow \cat{Ch}(\mathcal{A})$ as a functor $D_1(F)$ in $\cat{AbCat}_{\cat{Ch}}$. This construction will coincie, up to homotopy, with the homotopy fiber of the map $q_1:P_1(F)\to P_0(F)$. All of the properties we will describe for $D_1(F)$ are developed only up to \rd{natural} chain homotopy equivalence.

\begin{defn}[label=defn:linearization]
    The \textbf{linearization} of $F:\mathcal{B}\rightarrow \cat{Ch}(\mathcal{A})$ is the functor $D_1(F):\mathcal{B}\rightarrow \mathcal{A}$ given as the totalization of the explicit chain complex of chain complexes $(D_1(F)_\bullet,\partial_\bullet)$ where:
    %%
    \begin{equation*}
        D_1(F)_k := \left\{\begin{array}{cc} C_2^{\times k}(F) & k \geq 1 \\ \text{cr}_1(F) & k = 0 \\ 0 & \text{else} \end{array}\right.
    \end{equation*}
    %%
    and the differential $\partial_1:D_1(F)_1\rightarrow D_1(F)_0$ is given by $\rho_1$, while for $k \geq 2$, $\partial_k:D_1(F)_k\rightarrow D_1(F)_{k-1}$ is given by the alternating sum $\sum_{i=0}^{k-1}(-1)^iC_2^{\times i}\epsilon_{C_2^{\times (k-1-i)}}$. Note $\rho_1 = \epsilon$ since $C_2(F) \cong C_2(\text{cr}_1(F))$.
\end{defn}

Recall that $P_0(F)$ is naturally chain homotopy equivalent to $F(0)$. Using this model the induced map $q_1:P_1(F)\to P_0(F)\to F(0)$ is a degree-wise epimorphism, and hence a fibration in the standard model structure on non-negatively graded chain complexes.

\begin{proof}[Proof of Epimorphism]
    Note by our explicit construction of $q_1$, $q_1\circ p_1:F\to F(0)$ is exactly the projection for the isomorphism $F \cong F(0)\oplus \text{cr}_1(F)$. But this is epi as a map of chain complexes in each degree, so $q_1$ must also be epi.
\end{proof}

This implies that the kernel of $q_1$ is a model of the homotopy fiber, when this structure makes sense. Note that our definition of $D_1(F)$ coincides with $P_1(\text{cr}_1(F))$ due to Lemma \ref{lem:idempotCr1}, which is exactly the kernel of $q_1:P_1(F)\to F(0)$. 

To start understanding the linear approximation we consider an example:

\begin{eg}{}
    We consider the affine functor $F(X) = A\oplus X$. Recall $\text{cr}_2(F) \cong 0$ and $\text{cr}_1(F) \cong \id$. This implies that $D_1(F)$ is chain homotopy equivalent to $\id$ in degree $0$.
\end{eg}

We begin by inspecting properties of $D_1$. First we obtain immediate results since $D_1\cong P_1\circ \text{cr}_1$.

\begin{prop}[label=prop:D1Exact]
    \begin{itemize}
        \item[(i)] $D_1:\text{Fun}(\mathcal{B},\cat{Ch}(\mathcal{A}))\to \text{Fun}(\mathcal{B},\cat{Ch}(\mathcal{A}))$ is exact
        \item[(ii)] $D_1$ preserves (natural) chain homotopies, chain homotopy equivalences, and contractibility.
    \end{itemize}
\end{prop}
\begin{proof}
    These results follow from Propositions~\ref{prop:exactPol} and~\ref{prop:exactCross} using the isomorphism $D_1\cong P_1\circ \text{cr}_1$.
\end{proof}

Relaxing some of our previous constraints to the level of \rd{natural} chain homotopy equivalence, we define what it means for a functor to be linear.

\begin{defn}[label=defn:linear]
    A functor $F:\mathcal{B}\to \cat{Ch}(\mathcal{A})$ is said to be \textbf{linear} if it is degree one and weakly reduced, so $F(0)$ is naturally contractible.
\end{defn}

This definition is equivalent to a characterization in terms of finite direct sums. In order to establish this equivalence we must first prove results on how direct sums interact with natural chain homotopy equivalence. These results are a generalization of Lemma~\ref{lem:biprod}.

\begin{lem}[label=lem:oplusPres]
    Let $F,G,H:\mathcal{B}\to \cat{Ch}(\mathcal{A})$ be functors and let $f,g:F\Rightarrow G$ be chain homotopic maps. Then $f\oplus 1_H$ is chain homotopic to $g\oplus 1_H$.
\end{lem}
\begin{proof}
    Note that under the isomorphism $\text{Fun}^{\cat{Ch}}:\cat{Ch}(\text{Fun}(\mathcal{B},\mathcal{A}))\to \text{Fun}(\mathcal{B},\cat{Ch}(\mathcal{A}))$, it is sufficient to show for $F_\bullet,G_\bullet,H_\bullet \in \cat{Ch}(\text{Fun}(\mathcal{B},\mathcal{A}))$, with $f,g:F_\bullet\Rightarrow G_\bullet$, and let for each $n \in \Z$, $s_n:F_n\Rightarrow G_{n+1}$ denote the component natural transformations for the homotopy. Note that $-\oplus H_\bullet:\cat{Ch}(\text{Fun}(\mathcal{B},\mathcal{A}))\to \cat{Ch}(\text{Fun}(\mathcal{B},\mathcal{A}))$ is a functor. We consider $s_n\oplus 0:F_n\oplus H_n\Rightarrow G_{n+1}\oplus H_{n+1}$. Then 
    %%
    \begin{align*}
        (\partial_{n+1}^G\oplus \partial_{n+1}^H)\circ (s_n\oplus 0)+(s_{n-1}\oplus 0)\circ (\partial_n^F\oplus \partial_n^H) &= (\partial_{n+1}^G\circ s_n+s_{n-1}\circ \partial_n^F)\oplus 0  \\
        &= (f_n-g_n)\oplus (1_{H_n}-1_{H_n}) \\
        &= f_n\oplus 1_{H_n}-g_n\oplus 1_{H_n}
    \end{align*}
    %%
    as desired.
\end{proof}

We also have the converse result.

\begin{lem}[label=lem:homotopCancel]
    Let $F,G,H:\mathcal{B}\to \cat{Ch}(\mathcal{A})$ be functors and let $f,g:F\Rightarrow G$ be maps such that $f\oplus 1_H$ and $g\oplus 1_H$ are chain homotopic maps. Then $f$ is chain homotopic to $g$.
\end{lem}
\begin{proof}
    Once again we pass to $F_\bullet,G_\bullet,H_\bullet\in \cat{Ch}(\text{Fun}(\mathcal{B},\mathcal{A}))$ with $f,g:F_\bullet\Rightarrow G_\bullet$. Let 
    %%
    \begin{equation*}
        \begin{pmatrix} s_n^{1,1} & s_n^{1,2} \\ s_n^{2,1} & s_n^{2,2} \end{pmatrix}:F_n\oplus H_n\to G_{n+1}\oplus H_{n+1}
    \end{equation*}
    %%
    be the homotopy witnessing $f\oplus 1_H$ homotopic to $g\oplus 1_H$. Then the homotopy condition takes the form
    %%
    \begin{equation*}
        \begin{pmatrix} \partial_{n+1}^Gs_n^{1,1}+s_{n-1}^{1,1}\partial_n^G & \partial_{n+1}^Gs_n^{1,2}+s_{n-1}^{1,2}\partial_n^H \\ \partial_{n+1}^Hs_n^{2,1}+s_{n-1}^{2,1}\partial_n^G & \partial_{n+1}^Hs_n^{2,2}+s_{n-1}^{2,2}\partial_n^H \end{pmatrix} = \begin{pmatrix} f_n-g_n & 0 \\ 0 & 0 \end{pmatrix}
    \end{equation*}
    %%
    It follows that $s_n^{1,1}:F_n\to G_{n+1}$ forms a homotopy from $f$ to $g$.
\end{proof}

As an immediate corollary of Lemma~\ref{lem:oplusPres} and Lemma~\ref{lem:homotopCancel} we have that chain homotopy equivalences are preserved by direct sum and satisfy the cancellative property.

We can now prove the previously asserted equivalence.

\begin{prop}[label=prop:linearEquiv]
    A functor $F:\mathcal{B}\to \cat{Ch}(\mathcal{A})$ is linear if and only if the natural map
    %%
    \begin{equation*}
        F(X)\oplus F(Y)\hookrightarrow F(X\oplus Y)
    \end{equation*}
    %%
    is a natural chain homotopy equivalence.
\end{prop}
\begin{proof}
    First suppose $F:\mathcal{B}\to \cat{Ch}(\mathcal{A})$ is linear. Then we have that $\text{cr}_2(F)\simeq_{\cat{Ch}}0$ and $F(0)\simeq_{\cat{Ch}}0$. Recall from the inductive definition of the cross effect that 
    %%
    \begin{equation*}
        \text{cr}_1(F)(X\oplus Y) \cong \text{cr}_1(F)(X)\oplus \text{cr}_1(F)(Y)\oplus \text{cr}_2(F)(X,Y)
    \end{equation*}
    %%
    naturally in $X$ and $Y$. Then by Lemma~\ref{lem:oplusPres} we have that \[\text{cr}_1(F)(X)\oplus \text{cr}_1(F)(Y)\oplus \text{cr}_2(F)(X,Y)\simeq_{\cat{Ch}}\text{cr}_1(F)(X)\oplus \text{cr}_1(F)(Y)\oplus 0 \cong \text{cr}_1(F)(X)\oplus \text{cr}_1(F)(Y)\] 
    so $\text{cr}_1(F)$ preserves direct sums up to natural chain homotopy equivalence. Then since $F(0)\simeq_{\cat{Ch}}0$ and $F\cong \text{cr}_1(F)\oplus F(0)$, it follows that $F$ also preserves direct sums up to natural chain homotopy equivalence.

    \vspace{10pt}

    Conversely, if $F$ preserves direct sums up to natural chain homotopy equivalence we have the identity
    %%
    \begin{equation*}
        F(0)\cong F(0\oplus 0)\simeq_{\cat{Ch}}F(0)\oplus F(0)
    \end{equation*}
    %%
    so by Lemma~\ref{lem:homotopCancel} it follows that $F(0) \simeq_{\cat{Ch}} 0$. Similarly, using the inductive formula for $\text{cr}_2(F)$ and Lemma~\ref{lem:homotopCancel} we have that $\text{cr}_2(F)$ is naturally contractible, so $F$ is degree 1.
\end{proof}

Note that since $D_1$ is an exact functor it is linear in the sense that it preserves the zero object and direct sums of functors up to natural isomorphism. Additionally, $D_1(F)$ is also a linear functor for any $F$.

\begin{lem}[label=lem:5.6]
    Let $F:\mathcal{B}\to \cat{Ch}(\mathcal{A})$. Then $D_1(F):\mathcal{B}\to \cat{Ch}(\mathcal{A})$ is strictly reduced and linear. 
\end{lem}
\begin{proof}
    Recall $D_1(F) \cong P_1(\text{cr}_1(F))$. Then as $\text{cr}_1(F)$ is strictly reduced and each $C_2(\text{cr}_1(F))$ is strictly reduced, it follows that $P_1(\text{cr}_1(F))$ is strictly reduced.

    \vspace{10pt}

    By Proposition~\ref{prop:4.5} we have that $P_1(\text{cr}_1(F))$ is degree 1, which completes the proof.
\end{proof}


The next main result we wish to show is that $D_1$ preserves composition up to \rd{naturally} chain homotopy, upgrading the original pointwise chain homotopy results in~\cite{BJORT}. The primary work and lemmas for this result are contained in Section~\ref{sec:Lotswork}. In particular, we will prove the following result in Section~\ref{sec:Lotswork}.

\begin{prop}[label=prop:5.7]
    If $F:\mathcal{B}\to \cat{Ch}(\mathcal{A})$ and $G:\mathcal{C}\to \cat{Ch}(\mathcal{B})$ are composible functors in $\cat{AbCat}_{\cat{Ch}}$ such that $G$ is reduced, Then
    %%
    \begin{equation*}
        D_1(F\lhd G) \simeq_{\cat{Ch}} D_1(F)\lhd D_1(G)
    \end{equation*}
    %%
\end{prop}

In this section we will extend this result to non-reduced $G$. First, by Lemma~\ref{lem:idempotCr1} we have that $\text{cr}_1(\text{cr}_1(F))\cong \text{cr}_1(F)$, so as $D_1(F) \cong P_1(\text{cr}_1(F))$ and post-composition preserves natural isomorphisms we obtain that $D_1(F) \cong D_1(\text{cr}_1(F))$. Before extending the proposition we first prove some preliminary results on the relationship between the direct sum operation on functors and our composition, $\lhd$, in $\cat{AbCat}_{\cat{Ch}}$. 

\begin{lem}[label=lem:compDirSum]
    Let $F,G:\mathcal{B}\to \cat{Ch}(\mathcal{A})$ and $H:\mathcal{C}\to \cat{Ch}(\mathcal{B})$. Then there is a natural isomorphism
    %%
    \begin{equation*}
        (F\oplus G)\lhd H\cong (F\lhd H)\oplus (G\lhd H)
    \end{equation*}
    %%
\end{lem}
\begin{proof}
    First, note that since $\Gamma$ preserves direct sums, $\Gamma_\mathcal{A}\circ (F\oplus G)\cong (\Gamma_\mathcal{A}\circ F)\oplus (\Gamma_\mathcal{A}\circ G)$. Then by definition of the direct sum of functors
    %%
    \begin{equation*}
        ((\Gamma_\mathcal{A}\circ F)\oplus (\Gamma_\mathcal{A}\circ G))_*\Gamma_\mathcal{B}H = (\Gamma_\mathcal{A}\circ F)_*\Gamma_\mathcal{B}H\oplus (\Gamma_\mathcal{A}\circ G)_*\Gamma_\mathcal{B}H
    \end{equation*}
    %%
    Finally, since $\Delta_\mathcal{A}$ and $N_\mathcal{A}$ also preserve finite limits, we have that 
    %%
    \begin{equation*}
        (F\oplus G)\lhd H \cong (F\lhd H)\oplus (G\lhd H)
    \end{equation*}
    %%
    as desired.
\end{proof}

As a simple corollary we find that post-composing by a $0$ functor is zero.

\begin{cor}[label=cor:contractLHD]
    Let $H:\mathcal{C}\to \cat{Ch}(\mathcal{B})$ and let $0:\mathcal{B}\to \cat{Ch}(\mathcal{A})$ be a zero functor. Then $0\lhd H$ is a zero functor.
\end{cor}
\begin{proof}
    Note that by Lemma~\ref{lem:compDirSum} we have that $0\lhd H \cong (0\lhd H)\oplus (0\lhd H)$, so from general properties of Abelian categories we have that $0\lhd H \cong 0$.
\end{proof}

\begin{lem}[label=lem:constComp]
    Let $F:\mathcal{B}\to \cat{Ch}(\mathcal{A})$ and $H:\mathcal{C}\to \cat{Ch}(\mathcal{B})$. If $F$ or $H$ are constant functors, then so is $F\lhd H$.
\end{lem}
\begin{proof}
    First, suppose $H$ is a constant functor associated to $B_\bullet \in \cat{Ch}(\mathcal{B})$. Then for any $C \in \mathcal{C}_0$ we have that 
    %%
    \begin{equation*}
        (F\lhd H)(C) = N_\mathcal{A}\Delta_\mathcal{A}(\Gamma_\mathcal{A})_*F_*\Gamma_\mathcal{B} H(C) = N_\mathcal{A}(\Delta_\mathcal{A}((\Gamma_\mathcal{A})_*\circ F_*\circ \Gamma_\mathcal{B}(B_\bullet))) 
    \end{equation*}
    %%
    with maps sent to identities. Similarly, if $F$ is constant determined by $A_\bullet \in \cat{Ch}(\mathcal{B})$, then 
    %%
    \begin{align*}
        (F\lhd H)(C) &= N_\mathcal{A}\Delta_\mathcal{A}(\Gamma_\mathcal{A})_*F_*\Gamma_\mathcal{B} H(C) \\
        &= N_\mathcal{A}(\Delta_\mathcal{A}((\Gamma_\mathcal{A})_*\circ F_*\circ \Gamma_\mathcal{B}(B_\bullet)))  \\
        &= N_\mathcal{A}(\Delta_\mathcal{A}((\Gamma_\mathcal{A})_*\circ \Delta_{A_\bullet})) \\
        &= N_\mathcal{A}\Gamma_\mathcal{A}(A_\bullet) \\
        &\cong A_\bullet
    \end{align*}
    %%
    using the Dold-Kan equivalence, where $\Delta_{A_\bullet}:\Delta^{op}\to \cat{Ch}(\mathcal{A})$ is the constant simplicial object at $A_\bullet$.
\end{proof}

\begin{lem}[label=lem:dirComp]
    Let $F:\mathcal{B}\to \cat{Ch}(\mathcal{A})$ and $H_1,...,H_n:\mathcal{C}\to \mathcal{B}$. Then 
    %%
    \begin{align*}
        \text{cr}_{n-1}(F)(H_1\oplus H_2,H_3,...,H_n) \cong \text{cr}_{n-1}(F)(H_1,H_3,...,H_n)&\oplus \text{cr}_{n-1}(F)(H_2,...,H_n) \\
        &\oplus \text{cr}_n(F)(H_1,...,H_n) 
    \end{align*}
\end{lem}
\begin{proof}
    Note that we have functors $H_1\times \cdots \times H_n:\mathcal{C}^n\to \mathcal{B}^n$. Then the desired isomorphism is obtained from the isomorphism in Equation~\eqref{eq:natN} by whiskering by $H_1\times \cdots \times H_n$.
\end{proof}

\begin{lem}[label=lem:dirCompLhd]
    Let $F:\mathcal{B}\to \cat{Ch}(\mathcal{A})$ and $H_1,...,H_n:\mathcal{C}\to \cat{Ch}(\mathcal{B})$. Then 
    %%
    \begin{align*}
        \text{cr}_{n-1}(F)\lhd (H_1\oplus H_2,H_3,...,H_n) \cong \text{cr}_{n-1}(F)\lhd (H_1,H_3,...,H_n)&\oplus \text{cr}_{n-1}(F)\lhd (H_2,...,H_n) \\
        &\oplus \text{cr}_n(F)\lhd (H_1,...,H_n) 
    \end{align*}
\end{lem}
\begin{proof}
    Since $\Gamma_\mathcal{B}$ preserves finite limits and colimits, we have that \textbf{TBD}
    %%
    \begin{align*}
        \text{cr}_{n-1}(F)_*\Gamma_{\mathcal{B}^{n-1}}(H_1\oplus H_2,H_3,...,H_n) &\cong 
    \end{align*}
\end{proof}

In order to extend the proposition we require one additional result.

\begin{lem}[label=lem:compGen]
    If $F:\mathcal{B}\to \cat{Ch}(\mathcal{A})$ and $G:\mathcal{C}\to \cat{Ch}(\mathcal{B})$ are composible functors, then 
    %%
    \begin{equation*}
        \text{cr}_1(F\lhd G)(X) \cong (\text{cr}_1(F)\lhd \text{cr}_1(G))(X)\oplus (\text{cr}_2(F)\lhd (G(0)\oplus \text{cr}_1(G)(X)))
    \end{equation*}
    %%
\end{lem}
\begin{proof}
    First, observe that $\text{cr}_1(F\lhd G)\oplus (F\lhd G)(0) \cong F\lhd G$ by construction of the cross-effect. Next, using the isomorphism again but now on $F$ and $G$ individually,
    %%
    \begin{equation*}
        F\lhd G \cong F\lhd (G(0)\oplus \text{cr}_1(G)) \cong (F(0)\oplus \text{cr}_1(F))\lhd (G(0)\oplus \text{cr}_1(G))
    \end{equation*}
    %%
    From our Lemmas~\ref{lem:compDirSum} and~\ref{lem:constComp} we have that 
    %%
    \begin{equation*}
        (F(0)\oplus \text{cr}_1(F))\lhd (G(0)\oplus \text{cr}_1(G)) \cong F(0)\oplus (\text{cr}_1(F)\lhd (G(0)\oplus \text{cr}_1(G)))
    \end{equation*}
    %%
    Next applying Lemma~\ref{lem:dirCompLhd} we obtain that 
    %%
    \begin{equation*}
        \text{cr}_1(F\lhd G)\oplus (F\lhd G)(0) \cong F(0)\oplus (\text{cr}_1(F)\lhd G(0))\oplus (\text{cr}_1(F)\lhd \text{cr}_1(G))\oplus (\text{cr}_2(F)\lhd (G(0)\oplus \text{cr}_1(G)))
    \end{equation*}
    %%
    Expanding $(F\lhd G)(0) \cong F(0)\oplus \text{cr}_1(F)\lhd G(0)$ and using the cancellative property of the direct sum, we obtain the desired isomorphism. 
\end{proof}

We now extend Proposition~\ref{prop:5.7}.

\begin{prop}[label=prop:5.10]
    If $F:\mathcal{B}\to \cat{Ch}(\mathcal{A})$ and $G:\mathcal{C}\to \cat{Ch}(\mathcal{B})$, then there is a \rd{natural} chain homotopy equivalence
    %%
    \begin{equation*}
        D_1(F\lhd G)\simeq_{\cat{Ch}}(D_1(F)\lhd D_1(G))\oplus D_1((\text{cr}_2(F)\lhd (G(0)\oplus \text{cr}_1(G))))
    \end{equation*}
    %%
\end{prop}
\begin{proof}
    Using Lemma~\ref{lem:compGen} we have that 
    %%
    \begin{equation*}
        D_1(F\lhd G) \cong D_1(\text{cr}_1(F\lhd G)) \cong D_1((\text{cr}_1(F)\lhd \text{cr}_1(G))\oplus (\text{cr}_2(F)\lhd (G(0)\oplus \text{cr}_1(G))))
    \end{equation*}
    %%
    Using the linearity of $D_1$ we have that 
    %%
    \begin{align*}
        &D_1((\text{cr}_1(F)\lhd \text{cr}_1(G))\oplus (\text{cr}_2(F)\lhd (G(0)\oplus \text{cr}_1(G)))) \\
        &\cong D_1(\text{cr}_1(F)\lhd \text{cr}_1(G))\oplus D_1(\text{cr}_2(F)\lhd (G(0)\oplus \text{cr}_1(G)))
    \end{align*}
    %%
    Since $\text{cr}_1(G)$ is reduced, we can apply Proposition~\ref{prop:5.7} and Lemma~\ref{lem:oplusPres} we have a \rd{natural} chain homotopy equivalence
    %%
    \begin{align*}
        &D_1(\text{cr}_1(F)\lhd \text{cr}_1(G))\oplus D_1(\text{cr}_2(F)\lhd (G(0)\oplus \text{cr}_1(G))) \\
        &\simeq_{\cat{Ch}} (D_1(\text{cr}_1(F))\lhd D_1(\text{cr}_1(G)))\oplus D_1(\text{cr}_2(F)\lhd (G(0)\oplus \text{cr}_1(G)))
    \end{align*}
    %%
    Applying the isomorphism $D_1\circ \text{cr}_1 \cong D_1$ we have that 
    %%
    \begin{equation*}
        D_1(F\lhd G) \cong D_1(\text{cr}_1(F\lhd G)) \simeq_{\cat{Ch}} (D_1(F)\lhd D_1(G))\oplus D_1(\text{cr}_2(F)\lhd (G(0)\oplus \text{cr}_1(G)))
    \end{equation*}
    %%
    as desired.
\end{proof}


We now move on to multilinearization of multivariable functors, $F:\mathcal{B}^n\to \cat{Ch}(\mathcal{A})$. We introduce the same convention as in~\cite[Conv 5.11]{BJORT} where for $X_1,...,X_{i-1},X_{i+1},...,X_n \in \mathcal{B}$ fixed we define $F_i:\mathcal{B}\to \cat{Ch}(\mathcal{A})$ to be the functor given by 
%%
\begin{equation*}
    F_i(Y) := F(X_1,...,X_{i-1},Y,X_{i+1},...,X_n)
\end{equation*}
%%
In general we define the following.

\begin{defn}[label=defn:multivariateLin]
    Let $1 \leq i_1 < i_2 < \cdots < i_k\leq n$ be a set of distinct increasing indices. Let $\pi_{i_1\times \cdots \times i_k}:\mathcal{B}^n\to \mathcal{B}^k$ and let $\iota_{i_1\times \cdots \times i_k}:\mathcal{B}^{n-k}\times\text{Fun}(\mathcal{B}^n,\mathcal{A})\to \text{Fun}(\mathcal{B}^k,\mathcal{A})$ be the functor defined by 
    %%
    \begin{equation*}
        \iota_{i_1\times\cdots\times i_k}((B_j)_{j\neq i_\ell},F)((B_{i_\ell})_{1\leq \ell\leq k}) := F((B_i)_{1\leq i \leq n})
    \end{equation*}
    %%
    with natural action on maps. We define a functor $D_1^{i_1\times \cdots \times i_k}:\text{Fun}(\mathcal{B}^n,\mathcal{A})\to \text{Fun}(\mathcal{B}^n,\mathcal{A})$ by 
    %%
    \begin{equation*}
        D_1^{i_1\times \cdots \times i_k}(F)((B_i)_{1\leq i \leq n}) := D_1(\iota_{i_1\times\cdots\times i_k}((B_j)_{j\neq i_\ell},F))((B_{i_\ell})_{1\leq \ell\leq k})
    \end{equation*}
    %%
    where on maps for $f_i:B_i\to B_i'$, we have that 
    %%
    \begin{equation*}
        D_1^{i_1\times \cdots \times i_k}(F)((f_i)_{1\leq i \leq n}) := D_1(\iota_{i_1\times\cdots \times i_k}((f_j)_{j\neq i_\ell},F))_{(B_{i_\ell}')_{1\leq \ell\leq k}}\circ D_1(\iota_{i_1\times \cdots \times i_k}((B_j)_{j\neq i_\ell},F))((f_{i_\ell})_{1\leq \ell\leq k})
    \end{equation*}
    %%
    which is functorial by definition of composition of natural transformations and functoriality of $D_1,F$, and $\iota_{i_1\times \cdots \times i_k}$. Now, if $\alpha:F\Rightarrow G$ is a natural transformation, we define 
    %%
    \begin{equation*}
        D_1^{i_1\times \cdots \times i_k}(\alpha)_{(B_i)_{1\leq i \leq n}} := D_1(\iota_{i_1\times \cdots \times i_k}((B_j)_{j\neq i_\ell},\alpha))_{(B_{i_\ell})_{1\leq \ell\leq k}}
    \end{equation*}
    %%
\end{defn}

Note that we can also apply this multilinearization procedure sequentially by applying $D_1^i$ terms. When we are multilinearing over all variables in this way we write
%%
\begin{equation*}
    D_1^{(n)}(F) := D_1^n\circ \cdots D_1^1(F)
\end{equation*}
%%


We proceed to prove a number of lemmas on the behaviour of these multilinearization operations.


\begin{lem}[label=lem:5.12]
    Let $H:\mathcal{B}^n\to \cat{Ch}(\mathcal{A})$ be a strictly multi-reduced functor. Then for any $1 \leq i_1< i_2 <\cdots < i_k \leq n$, $D_1^{i_1\times \cdots \times i_k}(H)$ is contractible.
\end{lem}
\begin{proof}
    Let $B_1,...,B_n \in \mathcal{B}$ and let $C_{i_1},...,C_{i_k} \in \mathcal{B}$. For $1 \leq i \leq n$ let $A_i'$ be $B_{i_j} \oplus C_{i_j}$ if there exists $1 \leq j \leq k$ such that $i = i_j$, and $B_i$ otherwise. Then we have that 
    %%
    \begin{equation*}
        D_1^{i_1\times \cdots \times i_k}(H)((A_i')_{1\leq i \leq n}) = D_1(\iota_{i_1\times \cdots \times i_k}((B_j)_{j\neq i_\ell},H))((B_{i_j}\oplus C_{i,j})_{1\leq j \leq k})
    \end{equation*}
    %%
    But by Lemma~\ref{lem:5.6} $D_1$ of a functor is linear, so we have a natural chain homotopy
    %%
    \begin{equation*}
        D_1(\iota_{i_1\times \cdots \times i_k}((B_j)_{j\neq i_\ell},H))((B_{i_j}\oplus C_{i_j})_{1\leq j \leq k})\simeq_{\cat{Ch}}D_1(\iota_{i_1\times \cdots \times i_k}((B_j)_{j\neq i_\ell},H))((B_{i_j})_{1\leq j \leq k})\oplus D_1(\iota_{i_1\times \cdots \times i_k}((B_j)_{j\neq i_\ell},H))((C_{i_j})_{1\leq j \leq k})
    \end{equation*}
    %%
    Note that in the special case where $B_{i_j} = 0$ for $1 < j \leq k$ and $C_{i_1} = 0$, the natural chain homotopy equivalence implies that $D_1^{i_1\times \cdots \times i_k}(H)$ is contractible since $D_1$ of a multireduced functor is multireduced by Lemma~\ref{lem:multiRed}.
\end{proof}


An important form of Lemma~\ref{lem:5.12} is the following.

\begin{cor}[label=cor:5.13]
    Let $F:\mathcal{B}\to \cat{Ch}(\mathcal{A})$ and $H:\mathcal{B}^n\to \cat{Ch}(\mathcal{A})$ be functors such that $F = H\circ \Delta$, where $H$ is strictly multi-reduced. Then $D_1(F)$ is contractible.
\end{cor}
\begin{proof}
    By Proposition~\ref{prop:5.7} we have that $D_1(F)$ is naturally chain homotopy equivalent to $D_1(H)\lhd D_1(\Delta)$ since $\Delta$ is strictly reduced. But $D_1(H) = D_1^{1\times \cdots \times n}(H)$, so by Lemma~\ref{lem:5.12} $D_1(H)$ is naturally contractible. Further, since $\Delta$ is degree $1$ we have that $\Delta$ is naturally chain homotopy equivalent to $P_1(\Delta) \cong P_1(\text{cr}_1(\Delta)) \cong D_1(\Delta)$ by Proposition~\ref{prop:4.5}. 

    \vspace{10pt}

    Since $\lhd$ preserves natural chain homotopy equivalences we have that 
    %%
    \begin{equation*}
        D_1(H)\lhd D_1(\Delta) \simeq_{\cat{Ch}}D_1(H)\lhd \Delta \simeq_{\cat{Ch}} 0 \lhd \Delta \cong 0
    \end{equation*}
    %%
    where the last isomorphism is by Corollary~\ref{cor:contractLHD}.
\end{proof}

Next we will provide an isomorphism for sequential linearization, analogous to the commutivity of partial differentiation from calculus when dealing with smooth functions.

\begin{lem}[label=lem:5.15]
    For any $F:\mathcal{B}^n\to \cat{Ch}(\mathcal{A})$ and any $1 \leq i,j\leq n$ we have a natural isomorphism
    %%
    \begin{equation*}
        D_1^i(D_1^j(F)) \cong D_1^j(D_1^i(F))
    \end{equation*}
    %%
\end{lem}
\begin{proof}
    \textbf{TBD} 
\end{proof}

Next, we show that linearization preserves strictly linear functors up to isomorphism.

\begin{prop}[label=prop:strictLin]
    Let $F:\mathcal{B}\to \cat{Ch}(\mathcal{A})$ be a functor that is strictly linear. Then $D_1(F)\cong F$.
\end{prop}
\begin{proof}
    Since $F$ is strictly linear $\text{cr}_2(F) \cong 0$ and $\text{cr}_1(F) \cong F$. Thus, the bicomplex defining $D_1(F)$ contains $F$ in the zeroth column and zeros elsewhere, so its totalization is isomorphic to $F$.
\end{proof}

As a simple consequence we have that the projection functor $\pi_i:\mathcal{B}^n\to \mathcal{B}$ is its own linearization, which is to say $D_1(\pi_i)\cong \pi_i$.


\begin{cor}[label=cor:constLinearization]
    Let $F:\mathcal{B}\to \cat{Ch}(\mathcal{A})$. Then $D_1(F\circ \pi_1)(B_1,B_2)\cong D_1(F)(B_1)$ for all $B_1,B_2 \in \mathcal{B}$.
\end{cor}
\begin{proof}
    By the kernel definition of the cross-effect we have that $\text{cr}_1(F\circ \pi_1)\circ \iota_{B_2}\cong \text{cr}_1(F)$ for any $B_2 \in \mathcal{B}$. Similarly, from our kernel description of the second cross-effect $\text{cr}_2(F)\cong \text{cr}_2(F\circ \pi_1)\circ (\iota_{B_2}\times \iota_{B_1})$. Then since these are natural isomorphisms we conclude that $D_1(F\circ \pi_1)\circ \iota_{B_2}\cong D_1(F)$ for all $B_2 \in \mathcal{B}$.
\end{proof}

Finally, using this result we can obtain a commutativity between functors into products and linearization.

\begin{lem}[label=lem:5.18]
    Let $F:\mathcal{C}\to \cat{Ch}(\mathcal{A})$ and $G:\mathcal{C}\to \cat{Ch}(\mathcal{B})$ and consider $\langle F,G\rangle:\mathcal{C}\to \cat{Ch}(\mathcal{A})\times \cat{Ch}(\mathcal{B})$. Under the isomorphism $\cat{Ch}(\mathcal{A})\times \cat{Ch}(\mathcal{B}) \cong \cat{Ch}(\mathcal{A}\times \mathcal{B})$ we obtain an isomorphisms
    %%
    \begin{equation*}
        D_1(\langle F,G\rangle) \cong \langle D_1(F),D_1(G)\rangle
    \end{equation*}
    %%
\end{lem}
\begin{proof}
    Since limits are computed componentwise in the product category $\mathcal{A}\times \mathcal{B}$ we have that $\text{cr}_n(\langle F,G\rangle) \cong \langle \text{cr}_n(F),\text{cr}_n(G)\rangle$, from which the result follows since the bicomplex defining $D_1$ is expressed in terms of cross-effects.
\end{proof}

