\section{Linear Approximations}


In this section we define the linear approximation of a functor $F:\mathcal{B}\rightarrow \cat{Ch}(\mathcal{A})$ as a functor $D_1(F)$ in $\cat{AbCat}_{\cat{Ch}}$. This construction will coincie, up to homotopy, with the homotopy fiber ofthe map $q_1:P_1(F)\to P_0(F)$. All of the properties we will describe for $D_1(F)$ are developed only up to \rd{natural} chain homotopy equivalence.

\begin{defn}[label=defn:linearization]
    The \textbf{linearization} of $F:\mathcal{B}\rightarrow \cat{Ch}(\mathcal{A})$ is the functor $D_1(F):\mathcal{B}\rightarrow \mathcal{A}$ given as the totalization of the explicit chain complex of chain complexes $(D_1(F)_\bullet,\partial_\bullet)$ where:
    %%
    \begin{equation*}
        D_1(F)_k := \left\{\begin{array}{cc} C_2^{\times k}(F) & k \geq 1 \\ \text{cr}_1(F) & k = 0 \\ 0 & \text{else} \end{array}\right.
    \end{equation*}
    %%
    and the differential $\partial_1:D_1(F)_1\rightarrow D_1(F)_0$ is given by $\rho_1$, while for $k \geq 2$, $\partial_k:D_1(F)_k\rightarrow D_1(F)_{k-1}$ is given by the alternating sum $\sum_{i=0}^{k-1}(-1)^iC_2^{\times i}\epsilon_{C_2^{\times (k-1-i)}}$. Note $\rho_1 = \epsilon$ since $C_2(F) \cong C_2(\text{cr}_1(F))$.
\end{defn}

Recall that $P_0(F)$ is naturally chain homotopy equivalent to $F(0)$. Using this model the induced map $q_0:P_1(F)\to P_0(F)\to F(0)$ is surjective.

\begin{proof}{Proof of Surjectivity}

\end{proof}

This implies that the kernel of $q_1$ is a model of the homotopy fiber (\textbf{WILL ADD SECTION ON HOMOTOPY FIBERS}). Note that our definition of $D_1(F)$ coincides with $P_1(\text{cr}_1(F))$ due to Lemma \ref{lem:idempotCr1}. 

To start understanding the linear approximation we consider an example:

\begin{eg}
    We consider the affine functor $F(X) = A\oplus X$. Recall $\text{cr}_2(F) \cong 0$ and $\text{cr}_1(F) \cong \id$. This implies that $D_1(F)$ is chain homotopy equivalent to $\id$.
\end{eg}

We begin by inspecting properties of $D_1$. First we obtain immediate results since $D_1\cong P_1\circ \text{cr}_1$.

\begin{prop}[label=prop:D1Exact]
    \begin{itemize}
        \item[(i)] $D_1:\text{Fun}(\mathcal{B},\mathcal{A})\to \text{Fun}(\mathcal{B},\mathcal{A})$ is exact
        \item[(ii)] $D_1$ preserves (natural) chain homotopies, chain homotopy equivalences, and contractibility.
    \end{itemize}
\end{prop}
