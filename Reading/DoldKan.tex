In this section we collect properties related to the Dold-Kan equivalence, and unify notation for use in the remainder of the notes. We first recall the statement of the general Dold-Kan equivalence from \cite[Thm 14.24.3]{StacksProject}.

\begin{thm}[label=thm:DoldKanEquiv]{}
    For $\mathcal{A}$ an abelian category, there is an equivalence of categories $N :\mathcal{A}^{\Delta^{op}}\simeq \cat{Ch}(\mathcal{A}): \Gamma$ \cite{StacksProject}. Let $\eta:1_{\cat{Ch}(\mathcal{A})}\Rightarrow N\circ \Gamma$ and $\varepsilon:\Gamma\circ N\Rightarrow 1_{\mathcal{A}^{\Delta^{op}}}$ be the unit and counit for the equivalence, which can be chosen to satisfy the triangle identities.

    $N$ is given explicitly on a simplicial object $X$ by
    %%
    \begin{equation*}
        N(X)_n := \left\{\begin{array}{cc} \bigcap_{i=0}^{n-1}\ker(d_n^i) & n \geq 1 \\ X_0 & n = 0 \\ 0 & n < 0 \end{array}\right.
    \end{equation*}
    %%
    with differential given by $(-1)^nd_n^n:N(X)_n\rightarrow N(X)_{n-1}$. $N$ is given on arrows by restriction. On the other hand, for a chain complex $A_\bullet$, with boundary maps $d_{A,n}$, $\Gamma(A_\bullet)$ is given on objects by 
    %%
    \begin{equation*}
        \Gamma(A_\bullet)_n = \bigoplus_{\alpha\in I_n}A_{k(\alpha)}
    \end{equation*}
    %%
    where $I_n = \{\alpha:[n]\rightarrow \N\vert \ran(\alpha) = [k(\alpha)]\}$ where $k(\alpha)$ is the maximum element in the image. For a monotonic map $\varphi:[m]\rightarrow [n]$, we define $\Gamma$ using the universal property of the biproduct by
    %%
    \begin{equation*}
        \pi_\beta\circ \Gamma(A_\bullet)(\varphi)\circ \iota_\alpha := \left\{\begin{array}{cc} 0 & \alpha\circ \varphi \notin I_m \\ 0 & \alpha\circ \varphi \in I_m, k(\alpha\circ \varphi) \neq k(\alpha),k(\alpha)-1 \\ 0 & k(\alpha\circ \varphi) \neq \beta \\ 1_{A_{k(\alpha)}} & \alpha\circ \varphi \in I_m, k(\alpha\circ \varphi) = k(\alpha) \\  (-1)^{k(\alpha)}d_{A,k(\alpha)} & \alpha\circ \varphi \in I_m, k(\alpha\circ \varphi) = k(\alpha)-1\end{array}\right.
    \end{equation*}
    %%
\end{thm}

Occasionally we will denote the Dold-Kan equivalence functors for a category $\mathcal{A}$ by $N_\mathcal{A}$ and $\Gamma_\mathcal{A}$ if multiple categories are involved, or if the category isn't clear from the context. We begin by reciting certain properties that the functors in the Dold-Kan equivalence satisfy \cite[Section 14.24]{StacksProject}.

\begin{lem}[label=lem:refl]
    The functor $N$ reflects isomorphisms, injections, and surjections.
\end{lem}

Recall since $N$ and $\Gamma$ are equivalences, in particular this means that they are full, faithful, and essentially surjective. They also satisfy a number of other properties.

%%
\begin{lem}[label=lem:doldKanProps]
    For any abelian category $\mathcal{A}$, the Dold-Kan functors satisfy the following properties:
    %%
    \begin{itemize}
        \item $N$ is exact
        \item $N$ sends simplicial homotopies to chain homotopies, and hence sends simplicially homotopic maps to chain homotopic maps
        \item $N$ reflects chain homotopies
        \item $\Gamma$ sends chain homotopies to simplicial homotopies
    \end{itemize}
\end{lem}

The majority of our results will involve the interaction of $N$ and $\Gamma$ with standard functors and natural transformations, which we collect the definitions of here for simplicity:

\begin{defn}[label=defn:Diag]{}
    Let $(-)\Sob:2\cat{Ab}\rightarrow 2\cat{Ab}$ denote the pseudomonad constructed in Section \ref{sec:simpMon} which sends an abelian category to its category of simplicial objects, a functor to its post composition, and natural transformation to its post composition through whiskering with simplicial objects.

    From Section \ref{sec:simpMon} we also have a natural transformation, $\Delta_{(-)}:(-)\Sob\circ (-)\Sob\Rightarrow (-)\Sob$, which gives the multiplication of the pseudomonad, and on objects sends a bisimplicial complex to its diagonal.


    Another important pseudonatural transformation is given by $\iota_{(-)}:1_{2\cat{Ab}}\rightarrow (-)\Sob$, which on an abelian category $\mathcal{A}$ has component $\iota_\mathcal{A}$ which sends objects to the constant simplicial object for them.
\end{defn}

The first result we give is used in Section \ref{sec:ptwiseNat} during the proof that $\lhd$ gives a well-defined composition on $\cat{AbCat}_{\cat{Ch}}$.

\begin{lem}[label=lem:gammaDeg]
    For any abelian category $\mathcal{A}$, $\Gamma_\mathcal{A}\circ \deg_0^\mathcal{A} = \iota_\mathcal{A}$.
\end{lem}
\begin{proof}
    Let $A \in \mathcal{A}$. Then $\Gamma_\mathcal{A}(\deg_0^\mathcal{A}(A))([n]) = A$ for all $n$, since $\deg_0^\mathcal{A}(A)$ contains $A$ concentrated in degree $0$ and their is a unique $\alpha:[n]\rightarrow [0]$ for each $n$. Next, for each $\alpha:[m]\rightarrow [n]$, $\Gamma(\deg_0^\mathcal{A}(A))(\alpha) = 1_A$ from the piecewise definition of $\Gamma$ on arrows.

    Next, let $f:A\rightarrow B$ be a map in $\mathcal{A}$. Then $\deg_0^\mathcal{A}(f)$ is the map concentrated in degree $0$. Then $\Gamma(\deg_0^\mathcal{A}(f))([n]) = f$ for each $n$. It follows that $\Gamma_\mathcal{A}\circ \deg_0^\mathcal{A} = \iota_\mathcal{A}$, as claimed.
\end{proof}



In addition to the Dold-Kan correspondence, we have two other natural functors from simplicial categories to categories of chain complexes.

\begin{prop}[label=prop:simpToNiceCh]
    Let $\mathcal{A}$ be an abelian category. Then there is a functor 
    %%
    \begin{equation*}
        \mathfrak{C}:\mathcal{A}\Sob\to \cat{Ch}(\mathcal{A})
    \end{equation*}
    %%
\end{prop}
\begin{proof}
    Let $A \in \mathcal{A}_0$. We define $\mathfrak{C}(A)_n := A_n$ for $n \in \N$, with boundary map
    %%
    \begin{equation*}
        \partial_n^A:A_n\xrightarrow{\sum_{i=0}^n(-1)^iA(d_i^n)} A_{n-1}
    \end{equation*}
    %%
    To show that this indeed gives a boundary map observe that
    %%
    \begin{align*}
        \left[\sum_{i=0}^n(-1)^iA(d_i^n)\right]\circ \left[\sum_{i=0}^{n+1}(-1)^iA(d_i^{n+1})\right] &= \sum_{i=0}^n\sum_{j=0}^{n+1}(-1)^{i+j}A(d_i^n)\circ A(d_j^{n+1}) \\ 
        &= \sum_{i=0}^n\sum_{j > i}^{n+1}(-1)^{i+j}A(d_i^n)\circ A(d_j^{n+1}) \\
        &+ \sum_{i=0}^n\sum_{j\leq i}(-1)^{i+j}A(d_i^n)\circ A(d_j^{n+1}) \\
        &= \sum_{i=0}^n\sum_{j > i}^{n+1}(-1)^{i+j}A(d_{j-1}^n)\circ A(d_i^{n+1}) \\
        &+ \sum_{i=0}^n\sum_{j\leq i}(-1)^{i+j}A(d_i^n)\circ A(d_j^{n+1}) \\ 
        &= \sum_{i=0}^n\sum_{j \geq i}^{n}(-1)^{i+j+1}A(d_{j}^n)\circ A(d_i^{n+1}) \\
        &+ \sum_{i=0}^n\sum_{j\leq i}(-1)^{i+j}A(d_i^n)\circ A(d_j^{n+1}) \\ 
        &= -\sum_{j=0}^n\sum_{i\leq j}(-1)^{i+j}A(d_{j}^n)\circ A(d_i^{n+1}) \\
        &+ \sum_{i=0}^n\sum_{j\leq i}(-1)^{i+j}A(d_i^n)\circ A(d_j^{n+1}) \\
        &= 0
    \end{align*}
    %%
    For a map $f:A\to B$ of simplicial objects, $\mathfrak{C}(f)_n := f_n$, which commutes with the differentials since the composition is bilinear in an abelian category and $f$ commutes with the face operators being a map of simplicial sets. Since the components of $\mathfrak{C}(f)$ are given by the components of $f$, $\mathfrak{C}$ is functorial, completing the construction.
\end{proof}

Finally, we have a functor $D:\mathcal{A}\Sob\to \cat{Ch}(\mathcal{A})$ given by $D(A)_n$ being the colimit over the images of the degeneracies $s_i^{n-1}$. From~\cite[Lem 8.3.7]{weibel_1994} we have that $\mathfrak{C}\cong D\oplus N$. Further, for all simplicial objects $A \in \mathcal{A}_0$, $D(A)$ is acyclic by~\cite[Thm 8.3.8]{weibel_1994}.


\begin{prop}[label=prop:constDoldKan]
    Let $A_\bullet \in \mathcal{A}\Sob$ be the constant simplicial object at an object $A \in \mathcal{A}_0$. Then $N(A_\bullet)\cong  \deg_0^\mathcal{A}(A)$.
\end{prop}
\begin{proof}
    By definition we have that for any $n \geq 1$, $N(A_\bullet)_n = \bigcap_{i=0}^{n-1}\ker(1_A) \cong 0$, while $N(A_\bullet)_0 = A_0 = A$. Thus as all boundary maps are vacuously zero maps, $N(A_\bullet)\cong \deg_0^\mathcal{A}(A)$. 
\end{proof}

To compare with the definition in~\cite{Johnson2003DerivingCW} we must introduce the concept of a mapping cone for chain complexes.

\begin{defn}{ \cite[Sec. 1.5.1]{weibel_1994}}
    The mapping cone for a chain map $f_\bullet:B_\bullet\to A_\bullet$ is given by the totalization of the bicomplex
    \[\begin{tikzcd}
    	\cdots & 0 & 0 & 0 & 0 & \cdots \\
    	\cdots & 0 & {B_0} & {B_1} & {B_2} & \cdots \\
    	\cdots & 0 & {A_0} & {A_1} & {A_2} & \cdots \\
    	\cdots & 0 & 0 & 0 & 0 & \cdots
    	\arrow[from=3-3, to=3-2]
    	\arrow[from=3-2, to=3-1]
    	\arrow["{\partial_1^A}", from=3-4, to=3-3]
    	\arrow["{\partial_2^A}", from=3-5, to=3-4]
    	\arrow[from=3-6, to=3-5]
    	\arrow[from=3-3, to=4-3]
    	\arrow[from=3-4, to=4-4]
    	\arrow[from=3-5, to=4-5]
    	\arrow[from=3-2, to=4-2]
    	\arrow[from=4-3, to=4-2]
    	\arrow[from=4-2, to=4-1]
    	\arrow[from=4-4, to=4-3]
    	\arrow[from=4-5, to=4-4]
    	\arrow[from=4-6, to=4-5]
    	\arrow["{\partial_1^B}"', from=2-4, to=2-3]
    	\arrow["{\partial_2^B}"', from=2-5, to=2-4]
    	\arrow[from=2-6, to=2-5]
    	\arrow[from=2-3, to=2-2]
    	\arrow[from=2-2, to=3-2]
    	\arrow[from=2-2, to=2-1]
    	\arrow["{f_0}"', from=2-3, to=3-3]
    	\arrow["{f_1}"', from=2-4, to=3-4]
    	\arrow["{f_2}"', from=2-5, to=3-5]
    	\arrow[from=1-2, to=1-1]
    	\arrow[from=1-6, to=1-5]
    	\arrow[from=1-5, to=2-5]
    	\arrow[from=1-5, to=1-4]
    	\arrow[from=1-4, to=2-4]
    	\arrow[from=1-4, to=1-3]
    	\arrow[from=1-3, to=2-3]
    	\arrow[from=1-3, to=1-2]
    	\arrow[from=1-2, to=2-2]
    \end{tikzcd}\]
    where $A_\bullet$ is located in the $0$th row and $B_\bullet$ is located in the $1$st row.
\end{defn}

The boundary maps for the mapping cone can be described by a simple $2 \times 2$ matrix.
%%
\begin{equation*}
	{B_{n-1}\oplus A_n} \xrightarrow{\begin{pmatrix} \partial^B_{n-1} & 0 \\ (-1)^{n-1}f_{n-1} & \partial_n^A \end{pmatrix}} {B_{n-2}\oplus A_{n-1}}
\end{equation*}
%%

\begin{rmk}
    Let $f_\bullet:B_\bullet \to \deg_0^\mathcal{A}(A)$ be a chain map for $A \in \mathcal{A}_0$. Then $\text{cone}(f_\bullet)$ is the chain complex that in degree $n \geq 1$ is $B_{n-1}$, and in degree $0$ is $A$. For $n \geq 1$ the $n+1$st boundary map is $B_n\xrightarrow{\partial_n^B}B_{n-1}$, while the $1$st boundary map is $B_0\xrightarrow{f_0}A$. 
\end{rmk}

\begin{prop}[label=prop:mapBetweenCones]
    Let $f_\bullet:B_\bullet\to \deg_0^\mathcal{A}(A)$, $g_\bullet:C_\bullet\to \deg_0^\mathcal{A}(A)$, and $h_\bullet:B_\bullet\to C_\bullet$ be chain maps such that $g_\bullet\circ h_\bullet = f_\bullet$. Then we have a map $\overline{h}_\bullet:\text{cone}(f_\bullet)\to \text{cone}(g_\bullet)$ such that $\overline{h}_n = h_{n-1}:B_{n-1}\to C_{n-1}$ for $n \geq 1$, and $\overline{h}_0 = 1_A$.
\end{prop}
\begin{proof}
    Since $h_\bullet$ is a chain map, the described map is a chain map as 
    \[\begin{tikzcd}
        {B_0} & A \\
        {C_0} & A
        \arrow["{f_0}", from=1-1, to=1-2]
        \arrow["{h_0}"', from=1-1, to=2-1]
        \arrow["{g_0}"', from=2-1, to=2-2]
        \arrow[Rightarrow, no head, from=1-2, to=2-2]
    \end{tikzcd}\]
    commutes by assumption.
\end{proof}

By Proposition~\ref{prop:simpMonMapToId} we have for a comonad $(C,\epsilon,\delta)$ on a category $\mathcal{C}$ with a natural transformation $\overline{\epsilon}:C^{\bullet+1}\to 1_\mathcal{C}^{\bullet+1}$ to the functor which assigns the constant complex associated to each object $A \in \mathcal{C}_0$ given by successive application of the counit. Composing with the Dold-Kan equivalence $N_\mathcal{C}$ we obtain a functor
%%
\begin{equation*}
    N_\mathcal{C}\circ C^{\bullet+1}:\mathcal{C}\to \cat{Ch}(\mathcal{C}),\;\; N_\mathcal{C}(C^{\bullet+1}(A))_n = \left\{\begin{array}{cc} \bigcap_{i=0}^{n-1}\ker(C^i\epsilon_{C^{n-i}(A)}), & n \geq 1 \\ C(A), & n = 0 \end{array}\right.
\end{equation*}
%%
with boundary maps
%%
\begin{equation*}
    \bigcap_{i=0}^{n}\ker(C^i\epsilon_{C^{n+1-i}(A)})\xrightarrow{(-1)^{n+1}C^{n+1}\epsilon_{A}} \bigcap_{i=0}^{n-1}\ker(C^i\epsilon_{C^{n-i}(A)}),\; n \geq 1
\end{equation*}
%%
and 
%%
\begin{equation*}
    \ker(\epsilon_{C(A)})\xrightarrow{-C\epsilon_A} C(A)
\end{equation*}
%%
The map $N_\mathcal{C}(\overline{\epsilon}):N_\mathcal{C}\circ C^{\bullet+1}\to N_\mathcal{C}\circ 1_\mathcal{C}^{\bullet+1}$ is then given by $N_\mathcal{C}(\overline{\epsilon})_A$ which has $n$th component $0$ for $n \geq 1$, and $0$th component is given by the natural transformation $\epsilon_A$. From the work above the chain complex $\text{cone}(\overline{\epsilon})$ is given by 
%%
\[\begin{tikzcd}
	\hspace{-30pt} {\bigcap_{i=0}^{n-1}\ker(C^i\epsilon_{C^{n-i}(A)})} & \cdots & {\ker(\epsilon_{C^2(A)})\cap \ker(C\epsilon_{C(A)})} & {\ker(\epsilon_{C(A)})} & {C(A)} & A
	\arrow["{-C\epsilon_A}", from=1-4, to=1-5]
	\arrow["{\epsilon_A}", from=1-5, to=1-6]
	\arrow["{C^2\epsilon_A}", from=1-3, to=1-4]
	\arrow["{-C^3\epsilon_A}", from=1-2, to=1-3]
	\arrow["{(-1)^nC^n\epsilon_A}", from=1-1, to=1-2]
\end{tikzcd}\]
%%

On the other hand, using the functor $\mathfrak{C}$ we also obtain
%%
\begin{equation*}
    \mathfrak{C}_\mathcal{C}\circ C^{\bullet+1}:\mathcal{C}\to\cat{Ch}(\mathcal{C}),\;\; \mathfrak{C}_\mathcal{C}(C^{\bullet+1}(A))_n = C^{n+1}(A)
\end{equation*}
%%
with boundary maps
%%
\begin{equation*}
    C^{n+2}(A)\xrightarrow{\sum_{i=0}^n(-1)^iC^i\epsilon_{C^{n-i}(A)}} C^{n+1}(A)
\end{equation*}
%%
Note that for $n = 0$ we have $C^2(A)\xrightarrow{\epsilon_{C(A)}-C\epsilon_A}C(A)$. We then have a natural transformation 
%%
\begin{equation*}
    \mathfrak{C}_\mathcal{C}\circ C^{\bullet+1}\to N_\mathcal{C}\circ 1_\mathcal{C}^{\bullet+1}
\end{equation*}
%%
Given by zero maps in degrees $n \geq 1$, and $\epsilon_A$ in degree $0$, which is a map of chain complexes for each $A$ since 
%%
\begin{equation*}
    \epsilon_A\circ (\epsilon_{C(A)}-C\epsilon_A) = \epsilon_A\circ \epsilon_{C(A)}-\epsilon_A\circ C\epsilon_A = 0
\end{equation*}
%%
by naturality of $\epsilon$. Let us denote this map by $\widetilde{\epsilon}:\mathfrak{C}_\mathcal{C}\circ C^{\bullet+1}\to N_\mathcal{C}\circ 1_{\mathcal{C}}^{\bullet+1}$. Taking the mapping cone we obtain the chain complex $\text{cone}(\widetilde{\epsilon})$ which at $A$ is given by 
%%
\begin{equation*}
    \cdots \to C^3(A)\xrightarrow{\epsilon_{C^2(A)}-C\epsilon_{C(A)}+C^2\epsilon_A}C^2(A)\xrightarrow{\epsilon_{C(A)}-C\epsilon_A}C(A)\xrightarrow{\epsilon_A}A
\end{equation*}
%%
Note that this is precisely the functor $C^{\cat{Ch}}$. 


Let $\iota_{\mathcal{C},C^{\bullet+1}}:N_\mathcal{C}\circ C^{\bullet+1}\to \mathfrak{C}_\mathcal{C}\circ C^{\bullet+1}$ denote the natural inclusion of the normalized chain complex into the full chain complex, and let $\pi_{\mathcal{C},C^{\bullet+1}}:\mathfrak{C}\circ C^{\bullet+1}\to N_\mathcal{C}\circ C^{\bullet+1}$ be the natural projection. Then $\pi_\mathcal{C}\circ \iota_\mathcal{C} = 1_{N_\mathcal{C}\circ C^{\bullet+1}}$. From our above work we obtain corresponding maps of augmented chain complexes with one direction composing to the identity and the other giving the composite
\[\begin{tikzcd}
	\cdots & {C^3(A)} & {C^2(A)} & {C(A)} & A \\
	\cdots & {\ker(\epsilon_{C^2(A)})\cap \ker(C\epsilon_{C(A)})} & {\ker(\epsilon_{C(A)})} & {C(A)} & A \\
	\cdots & {C^3(A)} & {C^2(A)} & {C(A)} & A
	\arrow["{-C\epsilon_A}", from=2-3, to=2-4]
	\arrow["{\epsilon_A}", from=2-4, to=2-5]
	\arrow["{C^2\epsilon_A}", from=2-2, to=2-3]
	\arrow[from=2-1, to=2-2]
	\arrow[Rightarrow, no head, from=2-5, to=3-5]
	\arrow["{\epsilon_A}"', from=3-4, to=3-5]
	\arrow[Rightarrow, no head, from=2-4, to=3-4]
	\arrow["{\epsilon_{C(A)}-C\epsilon_A}"', from=3-3, to=3-4]
	\arrow["{\iota_{\mathcal{C},C^{\bullet+1}(A)_2}}"', from=2-3, to=3-3]
	\arrow["{\iota_{\mathcal{C},C^{\bullet+1}(A)_3}}"', from=2-2, to=3-2]
	\arrow["{\epsilon_{C^2(A)}-C\epsilon_{C(A)}+\epsilon_{C^2(A)}}"', from=3-2, to=3-3]
	\arrow[from=3-1, to=3-2]
	\arrow[from=1-1, to=1-2]
	\arrow[Rightarrow, no head, from=1-5, to=2-5]
	\arrow[Rightarrow, no head, from=1-4, to=2-4]
	\arrow["{\epsilon_A}", from=1-4, to=1-5]
	\arrow["{\epsilon_{C(A)}-C\epsilon_A}", from=1-3, to=1-4]
	\arrow["{\pi_{\mathcal{C},C^{\bullet+1}(A)_2}}"', from=1-3, to=2-3]
	\arrow["{\pi_{\mathcal{C},C^{\bullet+1}(A)_3}}"', from=1-2, to=2-2]
	\arrow["{\epsilon_{C^2(A)}-C\epsilon_{C(A)}+\epsilon_{C^2(A)}}", from=1-2, to=1-3]
\end{tikzcd}\]
We wish to show that this gives a natural chain homotopy equivalence. However this follows from more general results.

\begin{prop}[label=prop:chainHomotopEquivOfAug]
    Let $f_\bullet:B_\bullet\to \deg_0^\mathcal{A}(A)$, $g_\bullet:C_\bullet\to \deg_0^\mathcal{A}(A)$, and $h_\bullet,k_\bullet:B_\bullet\to C_\bullet$ be chain maps such that $g_\bullet\circ h_\bullet = f_\bullet = g_\bullet\circ k_\bullet$. If $h_\bullet$ and $k_\bullet$ are chain homotopy equivalent, then $\overline{h}_\bullet$ and $\overline{k}_\bullet$ are also chain homotopy equivalent.
\end{prop}
\begin{proof}
    Let $s_n:B_n\to C_{n+1}$ be chain homotopies from $h_\bullet$ to $k_\bullet$, so 
    %%
    \begin{equation*}
        \partial_{n+1}^C \circ s_n+s_{n-1} \circ \partial_n^B = h_n-k_n
    \end{equation*}
    %%
    Then we define $\overline{s}_n:B_{n-1}\to C_n$ by $\overline{s}_{n+1} = s_n$ for $n \geq 0$, and $\overline{s}_0:A\to C_0$ equal to $0$ so that $g_0\circ 0 = 1_A-1_A = 0$, and 
    %%
    \begin{equation*}
        \partial_1^C\circ s_0 + f_0\circ 0 = \partial_1^C\circ s_0 = h_0-k_0
    \end{equation*}
    %%
    Thus we obtain a homotopy from $\overline{h}_\bullet$ to $\overline{k}_\bullet$, as desired.
\end{proof}

The desired result now follows by~\cite[Thm 2.5]{goerss-jardine}. 

\begin{lem}[label=lem:natChainHomotop]
    The inclusion $\iota_{\mathcal{C}}:N_\mathcal{C}\to \mathfrak{C}_\mathcal{C}$ and projection $\pi_{\mathcal{C}}:\mathfrak{C}_\mathcal{C}\to N_\mathcal{C}$ constitute a natural chain homotopy equivalence.
\end{lem}

We will show this explicitly in the case of general abelian categories \textbf{TBC}.



\subsection{General Polynomial Approximations from Cotriples}

In this section we develop of a general theory of polynomial approximations associated with additive adjunctions between abelian categories. To this end let
\[\begin{tikzcd}
	{(\eta,\epsilon):} & {\mathcal{A}} & {\mathcal{B}}
	\arrow[""{name=0, anchor=center, inner sep=0}, "R"', curve={height=6pt}, from=1-2, to=1-3]
	\arrow[""{name=1, anchor=center, inner sep=0}, "L"', curve={height=6pt}, from=1-3, to=1-2]
	\arrow["\dashv"{anchor=center, rotate=-90}, draw=none, from=1, to=0]
\end{tikzcd}\]
be an adjunction between abelian categories such that $L$ and $R$ are additive functors. Let $C = L\circ R$ be the comonad with co-unit $\epsilon:L\circ R\to 1_\mathcal{A}$ and co-multiplication $\delta = L\eta_R:L\circ R\to L\circ R\circ L\circ R$. Note $C$ is additive being the composite of additive functors. By Lemmas~\ref{lem:funcActChainHori}~\ref{lem:addFuncPres}~\ref{lem:ChConsPres} we have another additive functor $C^{\cat{Ch}}$ and an adjunction 
\[\begin{tikzcd}
	{\cat{Ch}(\mathcal{A})} & {\cat{Ch}(\mathcal{B})}
	\arrow[""{name=0, anchor=center, inner sep=0}, "{\cat{Ch}(R)}"', curve={height=6pt}, from=1-1, to=1-2]
	\arrow[""{name=1, anchor=center, inner sep=0}, "{\cat{Ch}(L)}"', curve={height=6pt}, from=1-2, to=1-1]
	\arrow["\dashv"{anchor=center, rotate=-90}, draw=none, from=1, to=0]
\end{tikzcd}\]
with comonad $\cat{Ch}(C)$ which preserves chain homotopies. As in the case of cross-effects we define the notion of degree.

\begin{defn}[label=defn:degreeC]
    We say a chain complex $A_\bullet \in \cat{Ch}(\mathcal{A})$ is of degree $C$ if $\cat{Ch}(R)(A_\bullet)$ is chain contractible.
\end{defn}

Next we can define the polynomial approximation associated with the comonad $C$.

\begin{defn}[label=defn:polCApprox]
    We define the polynomial approximation for $C$, $P_C:\cat{Ch}(\mathcal{A})\to \cat{Ch}(\mathcal{A})$, by the composite
    %%
    \begin{equation*}
        P_C = \text{Tot}_\mathcal{A}\circ \cat{Ch}(C^{\cat{Ch}})
    \end{equation*}
    %%
\end{defn}

By Lemmas~\ref{lem:addFuncPres}~\ref{lem:ChConsPres} $P_C$ is additive, and further preserves chain homotopies. Note that we have a natural transformation $I_C:\deg_0^\mathcal{A}\to C^{\cat{Ch}}$ such that at $A \in \mathcal{A}_0$, $I_{C,A}$ is 
%%
\[\begin{tikzcd}
	\cdots & 0 & 0 & A \\
	\cdots & {C^2(A)} & {C(A)} & A
	\arrow[from=1-2, to=1-3]
	\arrow[from=1-3, to=1-4]
	\arrow[from=1-1, to=1-2]
	\arrow[Rightarrow, no head, from=1-4, to=2-4]
	\arrow["{\epsilon_A}"', from=2-3, to=2-4]
	\arrow[from=1-3, to=2-3]
	\arrow[from=1-2, to=2-2]
	\arrow[from=2-1, to=2-2]
	\arrow["{\epsilon_{C(A)}-C\epsilon_A}"', from=2-2, to=2-3]
\end{tikzcd}\]
%%
Then we define $p_{C}:= \text{Tot}_\mathcal{A}(\cat{Ch}(I_C)):1_{\cat{Ch}(\mathcal{A})}\to P_C$ using the isomorphism $\text{Tot}_\mathcal{A}\circ \deg_0^{\cat{Ch}(\mathcal{A})}\cong 1_{\cat{Ch}(\mathcal{A})}$. 


Observe that
%%
\begin{align*}
	P_C(p_C):P_C\to P_C^2,\;\; P_C(p_C) &= \text{Tot}_\mathcal{A}\cat{Ch}(C^{\cat{Ch}})\text{Tot}_\mathcal{A}\cat{Ch}(I_C)
\end{align*}
%%
First applying $\cat{Ch}(I_C)$ to $A_\bullet$ we obtain the map of bicomplexes
\[\begin{tikzcd}
	& 0 && 0 && {A_2} \\
	{C^2(A_2)} && {C(A_2)} && {A_2} \\
	& 0 && 0 && {A_1} \\
	{C^2(A_1)} && {C(A_1)} && {A_1} \\
	& 0 && 0 && {A_0} \\
	{C^2(A_0)} && {C(A_0)} && {A_0}
	\arrow[from=3-6, to=5-6]
	\arrow[from=1-6, to=3-6]
	\arrow["{\epsilon_{C(A_0)}-C\epsilon_{A_0}}"', from=6-1, to=6-3]
	\arrow["{\epsilon_{A_0}}"', from=6-3, to=6-5]
	\arrow[Rightarrow, no head, from=5-6, to=6-5]
	\arrow[""{name=0, anchor=center, inner sep=0}, from=4-5, to=6-5]
	\arrow[Rightarrow, no head, from=3-6, to=4-5]
	\arrow[Rightarrow, no head, from=1-6, to=2-5]
	\arrow[""{name=1, anchor=center, inner sep=0}, from=2-5, to=4-5]
	\arrow[from=1-4, to=1-6]
	\arrow[from=1-2, to=1-4]
	\arrow[from=4-1, to=6-1]
	\arrow[""{name=2, anchor=center, inner sep=0}, from=4-3, to=6-3]
	\arrow[""{name=3, anchor=center, inner sep=0}, "{\epsilon_{C(A_1)}-C\epsilon_{A_1}}", from=4-1, to=4-3]
	\arrow[""{name=4, anchor=center, inner sep=0}, "{\epsilon_{A_1}}", from=4-3, to=4-5]
	\arrow[""{name=5, anchor=center, inner sep=0}, "{\epsilon_{A_2}}", from=2-3, to=2-5]
	\arrow[""{name=6, anchor=center, inner sep=0}, "{\epsilon_{C(A_2)}-C\epsilon_{A_2}}", from=2-1, to=2-3]
	\arrow[from=2-1, to=4-1]
	\arrow[""{name=7, anchor=center, inner sep=0}, from=2-3, to=4-3]
	\arrow[from=3-4, to=4-3]
	\arrow[from=3-2, to=4-1]
	\arrow[from=1-2, to=2-1]
	\arrow[from=1-4, to=2-3]
	\arrow[from=5-4, to=6-3]
	\arrow[from=5-2, to=6-1]
	\arrow[shorten >=11pt, no head, from=1-4, to=5]
	\arrow[shorten <=3pt, from=5, to=3-4]
	\arrow[shorten >=10pt, no head, from=1-2, to=6]
	\arrow[shorten >=5pt, no head, from=3-4, to=1]
	\arrow[shorten <=5pt, from=1, to=3-6]
	\arrow[shorten >=5pt, no head, from=5-4, to=0]
	\arrow[shorten <=5pt, from=0, to=5-6]
	\arrow[shorten <=3pt, from=6, to=3-2]
	\arrow[shorten >=5pt, no head, from=3-2, to=7]
	\arrow[shorten <=5pt, from=7, to=3-4]
	\arrow[shorten >=11pt, no head, from=3-4, to=4]
	\arrow[shorten <=3pt, from=4, to=5-4]
	\arrow[shorten >=5pt, no head, from=5-2, to=2]
	\arrow[shorten <=5pt, from=2, to=5-4]
	\arrow[shorten >=10pt, no head, from=3-2, to=3]
	\arrow[shorten <=3pt, from=3, to=5-2]
\end{tikzcd}\]
where the vertical maps are the $A_\bullet$ boundary maps possibly after applying $C^n$. Taking the totalization we obtain the map of chain complexes
%%
\[\begin{tikzcd}
	\cdots & {A_2} & {A_1} & {A_0} \\
	\cdots & {C^2(A_0)\oplus C(A_1)\oplus A_2} & {C(A_0)\oplus A_1} & {A_0}
	\arrow["{\partial_2^A}", from=1-2, to=1-3]
	\arrow[from=1-1, to=1-2]
	\arrow["{\partial_1^A}", from=1-3, to=1-4]
	\arrow[Rightarrow, no head, from=1-4, to=2-4]
	\arrow["{i_2}", from=1-3, to=2-3]
	\arrow["{i_3}", from=1-2, to=2-2]
	\arrow[from=2-1, to=2-2]
	\arrow["{\zeta_2}"', from=2-2, to=2-3]
	\arrow["{\zeta_1}"', from=2-3, to=2-4]
\end{tikzcd}\]
%%
where $\zeta_n$ is the $n\times n+1$ matrix 
%%
\begin{equation*}
    \begin{pmatrix}
        \sum_{i=0}^{n-1}(-1)^iC^i\epsilon_{C^{n-1-i}(A_0)} & (-1)^{n-1}C^{n-1}(\partial_1^A) & 0 & \cdots & 0 \\
        0 & \sum_{i=0}^{n-2}(-1)^iC^i\epsilon_{C^{n-2-i}(A_1)} & (-1)^{n-2}C^{n-2}(\partial_2^A) & \cdots & 0 \\
        \vdots & \vdots & \ddots & \ddots & \vdots \\
        0 & 0 & \cdots & \epsilon_{A_{n-1}} & \partial_{n}^A
    \end{pmatrix}
\end{equation*}
%%
Applying $\cat{Ch}(C^{\cat{Ch}})$ we obtain the map of bicomplexes, where we use the fact that $C$ is additive so preserves direct sums, and by Lemma \textbf{TBD} the co-unit turns into a direct sum of co-units.
%%
\[\hspace{-70pt}\begin{tikzcd}[column sep=8pt]
	& {C^2(A_2)} && {C(A_2)} && {A_2} \\
	{C^4(A_0)\oplus C^3(A_1)\oplus C^2(A_2)} && {C^3(A_0)\oplus C^2(A_1)\oplus C(A_2)} && {C^2(A_0)\oplus C(A_1)\oplus A_2} \\
	& {C^2(A_1)} && {C(A_1)} && {A_1} \\
	{C^3(A_0)\oplus C^2(A_1)} && {C^2(A_0)\oplus C(A_1)} && {C(A_0)\oplus A_1} \\
	& {C^2(A_0)} && {C(A_0)} && {A_0} \\
	{C^2(A_0)} && {C(A_0)} && {A_0}
	\arrow["{\partial_1^A}", from=3-6, to=5-6]
	\arrow["{\partial_2^A}", from=1-6, to=3-6]
	\arrow["{\epsilon_{C(A_0)}-C\epsilon_{A_0}}"', from=6-1, to=6-3]
	\arrow["{\epsilon_{A_0}}"', from=6-3, to=6-5]
	\arrow[Rightarrow, no head, from=5-6, to=6-5]
	\arrow[""{name=0, anchor=center, inner sep=0}, "{\zeta_1}"{description}, from=4-5, to=6-5]
	\arrow["{i_2}", from=3-6, to=4-5]
	\arrow["{i_3}", from=1-6, to=2-5]
	\arrow[""{name=1, anchor=center, inner sep=0}, "{\zeta_2}"{description}, from=2-5, to=4-5]
	\arrow["{\epsilon_{A_2}}", from=1-4, to=1-6]
	\arrow["{\epsilon_{C(A_2)}-C\epsilon_{A_2}}", from=1-2, to=1-4]
	\arrow[""{name=2, anchor=center, inner sep=0}, "{C(\zeta_1)}"{description}, from=4-3, to=6-3]
	\arrow[""{name=3, anchor=center, inner sep=0}, "{(\epsilon_{C^2(A_0)}-C\epsilon_{C(A_0)})\oplus (\epsilon_{C(A_1)}-C\epsilon_{A_1})}"', from=4-1, to=4-3]
	\arrow[""{name=4, anchor=center, inner sep=0}, "{\epsilon_{C(A_0)}\oplus \epsilon_{A_1}}", from=4-3, to=4-5]
	\arrow[""{name=5, anchor=center, inner sep=0}, "{\bigoplus_{j=0}^2\epsilon_{C^{2-j}(A_j)}}", from=2-3, to=2-5]
	\arrow[""{name=6, anchor=center, inner sep=0}, "{\bigoplus_{j=0}^2(\epsilon_{C^{3-j}(A_j)}-C\epsilon_{C^{2-j}(A_j)})}", shift left, from=2-1, to=2-3]
	\arrow["{C^2(\zeta_2)}"', from=2-1, to=4-1]
	\arrow[""{name=7, anchor=center, inner sep=0}, "{C(\zeta_2)}"{description}, from=2-3, to=4-3]
	\arrow["{i_2}", from=3-4, to=4-3]
	\arrow["{i_2}"', from=3-2, to=4-1]
	\arrow["{i_3}"', from=1-2, to=2-1]
	\arrow["{i_3}", from=1-4, to=2-3]
	\arrow[Rightarrow, no head, from=5-4, to=6-3]
	\arrow[Rightarrow, no head, from=5-2, to=6-1]
	\arrow["{C^2(\zeta_1)}"', from=4-1, to=6-1]
	\arrow[shorten >=11pt, no head, from=1-4, to=5]
	\arrow[shorten <=3pt, from=5, to=3-4]
	\arrow[shorten >=9pt, no head, from=1-2, to=6]
	\arrow["{\epsilon_{A_1}}", shorten >=10pt, no head, from=3-4, to=1]
	\arrow[shorten <=11pt, from=1, to=3-6]
	\arrow["{\epsilon_{A_0}}"', shorten >=10pt, no head, from=5-4, to=0]
	\arrow[shorten <=11pt, from=0, to=5-6]
	\arrow[shorten <=3pt, from=6, to=3-2]
	\arrow["{\epsilon_{C(A_1)}-C\epsilon_{A_1}}"{pos=0.4}, shift left, shorten >=12pt, no head, from=3-2, to=7]
	\arrow[shorten <=12pt, from=7, to=3-4]
	\arrow[shorten <=3pt, from=4, to=5-4]
	\arrow["{\epsilon_{C(A_0)}-C\epsilon_{A_0}}"'{pos=0.2}, shift right, shorten >=12pt, no head, from=5-2, to=2]
	\arrow[shorten <=12pt, from=2, to=5-4]
	\arrow[shorten >=3pt, no head, from=3-2, to=3]
	\arrow[shorten <=11pt, from=3, to=5-2]
	\arrow[shorten >=11pt, no head, from=3-4, to=4]
\end{tikzcd}\]
%%
Finally, taking the totalization of this map of bicomplexes we obtain the map of complexes
%%
\[\hspace{-50pt}\begin{tikzcd}
	\cdots & {C^2(A_0)\oplus C(A_1)\oplus A_2} & {C(A_0)\oplus A_1} & {A_0} \\
	\cdots & {C^2(A_0)\oplus (C^2(A_0)\oplus C(A_1))\oplus (C^2(A_0)\oplus C(A_1)\oplus A_2)} & {C(A_0)\oplus (C(A_0)\oplus A_1)} & {A_0}
	\arrow[from=1-1, to=1-2]
	\arrow["{\zeta_2}", from=1-2, to=1-3]
	\arrow["{\zeta_1}", from=1-3, to=1-4]
	\arrow[Rightarrow, no head, from=1-4, to=2-4]
	\arrow["{i_1\oplus 1_{A_1}}", from=1-3, to=2-3]
	\arrow["{\omega_1}"', from=2-3, to=2-4]
	\arrow["{\omega_2}"', from=2-2, to=2-3]
	\arrow[from=2-1, to=2-2]
	\arrow["{i_1\oplus i_1\oplus 1_{A_2}}"', from=1-2, to=2-2]
\end{tikzcd}\]
%%
where $\omega_n$ is given by the formal $n\times n+1$ matrix
%%
\begin{equation*}
    \begin{pmatrix}
        \sum_{i=0}^{n-1}(-1)^iC^i\epsilon_{C^{n-1-i}(A_0)} & (-1)^{n-1}C^{n-1}(\zeta_1) & 0 & \cdots & 0 \\
        0 & \bigoplus_{j=0}^1\sum_{i=0}^{n-2}(-1)^iC^i\epsilon_{C^{n-1-i-j}(A_j)} & (-1)^{n-2}C^{n-2}(\zeta_2) & \cdots & 0 \\
        \vdots & \vdots & \ddots & \ddots & \vdots \\
        0 & 0 & \cdots & \bigoplus_{j=0}^{n-1}\epsilon_{C^{n-1-j}(A_j)} & \zeta_{n}
    \end{pmatrix}
\end{equation*}
%%
and $C^{n-i}(\zeta_i)$ is the $i\times i+1$ matrix 
%%
\begin{equation*}\hspace{-50pt}
    \begin{pmatrix}
        C^{n-i}\sum_{k=0}^{i-1}(-1)^kC^k\epsilon_{C^{i-1-k}(A_0)} & C^{n-i}(-1)^{i-1}C^{i-1}(\partial_1^A) & 0 & \cdots & 0 \\
        0 & C^{n-i}\sum_{k=0}^{i-2}(-1)^kC^k\epsilon_{C^{i-2-k}(A_1)} & C^{n-i}(-1)^{i-2}C^{i-2}(\partial_2^A) & \cdots & 0 \\
        \vdots & \vdots & \ddots & \ddots & \vdots \\
        0 & 0 & \cdots & C^{n-i}(\epsilon_{A_{i-1}}) & C^{n-i}(\partial_{i}^A)
    \end{pmatrix}
\end{equation*}
%%

%%
Note that the complex 
%%
\[\begin{tikzcd}
	\cdots & {C^2(A_0)\oplus C(A_1)\oplus A_2} & {C(A_0)\oplus A_1} & {A_0}
	\arrow["{\zeta_2}"', from=1-2, to=1-3]
	\arrow["{\zeta_1}"', from=1-3, to=1-4]
	\arrow[from=1-1, to=1-2]
\end{tikzcd}\]
%%
is precisely $\text{Tot}_\mathcal{A}\cat{Ch}(C^{\cat{Ch}})(A_\bullet) = P_C(A_\bullet)$.

\vspace{10pt}

On the other hand, $\cat{Ch}(I_C)$ at this complex gives the following map of bicomplexes, using the same properties as in the previous case:
%%
\[\hspace{-70pt}\begin{tikzcd}[column sep=8pt]
	& 0 && 0 && {C^2(A_0)\oplus C(A_1)\oplus A_2} \\
	{C^4(A_0)\oplus C^3(A_1)\oplus C^2(A_2)} && {C^3(A_0)\oplus C^2(A_1)\oplus C(A_2)} && {C^2(A_0)\oplus C(A_1)\oplus A_2} \\
	& 0 && 0 && {C(A_0)\oplus A_1} \\
	{C^3(A_0)\oplus C^2(A_1)} && {C^2(A_0)\oplus C(A_1)} && {C(A_0)\oplus A_1} \\
	& 0 && 0 && {A_0} \\
	{C^2(A_0)} && {C(A_0)} && {A_0}
	\arrow["{\zeta_2}", from=1-6, to=3-6]
	\arrow["{\zeta_1}", from=3-6, to=5-6]
	\arrow["{\epsilon_{C(A_0)}-C\epsilon_{A_0}}"', from=6-1, to=6-3]
	\arrow["{\epsilon_{A_0}}"', from=6-3, to=6-5]
	\arrow[Rightarrow, no head, from=5-6, to=6-5]
	\arrow[""{name=0, anchor=center, inner sep=0}, "{(\epsilon_{C^2(A_0)}-C\epsilon_{C(A_0)})\oplus (\epsilon_{C(A_1)}-C\epsilon_{A_1})}"', from=4-1, to=4-3]
	\arrow[""{name=1, anchor=center, inner sep=0}, "{\epsilon_{C(A_0)}\oplus \epsilon_{A_1}}"', from=4-3, to=4-5]
	\arrow[from=1-2, to=1-4]
	\arrow[from=1-4, to=1-6]
	\arrow[Rightarrow, no head, from=1-6, to=2-5]
	\arrow[Rightarrow, no head, from=3-6, to=4-5]
	\arrow[from=1-4, to=2-3]
	\arrow[from=1-2, to=2-1]
	\arrow[""{name=2, anchor=center, inner sep=0}, "{\epsilon_{C^2(A_0)}\oplus \epsilon_{C(A_1)}\oplus \epsilon_{A_2}}"', from=2-3, to=2-5]
	\arrow[""{name=3, anchor=center, inner sep=0}, "{\bigoplus_{i=0}^3(\epsilon_{C^{4-i}(A_i)}-C\epsilon_{C^{3-i}(A_i)})}", shift left, from=2-1, to=2-3]
	\arrow[""{name=4, anchor=center, inner sep=0}, "{\zeta_1}", from=4-5, to=6-5]
	\arrow[""{name=5, anchor=center, inner sep=0}, "{\zeta_2}", from=2-5, to=4-5]
	\arrow[""{name=6, anchor=center, inner sep=0}, "{C(\zeta_1)}", from=4-3, to=6-3]
	\arrow[""{name=7, anchor=center, inner sep=0}, "{C(\zeta_2)}", from=2-3, to=4-3]
	\arrow["{C^2(\zeta_2)}", from=2-1, to=4-1]
	\arrow["{C^2(\zeta_1)}", from=4-1, to=6-1]
	\arrow[from=5-2, to=6-1]
	\arrow[from=3-2, to=4-1]
	\arrow[from=3-4, to=4-3]
	\arrow[from=5-4, to=6-3]
	\arrow[shorten >=9pt, no head, from=1-2, to=3]
	\arrow[shorten <=3pt, from=3, to=3-2]
	\arrow[shorten >=3pt, no head, from=3-2, to=0]
	\arrow[shorten <=11pt, from=0, to=5-2]
	\arrow[shorten >=12pt, no head, from=5-2, to=6]
	\arrow[shorten <=36pt, from=6, to=5-4]
	\arrow[shorten >=11pt, no head, from=5-4, to=4]
	\arrow[shorten <=16pt, from=4, to=5-6]
	\arrow[shorten <=14pt, from=5, to=3-6]
	\arrow[shorten >=11pt, no head, from=3-4, to=5]
	\arrow[shorten <=30pt, from=7, to=3-4]
	\arrow[shorten >=12pt, no head, from=3-2, to=7]
	\arrow[shorten >=3pt, no head, from=1-4, to=2]
	\arrow[shorten <=10pt, from=2, to=3-4]
	\arrow[shorten >=3pt, no head, from=3-4, to=1]
	\arrow[shorten <=11pt, from=1, to=5-4]
\end{tikzcd}\]
%%
Finally, taking totalization this map becomes the following map of complexes:
%%
\[\begin{tikzcd}
	{C^2(A_0)\oplus C(A_1)\oplus A_2} & {C(A_0)\oplus A_1} & {A_0} \\
	{C^2(A_0)\oplus (C^2(A_0)\oplus C(A_1))\oplus (C^2(A_0)\oplus C(A_1)\oplus A_2)} & {C(A_0)\oplus (C(A_0)\oplus A_1)} & {A_0}
	\arrow[Rightarrow, no head, from=1-3, to=2-3]
	\arrow["{\zeta_1}", from=1-2, to=1-3]
	\arrow["{\omega_1}"', from=2-2, to=2-3]
	\arrow["{i_2\oplus 1_{A_1}}", from=1-2, to=2-2]
	\arrow["{\omega_2}"', from=2-1, to=2-2]
	\arrow["{\zeta_2}", from=1-1, to=1-2]
	\arrow["{i_3\oplus i_2\oplus 1_{A_2}}", from=1-1, to=2-1]
\end{tikzcd}\]
%%
Therefore, we have that $P_C(p_C)$ and $p_{C,P_C}$ are equal up to a natural isomorphism given by \textbf{TBC} 
% %%
% \[\hspace{-50pt}\begin{tikzcd}
% 	\cdots & {C^2(A_0)\oplus (C^2(A_0)\oplus C(A_1))\oplus (C^2(A_0)\oplus C(A_1)\oplus A_2)} & {C(A_0)\oplus (C(A_0)\oplus A_1)} & {A_0} \\
% 	\cdots & {C^2(A_0)\oplus (C^2(A_0)\oplus C(A_1))\oplus (C^2(A_0)\oplus C(A_1)\oplus A_2)} & {C(A_0)\oplus (C(A_0)\oplus A_1)} & {A_0}
% 	\arrow["{\omega_1}", from=1-3, to=1-4]
% 	\arrow[from=1-1, to=1-2]
% 	\arrow["{\omega_2}"', from=2-2, to=2-3]
% 	\arrow["{\omega_1}"', from=2-3, to=2-4]
% 	\arrow[Rightarrow, no head, from=1-4, to=2-4]
% 	\arrow["{i_{1,2}\oplus 1_{A_1}}", from=1-3, to=2-3]
% 	\arrow[from=2-1, to=2-2]
% 	\arrow["{i_{1,3}\oplus i_{1,2}\oplus 1_{A_2}}"', from=1-2, to=2-2]
% 	\arrow["{\omega_2}", from=1-2, to=1-3]
% \end{tikzcd}\]
% %%	



Before proving the main result we give a preliminary result on totalization.

\begin{lem}[label=lem:totComm]
    Let $F:\mathcal{A}\to \mathcal{B}$ be an additive functor. Then 
    %%
    \begin{equation*}
        \cat{Ch}(F)\circ \text{Tot}_\mathcal{A} \cong \text{Tot}_\mathcal{B}\circ \cat{Ch}^2(F)
    \end{equation*}
    %%
\end{lem}
\begin{proof}
    Let $A_{\bullet,\bullet} \in \cat{Ch}(\mathcal{A})$. Then since $F$ is additive it preserves finite direct sums, so 
    %%
    \begin{equation*}
        \cat{Ch}(F)(\text{Tot}_\mathcal{A}(A_{\bullet,\bullet}))_n = F\left(\bigoplus_{p+q=n}A_{p,q}\right) \cong \bigoplus_{p+q=n}F(A_{p,q}) = \text{Tot}_\mathcal{B}\circ \cat{Ch}^2(F)(A_{\bullet,\bullet})_n
    \end{equation*}
    %%
    and since $\cat{Ch}(F)$ and $\cat{Ch}^2(F)$ act term-wise on differentials, it follows that
    %%
    \begin{equation*}
        \cat{Ch}(F)(\text{Tot}_\mathcal{A}(A_{\bullet,\bullet})) \cong \text{Tot}_\mathcal{B}\circ \cat{Ch}^2(F)(A_{\bullet,\bullet})
    \end{equation*}
    %%
    Further, since this isomorphism is induced by the unique map between biproducts it follows that we have the desired natural isomorphism.
\end{proof}


We now prove a general proposition in analogy with~\cite[Prop 4.5]{BJORT}.

\begin{prop}[label=prop:4.5C]
    For $A_\bullet \in \cat{Ch}(\mathcal{A})$,
    %%
    \begin{itemize}
        \item[(i)] The chain complex $P_C(A_\bullet)$ is degree $C$.
        \item[(ii)] If $A_\bullet$ is degree $C$ then the map $p_{C,A_\bullet}:A_\bullet\to P_C(A_\bullet)$ is a chain homotopy equivalence.
        \item[(iii)] The pair $(P_C(A_\bullet),p_{C,A_\bullet}:A_\bullet\to P_C(A_\bullet))$ is universal up to chain homotopy equivalence with respect to degree $C$ chain complexes with maps from $A_\bullet$. 
    \end{itemize}
\end{prop}
\begin{proof}
    We proceed with the proof in parts.
    \begin{itemize}
        \item[(i)] By Lemma~\ref{lem:totComm} we have the natural isomorphism
            %%
            \begin{equation*}
                \cat{Ch}(R)\circ \text{Tot}_\mathcal{A}\circ \cat{Ch}(C^{\cat{Ch}}) \cong \text{Tot}_\mathcal{A}\circ \cat{Ch}^2(R)\circ \cat{Ch}(C^{\cat{Ch}})
            \end{equation*}
            %%
        Since $\text{Tot}$ preserves chain homotopies it is sufficient to show that for any $A_\bullet \in \cat{Ch}(\mathcal{A})$,
        %%
        \begin{equation*}
            \cat{Ch}^2(R)\circ \cat{Ch}(C^{\cat{Ch}})(A_\bullet)
        \end{equation*}
        %%
        is contractible. By Lemma~\ref{lem:contractHomotop} $\cat{Ch}(R)\circ C^{\cat{Ch}}$ has a natural chain homotopy $s_k:RC^k\to RC^{k+1}$ given by $\eta_{RC^k}$. Note that if $A_\bullet \in \cat{Ch}(\mathcal{A})$ is a chain complex, then by naturality the square 
        \[\begin{tikzcd}
            {R(C^m(A_n))} & {R(C^m(A_{n-1}))} \\
            {R(C^{m+1}(A_n))} & {R(C^{m+1}(A_{n-1}))}
            \arrow["{R(C^m(\partial^A_n))}", from=1-1, to=1-2]
            \arrow["{s_{m,A_n}}"', from=1-1, to=2-1]
            \arrow["{R(C^{m+1}(\partial_n^A))}"', from=2-1, to=2-2]
            \arrow["{s_{m,A_{n-1}}}", from=1-2, to=2-2]
        \end{tikzcd}\]
        commutes. In particular, this implies that $s_k$ defines a natural contraction of bicomplexes. Since $\text{Tot}$ preserves natural chain homotopies we obtain the desired result.
        \item[(ii)] Let $A_{\bullet} \in \cat{Ch}(\mathcal{A})$ be degree $C$, so that $\cat{Ch}(C)(A_\bullet)$ is chain contractible. Since $C$ is additive, $\cat{Ch}(C)$ preserves chain homotopies so $\cat{Ch}(C)^n(A_\bullet)$ is chain contractible for all $n \geq 1$. Then $\cat{Ch}(C^{\cat{Ch}})(A_\bullet)$ is a first-quadrant bicomplex such that every row except the zeroth row is contractible. Therefore, by Corollary~\ref{cor:A7} we have that the natural inclusion $p_{C,A_\bullet} = \text{Tot}_\mathcal{A}(\cat{Ch}(I_C)):A_\bullet \to P_C(A_\bullet)$ is a chain homotopy equivalence, as desired.
        \item[(iii)] To prove the final claim fix $A_\bullet,B_\bullet \in \cat{Ch}(\mathcal{A})$ such that $B_\bullet$ is degree $C$. Further, let $\tau:A_\bullet\to B_\bullet$ be a chain map. Then by naturality of $p_C$ we have the commuting square
        %%
        \[\begin{tikzcd}
            {A_\bullet} & {B_\bullet} \\
            {P_C(A_\bullet)} & {P_C(B_\bullet)}
            \arrow["\tau", from=1-1, to=1-2]
            \arrow["{p_{C,B_\bullet}}", from=1-2, to=2-2]
            \arrow["{p_{C,A_\bullet}}"', from=1-1, to=2-1]
            \arrow["{P_C(\tau)}"', from=2-1, to=2-2]
        \end{tikzcd}\]
        %%
        Let $s_C:P_C\to 1_{\cat{Ch}(\mathcal{A})}$ be the natural homotopy inverse of $p_C$ for degree $C$ complexes. Setting $\tau^\# = s_{C,B_\bullet}\circ P_C(\tau)$ we obtain
        %%
        \begin{equation*}
            \tau^\#\circ p_{C,A_\bullet} = s_{C,B_\bullet}\circ P_C(\tau)\circ p_{C,A_\bullet} = s_{C,B_\bullet}\circ p_{C,B_\bullet}\circ \tau \simeq_{\cat{Ch}}\tau
        \end{equation*}
        %%
        so $\tau$ factors through $p_{C,A_\bullet}$ up to chain homotopy. 

        \vspace{10pt}

        To show uniqueness for the universal property let $\sigma:P_C(A_\bullet)\to B_\bullet$ such that $\sigma\circ p_{C,A_\bullet} \simeq_{\cat{Ch}}\tau$. Observe that we have a commuting rectangle
        %%
        \[\begin{tikzcd}
            {A_\bullet} & {P_C(A_\bullet)} & {B_\bullet} \\
            {P_C(A_\bullet)} & {P_C(P_C(A_\bullet))} & {P_C(B_\bullet)}
            \arrow["{p_{C,A_\bullet}}"', from=1-1, to=2-1]
            \arrow["{P_C(p_{C,A_\bullet})}"', from=2-1, to=2-2]
            \arrow["{p_{C,A_\bullet}}", from=1-1, to=1-2]
            \arrow["{p_{C,P_C(A_\bullet)}}", from=1-2, to=2-2]
            \arrow["\sigma", from=1-2, to=1-3]
            \arrow["{p_{C,B_\bullet}}", from=1-3, to=2-3]
            \arrow["{P_C(\sigma)}"', from=2-2, to=2-3]
        \end{tikzcd}\]
        %%
        where from part (ii) the maps $p_{C,P_C(A_\bullet)}$ and $p_{C,B_\bullet}$ are chain homotopy equivalences. 
    \end{itemize}
\end{proof}

\textbf{To Do: Develop Theory, Check quasi-isomorphisms}