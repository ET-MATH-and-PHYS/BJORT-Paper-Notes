\section{A General Bicomplex Retraction}\label{sec:bicomplexes}

In order to construct certain explicit chain homotopy equivalences in the text we require a criteria for when total complexes of certain bicomplexes are chain homotopy equivalent. Throughout we consider $A_{\bullet,\bullet}$ to be denote a first-quadrant bicomplex. This is sufficient for our case since all our bicomplexes are constructed from chain complexes concentrated in non-negative degrees. As in \cite{BJORT} we proceed with bicomplexes having anti-commuting squares. To apply this to the work elsewhere all that must be done is the replacement of $d_h:A_{p,q}\to A_{p,q-1}$ by $(-1)^pd_h$.

\begin{defn}[label=defn:RowWiseStrngDefRetr]
    We say a morphism $\iota:A_{\bullet,\bullet}\to B_{\bullet,\bullet}$ \textbf{admits a row-wise strong deformation retraction} if for all $p \geq 0$ there exists a map $f_{p,\bullet}:B_{p,\bullet}\to A_{p,\bullet}$ such that
    %%
    \begin{itemize}
        \item[(i)] $f_{p,\bullet}\circ \iota_{p,\bullet} = 1_{A_{p,\bullet}}$
        \item[(ii)] there exist morphisms $s_h:B_{p,q}\to B_{p,q+1}$ such that $d_hs+sd_h = 1-\iota_{p,q}f_{p,q}$ and $s_h\circ \iota_{p,q} = 0$ (i.e.\ we have a strong chain homotopy equivalence between $A_{p,\bullet}$ and $B_{p,\bullet}$)
    \end{itemize}
\end{defn}

Throughout this section we will denote the horizontal differentials of a bicomplex by $d_h:A_{p,q}\to A_{p,q-1}$ and the vertical differentials by $d_v:B_{p,q}\to B_{p-1,q}$. Although our maps in definition \ref{defn:RowWiseStrngDefRetr} are given only for $p,q\geq 0$, they can easily be extended to all $p,q$ by setting ones with negative indices equal to zero. We record some commutativity equalities for use in the proofs to follow
%%
\begin{align*}
    d_h^2 = 0 &\;\; d_v^2=0 \;\; d_hd_v+d_vd_h = 0\;\; d_hs_h+s_hd_h=1-\iota \circ f\;\; s_h\circ \iota = 0 \\
    & f\circ \iota = 1\;\; f \circ d_h= d_h\circ f\;\; \iota \circ d_h= d_h\circ \iota \;\; \iota \circ d_v = d_v\circ \iota 
\end{align*}

We begin with the following lemma. (\textbf{Note:} Juxtaposition in the following lemma still denotes functional compositional ordering for the sake of preserving space).

%%
\begin{lem}[label=lem:A3]
    For any $k \geq 0$ we have the following equalities:
    %%
    \begin{itemize}
        \item[(i)] $d_v f(-d_vs_h)^k+d_hf (-d_vs_h)^{k+1}=f(-d_vs_h)^kd_v + f (-d_vs_h)^{k+1}d_h$
        \item[(ii)] $d_vs_h(-d_vs_h)^k+d_hs_h(-d_vs_h)^{k+1} = -\iota f(-d_vs_h)^{k+1}-s_hd_h(-d_vs_h)^{k+1}$ 
        \item[(iii)] $s_h(-d_vs_h)^{k+1}d_h+s_h(-d_vs_h)^kd_v = -(-s_hd_v)^{k+1}d_hs_h$ 
        \item[(iv)] $s_hd_h(-d_vs_h)^{k+1} = -(-s_hd_v)^{k+1}d_hs_h$
    \end{itemize}
\end{lem}
\begin{proof}
    We will prove each formula by induction.
    \begin{itemize}
        \item[(i)] If $k = 0$ we want to show 
        %%
        \begin{equation*}
            d_vf+d_hf(-d_vs_h) = fd_v + f(-d_vs_h)d_h
        \end{equation*}
        %%
        Using our relations
        \begin{align*}
            d_vf+d_hf(-d_vs_h) &= d_vf - fd_hd_vs_h \\
            &= d_vf+fd_vd_hs_h \\
            &= d_vf+fd_v(1-\iota f-s_hd_h) \\
            &= d_vf+fd_v-fd_v\iota f - fd_vs_hd_h \\
            &= d_vf+fd_v-d_vf + f(-d_vs_h)d_h \\
            &= fd_v+f(-d_vs_h)d_h
        \end{align*}
        as desired. Suppose now that the claim holds for some $k \geq 0$. Then
        %%
        \begin{align*}
            d_vf(-d_vs_h)^{k+1}+d_hf(-d_vs_h)^{k+2} &= [f(-d_vs_h)^kd_v+f(-d_vs_h)^{k+1}d_h](-d_vs_h) \\
            &= f(-d_vs_h)^kd_v(-d_vs_h)+f(-d_vs_h)^{k+1}d_h(-d_vs_h) \\
            &= f(-d_vs_h)^{k+1}d_v(1-s_hd_h-\iota f) \\
            &= f(-d_vs_h)^{k+1}d_v-f(-d_vs_h)^{k+1}d_vs_hd_h-f(-d_vs_h)^{k+1}d_v\iota f \\
            &= f(-d_vs_h)^{k+1}d_v-f(-d_vs_h)^{k+2}d_h-f(-d_vs_h)^{k+1}\iota d_vf \\
            &= f(-d_vs_h)^{k+1}d_v-f(-d_vs_h)^{k+2}d_h
        \end{align*}
        as desired.
        \item[(ii)] We can immediately compute
        %%
        \begin{align*}
            d_vs_h(-d_vs_h)^k + d_hs_h(-d_vs_h)^{k+1} &= [d_vs_h + d_hs_h(-d_vs_h)](-d_vs_h)^k \\
            &= [d_vs_h + (1-s_hd_h-\iota f)(-d_vs_h)](-d_vs_h)^k \\
            &= [-\iota f(-d_vs_h) - s_hd_h(-d_vs_h)](-d_vs_h)^k \\
            &= -\iota f(-d_vs_h)^{k+1} - s_hd_h(-d_vs_h)^{k+1}
        \end{align*}
        %%
        as desired.
        \item[(iii)] If $k = 0$ we compute
        %%
        \begin{align*}
            s_h(-d_vs_h)d_h+s_hd_v &= -s_hd_v(1-d_hs_h-\iota f)+s_hd_v \\
            &= s_hd_vd_hs_h+s_hd_v\iota f \\
            &= -(-s_hd_v)d_hs_h+s_h\iota d_vf \\
            &= -(-s_hd_v)d_hs_h
        \end{align*}
        %%
        Now if the claim holds for $k \geq 0$ we can compute
        %%
        \begin{align*}
            -(-s_hd_v)^{k+2}d_hs_h &= (-s_hd_v)[s_h(-d_vs_h)^{k+1}d_h+s_h(-d_vs_h)^kd_v] \\
            &= s_h(-d_vs_h)^{k+2}d_h+s_h(-d_vs_h)^{k+1}d_v
        \end{align*}
        %%
        as desired.
        \item[(iv)] If $k = 0$ we observe that
        %%
        \begin{align*}
            s_hd_h(-d_vs_h) &= s_hd_vd_hs_h = -(-s_hd_v)d_hs_h
        \end{align*}
        %%
        If the claim holds for some $k \geq 0$, then we can compute
        %%
        \begin{align*}
            s_hd_h(-d_vs_h)^{k+2} &= -(-s_hd_v)^{k+1}d_hs_h(-d_vs_h) \\
            &= (-s_hd_v)^{k+1}(1-s_hd_h-\iota f)d_vs_h \\
            &= (-s_hd_v)^{k+1}d_vs_h-(-s_hd_v)^{k+1}s_hd_hd_vs_h-(-s_hd_v)^{k+1}\iota fd_vs_h \\
            &= -(-s_hd_v)^{k+2}d_hs_h
        \end{align*}
        as desired since $d_v^2 = 0$, $d_v\iota = \iota d_v$, and $s_h\iota = 0$.
    \end{itemize}
\end{proof}

We recall that for first-quadrant bicomplexes the totalization at each degree is a finite direct sum, so its differentials can be described by finite matrices. Explicitly the $n$th differential $\text{Tot}(A_{\bullet,\bullet})_n\to \text{Tot}(A_{\bullet,\bullet})_{n-1}$ is given by the matrix
%%
\begin{equation*}
    \begin{pmatrix}
        d_v & d_h & 0 & \cdots & \cdots & 0 \\
        0 & d_v & d_h & 0 & \cdots & 0 \\
        \vdots & \ddots & \ddots & \ddots & \ddots & \vdots \\
        0 & \cdots & 0 & d_v & d_h & 0 \\
        0 & \cdots & \cdots & 0 & d_v & d_h
    \end{pmatrix}
\end{equation*}


\begin{prop}[label=prop:A5]
    Let $\iota:A_{\bullet,\bullet}\to B_{\bullet,\bullet}$ be a map of bicomplexes that admits a row-wise strong deformation retraction. Then the induced morphism of total complexes $\text{Tot}(\iota):\text{Tot}(A_{\bullet,\bullet})_\bullet\to\text{Tot}(B_{\bullet,\bullet})_{\bullet}$ admits a retraction $\rho:\text{Tot}(B_{\bullet,\bullet})_{\bullet}\to \text{Tot}(A_{\bullet,\bullet})_{\bullet}$ defined in degree $n$ by the $(n+1)\times (n+1)$ matrix
    %%
    \begin{equation*}
        \begin{pmatrix}
            f & 0 & \cdots & \cdots & 0 & 0 \\
            f(-d_vs_h) & f & 0 & \ddots & \ddots & 0 \\
            f(-d_vs_h)^2 & f(-d_vs_h) & f & 0 & \ddots & \vdots \\
            \vdots & \vdots & \vdots &\vdots & \ddots & \ddots & \vdots \\
            f(-d_vs_h)^n & f(-d_vs_h)^{n-1} & \cdots & \cdots & f(-d_vs_h) & f 
        \end{pmatrix}
    \end{equation*}
\end{prop}
\begin{proof}
    First, to see that $\rho$ is a chain map fix $n \geq 1$. Then $\partial_n\rho_n$ and $\rho_{n-1}\partial_n$ are $n\times (n+1)$ matrices with $i,j$ component given by 
    %%
    \begin{align*}
        (\partial_n\rho_n)_{i,j} = \left\{\begin{array}{cc} 
            0 & i+1 < j \\
            d_hf & i+1 = j \\
            d_hf(-d_vs_h)+d_vf & i = j \\
            d_hf(-d_vs_h)^{i-j+1}+d_vf(-d_vs_h)^{i-j} & i > j 
        \end{array}\right.
    \end{align*} 
    %%
    while
    %%
    \begin{align*}
        (\rho_{n-1}\partial_n)_{i,j} = \left\{\begin{array}{cc} 
            0 & i+1 < j \\
            fd_h & i+1 = j \\ 
            fd_v & i = j = 1 \\
            f(-d_vs_h)d_h+fd_v & i = j \neq 1 \\
            f(-d_vs_h)^{i-j+1}d_v+f(-d_vs_h)^{i-j}d_h & i > j 
        \end{array}\right.
    \end{align*}
    %%
    We have equality for $i > j$ by equation (i) of Lemma \ref{lem:A3}, equality for $i+1 < j$ vacuously, equality for $i+1 = j$ since $f$ is a chain map with respect to the horizontal differentials, equality for $i = j \neq 1$ is also from equation (i) of Lemma \ref*{lem:A3}, and $i = j = 1$ is equation (i) and the fact that $A_{n+1,-1} = 0$ as the bicomplex is concentrated in non-negative degree.

    \vspace{10pt}

    To show $\rho$ is a retraction it remains to show $\rho_n\iota_n = 1$. Observe that for $i,j$,
    %%
    \begin{equation*}
        (\rho_n\iota_n)_{i,j} = \left\{\begin{array}{cc} 
            0 & i > j \\
            f\circ \iota & i = j \\
            f(-d_vs_h)^{j-i}\circ \iota & i < j \\
        \end{array}\right.
    \end{equation*}
    %%
    But since $f$ is part of a row-wise strong deformation retraction, $f\circ \iota = 1$, and $s_h \circ \iota = 0$, so the matrix is the identity.
\end{proof}


It remains to show that $\rho$ is associated with a deformation retraction for $\iota$. In other words, we want to show that $\iota\circ \rho$ is chain homotopic to the identity.

\begin{prop}[label=prop:A6]
    The composite map $\iota\circ \rho:\text{Tot}(B)_\bullet\to \text{Tot}(B)_\bullet$ is chain homotopic to the identity via the chain homotopy $\sigma:\text{Tot}(B)_\bullet\to \text{Tot}(B)_{\bullet+1}$ defined in degree $n$ by the $n+2\times n+1$ matrix
    %%
    \begin{equation*}
        \begin{pmatrix}
                0 & 0 & \cdots & \cdots & 0 & 0 \\
                s_h & 0 & \ddots & \ddots & \ddots & 0 \\
                s_h(-d_vs_h) & s_h & 0 & \ddots & \ddots & \vdots \\
                s_h(-d_vs_h)^2 & s_h(-d_vs_h) & s_h & 0 & \ddots & \vdots \\
                \vdots & \vdots & \vdots & \ddots & \ddots & \vdots \\
                s_h(-d_vs_h)^{n-1} & \cdots & \cdots & s_h(-d_vs_h) & s_h & 0 \\
                s_h(-d_vs_h)^n & s_h(-d_vs_h)^{n-1} & \cdots & \cdots & s_h(-d_vs_h) & s_h
        \end{pmatrix}
    \end{equation*}
    %%
\end{prop}
\begin{proof}
    Let $\partial_n:\text{Tot}(B)_n\to \text{Tot}(B)_{n-1}$ be the total complex differential. Observe that that $1-\iota\circ \rho$ is the matrix 
    %%
    \begin{equation*}
            \begin{pmatrix} 
                1-\iota f & 0 & \cdots & \cdots & 0 \\
                -\iota f(-d_vs_h) & 1-\iota f & 0 & \cdots & 0 \\
                -\iota f(-d_vs_h)^2 & -\iota f(-d_vs_h) & 1-\iota f & \cdots & 0 \\
                \vdots & \vdots & \vdots & \ddots & \vdots \\
                -\iota f(-d_vs_h)^n & \cdots & -\iota f(-d_vs_h)^2 & -\iota f(-d_vs_h) & 1-\iota f \\
            \end{pmatrix}
    \end{equation*}
    %%
    On the other hand, we can compute for $1 \leq i,j\leq n+1$
    %%
    \begin{equation*}
        (\partial_{n+1}\sigma_n)_{i,j} = \left\{
            \begin{array}{cc}
                0 & i < j \\
                d_hs_h & i = j \\
                d_vs_h(-d_vs_h)^{i-j-1}+d_hs_h(-d_vs_h)^{i-j} & i > j 
            \end{array}
        \right.
    \end{equation*}
    %%
    and 
    %%
    \begin{equation*}
        (\sigma_{n-1}\partial_{n})_{i,j} = \left\{
            \begin{array}{cc}
                0 & i < j \\
                s_hd_h & i = j \\
                s_h(-d_vs_h)^{i-j}d_h + s_h(-d_vs_h)^{i-j-1}d_v & i > j 
            \end{array}
        \right.
    \end{equation*}
    %%
    Adding these together we observe that the case of $i=j$ gives equality since $d_hs_h + s_hd_h = 1-\iota f$. On the other hand, for $i > j$ we can use the relations in Lemma \ref{lem:A3}
    %%
    \begin{align*}
        d_vs_h(-d_vs_h)^{i-j-1}+d_hs_h(-d_vs_h)^{i-j}&+s_h(-d_vs_h)^{i-j}d_h + s_h(-d_vs_h)^{i-j-1}d_v \\
        &= -\iota f(-d_vs_h)^{i-j}-s_hd_h(-d_vs_h)^{i-j}\\
        &+s_h(-d_vs_h)^{i-j}d_h + s_h(-d_vs_h)^{i-j-1}d_v \tag{by (ii)} \\
        &= -\iota f(-d_vs_h)^{i-j}-s_hd_h(-d_vs_h)^{i-j}-(-s_hd_v)^{i-j}d_hs_h \tag{by (iii)} \\
        &= -\iota f(-d_vs_h)^{i-j}-s_hd_h(-d_vs_h)^{i-j}+s_hd_h(-d_vs_h)^{i-j} \tag{by (iv)} \\
        &= -\iota f(-d_vs_h)^{i-j}
    \end{align*}
    %%
    which is precisely the $i,j$ entry of $1-\iota \rho$, completing the proof.
\end{proof}


Together these lemmas prove the following result:

\begin{thm}[label=thm:A2]
    Let $\iota:A_{\bullet,\bullet}\to B_{\bullet,\bullet}$ be a morphism of first-quadrant bicomplexes that admit a row-wise strong deformation retraction. Then $\iota$ induces a chain homotopy equivalence of total complexes $\text{Tot}(A_{\bullet,\bullet})\to \text{Tot}(B_{\bullet,\bullet})$.  
\end{thm}

As a quick corollary we obtain the following sufficient condition for a chain homotopy equivalence between the degree $0$ inclusion of a chain complex into a first quadrant bicomplex with contractible rows and its totalization.

\begin{cor}[label=cor:A7]
    Let $A_{\bullet,\bullet}$ be a first-quadrant bicomplex so that every row except the zeroth row $A_{0,\bullet}$ is contractible. Then the natural inclusion $A_{0,\bullet}\hookrightarrow \text{Tot}(A)_\bullet$ is a chain homotopy equivalence.
\end{cor}
\begin{proof}
    Let $\iota:\deg_0(A_{0,\bullet})\to A_{\bullet,\bullet}$ denote the inclusion in degree $0$. Let $f_{p,\bullet}:A_{p,\bullet} \to \deg_0(A_{0,\bullet})_{p,\bullet}$ denote the chain complex map for the contraction for $p > 0$ (which is the zero map), the identity for $p = 0$, and zero for $p < 0$. Let $s_h:A_{p,q} \to A_{p,q+1}$ denote the contraction for $p \geq 1$, and $0$ for $p \leq 0$, so $d_hs_h+s_hd_h = 1-\iota f$ and $s_h\iota = 0$ since $\iota$ is zero for $p > 0$ and $s_h$ is zero for $p \leq 0$. By Theorem \ref{thm:A2} there exists a chain homotopy equivalence $\iota:A_{0,\bullet}\hookrightarrow \text{Tot}(A)_\bullet$ where the retraction $\rho:\text{Tot}(A)_\bullet\to A_{0,\bullet}$ is the $1\times (n+1)$ is given By
    %%
    \begin{equation*}
        \begin{pmatrix}
            f(-d_vs_h)^n & f(-d_vs_h)^{n-1} & \cdots & f(-d_vs_h) & f
        \end{pmatrix}
    \end{equation*}
    %%
    or equivalently
    %%
    \begin{equation*}
        \begin{pmatrix}
            (-d_vs_h)^n & (-d_vs_h)^{n-1} & \cdots & (-d_vs_h) & 1
        \end{pmatrix}
    \end{equation*}
    %%
    since $f$ is the identity on $A_{0,\bullet}$.
\end{proof}