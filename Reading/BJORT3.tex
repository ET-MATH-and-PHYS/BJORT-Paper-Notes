\section{A categorical context for abelian functor calculus}


Classically, abelian functor calculus deals with functors to abelian categories. In order to discuss universal properties ``up to homotopy" we replace abelian categories by some type of homotopical categories where weak universal properties replace strict ones. 


We denote the category of chain complexes of an abelian category $\mathcal{A}$ concentrated in non-negative degrees by $\cat{Ch}\mathcal{A}$. In this section we will construct a category which has as arrows maps from abelian categories to categories of chains. An important theorem for this construction is the Dold-Kan Equivalence, which is reviewed in Appendix \ref{sec:doldKan}.



\subsection{Pointwise versus Natural Equivalences}\label{sec:ptwiseNat}

\begin{rmk}
    If we choose pointwise isomorphisms for the functors in the definition of $\cat{AbCat}$, then composition in $\cat{AbCat}_{\cat{Ch}}$ will not be well-defined since we require that for any equivalent functors, $G$ and $H$, and any simplicial object $\hat{A}$ with codomain equal to the domain of $G$ and $H$, $G\circ \hat{A}\cong H\circ \hat{A}$ as simplicial objects.
\end{rmk}


Since the polynomial and linearization functors of \cite{JohnsonB.2004Dcwc} are only defined up to quasi-isomorphism in certain viewpoints, in order for them to be well-defined we must pass to the homotopy category $\cat{HoAbCat}_{\cat{Ch}}$. First, in this section we let composition in $\cat{AbCat}_{\cat{Ch}}$ be defined for $G:\mathcal{B}\rightarrow \cat{Ch}(\mathcal{C})$ and $F:\mathcal{A}\rightarrow \cat{Ch}(\mathcal{B})$ by
%%
\begin{equation*}
    G\lhd F := N_\mathcal{C}\Delta_\mathcal{C}\Gamma_* G_*\Gamma_\mathcal{B} F
\end{equation*}
%%
where $\Delta_\mathcal{C}:(\mathcal{C}\Sob)\Sob\rightarrow \mathcal{C}\Sob$ is the diagonal functor. It remains to show that this does define a categorical structure on $\cat{AbCat}_{\cat{Ch}}$, which we check through the following list of conditions:
%%
\begin{enumerate}
    \item Let $F:\mathcal{A}\rightarrow \cat{Ch}(\mathcal{B})$ be a functor. The identity is given by $\deg_0$. Indeed observe that
    %%
    \begin{equation*}
        [\deg_0^\mathcal{B}\lhd F] = [N_\mathcal{B}\Delta_\mathcal{B}\Gamma_*(\deg_0^\mathcal{B})_*\Gamma_\mathcal{B} F]
    \end{equation*}
    %%
    However, by Lemma \ref{lem:gammaDeg} $\Gamma_\mathcal{B}\circ \deg_0^\mathcal{B} = \iota_\mathcal{B}:\mathcal{B}\rightarrow \mathcal{B}\Sob$ is the constant functor. It follows that $\Delta_\mathcal{B}(\Gamma\circ \deg_0^\mathcal{B}\circ \Gamma_\mathcal{B}F) = \Gamma_\mathcal{B}F$, so
    %%
    \begin{equation*}
        [\deg_0^\mathcal{B}\lhd F] = [N_\mathcal{B}\Delta_\mathcal{B}\Gamma_*(\deg_0^\mathcal{B})_*\Gamma_\mathcal{B} F] = [N_\mathcal{B}\Gamma_\mathcal{B}F] = [F]
    \end{equation*}
    %%
    On the other hand,
    %%
    \begin{equation*}
        [F\lhd \deg_0^\mathcal{A}] = [N_\mathcal{B}\Delta_\mathcal{B}\Gamma_*F_*\Gamma_\mathcal{A}\deg_0^\mathcal{A}] = [N_\mathcal{B}\Delta_\mathcal{B}\Gamma_*F_*\iota_\mathcal{A}]
    \end{equation*}
    %%
    Observe that $F_*\iota_\mathcal{A} = \iota_{\cat{Ch}(\mathcal{B})}\circ F$. Similarly, $\Gamma_*\iota_{\cat{Ch}(\mathcal{B})} = \iota_{\mathcal{B}\Sob}\circ\Gamma_\mathcal{B}$. Finally,
    %%
    \begin{equation*}
        \Delta_\mathcal{B}(\iota_{\mathcal{B}\Sob}\circ\Gamma_\mathcal{B}\circ F) = \Gamma_\mathcal{B}\circ F
    \end{equation*}
    %%
    so
    %%
    \begin{equation*}
        [F\lhd \deg_0^\mathcal{A}] = [N_\mathcal{B}\Delta_\mathcal{B}\Gamma_*F_*\Gamma_\mathcal{A}\deg_0^\mathcal{A}] = [N_\mathcal{B}\Gamma_\mathcal{B}F] = [F]
    \end{equation*}
    %%
    \item It remains to show composition is associative, so consider $F:\mathcal{A}\rightarrow \cat{Ch}(\mathcal{B}),H:\mathcal{B}\rightarrow \cat{Ch}(\mathcal{C}),G:\mathcal{C}\rightarrow \cat{Ch}(\mathcal{D})$. Then we compute:
    %%
    \begin{align*}
        [(G\lhd H)\lhd F] &= [N_\mathcal{D}\Delta_\mathcal{D}\Gamma_*(G\lhd H)_*\Gamma_\mathcal{B}F] \\
        &= [N_\mathcal{D}\Delta_\mathcal{D}\Gamma_*(N_\mathcal{D}\Delta_\mathcal{D}\Gamma_*G_*\Gamma_\mathcal{C}H)_*\Gamma_\mathcal{B}F] \\
        &= [N_\mathcal{D}\Delta_\mathcal{D}(\Delta_\mathcal{D}\Gamma_*G_*)_*\Gamma_*H_*\Gamma_\mathcal{B}F] \\
    \end{align*}
    %%
    while
    %%
    \begin{align*}
        [G\lhd(H\lhd F)] &= [G\lhd(N_\mathcal{C}\Delta_\mathcal{C}\Gamma_*H_*\Gamma_\mathcal{B}F)] \\
        &= [N_\mathcal{D}\Delta_\mathcal{D}\Gamma_*G_*\Gamma_\mathcal{C}(N_\mathcal{C}\Delta_\mathcal{C}\Gamma_*H_*\Gamma_\mathcal{B}F)] \\
        &= [N_\mathcal{D}\Delta_\mathcal{D}\Gamma_*G_*\Delta_\mathcal{C}\Gamma_*H_*\Gamma_\mathcal{B}F] \\
    \end{align*}
    Hence, it is sufficient to show that $[\Gamma_*G_*\Delta_\mathcal{C}] = [(\Delta_\mathcal{D}\Gamma_*G_*)_*]$. Recall that $\Gamma_*G_* : \mathcal{C}\Sob \rightarrow \cat{Ch}(\mathcal{D})\Sob\rightarrow (\mathcal{D}\Sob)\Sob$ and $(\Gamma_*G_*)_* : (\mathcal{C}\Sob)\Sob\rightarrow ((\mathcal{D}\Sob)\Sob)\Sob$. Additionally, $(\Delta_\mathcal{D})_* = (-)\Sob\Delta_\mathcal{D} = \Delta_{\mathcal{D}\Sob}$, so this equality is exactly naturality of $\Delta_{(-)}$, which is shown in Section \ref{sec:simpMon}.
\end{enumerate}
Therefore, this composition provides the structure of a 1-category for $\cat{AbCat}_{\cat{Ch}}$.


\begin{defn}[label=defn:ChEquiv]{}
    Two functors $H,G:\mathcal{B}\rightarrow \cat{Ch}(\mathcal{A})$ are said to be \textbf{pointwise chain homotopy equivalent} if the chain complexes $H(X)$ and $G(X)$ are chain homotopy equivalent in $\cat{Ch}(\mathcal{A})$ for each $X \in \mathcal{B}_0$, and \textbf{naturally chain homotopy equivalent} if the chain homotopies are natural in $X$.

    Explicitly, naturally chain homotopy equivalent means that for each $X \in \mathcal{B}_0$ we have natural transformations $h:H\Rightarrow G, g:G\Rightarrow H$, together with homotopies $s_X:h_X\circ g_X\simeq 1_{G(X)}$ and $r_X:g_X\circ h_X\simeq 1_{H(X)}$ which are natural in the sense that for each $n$, $s_n:(-)_n\circ G\Rightarrow (-)_{n+1}\circ G$ and $r_n:(-)_n\circ H\Rightarrow (-)_{n+1}\circ H$ are natural transformations.
\end{defn}


Both of these notions define equivalence relations on the category $\cat{AbCat}_{\cat{Ch}}$. Indeed, any functor is naturally chain homotopy equivalent to itself through identity natural transformations and zero homotopies, and a natural chain homotopy equivalence from $H$ to $G$ is precisely the same as a natural chain homotopy equivalence from $G$ to $H$. It remains to show that if $H\simeq_{ChN}G\simeq_{ChN}F$, then $H\simeq_{ChN}F$. Let $(h,g,s,r)$ and $(g',f,s',r')$ be the quadruples witnessing the natural chain homotopy equivalence. Then we define a new quadruple by $(g'\circ h, g\circ f, g'\circ s\circ f+s', g\circ r'\circ h+r)$ which has all natural components since the composition of natural transformations is natural, and $+$ is functorial in an abelian category. To see that this does indeed define a chain homotopy equivalence observe that denoting the chain maps by $\partial$, and suppressing subscripts, we compute
%%
\begin{align*}
    \partial(g'\circ s\circ f+s')+(g'\circ s\circ f+s')\partial &= \partial g'sf+\partial s'+g'sf\partial + s'\partial \\
    &= g'\partial sf+g's\partial f+\partial s'+s'\partial \\
    &= g'(\partial s+s\partial)f+\partial s'+s'\partial \tag{using the fact $g'$ and $f$ are chain maps} \\
    &= g'(hg-1_G)f + (g'f-1_F) \tag{by definition of the homotopies $s$ and $s'$} \\
    &= g'hgf-1_F
\end{align*}
%%


Finally, it remains to show that this equivalence relation is well-defined for isomorphism classes of functors. It is sufficient to show that if $[H] = [H']$ and $H$ is (naturally) chain homotopy equivalent to $G$, then so is $H'$. Let $(h,g,s,r)$ witness the (natural) chain homotopy, and let $\alpha:H\rightarrow H'$ be a natural isomorphism witnessing the equivalence. Then I claim that the quadruple $(h\alpha^{-1},\alpha g, s, \alpha[+1]r\alpha^{-1})$ is a (natural) chain homotopy, where $\alpha[+1]$ is the induced natural isomorphism between $H[+1]$ and $H'[+1]$. 

First, observe that this definition preserves naturality, since $\alpha$ is natural, so it is sufficient in both cases to demonstrate that the components define a chain homotopy. Then for $X \in \mathcal{B}_0$ we can compute
%%
\begin{equation*}
    \partial s_X+s_X\partial = h_Xg_X-1_{G(X)} = (h_X\alpha_X^{-1})(\alpha_X g_X)-1_{G(X)}
\end{equation*}
%%
while for a given $n$ (to make the computation more tractable)
%%
\begin{align*}
    \partial_{n+1} ((\alpha_X[+1])_n(r_X)_n(\alpha_X^{-1})_n)&+((\alpha_X[+1])_{n-1}(r_X)_{n-1}(\alpha_X^{-1})_{n-1})\partial_n \\
    &= \partial_{n+1} ((\alpha_X)_{n+1}(r_X)_n(\alpha_X^{-1})_n)+((\alpha_X)_{n}(r_X)_{n-1}(\alpha_X^{-1})_{n-1})\partial_n \\
    &= (\alpha_X)_n\partial_{n+1}(r_X)_n(\alpha_X^{-1})_n+(\alpha_X)_n(r_X)_{n-1}\partial_n(\alpha_X^{-1})_{n} \\
    &= (\alpha_X)_n(\partial_{n+1}(r_X)_n+(r_X)_{n-1}\partial_n)(\alpha_X^{-1})_{n} \\
    &= (\alpha_X)_n((g_X)_n(h_X)_n-(1_{H(X)})_n)(\alpha_X^{-1})_{n} \\
    &= (\alpha_X)_n(g_X)_n(h_X)_n(\alpha_X^{-1})_n-(1_{H(X)})_n
\end{align*}
%%
so that the homotopies indeed hold. Note that these proofs thus far are independent of the composition we have chosen for $\cat{AbCat}_{\cat{Ch}}$. In order to define a homotopy category from these equivalence relations we must now show that they are in fact congruence relations. We begin by following \cite{BJORT} in showing this for the original definition of composition on $\cat{AbCat}_{\cat{Ch}}$.

\begin{lem}[label=lem:3.4]{(Lemma 3.4 \cite{BJORT})}
    If $G,H:\mathcal{C}\rightarrow \cat{Ch}(\mathcal{B})$ are (naturally) chain homotopy equivalent functors, then for any pair of functors $F:\mathcal{B}\rightarrow \cat{Ch}(\mathcal{A})$ and $K:\mathcal{D}\rightarrow \cat{Ch}(\mathcal{C})$, the composites
    %%
    \begin{equation*}
        F\circ G\circ K,F\circ H\circ K:\mathcal{D}\rightarrow \cat{Ch}(\mathcal{A})
    \end{equation*}
    %%
    are (naturally) chain homotopy equivalent.
\end{lem}
\begin{proof}
    Let $G,H$ be (naturally) chain homotopy equivalent, and let $F$ and $K$ be functors as in the question. We begin by showing that $G\circ K$ is (naturally) chain homotopy equivalent to $H\circ K$, where composition is in the ``Kleisli" category. 

    \vspace{10pt}

    % Observe that $[G\circ K] = [\text{Tot}_\mathcal{B}N_{\cat{Ch}(\mathcal{B})}G_*\Gamma_\mathcal{C}K]$. Since $N_{\cat{Ch}}$ preserves homotopy equivalences, it is sufficient to show that $[G_*\Gamma_\mathcal{C}K]$ is (naturally) homotopy equivalent to $[H_*\Gamma_\mathcal{C}K]$, in terms of simplicial homotopies. Finally, it is sufficient to show $[G_*\Gamma_\mathcal{C}]:\cat{Ch}(\mathcal{C})\rightarrow \mathcal{C}\Sob\rightarrow \cat{Ch}(\mathcal{B})\Sob$ and $[H_*\Gamma_\mathcal{C}]:\cat{Ch}(\mathcal{C})\rightarrow \mathcal{C}\Sob\rightarrow \cat{Ch}(\mathcal{B})\Sob$ are (naturally) homotopy equivalent.


    Let $(h,g,s,r):H\cong_{\cat{Ch}}G$ be a (natural) chain homotopy between our original functors. We prove preservation under pre-composition and post-composition separately
    %%
    \begin{itemize}
        \setlength{\itemindent}{2em}
        \item[[$-\circ K$]] We recall the compositions $G\circ K$ and $H\circ K$ in $\cat{AbCat}_{\cat{Ch}}$ are given by
        %%
        \begin{equation*}
            G\circ K = \text{Tot}_{\mathcal{B}}N_{\cat{Ch}(\mathcal{B})}G_*\Gamma_\mathcal{C}K\;\;\;H\circ K = \text{Tot}_{\mathcal{B}}N_{\cat{Ch}(\mathcal{B})}H_*\Gamma_\mathcal{C}K
        \end{equation*}
        %%
        \textbf{NOT DONE}
        \item[[$F\circ -$]] The compositions $F\circ G$ and $F \circ H$ in $\cat{AbCat}_{\cat{ch}}$ are given by
        %%
        \begin{equation*}
            F\circ G = \text{Tot}_{\mathcal{A}}N_{\cat{Ch}(\mathcal{A})}F_*\Gamma_\mathcal{B}G\;\;\;F\circ H = \text{Tot}_{\mathcal{A}}N_{\cat{Ch}(\mathcal{A})}F_*\Gamma_\mathcal{B}H
        \end{equation*}
        %%
        Using the fact that $\text{Tot}_{\cat{Ch}}$ preserves chain homotopy equivalences \cite[Remark 12.18.6]{StacksProject} and the results in Sections \ref{subsec:simpHomotop} and \ref{sec:doldKan} we obtain the following chain of equivalences:
        %%
        \begin{align*}
            H\simeq_{\cat{Ch}}G &\implies \Gamma_\mathcal{B}H\simeq_{Ho}\Gamma_\mathcal{B}G \\
            &\implies F_*\Gamma_\mathcal{B}H\simeq_{Ho}F_*\Gamma_\mathcal{B}G \\
            &\implies N_\cat{Ch}(\mathcal{A})F_*\Gamma_\mathcal{B}H\simeq_{\cat{Ch}}N_\cat{Ch}(\mathcal{A})F_*\Gamma_\mathcal{B}G \\
            &\implies \text{Tot}_\mathcal{A}N_\cat{Ch}(\mathcal{A})F_*\Gamma_\mathcal{B}H\simeq_{\cat{Ch}}\text{Tot}_\mathcal{A}N_\cat{Ch}(\mathcal{A})F_*\Gamma_\mathcal{B}G 
        \end{align*}
        %%
        where each implication preserves naturality as well.
    \end{itemize}
\end{proof}


Next we will show the equivalent result for our new composition, $\lhd$.

\begin{lem}[label=lem:3.4NewComp]
    If $G,H:\mathcal{C}\rightarrow \cat{Ch}(\mathcal{B})$ are (naturally) chain homotopy equivalent functors, then for any pair of functors $F:\mathcal{B}\rightarrow \cat{Ch}(\mathcal{A})$ and $K:\mathcal{D}\rightarrow \cat{Ch}(\mathcal{C})$, the composites
    %%
    \begin{equation*}
        F\lhd G\lhd K,F\lhd H\lhd K:\mathcal{D}\rightarrow \cat{Ch}(\mathcal{A})
    \end{equation*}
    %%
    are (naturally) chain homotopy equivalent.
\end{lem}
\begin{proof}
    As in the previous proof we separate this proof into two parts. Let $G,H$ be (naturally) chain homotopy equivalent, and let $F$ and $K$ be functors as in the question. Let $(h,g,s,r):H\simeq_{\cat{Ch}}G$ be a (natural) chain homotopy between our original functors. We begin the construction in the case of pre-composition by $K$:
    \begin{itemize}
        \setlength{\itemindent}{2em}
        \item[[$-\lhd K$]] From the definition of $\lhd$, the composites $H\lhd K$ and $G\lhd K$ are given by
        %%
        \begin{equation*}
            H\lhd K = N_\mathcal{B}\Delta_\mathcal{B}\Gamma_*H_*\Gamma_\mathcal{C}K\;\;\text{ and }\;\;G\lhd K = N_\mathcal{B}\Delta_\mathcal{B}\Gamma_*G_*\Gamma_\mathcal{C}K
        \end{equation*}
        %%
        First, note that if the homotopy is natural $H\lhd K\simeq G\lhd K$ by the results in Sections \ref{subsec:simpHomotop} and \ref{sec:doldKan}. Otherwise, if the homotopy is pointwise, showing $H\lhd K\simeq_{\cat{Ch}}G\lhd K$ for $K$ arbitrary is equivalent to showing $N_\mathcal{B}\Delta_\mathcal{B}(\Gamma_\mathcal{B}H)_* \simeq_{\cat{Ch}}N_\mathcal{B}\Delta_\mathcal{B}(\Gamma_\mathcal{B}G)_*$. Additionally, since $N_\mathcal{B}$ both reflects and preserves homotopies, this is equivalent to showing $\Delta_\mathcal{B}(\Gamma_\mathcal{B}H)*\simeq_{Ho}\Delta_\mathcal{B}(\Gamma_\mathcal{B}G)*$. Since $\Gamma$ also preserves homotopies, it is sufficient to prove Lemma \ref{lem:PrecompDiag} \textbf{YET TO BE SHOWN FOR POINTWISE}.
        \item[[$F\lhd -$]] The composites $F \lhd H$ and $F\lhd G$ are given by
        %%
        \begin{equation*}
            F\lhd H = N_\mathcal{A}\Delta_\mathcal{A}\Gamma_*F_*\Gamma_\mathcal{B}H\;\;\text{ and }\;\;F\lhd G = N_\mathcal{A}\Delta_\mathcal{A}\Gamma_*F_*\Gamma_\mathcal{B}G
        \end{equation*}
        %%
        Using the results in Sections \ref{sec:doldKan} and \ref{subsec:simpHomotop} we have the chain of implications
        %%
        \begin{align*}
            H \simeq_{\cat{Ch}}G &\implies \Gamma_\mathcal{B}H \simeq_{Ho} \Gamma_\mathcal{B}G \\
            &\implies F_*\Gamma_\mathcal{B}H \simeq_{Ho} F_*\Gamma_\mathcal{B}G \\
            &\implies \Gamma_*F_*\Gamma_\mathcal{B}H \simeq_{Ho} \Gamma_*F_*\Gamma_\mathcal{B}G \\
            &\implies \Delta_\mathcal{A}\Gamma_*F_*\Gamma_\mathcal{B}H \simeq_{Ho} \Delta_\mathcal{A}\Gamma_*F_*\Gamma_\mathcal{B}G \\
            &\implies N_\mathcal{A}\Delta_\mathcal{A}\Gamma_*F_*\Gamma_\mathcal{B}H \simeq_{\cat{Ch}} N_\mathcal{A}\Delta_\mathcal{A}\Gamma_*F_*\Gamma_\mathcal{B}G \\
        \end{align*}
        where each implication preserves naturality as well.
    \end{itemize}
\end{proof}


We now seek to upgrade the category $\cat{HoAbCat}_{\cat{Ch}}$ in \cite{BJORT} to natural chain homotopy equivalences.

\begin{defn}[label=defn:HoAbCat]{}
    There exists a (large) category $\cat{HoAbCat}_{\cat{Ch}}$ consisting of the following data:
    \begin{itemize}
        \item Objects are abelian categories
        \item Morphisms $\mathcal{B}\rightsquigarrow \mathcal{A}$ are natural chain homotopy equivalence classes of functors $\mathcal{B}\rightarrow \cat{Ch}(\mathcal{A})$
        \item Composition of maps $\mathcal{C}\rightsquigarrow \mathcal{B}$ and $\mathcal{B}\rightsquigarrow \mathcal{A}$, corresponding to functors $G:\mathcal{C}\rightarrow \cat{Ch}(\mathcal{B})$ and $F:\mathcal{B}\rightarrow \cat{Ch}(\mathcal{A})$ is defined by the equivalence class of the composite $F\lhd G$
    \end{itemize}
\end{defn}

This definition exists since the natural chain homotopy equivalences form a congruence relation on $\cat{AbCat}_{\cat{Ch}}$.

