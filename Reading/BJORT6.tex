\section{The first directional derivative}

With the linearization results in the previous section we can now introduce directional differentiation for functors. We follow the procedure used to define the directional derivative in~\cite{BJORT}, and then proceed to show that we obtain a cartesian differential structure on our homotopy category $\cat{HoAbCat}_{\cat{Ch}}$ before extending results to include $2$-categorical structure.

\begin{defn}[label=defn:DirectionalDeriv]
    Let $F:\mathcal{B}\to \cat{Ch}(\mathcal{B})$ and let $B,V \in \mathcal{B}$, where we consider $V$ to act analogously to a direction vector in a Banach space. We define a functor $\nabla F:\mathcal{B}\times \mathcal{B}\to \cat{Ch}(\mathcal{A})$ by 
    %%
    \begin{equation*}
        \nabla F(V;X) := D_1(F(X\oplus -))(V)
    \end{equation*}
    %%
\end{defn}

We can also define the directional derivative of $F:\mathcal{B}\to \cat{Ch}(\mathcal{A})$ using the limit formulation of the directional derivative. These two constructions are equivalent.

\begin{lem}[label=DirDerivEquiv]
    Let $F:\mathcal{B}\to \cat{Ch}(\mathcal{A})$. Then there is an isomorphism 
    %%
    \begin{equation*}
        D_1^V(\ker(F(X\oplus V)\xrightarrow{F(\pi_X)}F(X)))\cong D_1(F(X\oplus -))(V)
    \end{equation*}
    %%
    as well as an isomorphism 
    %%
    \begin{equation*}
        D_1^V(\ker(F(X\oplus V)\xrightarrow{F(\pi_X)}F(X))) \cong D_1(F)(V)\oplus D_1^V(\text{cr}_2(F))(X,V)
    \end{equation*}
    %%
\end{lem}
\begin{proof}
    First, note that the kernel described in the Lemma is exactly our definition of the first cross-effect of $F(X\oplus -)$, so we obtain an isomorphism 
    %%
    \begin{equation*}
        D_1^V(\ker(F(X\oplus V)\xrightarrow{\pi_X}F(X))) \cong D_1^V(\text{cr}_1(F(X\oplus -)))(V)
    \end{equation*}
    %%
    An alternate perspective on $\ker(F(X\oplus V)\xrightarrow{F(\pi_X)} F(X))$ uses the fact that $F(X)\xrightarrow{F(\iota_X)}F(X\oplus V)$ is a section for the projection. Now, from the isomorphisms
    %%
    \begin{align*}
        F(X\oplus V) &\cong F(0)\oplus \text{cr}_1(F)(X\oplus V) \\
        &\cong F(0)\oplus \text{cr}_1(F)(X)\oplus \text{cr}_1(F)(V)\oplus \text{cr}_2(F)(X,V) \\
        &\cong F(X)\oplus \text{cr}_1(F)(V)\oplus \text{cr}_2(F)(X,V)
    \end{align*}
    %%
    it follows that $\text{ker}(F(X\oplus V)\xrightarrow F(X))$ is isomorphic to $\text{cr}_1(F)(V)\oplus \text{cr}_2(F)(X,V)$. Then, since $D_1$ is strictly linear with respect to functors by Proposition~\ref{prop:D1Exact}, we obtain an isomorphism 
    %%
    \begin{align*}
        D_1^V(\ker(F(X\oplus V)\xrightarrow{\pi_X}F(X))) &\cong D_1^V(\text{cr}_1(F)(V)\oplus \text{cr}_2(F)(X,V)) \\
        &\cong D_1^V(\text{cr}_1(F))(V)\oplus D_1^V(\text{cr}_2(F))(X,V) \\
        &\cong D_1^V(F)(V)\oplus D_1^V(\text{cr}_2(F))(X,V)
    \end{align*}
    %%
    as desired.
\end{proof}

\textbf{CHECK THIS WITH KRISTINE AND FLORIAN}


Note that since $D_1$ preserves natural chain homotopy equivalences, so does $\nabla$, so for $F\simeq_{\cat{Ch}}G$, $\nabla F\simeq_{\cat{Ch}}\nabla G$.

We now will state the theorem that with this operation $\cat{HoAbCat}_{\cat{Ch}}$ becomes a cartesian differential category before proving each piece of the theorem through a sequence of lemmas.


\begin{thm}[label=thm:6.5]
    The category $\cat{AbCat}_{\cat{Ch}}$ together with $\nabla$ satosfies the following properties for functors $F,G:\mathcal{B}\to \cat{Ch}(\mathcal{A})$, $H:\mathcal{C}\to \cat{Ch}(\mathcal{B})$, and $K:\mathcal{B}\to \cat{Ch}(\mathcal{D})$:
    \begin{itemize}
        \item[(i)] $\nabla$ is linear in the sense that 
            %%
            \begin{equation*}
                \nabla(F\oplus G)\cong \nabla(F)\oplus \nabla(G)
            \end{equation*}
            %%
        \item[(ii)] $\nabla F$ is linear in the direction variable, which is to say $\nabla F$ is strictly reduced and degree 1, so 
            %%
            \begin{equation*}
                \nabla F(V\oplus W;X)\simeq_{\cat{Ch}}\nabla F(V;X)\oplus \nabla F(W;X)
            \end{equation*}
            %%
        and $\nabla F(0;X)\cong 0$.
        \item[(iii)] The directional derivative of the degree zero functor $\deg_0^\mathcal{A}:\mathcal{A}\to \cat{Ch}(\mathcal{A})$ is the projection onto the direction, which is to say 
            %%
            \begin{equation*}
                \nabla \deg_0^\mathcal{A}(V;X)\cong V
            \end{equation*}
            %%
        \item[(iv)] We have an isomorphism
            %%
            \begin{equation*}
                \nabla \langle F,K\rangle (V;X)\cong \langle \nabla F(V;X),\nabla G(V;X)\rangle
            \end{equation*}
            %%
        \item[(v)] There is a natural chain homotopy equivalence 
            %%
            \begin{equation*}
                \nabla (F\lhd H)(V;X)\simeq_{\cat{Ch}}\nabla F \lhd (\nabla G(V;X);G(X))
            \end{equation*}
            %%
        \item[(vi)] There is an isomorphism 
            %%
            \begin{equation*}
                \nabla (\nabla F)((Z;0);(0;X)) \cong \nabla F(Z;X)
            \end{equation*}
            %%
        \item[(vii)] There is a natural chain homotopy equivalence 
            %%
            \begin{equation*}
                \nabla (\nabla F)((Z;W);(V;X)) \simeq_{\cat{Ch}} \nabla (\nabla F)((Z;V);(W;X))
            \end{equation*}
            %%
    \end{itemize}
\end{thm}

As a result of Theorem~\ref{thm:6.5} we conclude that the homotopy category $\cat{HoAbCat}_{\cat{Ch}}$ is a cartesian differential category. We now proceed to the proof of Theorem~\ref{thm:6.5} in segments.

