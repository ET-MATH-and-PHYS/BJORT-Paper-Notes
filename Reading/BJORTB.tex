\section{Quasi-isomorphism for Composition}\label{sec:Lotswork}

In this section we will prove Proposition~\ref{prop:5.7} using the work in Section~\ref{sec:bicomplexes}. To this goal let $F:\mathcal{B}\to \cat{Ch}(\mathcal{A})$ and $G:\mathcal{C}\to \cat{Ch}(\mathcal{B})$ be composable functors with $G$ reduced. We first observe the following lemma which will allow us to reduce to the case that $F$ is also reduced.

\begin{lem}[label=lem:B.1]
    Let $F:\mathcal{B}\to \cat{Ch}(\mathcal{A})$ and $G:\mathcal{C}\to \cat{Ch}(\mathcal{B})$ be composable functors with $G$ reduced. Then $\text{cr}_1(F\lhd G)\cong \text{cr}_1(F)\lhd G$.
\end{lem}
\begin{proof}
    Observe that using our construction of the cross-effect we have isomorphisms
    %%
    \begin{equation*}
        F\lhd G \cong (F\lhd G)(0) \oplus \text{cr}_1(F\lhd G)
    \end{equation*}
    %%
    and by Lemma~\ref{lem:compDirSum} and Lemma~\ref{lem:constComp}
    %%
    \begin{equation*}
        F\lhd G\cong (\text{cr}_1(F)\oplus F(0))\lhd G \cong (\text{cr}_1(F)\lhd G)\oplus (F(0)\lhd G) \cong (\text{cr}_1(F)\lhd G)\oplus F(0)
    \end{equation*}
    %%
    Further, $(F\lhd G)(0) \cong F(0)$ since $G$ is reduced and $\Gamma_\mathcal{B}$ preserves zero objects. Thus taking direct sum complements we obtain the desired isomorphism.
\end{proof}

We can use this lemma to begin reducing our goal.

\begin{cor}[label=cor:B.2]
    Let $F:\mathcal{B}\to \cat{Ch}(\mathcal{A})$ and $G:\mathcal{C}\to \cat{Ch}(\mathcal{B})$ be composable functors with $G$ reduced. Then
    %%
    \begin{equation*}
        D_1(F\lhd G)\cong D_1(\text{cr}_1(F)\lhd G)
    \end{equation*}
\end{cor}
\begin{proof}
    Recall that $D_1\cong D_1\circ \text{cr}_1$ since the first cross-effect is idempotent. Then by Lemma~\ref{lem:B.1}
    %%
    \begin{equation*}
        D_1(F\lhd G) \cong D_1(\text{cr}_1(F\lhd G)) \cong D_1(\text{cr}_1(F)\lhd G)
    \end{equation*}
    %%
    as desired.
\end{proof}

Note that since $D_1(F)\lhd D_1(G)\cong D_1(\text{cr}_1(F))\lhd D_1(G)$, Corollary~\ref{cor:B.2} implies that it is sufficient to prove Proposition~\ref{prop:5.7} when both functors are reduced. Note that Lemma~\ref{lem:funcActChain} restricts to a functor $\cat{Ch}:\text{Fun}_*(\mathcal{A},\mathcal{C})\to \text{Fun}_*(\cat{Ch}(\mathcal{A}),\cat{Ch}(\mathcal{C}))$. 


\begin{rmk}
    Let $F:\mathcal{A}\to \cat{Ch}(\mathcal{B})$ be a strictly reduced functor. We define a comparison map $\text{sw}_F:\Gamma_{\cat{Ch}(\mathcal{B})}\circ \cat{Ch}(F)\to F_*\circ \Gamma_\mathcal{A}$ at $A_\bullet \in \cat{Ch}(\mathcal{A})$ and $n \in \N$,
    %%
    \begin{equation*}
        \text{sw}_{F,A_\bullet,n}:\bigoplus_{[n]\twoheadrightarrow [k]}F(A_k)\to F\left(\bigoplus_{[n]\twoheadrightarrow [k]}A_k\right)
    \end{equation*}
    %%
    given by the universal coproduct property of the biproduct. This is evidently natural in $n$, $A_\bullet$, and $F$. Then we have that $(\Gamma_{\mathcal{B}})_*\text{sw}_F$ is given by 
    %%
    \begin{equation*}
        (\Gamma_{\mathcal{B}})_*\text{sw}_{F,A_\bullet,n}:\Gamma_\mathcal{B}\left(\bigoplus_{[n]\twoheadrightarrow [k]}F(A_k)\right)\to \Gamma_\mathcal{B}\circ F\left(\bigoplus_{[n]\twoheadrightarrow [k]}A_k\right)
    \end{equation*} 
    %%
    which at $m$ is
    %%
    \begin{equation*}
        ((\Gamma_{\mathcal{B}})_*\text{sw}_{F,A_\bullet,n})_m:\bigoplus_{[m]\twoheadrightarrow[\ell]}\left(\bigoplus_{[n]\twoheadrightarrow [k]}F(A_k)_\ell\right)\to \bigoplus_{[m]\twoheadrightarrow[\ell]}F\left(\bigoplus_{[n]\twoheadrightarrow [k]}A_k\right)_\ell
    \end{equation*}
    Taking the diagonal we obtain at $A_\bullet$ and $n$ the map
    %%
    \begin{equation*}
        \bigoplus_{[n]\twoheadrightarrow[\ell]}\left(\bigoplus_{[n]\twoheadrightarrow[k]}F(A_k)_\ell\right) \to \bigoplus_{[n]\twoheadrightarrow[\ell]}F\left(\bigoplus_{[n]\twoheadrightarrow[k]}A_k\right)_\ell
    \end{equation*}
    %%
\end{rmk}

We now show some chain homotopy equivalence results for linear functors. 


\begin{lem}[label=lem:equivDef]
    Let $F:\mathcal{A}\to \cat{Ch}(\mathcal{B})$ be linear and strictly reduced. Then $\text{Tot}_*NF_*\Gamma_\mathcal{A}$ and $\text{Tot}_*\cat{Ch}(F)$ are \rd{naturally} chain homotopy equivalent.
\end{lem}
\begin{proof}
    Note that by assumption $F$ is linear and strictly reduced. Then by Proposition~\ref{prop:linearEquiv}, for each $A_\bullet \in \cat{Ch}(\mathcal{A})$ and each $n$ we have a natural chain homotopy equivalence
    %%
    \begin{equation*}
        F\left(\bigoplus_{[n]\twoheadrightarrow[k]}A_k\right) \simeq_{\cat{Ch}} \bigoplus_{[n]\twoheadrightarrow[k]}F(A_k)
    \end{equation*}
    %%
    which is natural in the $A_k$, and hence in $A_{\bullet}$. Note that in particular this is a natural chain homotopy equivalence of columns induced by the natural inclusion of bicomplexes from $\mathfrak{C}F_*\Gamma_\mathcal{A}$ into $\mathfrak{C}\Gamma_{\cat{Ch}(\mathcal{B})}\cat{Ch}(F)$. By Theorem~\ref{thm:A2} we obtain a natural chain homotopy equivalence between totalizations.
    

    By~\cite[Thm 2.5]{goerss-jardine} we have that $N$ and $\mathfrak{C}$ are natural chain homotopy equivalent. Thus, since natural chain homotopy equivalence is preserved by pre-composition and post-composition, and $\text{Tot}_*$ preserves natural chain homotopy equivalences, it follows that we obtain the natural chain homotopy equivalence
    %%
    \begin{equation*}
        \text{Tot}_*NF_*\Gamma_\mathcal{A}\simeq_{\cat{Ch}} \text{Tot}_*N\Gamma_{\cat{Ch}(\mathcal{B})}\cat{Ch}(F)
    \end{equation*}
    %%
    Finally, since $N\Gamma_{\cat{Ch}(\mathcal{B})} \cong 1_{\cat{Ch}^2(\mathcal{B})}$, it follows that 
    %%
    \begin{equation*}
        \text{Tot}_*NF_*\Gamma_\mathcal{A}\simeq_{\cat{Ch}} \text{Tot}_*\cat{Ch}(F)
    \end{equation*}
    %%
    as desired.
\end{proof}


Using the generalized Eilenberg-Zilber Theorem proved in~\cite[Chap. 7, Thm 4.1]{barr2002acyclic} we can relate this result on composition in~\cite{BJORT} to our current convention for composition in $\cat{AbCat}_{\cat{Ch}}$. In particular, the version of the generalized Eilenberg-Zilber Theorem in~\cite[Chap. 7, Thm 4.1]{barr2002acyclic} tells us that we have a natural chain homotopy equivalence between $\mathfrak{C}\Delta$ and $\text{Tot}_*\mathfrak{C}^2$ \textbf{Double check this}, or equivalently by~\cite[Thm 2.5]{goerss-jardine}, a natural chain homotopy equivalence between $N\Delta$ and $\text{Tot}_*N^2$. Thus, for $F:\mathcal{A}\to \cat{Ch}(\mathcal{B})$ and $G:\mathcal{B}\to \cat{Ch}(\mathcal{C})$, we have a natural chain homotopy equivalence 
%%
\begin{align*}
    N_\mathcal{C}\Delta_\mathcal{C}(\Gamma_\mathcal{C})_*G_*\Gamma_\mathcal{B}F &\simeq_{\cat{Ch}} (\text{Tot}_\mathcal{C})_*N_{\cat{Ch}(\mathcal{C})}(N_\mathcal{C})_*(\Gamma_\mathcal{C})_*G_*\Gamma_\mathcal{B}F \\
    &\cong (\text{Tot}_\mathcal{C})_*N_{\cat{Ch}(\mathcal{C})}G_*\Gamma_\mathcal{B}F 
\end{align*}
%%
which is exactly composition, as defined in~\cite{BJORT}. Thus, by Lemma~\ref{lem:equivDef} we have the natural chain homotopy equivalence when $G$ is linear and strictly reduced
%%
\begin{equation*}
    G\lhd F \simeq_{\cat{Ch}} (\text{Tot}_\mathcal{C})_*\cat{Ch}(G)F
\end{equation*}
%%



Before proving the final claim we prove a useful property of the diagonal functor.

\begin{lem}[label=lem:B.8]
    For any $G:\mathcal{C}\to \cat{Ch}(\mathcal{B})$ and any $F:\mathcal{B}\to \cat{Ch}(\mathcal{A})$, we have the commutative diagrams
    %%
    \begin{equation*}
        \begin{tikzcd}
            {\text{Fun}_*(\mathcal{C}^n,\cat{Ch}(\mathcal{B}))} & {\text{Fun}_*(\mathcal{C},\cat{Ch}(\mathcal{B}))} \\
            {\text{Fun}_*(\mathcal{C}^n,\cat{Ch}(\mathcal{A}))} & {\text{Fun}_*(\mathcal{C},\cat{Ch}(\mathcal{A}))}
            \arrow["\Delta", from=1-1, to=1-2]
            \arrow["{F_*}"', from=1-1, to=2-1]
            \arrow["{F_*}", from=1-2, to=2-2]
            \arrow["\Delta"', from=2-1, to=2-2]
        \end{tikzcd},\;\;\begin{tikzcd}
            {\text{Fun}_*(\mathcal{B}^n,\cat{Ch}(\mathcal{A}))} & {\text{Fun}_*(\mathcal{B},\cat{Ch}(\mathcal{A}))} \\
            {\text{Fun}_*(\mathcal{C}^n,\cat{Ch}(\mathcal{A}))} & {\text{Fun}_*(\mathcal{C},\cat{Ch}(\mathcal{A}))}
            \arrow["\Delta", from=1-1, to=1-2]
            \arrow["{G^*}"', from=1-1, to=2-1]
            \arrow["{G^*}", from=1-2, to=2-2]
            \arrow["\Delta"', from=2-1, to=2-2]
        \end{tikzcd}
    \end{equation*}
    %%
    where pre- and post-composition is done using the composition in $\cat{AbCat}_{\cat{Ch}}$.
\end{lem}
\begin{proof}
    First, let $H:\mathcal{C}^n\to \cat{Ch}(\mathcal{B})$. Then $F_*\Delta(H):\mathcal{C}\to \cat{Ch}(\mathcal{A})$ is given by 
    %%
    \begin{equation*}
        N_\mathcal{A}\Delta_\mathcal{A}(\Gamma_\mathcal{A})_*F_*\Gamma_\mathcal{B}\Delta(H)
    \end{equation*}
    %%
    while $\Delta(F_*H)$ is 
    %%
    \begin{equation*}
        \Delta(N_\mathcal{A}\Delta_\mathcal{A}(\Gamma_\mathcal{A})_*F_*\Gamma_\mathcal{B}H)
    \end{equation*}
    %%
    which, upon evaluating at an object $X$ and a morphism $f:X\to Y$ in $\mathcal{C}$, gives the same result.

    \vspace{10pt}

    Next, let $K:\mathcal{B}^n\to \cat{Ch}(\mathcal{A})$. Then $G^*(\Delta(K))$ is given by 
    %%
    \begin{equation*}
        N_\mathcal{A}\Delta_\mathcal{A}(\Gamma_\mathcal{A})_*\Delta(K)_*\Gamma_\mathcal{B}G
    \end{equation*}
    %%
    while $\Delta(G^*K)$ is given by 
    %%
    \begin{equation*}
        \Delta(N_\mathcal{A}\Delta_\mathcal{A}(\Gamma_\mathcal{A})_*K_*\Gamma_{\mathcal{B}^n}(G\times \cdots \times G))
    \end{equation*}
    %%
    \textbf{WRITE OUT FULLY}
\end{proof}

We now have enough to prove the main claim of the section, Proposition~\ref{prop:5.7}. In particular, we can prove the claim for both $F$ and $G$ reduced.

\begin{proof}[Proof of Proposition~\ref{prop:5.7}]
    \textbf{TBC}
\end{proof}