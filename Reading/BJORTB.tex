\section{Quasi-isomorphism for Composition}\label{sec:Lotswork}

In this section we will prove Proposition~\ref{prop:5.7} using the work in Section~\ref{sec:bicomplexes}. To this goal let $F:\mathcal{B}\to \cat{Ch}(\mathcal{A})$ and $G:\mathcal{C}\to \cat{Ch}(\mathcal{B})$ be composable functors with $G$ reduced. We first observe the following lemma which will allow us to reduce to the case that $F$ is also reduced.

\begin{lem}[label=lem:B.1]
    Let $F:\mathcal{B}\to \cat{Ch}(\mathcal{A})$ and $G:\mathcal{C}\to \cat{Ch}(\mathcal{B})$ be composable functors with $G$ reduced. Then $\text{cr}_1(F\lhd G)\cong \text{cr}_1(F)\lhd G$.
\end{lem}
\begin{proof}
    Observe that using our construction of the cross-effect we have isomorphisms
    %%
    \begin{equation*}
        F\lhd G \cong (F\lhd G)(0) \oplus \text{cr}_1(F\lhd G)
    \end{equation*}
    %%
    and by Lemma~\ref{lem:compDirSum} and Lemma~\ref{lem:constComp}
    %%
    \begin{equation*}
        F\lhd G\cong (\text{cr}_1(F)\oplus F(0))\lhd G \cong (\text{cr}_1(F)\lhd G)\oplus (F(0)\lhd G) \cong (\text{cr}_1(F)\lhd G)\oplus F(0)
    \end{equation*}
    %%
    Further, $(F\lhd G)(0) \cong F(0)$ since $G$ is reduced and $\Gamma_\mathcal{B}$ preserves zero objects. Thus taking direct sum complements we obtain the desired isomorphism.
\end{proof}

We can use this lemma to begin reducing our goal.

\begin{cor}[label=cor:B.2]
    Let $F:\mathcal{B}\to \cat{Ch}(\mathcal{A})$ and $G:\mathcal{C}\to \cat{Ch}(\mathcal{B})$ be composable functors with $G$ reduced. Then
    %%
    \begin{equation*}
        D_1(F\lhd G)\cong D_1(\text{cr}_1(F)\lhd G)
    \end{equation*}
\end{cor}
\begin{proof}
    Recall that $D_1\cong D_1\circ \text{cr}_1$ since the first cross-effect is idempotent. Then by Lemma~\ref{lem:B.1}
    %%
    \begin{equation*}
        D_1(F\lhd G) \cong D_1(\text{cr}_1(F\lhd G)) \cong D_1(\text{cr}_1(F)\lhd G)
    \end{equation*}
    %%
    as desired.
\end{proof}

Note that since $D_1(F)\lhd D_1(G)\cong D_1(\text{cr}_1(F))\lhd D_1(G)$, Corollary~\ref{cor:B.2} implies that it is sufficient to prove Proposition~\ref{prop:5.7} when both functors are reduced. Note that Lemma~\ref{lem:funcActChain} restricts to a functor $\cat{Ch}:\text{Fun}_*(\mathcal{A},\mathcal{C})\to \text{Fun}_*(\cat{Ch}(\mathcal{A}),\cat{Ch}(\mathcal{C}))$. 


\begin{rmk}
    Let $F:\mathcal{A}\to \cat{Ch}(\mathcal{B})$ be a strictly reduced functor. We define a comparison map $\text{sw}_F:\Gamma_{\cat{Ch}(\mathcal{B})}\circ \cat{Ch}(F)\to F_*\circ \Gamma_\mathcal{A}$ at $A_\bullet \in \cat{Ch}(\mathcal{A})$ and $n \in \N$,
    %%
    \begin{equation*}
        \text{sw}_{F,A_\bullet,n}:\bigoplus_{[n]\twoheadrightarrow [k]}F(A_k)\to F\left(\bigoplus_{[n]\twoheadrightarrow [k]}A_k\right)
    \end{equation*}
    %%
    given by the universal coproduct property of the biproduct. This is evidently natural in $n$, $A_\bullet$, and $F$. Next, observe that
    %%
    \begin{equation*}
        (\Gamma_\mathcal{B})_*\gamma_{\cat{Ch}(\mathcal{B})}\cat{Ch}(F)(A_\bullet)
    \end{equation*}
    Then we have that $(\Gamma_{\mathcal{B}})_*\text{sw}_F$ is given by 
    %%
    \begin{equation*}
        (\Gamma_{\mathcal{B}})_*\text{sw}_{F,A_\bullet,n}:\Gamma_\mathcal{B}\left(\bigoplus_{[n]\twoheadrightarrow [k]}F(A_k)\right)\to \Gamma_\mathcal{B}\circ F\left(\bigoplus_{[n]\twoheadrightarrow [k]}A_k\right)
    \end{equation*} 
    %%
    which at $m$ is
    %%
    \begin{equation*}
        ((\Gamma_{\mathcal{B}})_*\text{sw}_{F,A_\bullet,n})_m:\bigoplus_{[m]\twoheadrightarrow[\ell]}\left(\bigoplus_{[n]\twoheadrightarrow [k]}F(A_k)_\ell\right)\to \bigoplus_{[m]\twoheadrightarrow[\ell]}F\left(\bigoplus_{[n]\twoheadrightarrow [k]}A_k\right)_\ell
    \end{equation*}
    Taking the diagonal we obtain at $A_\bullet$ and $n$ the map
    %%
    \begin{equation*}
        \bigoplus_{[n]\twoheadrightarrow[\ell]}\left(\bigoplus_{[n]\twoheadrightarrow[k]}F(A_k)_\ell\right) \to \bigoplus_{[n]\twoheadrightarrow[\ell]}F\left(\bigoplus_{[n]\twoheadrightarrow[k]}A_k\right)_\ell
    \end{equation*}
    %%
\end{rmk}

We now show that under $N_\mathcal{B}$ this map becomes a chain homotopy equivalence when the functor $F$ is reduced and degree $1$.

\begin{lem}[label=lem:equivDef]
    Let $F:\mathcal{A}\to \cat{Ch}(\mathcal{B})$ be linear. Then $F_*\Gamma_\mathcal{A}$ and $\Gamma_{\cat{Ch}(\mathcal{B})}\cat{Ch}(F)$ are \rd{naturally} chain homotopy equivalent.
\end{lem}
\begin{proof}
    Note that by assumption $F$ is linear and strictly reduced. Then by Proposition~\ref{prop:linearEquiv}, for each $A_\bullet \in \cat{Ch}(\mathcal{A})$ and each $n$ we have a natural chain homotopy equivalence
    %%
    \begin{equation*}
        F\left(\bigoplus_{[n]\twoheadrightarrow[k]}A_k\right) \simeq_{\cat{Ch}} \bigoplus_{[n]\twoheadrightarrow[k]}F(A_k)
    \end{equation*}
    %%
    which is natural in the $A_k$. We want to enhance these to a natural chain homotopy $F_*\Gamma_\mathcal{A}\simeq_{\cat{Ch}}\Gamma_{\cat{Ch}(\mathcal{B})}\cat{Ch}(F)$. To this end let $\alpha_n,\beta_n,s^n,r^n$ denote such a natural chain homotopy for each $n$. 
\end{proof}