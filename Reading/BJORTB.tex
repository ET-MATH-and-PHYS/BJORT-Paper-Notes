\section{Quasi-isomorphism for Composition}\label{sec:Lotswork}

In this section we will prove Proposition~\ref{prop:5.7} using the work in Section~\ref{sec:bicomplexes}. To this goal let $F:\mathcal{B}\to \cat{Ch}(\mathcal{A})$ and $G:\mathcal{C}\to \cat{Ch}(\mathcal{B})$ be composable functors with $G$ reduced. We first observe the following lemma which will allow us to reduce to the case that $F$ is also reduced.

\begin{lem}[label=lem:B.1]
    Let $F:\mathcal{B}\to \cat{Ch}(\mathcal{A})$ and $G:\mathcal{C}\to \cat{Ch}(\mathcal{B})$ be composable functors with $G$ reduced. Then $\text{cr}_1(F\lhd G)\cong \text{cr}_1(F)\lhd G$.
\end{lem}
\begin{proof}
    Observe that using our construction of the cross-effect we have isomorphisms
    %%
    \begin{equation*}
        F\lhd G \cong (F\lhd G)(0) \oplus \text{cr}_1(F\lhd G)
    \end{equation*}
    %%
    and by Lemma~\ref{lem:compDirSum}
    %%
    \begin{equation*}
        F\lhd G\cong (\text{cr}_1(F)\oplus F(0))\lhd G \cong (\text{cr}_1(F)\lhd G)\oplus (F(0)\lhd G) \cong (\text{cr}_1(F)\lhd G)\cong F(0)
    \end{equation*}
    %%
    by Lemma \textbf{LEMMA ON CONSTANT FUNC COMP}. Further, \textbf{NEED RESULT ON COMPOSITION EVALUATION}
\end{proof}

We can use this lemma to begin reducing our goal.

\begin{cor}[label=cor:B.2]
    Let $F:\mathcal{B}\to \cat{Ch}(\mathcal{A})$ and $G:\mathcal{C}\to \cat{Ch}(\mathcal{B})$ be composable functors with $G$ reduced. Then
    %%
    \begin{equation*}
        D_1(F\lhd G)\cong D_1(\text{cr}_1(F)\lhd G)
    \end{equation*}
\end{cor}
\begin{proof}
    Recall that $D_1\cong D_1\circ \text{cr}_1$ since the first cross-effect is idempotent. Then by Lemma~\ref{lem:B.1}
    %%
    \begin{equation*}
        D_1(F\lhd G) \cong D_1(\text{cr}_1(F\lhd G)) \cong D_1(\text{cr}_1(F)\lhd G)
    \end{equation*}
    %%
    as desired.
\end{proof}

Note that since $D_1(F)\lhd D_1(G)\cong D_1(\text{cr}_1(F))\lhd D_1(G)$, Corollary~\ref{cor:B.2} implies that it is sufficient to prove Proposition~\ref{prop:5.7} when both functors are reduced. 