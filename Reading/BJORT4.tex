\section{The Taylor tower in abelian functor calculus}

With the appropriate categorical language built up, we can begin constructing Taylor towers, as in \cite{JohnsonB.2004Dcwc}. This process involves defining certain polynomial functors of \textbf{degree $n$} with natural transformations which go down by one in degree. In \cite{BJORT} many definitions are given in terms of pointwise chain homotopies. In order to introduce two-dimensional structures on the cat $\cat{HoAbCat}_{\cat{Ch}}$ we aim to upgrade these to natural homotopies. We distinguish where these changes are made by changing the text color to red for occurrences of natural which aren't originally in the text.

\begin{defn}[label=defn:4.1]{}
    A functor $F:\mathcal{B}\rightarrow \cat{Ch}(\mathcal{A})$ is \textbf{degree $n$} if $\text{cr}_{n+1}(F):\mathcal{B}^{n+1}\rightarrow \cat{Ch}(\mathcal{A})$ is \textbf{contractible}, i.e., \rd{natural} chain homotopy equivalent to zero. 
\end{defn}

Note that $\cat{cr}_k(F) \simeq_{\cat{Ch}} 0$ implies $\cat{cr}_\ell(F) \simeq_\cat{Ch} 0$ for any $\ell > k$. Indeed, since $G \oplus F \simeq_{\cat{Ch}} 0$ implies $F \simeq_{\cat{Ch}} 0$ from Section~\ref{subsec:chainHomotop}, this follows from the inductive definition. Consequently, functors of degree $k$ are also of degree $\ell$ for $\ell > k$.

From \cite[Defn 2.4]{JohnsonB.2004Dcwc}, chain complexes can be constructed from a pair of a comonad and an object in the category on which the comonad acts.

\begin{lem}[label=lem:comonadChain]
    Let $C:\text{Fun}(\mathcal{B},\mathcal{A})\rightarrow \text{Fun}(\mathcal{B},\mathcal{A})$ be a comonad on a functor category where $\mathcal{A}$ is an abelian category (so in particular $\text{Fun}(\mathcal{B},\mathcal{A})$ is abelian). Then $C$ induces a functor $$C^\cat{Ch}:\text{Fun}(\mathcal{B},\mathcal{A})\rightarrow \cat{Ch}(\text{Fun}(\mathcal{B},\mathcal{A}))$$
\end{lem}
\begin{proof}
    Let $\epsilon:C\rightarrow 1$ be the counit of the comonad. Then for $F :\mathcal{B}\rightarrow \mathcal{A}$, define $C^\cat{Ch}(F)$ to be the chain
    %%
    \begin{equation*}
        \cdots \rightarrow C^3(F)\xrightarrow{\epsilon_{C^2(F)}-C\epsilon_{C(F)}+C^2\epsilon_F}C^2(F)\xrightarrow{\epsilon_{C(F)}-C\epsilon_F}C(F)\xrightarrow{\epsilon_F} F
    \end{equation*}
    %%
    where the $k$th differential is defined by the alternating sum $\sum_{i=0}^{k-1}(-1)^iC^i\epsilon_{C^{(k-i)}}$. Note these differentials are indeed natural, and hence this defines a sequence of functors and natural transformations. To see that this sequence forms a chain complex observe that
    %%
    \begin{align*}
        \left(\sum_{i=0}^{n}(-1)^iC^i\epsilon_{C^{n-i}(F)}\right)&\circ \left(\sum_{i=0}^{n+1}(-1)^iC^i\epsilon_{C^{n+1-i}(F)}\right) \\
        &= \sum_{i=0}^n\sum_{j=0}^{n+1}(-1)^{i+j}C^i\epsilon_{C^{n-i}(F)}\circ C^j\epsilon_{C^{n+1-j}(F)} \\
        &= \sum_{i=0}^n\sum_{i < j}^{n+1}(-1)^{i+j}C^i\epsilon_{C^{n-i}(F)}\circ C^j\epsilon_{C^{n+1-j}(F)} \\
        &+ \sum_{i=0}^n\sum_{i\geq j}(-1)^{i+j}C^i\epsilon_{C^{n-i}(F)}\circ C^j\epsilon_{C^{n+1-j}(F)} \\
        &= \sum_{i=0}^n\sum_{i < j}^{n+1}(-1)^{i+j}C^i\left(\epsilon_{C^{n-i}(F)}\circ C^{j-i}\epsilon_{C^{n+1-j}(F)} \right)\\
        &+ \sum_{i=0}^n\sum_{i\geq j}(-1)^{i+j}C^j\left(C^{i-j}\epsilon_{C^{n-i}(F)}\circ \epsilon_{C^{n+1-j}(F)} \right)\\ 
        &= \sum_{i=0}^n\sum_{i\leq k}^n(-1)^{i+k+1}C^{i}\left(\epsilon_{C^{n-i}(F)}\circ C^{k+1-i}\epsilon_{C^{n-k}(F)} \right) \tag{Substituting $k = j-1$}\\
        &+ \sum_{i=0}^n\sum_{i\geq j}(-1)^{i+j}C^j\left(C^{i-j}\epsilon_{C^{n-i}(F)}\circ \epsilon_{C^{n+1-j}(F)} \right)\\ 
        &= \sum_{i=0}^n\sum_{i\leq k}^n(-1)^{i+k+1}C^{i}\left(C^{k-i}\epsilon_{C^{n-k}(F)}\circ \epsilon_{C^{n+1-i}(F)}\right) \tag{Naturality of $\epsilon$} \\
        &+ \sum_{i=0}^n\sum_{i\geq j}(-1)^{i+j}C^j\left(C^{i-j}\epsilon_{C^{n-i}(F)}\circ \epsilon_{C^{n+1-j}(F)} \right)\\ 
        &= -\sum_{k=0}^n\sum_{k\geq i}(-1)^{i+k}C^{i}\left(C^{k-i}\epsilon_{C^{n-k}(F)}\circ \epsilon_{C^{n+1-i}(F)}\right) \tag{Re-ordering the sum} \\
        &+ \sum_{i=0}^n\sum_{i\geq j}(-1)^{i+j}C^j\left(C^{i-j}\epsilon_{C^{n-i}(F)}\circ \epsilon_{C^{n+1-j}(F)} \right) \\
        &= 0
    \end{align*}
    Next, let $\alpha:F\Rightarrow G$ be a natural transformation. Then $C^\cat{Ch}(\alpha):C^\cat{Ch}(F)\rightarrow C^\cat{Ch}(G)$ is defined by $C^\cat{Ch}(\alpha)_n:= C^n\alpha:C^nF\rightarrow C^nG$. To see that this is a chain map observe that for $0 \leq i \leq n$ we have the commutative diagram
    \[\begin{tikzcd}
    	{C^{n+1}(F)} & {C^n(F)} \\
    	{C^{n+1}(G)} & {C^n(G)}
    	\arrow["{C^i\epsilon_{C^{n-i}(F)}}", from=1-1, to=1-2]
    	\arrow["{C^n(\alpha)}", from=1-2, to=2-2]
    	\arrow["{C^{n+1}(\alpha)}"', from=1-1, to=2-1]
    	\arrow["{C^i\epsilon_{C^{n-i}(G)}}"', from=2-1, to=2-2]
    \end{tikzcd}\]
    which commutes by naturality of $\epsilon$. Since composition is bilinear with respect to the group operation on hom sets in an abelian category we have that the $C^n(\alpha)$ form a chain map in $\cat{Ch}(\text{Fun}(\mathcal{B},\mathcal{A}))$. Further, since $C$ is a functor so is $C^\cat{Ch}$, completing the proof.
\end{proof}

Note that we can realize this functor as going to $\text{Fun}(\mathcal{B},\cat{Ch}(\mathcal{A}))$. Indeed, $\cat{Ch}(\text{Fun}(\mathcal{B},\mathcal{A}))\cong \text{Fun}(\mathcal{B},\cat{Ch}(\mathcal{A}))$, as we show in Lemma \ref{lem:funcChain}. We can use this technique to define the polynomial approximations in the Taylor tower for a functor.

\begin{defn}[label=defn:4.2]{}
    The \textbf{$n$th polynomial approximation} is the composite functor $P_n := (\text{Tot}_\mathcal{A})_*\circ \text{Fun}^\cat{ch}\circ C_{n+1}^{\cat{Ch}}:\text{Fun}(\mathcal{B},\cat{Ch}(\mathcal{A}))\rightarrow \text{Fun}(\mathcal{B},\cat{Ch}(\mathcal{A}))$. 
\end{defn}


Recall by Lemma \ref{lem:idempotCr1} if $n = 0$, $C_1(F) = \text{cr}_1(\text{cr}_1(F))=\text{cr}_1(F)$ (by our choice in defining $\text{cr}_1$), and so $C_1^{\times k}(F) = \text{cr}_1(F)$ for each $k \geq 1$. Further, the co-unit $\epsilon:\Delta^*(\text{cr}_1(\text{cr}_1(F)))(X) = \text{cr}_1(F)(X)\rightarrow F(X)$ is the kernel map in $\text{cr}_1(F)\rightarrow F(X)\rightarrow F(\star)$. In the case of $\epsilon_{C_1}$ the kernel map is the identity from our definition of $\text{cr}_1$, while by the definition of $\text{cr}_1$ on maps we have
\[\begin{tikzcd}
	{\text{cr}_1(F)(X)} & {\text{cr}_1(F)(X)} & 0 \\
	{\text{cr}_1(F)(X)} & {F(X)} & {F(\star)}
	\arrow[Rightarrow, no head, from=1-1, to=1-2]
	\arrow["{!}", from=1-2, to=1-3]
	\arrow["{\text{cr}_1(\epsilon)}"', from=1-1, to=2-1]
	\arrow["\epsilon"', from=1-2, to=2-2]
	\arrow[from=2-2, to=2-3]
	\arrow[from=1-3, to=2-3]
	\arrow["\epsilon"', tail, from=2-1, to=2-2]
\end{tikzcd}\]
so $C_1(\epsilon)$ is the identity as well since $\epsilon$ is monic. Hence, we obtain the chain complex of functors
%%
\begin{equation*}
    \cdots \xrightarrow{0}\text{cr}_1(F)\xrightarrow{\id}\text{cr}_1(F)\xrightarrow{0}\text{cr}_1(F)\xrightarrow{\epsilon}F
\end{equation*}
%%
Since $F \cong \text{cr}_1(F)\oplus F(\star)$, the chain complex defining $P_0(F)$ can be written as a direct sum of the two chain complexes
%%
\begin{equation*}
    \cdots \xrightarrow{0}\text{cr}_1(F)\xrightarrow{\id}\text{cr}_1(F)\xrightarrow{0}\text{cr}_1(F)\xrightarrow{\id}\text{cr}_1(F)
\end{equation*}
%%
and
%%
\begin{equation*}
    \cdots\rightarrow 0\rightarrow 0 \rightarrow 0 \rightarrow F(\star)
\end{equation*}
%%
Note that the chain complex in the top line is contractible. Recall the totalization commutes with direct sums. The totalization of the second complex is isomorphic to $F(\star)$ itself. On the other hand, the totalization of the first complex, $C_\bullet$, has $C_0 = \text{cr}_1(F)_0$, $C_1 = \text{cr}_1(F)_1\oplus \text{cr}_1(F)_0$, and in general
%%
\begin{equation*}
    C_n = \bigoplus_{i=0}^n\text{cr}_1(F)_i
\end{equation*}
%%
with differential $\partial_n:C_n\rightarrow C_{n-1}$ given by 
%%
\begin{equation*}
    \partial_n = (\delta_{n-i,even}\pi_{C_{i-1}}-\partial_i^{\text{cr}_1(F)}\circ \pi_{C_i})_{1\leq i \leq n}
\end{equation*}
%%
where $\delta_{n-i,even}$ is $1$ when $2\mid n-i$ and $0$ else. Then $P_0(F)$ is the direct sum of these two sequences. However, since the first complex before totalization is contractible, we can model $P_0(F)(X) \cong F(\star)$.

We test the above computation, and the results to follow, using the example of $\deg_0^\mathcal{A}:\mathcal{A}\rightarrow \cat{Ch}(\mathcal{A})$

\begin{eg}{}
    First, note that the chain complex $\deg_0^\mathcal{A}(0)$ is the zero complex. Since $\deg_0^\mathcal{A}$ is reduced we also have that $\text{cr}_1(\deg_0^\mathcal{A}) = \deg_0^\mathcal{A}$. It follows that $P_0(\deg_0^\mathcal{A})_n = 1_\mathcal{A}$ for each $n \geq 0$, and $\partial_n = \delta_{n-1,even}1_{\mathcal{A}}$, or in other words
    %%
    \begin{equation*}
        P_0(\deg_0^\mathcal{A}) := \cdots \xrightarrow{1_{1_\mathcal{A}}} 1_\mathcal{A}\xrightarrow{0} 1_\mathcal{A}\xrightarrow{1_{1_\mathcal{A}}} 1_\mathcal{A}
    \end{equation*}
    %%
    Note that this complex is contractible.
\end{eg}

Using the isomorphisms in Section~\ref{subsec:chainHomotop} we can now describe the cross-effect on functors into chain complexes explicitly. Next we show compatibility of this functor with the isomorphism $\text{Fun}^\cat{Ch}$.

\begin{lem}[label=lem:ChFuncCommute]
    For any pointed category $\mathcal{B}$ and abelian category $\mathcal{A}$, we have a natural isomorphism
    %%
    \begin{equation*}
        \text{cr}_n^{\mathcal{B},\cat{Ch}(A)} \cong \text{Fun}^\cat{Ch}\circ \cat{Ch}(\text{cr}_n^{\mathcal{B},\mathcal{A}})\circ (\text{Fun}^\cat{Ch})^{-1}
    \end{equation*}
    %%
\end{lem}
\begin{proof}
    Since the cross effect is additive we can apply $\cat{Ch}$, so the claim is well-posed. Note that finite limits in functor categories between abelian categories are computed pointwise, up to natural isomorphism, and the same holds for chain complexes, from Sections~\ref{sec:colimFuncs} and~\ref{subsec:chainHomotop}. It follows that 
    %%
    \begin{equation*}
        \text{cr}_n^{\mathcal{B},\cat{Ch}(A)} \cong \text{Fun}^\cat{Ch}\circ \cat{Ch}(\text{cr}_n^{\mathcal{B},\mathcal{A}})\circ (\text{Fun}^\cat{Ch})^{-1}
    \end{equation*}
    %%
\end{proof}

Note that in addition $\Delta^*\circ \text{Fun}^{\cat{Ch}} = \text{Fun}^{\cat{Ch}}\circ \cat{Ch}(\Delta^*)$, so we also have 
\begin{equation*}
    C_n^{\mathcal{B},\cat{Ch}(A)}\circ \text{Fun}^\cat{Ch} \cong \text{Fun}^\cat{Ch}\circ \cat{Ch}(C_n^{\mathcal{B},\mathcal{A}})
\end{equation*}

Finally, we show that the $C^{\cat{Ch}}$ construction also commutes with the chain complex functor when $C$ is additive.

\begin{lem}[label=lem:ChConsPtwise]
    Let $(C,\epsilon,\delta)$ be a comonad on $\mathcal{A}$ which is also an additive functor. Then we have the equality
    %%
    \begin{equation*}
        \cat{Ch}(C)^{\cat{Ch}} = \cat{Ch}(C^{\cat{Ch}})
    \end{equation*}
    %%
\end{lem}
\begin{proof}
    Let $A_\bullet \in \cat{Ch}(\mathcal{A})$. Then both $\cat{Ch}(C)^{\cat{Ch}}(A_\bullet)$ and $\cat{Ch}(C^{\cat{Ch}})(A_\bullet)$ give the same double complex after the isomorphism $\cat{Ch}(\cat{Ch}(\mathcal{A}))\to \cat{Ch}(\cat{Ch}(\mathcal{A}))$ which swaps the rows and columns (i.e. transposition)
    %%
    \[\begin{tikzcd}
        & \vdots && \vdots & \vdots \\
        \cdots & {C^2(A_{n+1})} && {C(A_{n+1})} & {A_{n+1}} \\
        \cdots & {C^2(A_n)} && {C(A_n)} & {A_n} \\
        \cdots & {C^2(A_{n-1})} && {C(A_{n-1})} & {A_{n-1}} \\
        & \vdots && \vdots & \vdots
        \arrow["{C^2(\partial_{n+1}^A)}"', from=2-2, to=3-2]
        \arrow["{C^2(\partial_n^A)}"', from=3-2, to=4-2]
        \arrow["{C(\partial_{n+1}^A)}", from=2-4, to=3-4]
        \arrow["{C(\partial_n^A)}", from=3-4, to=4-4]
        \arrow["{\partial_{n+1}^A}", from=2-5, to=3-5]
        \arrow["{\partial_n^A}", from=3-5, to=4-5]
        \arrow["{C\epsilon_{A_n}-\epsilon_{CA_n}}", from=3-2, to=3-4]
        \arrow["{\epsilon_{A_n}}", from=3-4, to=3-5]
        \arrow["{\epsilon_{A_{n-1}}}"', from=4-4, to=4-5]
        \arrow["{\epsilon_{A_{n+1}}}", from=2-4, to=2-5]
        \arrow["{C\epsilon_{A_{n+1}}-\epsilon_{CA_{n+1}}}", from=2-2, to=2-4]
        \arrow["{C\epsilon_{A_{n-1}}-\epsilon_{CA_{n-1}}}"', from=4-2, to=4-4]
        \arrow[from=4-2, to=5-2]
        \arrow[from=4-4, to=5-4]
        \arrow[from=4-5, to=5-5]
        \arrow[from=1-5, to=2-5]
        \arrow[from=1-4, to=2-4]
        \arrow[from=1-2, to=2-2]
        \arrow[from=2-1, to=2-2]
        \arrow[from=3-1, to=3-2]
        \arrow[from=4-1, to=4-2]
    \end{tikzcd}\]
    %%
\end{proof}

These results and Lemma~\ref{lem:addFuncPres} allow us to formalize preservation of chain homotopies for our approximation functors. Due to Proposition~\ref{prop:exactCross} we also obtain nice properties for the functors $P_n:\text{Fun}(\mathcal{B},\mathcal{A})\rightarrow \text{Fun}(\mathcal{B},\mathcal{A})$.

\begin{prop}[label=prop:exactPol]
    For any $n \geq 0$,
    \begin{itemize}
        \item[(i)] $P_n:\text{Fun}(\mathcal{B},\cat{Ch}(\mathcal{A}))\rightarrow \text{Fun}(\mathcal{B},\cat{Ch}(\mathcal{A}))$ is exact
        \item[(ii)] $P_n$ preserves \rd{natural} chain homotopies, chain homotopy equivalences, and contractibility. 
    \end{itemize}
\end{prop}
\begin{proof}
    It is sufficient to prove (i) for short exact sequences. Let $0 \rightarrow F\rightarrow G\rightarrow H\rightarrow 0$ be a SES of functors in $\cat{AbCat}_{\cat{Ch}}$. By Proposition~\ref{prop:exactCross} we obtain a SES of bicomplexes in the definition of the $n$th polynomial approximation. Since totalization is exact we obtain a SES $0 \rightarrow P_n(F)\rightarrow P_n(G)\rightarrow P_n(H)\rightarrow 0$. 
    
    \vspace{10pt}
    
    For (ii), observe that from our work in Sections~\ref{subsec:chainHomotop} and~\ref{sec:colimFuncs}
    %%
    \begin{align*}
        (\text{Tot}_{\mathcal{A}})_*\circ \text{Fun}^{\cat{Ch}}\circ C_{n+1}^{\cat{Ch}} &\cong (\text{Tot}_{\mathcal{A}})_*\circ \text{Fun}^{\cat{Ch}}\circ \text{Fun}^{\cat{Ch}} \circ \cat{Ch}(C_{n+1}^{\cat{Ch}})  \circ (\text{Fun}^{\cat{Ch}})^{-1} \\
        &\cong \text{Fun}^{\cat{Ch}}\circ \text{Tot}_{\text{Fun}(\mathcal{B},\mathcal{A})}\circ \cat{Ch}(C_{n+1})^{\cat{Ch}}\circ (\text{Fun}^{\cat{Ch}})^{-1}
    \end{align*}
    %%
    Then since $\cat{Ch}(C_{n+1})^{\cat{Ch}}=\cat{Ch}(C_{n+1}^{\cat{Ch}})$ preserves chain homotopies and by~\cite[12.18]{StacksProject} $\text{Tot}_{\text{Fun}(\mathcal{B},\mathcal{A})}$ also preserves chain homotopies, it follows by Lemma~\ref{lem:natHomotopIsChainFunctHomotop} that $P_n$ preserves \rd{natural} chain homotopies. 
\end{proof}

For each $F:\mathcal{B}\rightarrow \cat{Ch}(\mathcal{A})$, the functor $P_n(F)$ comes equipped with a natural transformation $p_n:F\rightarrow P_n(F)$ defined by inclusion into the degree zero part of the chain complex $P_n(F)$. Explicitly, can define $p_n:\mathbb{1}\Rightarrow P_n$ as done by Jason Parker:

\begin{rmk}[label=defn:littlepN]
    First we define a natural transformation $i:(\deg^{\cat{Ch}(\mathcal{A})})_*\Rightarrow \text{Fun}^{\cat{Ch}}\circ C_{n+1}^\cat{Ch}:\text{Fun}(\mathcal{B},\cat{Ch}(\mathcal{A}))\rightarrow \text{Fun}(\mathcal{B},\cat{Ch}^2(\mathcal{A}))$, and then we define $p_n := (\text{Tot}_\mathcal{A})_*\circ i$ along with Lemma \ref{lem:compIdTot}. 


    For $F:\mathcal{B}\rightarrow \cat{Ch}(\mathcal{A})$ we define $i_F:\deg^{\cat{Ch}(\mathcal{A})}\circ F\Rightarrow \text{Fun}^{\cat{Ch}}\circ C_{n+1}^\cat{Ch}(F):\mathcal{B}\Rightarrow \cat{Ch}^2(\mathcal{A})$ where for each $B \in \mathcal{B}_0$, we define $i_{F,B}:\deg^{\cat{Ch}(\mathcal{A})}(FB)\rightarrow (\text{Fun}^{\cat{Ch}}(C_{n+1}^\cat{Ch}(F)))B$ in $\cat{Ch}^2(\mathcal{A})$ by saying for all $m \geq 0$,
    %%
    \begin{equation*}
        (i_{F,B})_m = \left\{\begin{array}{cc} 0 & m > 0 \\
        1_{FB}:FB\rightarrow FB & m = 0 \end{array}\right.
    \end{equation*}
    %%
    since $\text{Fun}^{\cat{Ch}}(C_{n+1}^\cat{Ch}(F)))(B)_0 = C_{n+1}^0(F)(B) = F(B)$. These form appropriate chain maps which are natural in $B$ and $F$ as all components are either zero or identities. Then explicitly, 
    %%
    \begin{equation*}
        (p_n)_{F,B,m} = F(B)_m\to \bigoplus_{p+q=m}C_{n+1}^{p}(F)(B)_q
    \end{equation*}
    %%
    is the inclusion into the direct summand $F(B)_m$.
\end{rmk}


\begin{lem}[label=lem:compIdTot]
    We have a natural isomorphism
    \begin{equation*}
        (\text{Tot}_\mathcal{A})_*\circ \deg^{\cat{Ch}(\mathcal{A})} \cong \mathbb{1}_{\cat{Ch}(\mathcal{A})}
    \end{equation*}
\end{lem}
\begin{proof}
    Let $A \in \cat{Ch}(\mathcal{A})$. Then for $n \geq 0$
    %%
    \begin{equation*}
        (\text{Tot}_\mathcal{A})_*\circ \deg^{\cat{Ch}(\mathcal{A})}(A)_n = \bigoplus_{i+j=n}\deg^{\cat{Ch}((\mathcal{A})}(A)_i)_j \cong A_n
    \end{equation*}
    %%
    since all other terms are zero. This isomorphism is given uniquely by the universal property of the biproduct and zero map, so induces the desired isomorphism in the statement of the Lemma.
\end{proof}

In order to show some basic properties of the approximation functor we first prove the following Lemma related to the behaviour of the cross effect functor.

\begin{lem}[label=lem:crossNatTrans]
    Let $\mathcal{B}$ be a pointed category and let $\mathcal{A}$, $\mathcal{C}$ be abelian categories. Then if $F:\mathcal{A}\rightarrow \mathcal{C}$ is an exact functor we have a natural isomorphism
    %%
    \begin{equation*}
        \text{cr}_n^{\mathcal{B},\mathcal{C}}\circ F_*\cong F_*\circ \text{cr}_n^{\mathcal{B},\mathcal{A}}
    \end{equation*}
    where $\text{cr}_n^{\mathcal{B},-}:\text{Fun}(\mathcal{B},-)\rightarrow \text{Fun}_*(\mathcal{B}^n,-)$ specifies the codomain category.
\end{lem}
Actually, this statement is true whenever $F$ preserves direct sums. However we will only use it for exact functors.
\begin{proof}
    We will prove this by induction. For the base case on objects let $G: \mathcal B \to \mathcal A$ be a functor. Then (by the definition of the cross-effect)  $G(X)\cong G(*) \oplus\text{cr}_1^{\mathcal B, \mathcal A}G(X)$ for every object $X$ of $\mathcal B$. Applying $F$ to this equality and using that $F$ preserves direct sums, we obtain that 
    $$
    F\circ G (X) \cong F(G(X)\cong G(*)\oplus  \text{cr}_1^{\mathcal B, \mathcal A}G(X)) \cong F( G(*)) \oplus F(\text{cr}_1^{\mathcal B, \mathcal A}G(X))
    $$
    Applying the definition of the cross-effect to this we obtain that
    $$\text{cr}_n(F \circ G)(X) \cong F(\text{cr}_1^{\mathcal B, \mathcal A}G(X)).$$
    As $\text{cr}_n(F \circ G)$ send a morphisms $f$ of $\mathcal B$ to the unique induced map into the limit that is the cross-effect and $F(\text{cr}_1^{\mathcal B, \mathcal A}G(f)$ gives one such map, $\text{cr}_n^{\mathcal B, \mathcal C} \circ F_*(G) \cong F_* \circ \text{cr}_n^{\mathcal B, \mathcal A}(G)$ as functors $\mathcal B \to \mathcal C$.

    For the base case on morphisms, let $\varphi: G \Rightarrow G'$ be a natural transformation
    between functors $G,G': \mathcal B \to \mathcal A$. Then the component $\varphi_X$ corresponds to $\varphi_* \oplus \text{cr}_1^{\mathcal B, \mathcal A}\varphi_X$ in the sense that 
% https://q.uiver.app/#q=WzAsNCxbMCwwLCJHKFgpIl0sWzAsMSwiRygqKSBcXG9wbHVzIFxcdGV4dHtjcn1fMV57XFxtYXRoY2FsIEIsIFxcbWF0aGNhbCBBfUcoWCkiXSxbMiwxLCJHJygqKSBcXG9wbHVzIFxcdGV4dHtjcn1fMV57XFxtYXRoY2FsIEIsIFxcbWF0aGNhbCBBfUcnKFgpIl0sWzIsMCwiRycoWCkiXSxbMCwzLCJcXHZhcnBoaV9YIl0sWzEsMiwiXFx2YXJwaGlfKiBcXG9wbHVzIChcXHRleHR7Y3J9XzFee1xcbWF0aGNhbCBCLCBcXG1hdGhjYWwgQX1cXHZhcnBoaSlfWCJdLFswLDEsIlxcY29uZyIsMl0sWzMsMiwiXFxjb25nIl1d
\[\begin{tikzcd}
	{G(X)} && {G'(X)} \\
	{G(*) \oplus \text{cr}_1^{\mathcal B, \mathcal A}G(X)} && {G'(*) \oplus \text{cr}_1^{\mathcal B, \mathcal A}G'(X)}
	\arrow["{\varphi_X}", from=1-1, to=1-3]
	\arrow["{\varphi_* \oplus (\text{cr}_1^{\mathcal B, \mathcal A}\varphi)_X}", from=2-1, to=2-3]
	\arrow["\cong"', from=1-1, to=2-1]
	\arrow["\cong", from=1-3, to=2-3]
\end{tikzcd}\]
    commutes (this is how the cross-effect is defined on morphisms). Applying $F$ to this we can read off that $F(\varphi)_X$ corresponds to $F(\varphi_*) \oplus F(\text{cr}_1^{\mathcal B, \mathcal A}(\varphi)_X)$, so by the definition of the cross-effect $\text{cr}_1^{\mathcal B, \mathcal C}(F \circ \varphi) \cong F(\text{cr}_1^{\mathcal B, \mathcal A}(\varphi))$.
    
    For the inductive step, let the statement be true for $\text{cr}_n-1$ (as it will be analogous to the base case we will only sketch this part). Then the definition of the cross-effect tells us 
    $$
    \text{cr}^{\mathcal B, \mathcal A}_{n-1}G(X_1 \vee X_2 , X_3 ,...) = 
    \text{cr}^{\mathcal B, \mathcal A}_{n-1}G(X_1 , X_3 ,...) \oplus \text{cr}^{\mathcal B, \mathcal A}_{n-1}G(X_1 , X_3 ,...) \oplus 
    \text{cr}^{\mathcal B, \mathcal A}_{n}G(X_1, X_2 , X_3 ,...).
    $$
    Applying $F$ and the inductive hypothesis, we obtain
    $$
    \text{cr}^{\mathcal B, \mathcal C}_{n-1}(F \circ G)(X_1 \vee X_2 , X_3 ,...) = 
    \text{cr}^{\mathcal B, \mathcal C}_{n-1}(F \circ G)(X_1 , X_3 ,...) \oplus \text{cr}^{\mathcal B, \mathcal C}_{n-1}(F \circ G)(X_1 , X_3 ,...) \oplus 
    F(\text{cr}^{\mathcal B, \mathcal A}_{n}G(X_1, X_2 , X_3 ,...)).
    $$    
    from which we can see (by the definition of the cross-effect) that
    $$
    \text{cr}_n^{\mathcal B, \mathcal C}(F \circ G)(X_1, X_2 , X_3 ,...) =
    F(\text{cr}^{\mathcal B, \mathcal A}_{n}G(X_1, X_2 , X_3 ,...))$$
    Again the uniqueness of the induced map into a limit gives us that 
    $\text{cr}_n^{\mathcal B, \mathcal C}(F \circ G)=
    F \circ \text{cr}^{\mathcal B, \mathcal A}_{n}G$. 
    In order to do the induction step on morphisms, let $\varphi: G \to G'$ be a natural transformation. Then, by the definition of the cross-effect on morphisms, $    (\text{cr}^{\mathcal B, \mathcal A}_{n-1} \varphi)_{X_1 \vee X_2 , ...} $ corresponds to $(\text{cr}^{\mathcal B, \mathcal A}_{n-1} \varphi)_{X_1 , X_3 ...} \oplus (\text{cr}^{\mathcal B, \mathcal A}_{n-1} \varphi)_{X_2 , X_3 ...} \oplus (\text{cr}^{\mathcal B, \mathcal A}_{n} \varphi)_{X_1, X_2 , ...}$. Applying $F$ and using the inductive hypothesis, we can read off that 
    $$
  (\text{cr}^{\mathcal B, \mathcal A}_{n} F(\varphi))_{X_1, X_2 , ...} =   F (\text{cr}^{\mathcal B, \mathcal A}_{n} \varphi)_{X_1, X_2 , ...}
    $$
    which proves the inductive step for morphisms.
\end{proof}

The basic properties of this approximation are given in the following proposition.

\begin{prop}[label=prop:4.5]
    For $F:\mathcal{B}\rightarrow \cat{Ch}(\mathcal{A})$,
    %%
    \begin{itemize}
        \item[(i)] The functor $P_n(F)$ is degree $n$
        \item[(ii)] If $F$ is degree $n$, then the map $p_n:F\rightarrow P_n(F)$ is a chain homotopy equivalence (\rd{natural})
        \item[(iii)] The pair $(P_n(F),p_n:F\rightarrow P_n(F))$ is universal up to chain homotopy equivalence with respect to degree $n$ functors receiving natural transformations from $F$.
    \end{itemize}
\end{prop}


Finally, one last result we will require that is used in the proof of (iii) is the following computation for a functor $F:\mathcal{B}\rightarrow \cat{Ch}(\mathcal{A})$.

\begin{rmk}
    By construction $p_{n,P_n(F)}$ is the inclusion of $P_n(F)$ into $P_n(P_n(F))$ via the totalization after inclusion into the degree zero part of the bicomplex defining $P_n(P_n(F))$. On the other hand, applying $P_n$ to $p_{n,F}$ \textbf{TBC}
\end{rmk}

\begin{proof}[Proof of Proposition \ref{prop:4.5}]
    Let $F_k:\mathcal{B}\rightarrow \mathcal{A}$ be the $k$th degree component of $F:\mathcal{B}\rightarrow \cat{Ch}(\mathcal{A})$ (under the isomorphism $\text{Fun}^\cat{Ch}$). To prove (i) we show $\text{cr}_{n+1}(P_n(F))$ is contractible (i.e. \rd{naturally} contractible). By Lemma \ref{lem:crossNatTrans} 
    %%
    \begin{equation*}
        \text{cr}_{n+1}^{\mathcal{B},\cat{Ch}(\mathcal{A})}\circ (\text{Tot}_\mathcal{A})_*\circ \text{Fun}^\cat{Ch}\circ C_{n+1}^\cat{Ch} \cong (\text{Tot}_\mathcal{A})_*\circ \text{cr}_{n+1}^{\mathcal{B},\cat{Ch}^2(\mathcal{A})}\circ \text{Fun}^\cat{Ch}\circ C_{n+1}^\cat{Ch}
    \end{equation*}
    %%
    Since $\text{Tot}$ preserves natural homotopies it is sufficient to show that the cross effect for the bicomplex $\text{cr}_{n+1}^{\mathcal{B},\cat{Ch}^2(\mathcal{A})}\circ \text{Fun}^\cat{Ch}\circ C_{n+1}^\cat{Ch}(F)$ defining $P_n(F)$ is contractible. By Lemma \ref{lem:ChFuncCommute} this is equivalent to showing $\text{Fun}^\cat{Ch}\circ\cat{Ch}(\text{cr}_{n+1}^{\mathcal{B},\cat{Ch}(\mathcal{A})})\circ C_{n+1}^\cat{Ch}(F)$ is contractible, or since the naturally homotopies on either side of the isomorphism agree, it is sufficient to show $\cat{Ch}(\text{cr}_{n+1}^{\mathcal{B},\cat{Ch}(\mathcal{A})})\circ C^{\cat{Ch}}_{n+1}(F) \in \cat{Ch}(\text{Fun}(\mathcal{B},\cat{Ch}(\mathcal{A})))$ is contractible. In particular, we can consider the corresponding double complex $\cat{Ch}^2(\text{Fun}(\mathcal{B},\mathcal{A}))$, so we can apply the results in Appendix~\ref{sec:bicomplexes}. Then the $k$th row of this bicomplex is given by
    %%
    \begin{equation*}
        \cdots \rightarrow\text{cr}_{n+1}^{\mathcal{B},\mathcal{A}}C_{n+1}^{\times2}(F_k)\xrightarrow{\text{cr}_{n+1}^{\mathcal{B},\mathcal{A}}(\epsilon_{C_{n+1}}-C_{n+1}\epsilon)}\text{cr}_{n+1}^{\mathcal{B},\mathcal{A}}C_{n+1}(F_k)\xrightarrow{\text{cr}_{n+1}^{\mathcal{B},\mathcal{A}}\epsilon}\text{cr}_{n+1}^{\mathcal{B},\mathcal{A}}(F_k)
    \end{equation*}
    %%
    By Lemma~\ref{lem:contractHomotop} we have a family of horizontal contractions for each row after applying $\text{cr}_{n+1}$, denoted $s^{k,h}$. Setting the vertical contractions, $s^v$, to be zero, we obtain a natural contraction for the bicomplex, so under the totalization we obtain a natural contraction for the chain complex $\text{cr}_{n+1}(P_n(F))$, as desired. 

    \vspace{10pt}

    For (ii) let $F$ be of degree $n$, so $\text{cr}_{n+1}(F)$ is naturally contractible. Recall the $k$th column of the bicomplex defining $P_n(F)$ is $C_{n+1}^{\times k}(F) = (\Delta^*\text{cr}_{n+1})^k(F)$. The map $p_n:F\rightarrow P_n(F)$ is the natural inclusion of the $0$th column into the totalization. Note that $C_{n+1}^{\times k}(F)$ is contractible for each $k\geq 1$ since $C_{n+1}\cong \text{Fun}^{\cat{Ch}}\circ \cat{Ch}(C_n)\circ (\text{Fun}^\cat{Ch})^{-1}$ preserves chain homotopies and $F$ is of degree $n$. Under the isomorphisms in Section~\ref{subsec:chainHomotop}, $p_n$ becomes the inclusion of $F_\bullet$ into $\text{Tot}_{\cat{Fun}(\mathcal{B},\mathcal{A})}(C_{n+1}^{\cat{Ch}}(F))$, so by Corollary~\ref{cor:A7} we obtain a chain homotopy equivalence, which becomes a natural chain homotopy equivalence under the desired isomorphisms.

    \vspace{10pt}

    To show (iii) let $\tau:F\rightarrow G$ be a natural transformation transformation where $G$ is a functor of degree $n$. By naturality of $p_n$ in the functor $F$ we have a commutative diagram
    %%
    \[\begin{tikzcd}
    	F & G \\
    	{P_n(F)} & {P_n(G)}
    	\arrow["\tau", from=1-1, to=1-2]
    	\arrow["{P_n(\tau)}"', from=2-1, to=2-2]
    	\arrow["{p_{n,F}}"', from=1-1, to=2-1]
    	\arrow["{p_{n,G}}", from=1-2, to=2-2]
    \end{tikzcd}\]
    %%
    where the right hand $p_{n,G}$ is a natural chain homotopy equivalence by (ii). Let $s_{n,G}$ denote a natural chain homotopy inverse of $p_{n,G}$. Setting $\tau^\# = s_{n,G}\circ P_n(\tau)$ we have that
    %%
    \begin{equation*}
        \tau^\#\circ p_{n,F} = s_{n,G}\circ P_n(\tau)\circ p_{n,F} = s_{n,G}\circ p_{n,G}\circ \tau \simeq_{\cat{Ch}} \tau
    \end{equation*}
    %%
    This shows $\tau$ factors through $p_n:F\rightarrow P_n(F)$ up to \rd{natural} chain homotopy equivalence. 
    
    
    To show uniqueness suppose $\sigma:P_n(F)\rightarrow G$ is another map such that $\tau$ is naturally chain homotopy equivalent to $\sigma\circ p_{n,F}$. Then by naturality of the $p_n$, we have a commuting diagram
    %%
    \[\begin{tikzcd}
    	F & {P_n(F)} & G \\
    	{P_n(F)} & {P_n(P_n(F))} & {P_n(G)}
    	\arrow["{p_{n,F}}"', from=1-1, to=2-1]
    	\arrow["{P_np_{n,F}}"', from=2-1, to=2-2]
    	\arrow["{p_{n,F}}", from=1-1, to=1-2]
    	\arrow["{p_{n,P_n(F)}}", from=1-2, to=2-2]
    	\arrow["\sigma", from=1-2, to=1-3]
    	\arrow["{p_{n,G}}", from=1-3, to=2-3]
    	\arrow["{P_n(\sigma)}"', from=2-2, to=2-3]
    \end{tikzcd}\]
    %%
    where $p_{n,P_n(F)}$ and $p_{n,G}$ are natural chain homotopy equivalences by (ii). 
    % Further, observe that $P_n(p_{n,F})$ is the totalization of the map of bicomplexes
    % %%
    % \[\begin{tikzcd}
    %     \cdots & {C_{n+1}(F)} & F \\
    %     \cdots & {C_{n+1}(P_n(F))} & {P_n(F)}
    %     \arrow[from=1-1, to=1-2]
    %     \arrow["{\epsilon_F}", from=1-2, to=1-3]
    %     \arrow["{p_{n,F}}", from=1-3, to=2-3]
    %     \arrow["{C_{n+1}(p_{n,F})}"', from=1-2, to=2-2]
    %     \arrow["{\epsilon_{P_n(F)}}"', from=2-2, to=2-3]
    %     \arrow[from=2-1, to=2-2]
    % \end{tikzcd}\]
    % %%
    % Further, from Lemma~\ref{lem:crossNatTrans} we have that $C_{n+1}^k(P_n(F))\cong P_n(C_{n+1}^k(F))$ for all $k \geq 1$, and under this isomorphism $C_{n+1}^k(p_{n,F})$ corresponds to $p_{n,C_{n+1}^k(F)}$. Thus, $P_n(p_{n,F})$ is homotopy equivalent to the totalization of 
    % \[\begin{tikzcd}
    %     \cdots & {C_{n+1}(F)} & F \\
    %     \cdots & {P_n(C_{n+1}(F))} & {P_n(F)}
    %     \arrow[from=1-1, to=1-2]
    %     \arrow["{\epsilon_F}", from=1-2, to=1-3]
    %     \arrow["{p_{n,F}}", from=1-3, to=2-3]
    %     \arrow["{p_{n,C_{n+1}(F)}}"', from=1-2, to=2-2]
    %     \arrow["{P_n\epsilon_F}"', from=2-2, to=2-3]
    %     \arrow[from=2-1, to=2-2]
    % \end{tikzcd}\]
    % where by part (ii) each $p_{n,C_{n+1}^k(F)}$ is a chain homotopy equivalence for $k \geq 1$. 
    
    By Lemma \textbf{REF} we have that $P_n(p_{n,F})$ is a natural chain homotopy equivalence with inverse $S_{n,F}$. Observe that
    %%
    \begin{align*}
        \sigma &\simeq_{\cat{Ch}} s_{n,G}\circ p_{n,G} \circ \sigma  \\
        &= s_{n,G}\circ P_n(\sigma)\circ p_{n,P_n(F)}
    \end{align*}
    %%
    and 
    %%
    \begin{equation*}
        P_n(\sigma) \simeq_{\cat{Ch}} P_n(\sigma)\circ P_n(p_{n,F})\circ S_{n,F} \simeq_{\cat{Ch}} P_n(\tau)\circ S_{n,F}
    \end{equation*}
    %%
    It follows that
    %%
    \begin{equation*}
        \sigma \simeq_{\cat{Ch}} s_{n,G}\circ P_n(\tau)\circ S_{n,F}\circ p_{n,P_n(F)}
    \end{equation*}
    %%
    Since this is independent of $\sigma$, in particular we have that $\sigma \simeq_{\cat{Ch}} \tau^\#$.
\end{proof}

Note that the universal property in bullet (iii) of Proposition \ref{prop:4.5}, along with the fact that degrees of a functor are characterized by their minimal value, we obtain a factorization (up to homotopy in general)
%%
\[\begin{tikzcd}
	&& F \\
	\cdots & {P_{n+1}(F)} & {P_n(F)} & {P_{n-1}(F)} & \cdots & {P_0(F)}
	\arrow["{p_0}", from=1-3, to=2-6]
	\arrow["{q_1}"', from=2-5, to=2-6]
	\arrow["{p_n}"', from=1-3, to=2-3]
	\arrow["{p_{n-1}}"{description}, from=1-3, to=2-4]
	\arrow[from=2-4, to=2-5]
	\arrow["{q_n}"', from=2-3, to=2-4]
	\arrow["{p_{n+1}}"{description}, from=1-3, to=2-2]
	\arrow["{q_{n+1}}"', from=2-2, to=2-3]
	\arrow[from=1-3, to=2-1]
	\arrow[from=2-1, to=2-2]
\end{tikzcd}\]
%%
which is known as the \textbf{algebraic Taylor tower} of the functor $F$, where the $q_n$ are determined uniquely by the universal property of the $P_n(F)$. We can also realize the $q_n:P_n(F)\rightarrow P_{n-1}(F)$ as being induced by a natural transformation $\rho_n:C_{n+1}\Rightarrow C_n:\text{Fun}(\mathcal{B},\mathcal{A})\rightarrow \text{Fun}(\mathcal{B},\mathcal{A})$, with components given by
%%
\[\begin{tikzcd}
	{C_{n+1}(F)(X)=\text{cr}_{n+1}(\Delta^*F)(X)} & {\text{cr}_n(F)(X\oplus X,X,...,X)} & {\text{cr}_n(\Delta^*F)(X)=C_n(F)(X)}
	\arrow[tail, from=1-1, to=1-2]
	\arrow["{\text{cr}_n(F)(+,1_X,...,1_X)}", outer sep = 5pt, from=1-2, to=1-3]
\end{tikzcd}\]
%%
\begin{proof}[Proof of construction of $q_n$]
    Let $\rho_n:C_{n+1}\Rightarrow C_n$ be the natural transformation described above. Then we have natural transformations $\rho_n([k]):C_{n+1}^{\times (k+1)}\Rightarrow C_n^{\times (k+1)}$ given by
    %%
    \begin{equation*}
        \rho_n([k]) = C_n^{\times k}\rho_n \circ C_n^{\times (k-1)}(\rho_n)_{C_{n+1}}\circ \cdots \circ C_n(\rho_n)_{C_{n+1}^{\times (k-1)}}\circ(\rho_n)_{C_{n+1}^{\times k}}
    \end{equation*}
    %%
    These natural transformations define a natural map of the bicomplexes defining $P_n$ and $P_{n-1}$. Applying totalization we obtain the natural transformation $q_n:P_n\Rightarrow P_{n-1}$ described. Since the map of bicomplexes defining $q_n$ is the identity on the zeroth column given by the identity, it follows that the desired triangle commutes.
\end{proof}
%%


\begin{eg}{}
    Note that for $\deg_0^\mathcal{A}$, since $\text{cr}_1(\deg_0^\mathcal{A}) = \deg_0^\mathcal{A}$ is the identity concentrated in degree zero, by the inductive definition $\text{cr}_n(\deg_0^\mathcal{A})\cong 0$ for $n \geq 2$. Then the bicomplex defining $P_n(\deg_0^\mathcal{A})$ has $\deg_0^\mathcal{A}$ as the $0$th column, with all other columns zero. This implies that $P_n(\deg_0^\mathcal{A}) = \deg_0^\mathcal{A}$ for $n \geq 1$.

    The map $q_n(\deg_0^\mathcal{A}):P_n(\deg_0^\mathcal{A})\rightarrow P_{n-1}(\deg_0^\mathcal{A})$ for $n \geq 2$ is the identity. For $n = 1$ the $\rho_1$ defining $q_1$ has $0$th component the identity and $k$th component the zero map for $k \geq 1$. It follows that $q_1(\deg_0^\mathcal{A}):P_1(\deg_0^\mathcal{A})\rightarrow P_0(\deg_0^\mathcal{A})$ is the natural inclusion.
\end{eg}
