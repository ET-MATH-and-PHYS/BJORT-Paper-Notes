\section{The Taylor tower in abelian functor calculus}

With the appropriate categorical language built up, we can begin constructing Taylor towers, as in \cite{JohnsonB.2004Dcwc}. This process involves defining certain polynomial functors of \textbf{degree $n$} with natural transformations which go down by one in degree. In \cite{BJORT} many definitions are given in terms of pointwise chain homotopies. In order to introduce two-dimensional structures on the cat $\cat{HoAbCat}_{\cat{Ch}}$ we aim to upgrade these to natural homotopies. We distinguish where these changes are made by changing the text color to red for occurrences of natural which aren't originally in the text.

\begin{defn}[label=defn:4.1]{}
    A functor $F:\mathcal{B}\rightarrow \cat{Ch}(\mathcal{A})$ is \textbf{degree $n$} if $\text{cr}_{n+1}(F):\mathcal{B}^{n+1}\rightarrow \cat{Ch}(\mathcal{A})$ is \textbf{contractible}, i.e., \rd{natural} chain homotopy equivalent to zero. 
\end{defn}

Note that $\cat{cr}_k(F) \simeq_{\cat{Ch}} 0$ implies $\cat{cr}_\ell(F) \simeq_\cat{Ch} 0$ for any $\ell > k$. Indeed, since $G \oplus F \simeq_{\cat{Ch}} 0$ implies $F \simeq_{\cat{Ch}} 0$, this follows from the inductive definition. Consequently, functors of degree $k$ are also of degree $\ell$ for $\ell > k$.

From \cite[Defn 2.4]{JohnsonB.2004Dcwc}, chain complexes can be constructed from a pair of a comonad and an object in the category on which the comonad acts.
\begin{lem}[label=lem:comonadChain]
    Let $C:\text{Fun}(\mathcal{B},\mathcal{A})\rightarrow \text{Fun}(\mathcal{B},\mathcal{A})$ be a comonad on a functor category where $\mathcal{A}$ is an abelian category (so in particular $\text{Fun}(\mathcal{B},\mathcal{A})$ is abelian). Then $C$ induces a functor $$C^\cat{Ch}:\text{Fun}(\mathcal{B},\mathcal{A})\rightarrow \cat{Ch}(\text{Fun}(\mathcal{B},\mathcal{A}))$$
\end{lem}
\begin{proof}
    Let $\epsilon:C\rightarrow 1$ be the counit of the comonad. Then for $F :\mathcal{B}\rightarrow \mathcal{A}$, define $C^\cat{Ch}(F)$ to be the chain
    %%
    \begin{equation*}
        \cdots \rightarrow C^3(F)\xrightarrow{\epsilon_{C^2(F)}-C\epsilon_{C(F)}+C^2\epsilon_F}C^2(F)\xrightarrow{\epsilon_{C(F)}-C\epsilon_F}C(F)\xrightarrow{\epsilon_F} F
    \end{equation*}
    %%
    where the $k$th differential is defined by the alternating sum $\sum_{i=0}^{k-1}(-1)^iC^i\epsilon_{C^{(k-i)}}$. Note these differentials are indeed natural, and hence this defines a sequence of functors and natural transformations. To see that this sequence forms a chain complex observe that
    %%
    \begin{align*}
        \left(\sum_{i=0}^{n}(-1)^iC^i\epsilon_{C^{n-i}(F)}\right)&\circ \left(\sum_{i=0}^{n+1}(-1)^iC^i\epsilon_{C^{n+1-i}(F)}\right) \\
        &= \sum_{i=0}^n\sum_{j=0}^{n+1}(-1)^{i+j}C^i\epsilon_{C^{n-i}(F)}\circ C^j\epsilon_{C^{n+1-j}(F)} \\
        &= \sum_{i=0}^n\sum_{i < j}^{n+1}(-1)^{i+j}C^i\epsilon_{C^{n-i}(F)}\circ C^j\epsilon_{C^{n+1-j}(F)} \\
        &+ \sum_{i=0}^n\sum_{i\geq j}(-1)^{i+j}C^i\epsilon_{C^{n-i}(F)}\circ C^j\epsilon_{C^{n+1-j}(F)} \\
        &= \sum_{i=0}^n\sum_{i < j}^{n+1}(-1)^{i+j}C^i\left(\epsilon_{C^{n-i}(F)}\circ C^{j-i}\epsilon_{C^{n+1-j}(F)} \right)\\
        &+ \sum_{i=0}^n\sum_{i\geq j}(-1)^{i+j}C^j\left(C^{i-j}\epsilon_{C^{n-i}(F)}\circ \epsilon_{C^{n+1-j}(F)} \right)\\ 
        &= \sum_{i=0}^n\sum_{i\leq k}^n(-1)^{i+k+1}C^{i}\left(\epsilon_{C^{n-i}(F)}\circ C^{k+1-i}\epsilon_{C^{n-k}(F)} \right) \tag{Substituting $k = j-1$}\\
        &+ \sum_{i=0}^n\sum_{i\geq j}(-1)^{i+j}C^j\left(C^{i-j}\epsilon_{C^{n-i}(F)}\circ \epsilon_{C^{n+1-j}(F)} \right)\\ 
        &= \sum_{i=0}^n\sum_{i\leq k}^n(-1)^{i+k+1}C^{i}\left(C^{k-i}\epsilon_{C^{n-k}(F)}\circ \epsilon_{C^{n+1-i}(F)}\right) \tag{Naturality of $\epsilon$} \\
        &+ \sum_{i=0}^n\sum_{i\geq j}(-1)^{i+j}C^j\left(C^{i-j}\epsilon_{C^{n-i}(F)}\circ \epsilon_{C^{n+1-j}(F)} \right)\\ 
        &= -\sum_{k=0}^n\sum_{k\geq i}(-1)^{i+k}C^{i}\left(C^{k-i}\epsilon_{C^{n-k}(F)}\circ \epsilon_{C^{n+1-i}(F)}\right) \tag{Re-ordering the sum} \\
        &+ \sum_{i=0}^n\sum_{i\geq j}(-1)^{i+j}C^j\left(C^{i-j}\epsilon_{C^{n-i}(F)}\circ \epsilon_{C^{n+1-j}(F)} \right) \\
        &= 0
    \end{align*}
    Next, let $\alpha:F\Rightarrow G$ be a natural transformation. Then $C^\cat{Ch}(\alpha):C^\cat{Ch}(F)\rightarrow C^\cat{Ch}(G)$ is defined by $C^\cat{Ch}(\alpha)_n:= C^n\alpha:C^nF\rightarrow C^nG$. To see that this is a chain map observe that for $0 \leq i \leq n$ we have the commutative diagram
    \[\begin{tikzcd}
    	{C^{n+1}(F)} & {C^n(F)} \\
    	{C^{n+1}(G)} & {C^n(G)}
    	\arrow["{C^i\epsilon_{C^{n-i}(F)}}", from=1-1, to=1-2]
    	\arrow["{C^n(\alpha)}", from=1-2, to=2-2]
    	\arrow["{C^{n+1}(\alpha)}"', from=1-1, to=2-1]
    	\arrow["{C^i\epsilon_{C^{n-i}(G)}}"', from=2-1, to=2-2]
    \end{tikzcd}\]
    which commutes by naturality of $\epsilon$. Since composition is bilinear with respect to the group operation on hom sets in an abelian category we have that the $C^n(\alpha)$ form a chain map in $\cat{Ch}(\text{Fun}(\mathcal{B},\mathcal{A}))$. Further, since $C$ is a functor so is $C^\cat{Ch}$, completing the proof.
\end{proof}

Note that we can realize this functor as going to $\text{Fun}(\mathcal{B},\cat{Ch}(\mathcal{A}))$. Indeed, $\cat{Ch}(\text{Fun}(\mathcal{B},\mathcal{A}))\cong \text{Fun}(\mathcal{B},\cat{Ch}(\mathcal{A}))$, as we show in Lemma \ref{lem:funcChain}

\begin{lem}[label=lem:funcChain]
    For $\mathcal{A}$ an abelian category, we have an isomorphism of categories
    \begin{equation}\label{eq:ChainFunc}
        \cat{Ch}(\text{Fun}(\mathcal{B},\mathcal{A}))\cong \text{Fun}(\mathcal{B},\cat{Ch}(\mathcal{A}))
    \end{equation}
\end{lem}
\begin{proof}
    Define a functor $\gamma:\cat{Ch}(\text{Fun}(\mathcal{B},\mathcal{A}))\rightarrow \text{Fun}(\mathcal{B},\cat{Ch}(\mathcal{A}))$ given on a chain complex of functors $F_\bullet$ by
    %%
    \begin{equation*}
        \gamma(F_\bullet)(B)_n := F_n(B),\;\forall B \in \mathcal{B}
    \end{equation*}
    %%
    where the differentials are given by the natural transformation differentials in $F_\bullet$ evaluated at $B$. Given a map of chain complexes $\alpha_\bullet:F_\bullet\rightarrow G_\bullet$ we set
    %%
    \begin{equation*}
        (\gamma(\alpha_\bullet)_B)_n := (\alpha_n)_B
    \end{equation*}
    %%
    This defines a chain map $\gamma(F_\bullet)(B)\rightarrow \gamma(G_\bullet)(B)$ since $\alpha_\bullet$ is a chain map of natural transformations, so all squares with differentials commute. Further, $\gamma(\alpha_\bullet)$ is natural in $B$ since if $f:B\rightarrow B'$ is a map in $\mathcal{B}$, then in
    \[\begin{tikzcd}
    	& {F_{n+1}(B')} &&& {F_n(B')} \\
    	{F_{n+1}(B)} && {F_n(B)} \\
    	& {G_{n+1}(B')} &&& {G_n(B')} \\
    	{G_{n+1}(B)} && {G_n(B)}
    	\arrow["{\partial_{n+1}}"{pos=0.7}, from=2-1, to=2-3]
    	\arrow["{(\alpha_{n+1})_B}"', from=2-1, to=4-1]
    	\arrow["{\partial_{n+1}}"', from=4-1, to=4-3]
    	\arrow["{(\alpha_n)_B}"{pos=0.3}, from=2-3, to=4-3]
    	\arrow["{F_{n+1}(f)}", from=2-1, to=1-2]
    	\arrow["{F_n(f)}"', from=2-3, to=1-5]
    	\arrow["{\partial_{n+1}}", from=1-2, to=1-5]
    	\arrow["{(\alpha_n)_{B'}}", from=1-5, to=3-5]
    	\arrow["{G_n(f)}"', from=4-3, to=3-5]
    	\arrow["{G_{n+1}(f)}", from=4-1, to=3-2]
    	\arrow["{(\alpha_{n+1})_{B'}}"'{pos=0.6}, from=1-2, to=3-2]
    	\arrow["{\partial_{n+1}}"', from=3-2, to=3-5]
    \end{tikzcd}\]
    the front and back faces commute since $\alpha_\bullet$ is a chain map, the top and bottom faces commute by naturality of the boundary maps, and the side faces commute by naturality of the $\alpha_n$. Since this definition is in terms of the components of $\alpha_\bullet$ it is inherently functorial. 

    Next we must witness an inverse $\rho:\text{Fun}(\mathcal{B},\cat{Ch}(\mathcal{A}))\rightarrow \cat{Ch}(\text{Fun}(\mathcal{B},\mathcal{A}))$ functor. Given $F:\mathcal{B}\rightarrow \cat{Ch}(\mathcal{A})$ we set $\rho(F)$ to have $n$th component $(-)_n\circ F$ and differential $\partial_n$ given by components the $n$th differential of $F$ evaluated at $B  \in \mathcal{B}$. Naturality of the differential equates to the commutivity of 
    \[\begin{tikzcd}
    	{F(B)_n} & {F(B)_{n-1}} \\
    	{F(B')_n} & {F(B')_{n-1}}
    	\arrow["{F(f)_n}"', from=1-1, to=2-1]
    	\arrow["{\partial_n(B)}", from=1-1, to=1-2]
    	\arrow["{\partial_n(B')}"', from=2-1, to=2-2]
    	\arrow["{F(f)_{n-1}}", from=1-2, to=2-2]
    \end{tikzcd}\]
    for any $f:B\rightarrow B'$, which follows since $F(f)$ is a chain map. Next, if $\alpha:F\rightarrow G$ is a natural transformation between two such functors we set $\rho(\alpha)$ such that $\rho(\alpha)_n$ is the natural transformation defined by $(\rho(\alpha)_n)_B := (\alpha_B)_n$. Naturality and the chain condition follow by the commutivity of 
    \[\begin{tikzcd}
    	& {F(B')_{n+1}} &&& {F(B')_n} \\
    	{F(B)_{n+1}} && {F(B)_n} \\
    	& {G(B')_{n+1}} &&& {G(B')_n} \\
    	{G(B)_{n+1}} && {G(B)_n}
    	\arrow["{\partial_{n+1}(B)}"{pos=0.7}, from=2-1, to=2-3]
    	\arrow["{(\alpha_B)_{n+1}}"', from=2-1, to=4-1]
    	\arrow["{\partial_{n+1}(B)}"', from=4-1, to=4-3]
    	\arrow["{(\alpha_B)_n}"{pos=0.2}, from=2-3, to=4-3]
    	\arrow["{F(f)_{n+1}}", from=2-1, to=1-2]
    	\arrow["{F(f)_n}"', from=2-3, to=1-5]
    	\arrow["{\partial_{n+1}(B')}", from=1-2, to=1-5]
    	\arrow["{(\alpha_{B'})_n}", from=1-5, to=3-5]
    	\arrow["{G(f)_n}"', from=4-3, to=3-5]
    	\arrow["{G(f)_{n+1}}", from=4-1, to=3-2]
    	\arrow["{(\alpha_{B'})_{n+1}}"'{pos=0.6}, from=1-2, to=3-2]
    	\arrow["{\partial_{n+1}(B')}"', from=3-2, to=3-5]
    \end{tikzcd}\]
    where the bottom and top faces are the fact $G(f)$ and $F(f)$ are chain maps, the front and back faces are the fact $\alpha_B$ is a chain map, and finally the side faces are naturality of $\alpha$. Once again, since $\rho(\alpha)$ is defined in terms of the components of $\alpha$ the assignment is inherently functorial. Further, these operations are exactly inverse of each other as they correspond to swapping the element and natural number indices (in particular, on the other side of the Dold-Kan Equivalence this is simply the swap natural isomorphism on functors of two variables).
\end{proof}


Moving forward we write $\text{Fun}^\cat{Ch}$ for the isomorphism $\gamma$ in the proof. We can use this technique to define the polynomial approximations in the Taylor tower for a functor.

\begin{defn}[label=defn:4.2]{}
    The \textbf{$n$th polynomial approximation} is the composite functor $P_n := (\text{Tot}_\mathcal{A})_*\circ \text{Fun}^\cat{ch}\circ C_{n+1}^{\cat{Ch}}:\text{Fun}(\mathcal{B},\cat{Ch}(\mathcal{A}))\rightarrow \text{Fun}(\mathcal{B},\cat{Ch}(\mathcal{A}))$. 
\end{defn}


Recall by Lemma \ref{lem:idempotCr1} if $n = 0$, $C_1(F) = \text{cr}_1(\text{cr}_1(F))=\text{cr}_1(F)$ (by our choice in defining $\text{cr}_1$), and so $C_1^{\times k}(F) = \text{cr}_1(F)$ for each $k \geq 1$. Further, the co-unit $\epsilon:\Delta^*(\text{cr}_1(\text{cr}_1(F)))(X) = \text{cr}_1(F)(X)\rightarrow F(X)$ is the kernel map in $\text{cr}_1(F)\rightarrow F(X)\rightarrow F(\star)$. In the case of $\epsilon_{C_1}$ the kernel map is the identity from our definition of $\text{cr}_1$, while by the definition of $\text{cr}_1$ on maps we have
\[\begin{tikzcd}
	{\text{cr}_1(F)(X)} & {\text{cr}_1(F)(X)} & 0 \\
	{\text{cr}_1(F)(X)} & {F(X)} & {F(\star)}
	\arrow[Rightarrow, no head, from=1-1, to=1-2]
	\arrow["{!}", from=1-2, to=1-3]
	\arrow["{\text{cr}_1(\epsilon)}"', from=1-1, to=2-1]
	\arrow["\epsilon"', from=1-2, to=2-2]
	\arrow[from=2-2, to=2-3]
	\arrow[from=1-3, to=2-3]
	\arrow["\epsilon"', tail, from=2-1, to=2-2]
\end{tikzcd}\]
so $C_1(\epsilon)$ is the identity as well since $\epsilon$ is monic. Hence, we obtain the chain complex of functors
%%
\begin{equation*}
    \cdots \xrightarrow{0}\text{cr}_1(F)\xrightarrow{\id}\text{cr}_1(F)\xrightarrow{0}\text{cr}_1(F)\xrightarrow{\epsilon}F
\end{equation*}
%%
Since $F \cong \text{cr}_1(F)\oplus F(\star)$, the chain complex defining $P_0(F)$ can be written as a direct sum of the two chain complexes
%%
\begin{equation*}
    \cdots \xrightarrow{0}\text{cr}_1(F)\xrightarrow{\id}\text{cr}_1(F)\xrightarrow{0}\text{cr}_1(F)\xrightarrow{\id}\text{cr}_1(F)
\end{equation*}
%%
and
%%
\begin{equation*}
    \cdots\rightarrow 0\rightarrow 0 \rightarrow 0 \rightarrow F(\star)
\end{equation*}
%%
Note that the chain complex in the top line is contractible. Recall the totalization commutes with direct sums. The totalization of the second complex is isomorphic to $F(\star)$ itself. On the other hand, the totalization of the first complex, $C_\bullet$, has $C_0 = \text{cr}_1(F)_0$, $C_1 = \text{cr}_1(F)_1\oplus \text{cr}_1(F)_0$, and in general
%%
\begin{equation*}
    C_n = \bigoplus_{i=0}^n\text{cr}_1(F)_i
\end{equation*}
%%
with differential $\partial_n:C_n\rightarrow C_{n-1}$ given by 
%%
\begin{equation*}
    \partial_n = (\delta_{n-i,even}\pi_{C_{i-1}}-\partial_i^{\text{cr}_1(F)}\circ \pi_{C_i})_{1\leq i \leq n}
\end{equation*}
%%
where $\delta_{n-i,even}$ is $1$ when $2\mid n-i$ and $0$ else. Then $P_0(F)$ is the direct sum of these two sequences. However, since the first complex before totalization is contractible, we can model $P_0(F)(X) \cong F(\star)$.

\begin{rmk}[label=rmk:ExplicitPn]
    
\end{rmk}

We test the above computation, and the results to follow, using the example of $\deg_0^\mathcal{A}:\mathcal{A}\rightarrow \cat{Ch}(\mathcal{A})$

\begin{eg}{}
    First, note that the chain complex $\deg_0^\mathcal{A}(0)$ is the zero complex. Since $\deg_0^\mathcal{A}$ is reduced we also have that $\text{cr}_1(\deg_0^\mathcal{A}) = \deg_0^\mathcal{A}$. It follows that $P_0(\deg_0^\mathcal{A})_n = 1_\mathcal{A}$ for each $n \geq 0$, and $\partial_n = \delta_{n-1,even}1_{\mathcal{A}}$, or in other words
    %%
    \begin{equation*}
        P_0(\deg_0^\mathcal{A}) := \cdots \xrightarrow{1_{1_\mathcal{A}}} 1_\mathcal{A}\xrightarrow{0} 1_\mathcal{A}\xrightarrow{1_{1_\mathcal{A}}} 1_\mathcal{A}
    \end{equation*}
    %%
    Note that this complex is contractible.
\end{eg}

We now prove a preliminary result on exactness of totalization.

\begin{lem}[label=lem:TotExact]
    The totalization functor $\text{Tot}:\cat{Ch}^2(\mathcal{A})\rightarrow \cat{Ch}(\mathcal{A})$ is exact.
\end{lem}
\begin{proof}
    Let 
    %%
    \begin{equation*}
        0\rightarrow A_1\xrightarrow{f_1} A_2\xrightarrow{f_2} A_3\rightarrow 0
    \end{equation*}
    %%
    be a short exact sequence of bicomplexes in $\mathcal{A}$. This becomes a sequence of complexes 
    %%
    \begin{equation*}
        \text{Tot}(A_1)\xrightarrow{\text{Tot}(f_1)}\text{Tot}(A_2)\xrightarrow{\text{Tot}(f_2)}\text{Tot}(A_3)
    \end{equation*}
    %%
    where at a given $n$,
    %%
    \begin{equation*}
        \text{Tot}(A_i)_n = \bigoplus_{j=0}^n(A_i)_{j,n-j}
    \end{equation*}
    %%
    and
    %%
    \begin{equation*}
        \text{Tot}(f_i)_n = \bigoplus_{j=0}^n (f_i)_{j,n-j}
    \end{equation*}
    %%
    Note that the sequence of bicomplexes being exact means that each component sequence in $\mathcal{A}$ is exact. Then the component sequence of the totalization at $n$ is a finite direct sum of exact sequences, and hence exact.
\end{proof}

\begin{lem}[label=pushForward]
    Let $F:\mathcal{B}\rightarrow \mathcal{C}$ be an exact functor between abelian categories. Then for a category $\mathcal{A}$, $F_*:\text{Fun}(\mathcal{A},\mathcal{B})\rightarrow \text{Fun}(\mathcal{A},\mathcal{C})$ is exact.
\end{lem}
\begin{proof}
    Let $0\rightarrow G_1\xrightarrow{\eta_1} G_2\xrightarrow{\eta_2} G_3\rightarrow 0$ be a short exact sequence of functors from $\mathcal{A}$ to $\mathcal{B}$. Since abelian categories are finitely complete and cocomplete, finite limits and colimits in $\text{Fun}(\mathcal{A},\mathcal{B})$ and $\text{Fun}(\mathcal{A},\mathcal{C})$ are computed pointwise, so it is sufficient to prove the lemma at a given $A \in \mathcal{A}$. This follows by exactness of $F$.
\end{proof}

Due to Proposition \ref{prop:exactCross} we obtain nice properties for the functors $P_n:\text{Fun}(\mathcal{B},\mathcal{A})\rightarrow \text{Fun}(\mathcal{B},\mathcal{A})$.

\begin{prop}[label=prop:exactPol]
    For any $n \geq 0$,
    \begin{itemize}
        \item[(i)] $P_n:\text{Fun}(\mathcal{B},\cat{Ch}(\mathcal{A}))\rightarrow \text{Fun}(\mathcal{B},\cat{Ch}(\mathcal{A}))$ is exact
        \item[(ii)] $P_n$ preserves chain homotopies, chain homotopy equivalences, and contractibility. \textbf{(only pointwise or also natural?)}
    \end{itemize}
\end{prop}
\begin{proof}
    It is sufficient to prove (i) for short exact sequences. Let $0 \rightarrow F\rightarrow G\rightarrow H\rightarrow 0$ be a SES of functors in $\cat{AbCat}_{\cat{Ch}}$. By Proposition \ref{prop:exactCross} we obtain a SES of bicomplexes in the definition of the $n$th polynomial approximation. Since totalization is exact we obtain a SES $0 \rightarrow P_n(F)\rightarrow P_n(G)\rightarrow P_n(H)\rightarrow 0$. (ii) follows from (i) since exact functors preserve chain homotopies and zero chain complexes.
\end{proof}

For each $F:\mathcal{B}\rightarrow \cat{Ch}(\mathcal{A})$, the functor $P_n(F)$ comes equipped with a natural transformation $p_n:F\rightarrow P_n(F)$ defined by inclusion into the degree zero part of the chain complex $P_n(F)$. Explicitly, can define $p_n:\mathbb{1}\Rightarrow P_n$ as done by Jason Parker:

\begin{rmk}[label=defn:littlepN]
    First we define a natural transformation $i:(\deg^{\cat{Ch}(\mathcal{A})})_*\Rightarrow \text{Fun}^{\cat{Ch}}\circ C_{n+1}^\cat{Ch}:\text{Fun}(\mathcal{B},\cat{Ch}(\mathcal{A}))\rightarrow \text{Fun}(\mathcal{B},\cat{Ch}^2(\mathcal{A}))$, and then we define $p_n := (\text{Tot}_\mathcal{A})_*\circ i$ along with Lemma \ref{lem:compIdTot}. 


    For $F:\mathcal{B}\rightarrow \cat{Ch}(\mathcal{A})$ we define $i_F:\deg^{\cat{Ch}(\mathcal{A})}\circ F\Rightarrow \text{Fun}^{\cat{Ch}}\circ C_{n+1}^\cat{Ch}(F):\mathcal{B}\Rightarrow \cat{Ch}^2(\mathcal{A})$ where for each $B \in \mathcal{B}_0$, we define $i_{F,B}:\deg^{\cat{Ch}(\mathcal{A})}(FB)\rightarrow (\text{Fun}^{\cat{Ch}}(C_{n+1}^\cat{Ch}(F)))B$ in $\cat{Ch}^2(\mathcal{A})$ by saying for all $m \geq 0$,
    %%
    \begin{equation*}
        (i_{F,B})_m = \left\{\begin{array}{cc} 0 & m > 0 \\
        1_{FB}:FB\rightarrow FB & m = 0 \end{array}\right.
    \end{equation*}
    %%
    since $\text{Fun}^{\cat{Ch}}(C_{n+1}^\cat{Ch}(F)))(B)_0 = C_{n+1}^0(F)(B) = F(B)$. These form appropriate chain maps which are natural in $B$ and $F$ (\textbf{Maybe expand on this when have time?}). Then explicitly
\end{rmk}


\begin{lem}[label=lem:compIdTot]
    We have a natural isomorphism
    \begin{equation*}
        (\text{Tot}_\mathcal{A})_*\circ \deg^{\cat{Ch}(\mathcal{A})} \cong \mathbb{1}_{\cat{Ch}(\mathcal{A})}
    \end{equation*}
\end{lem}
\begin{proof}
    Let $A \in \cat{Ch}(\mathcal{A})$. Then for $n \geq 0$
    %%
    \begin{equation*}
        (\text{Tot}_\mathcal{A})_*\circ \deg^{\cat{Ch}(\mathcal{A})}(A)_n = \bigoplus_{i+j=n}\deg^{\cat{Ch}((\mathcal{A})}(A)_i)_j \cong A_n
    \end{equation*}
    %%
    since all other terms are zero. This isomorphism is given uniquely by the universal property of the biproduct and zero map, so induces the desired isomorphism in the statement of the Lemma.
\end{proof}

In order to show some basic properties of the approximation functor we first prove the following Lemma related to the behaviour of the cross effect functor.

\begin{lem}[label=lem:crossNatTrans]
    Let $\mathcal{B}$ be a pointed category and let $\mathcal{A}$, $\mathcal{C}$ be abelian categories. Then if $F:\mathcal{A}\rightarrow \mathcal{C}$ is an exact functor we have a natural isomorphism
    %%
    \begin{equation*}
        \text{cr}_n^{\mathcal{B},\mathcal{C}}\circ F_*\cong F_*\circ \text{cr}_n^{\mathcal{B},\mathcal{A}}
    \end{equation*}
    where $\text{cr}_n^{\mathcal{B},-}:\text{Fun}(\mathcal{B},-)\rightarrow \text{Fun}_*(\mathcal{B}^n,-)$ specifies the codomain category.
\end{lem}
Actually, this statement is true whenever $F$ preserves direct sums. However we will only use it for exact functors.
\begin{proof}
    We will prove this by induction. For the base case on objects let $G: \mathcal B \to \mathcal A$ be a functor. Then (by the definition of the cross-effect)  $G(X)\cong G(*) \oplus\text{cr}_1^{\mathcal B, \mathcal A}G(X)$ for every object $X$ of $\mathcal B$. Applying $F$ to this equality and using that $F$ preserves direct sums, we obtain that 
    $$
    F\circ G (X) \cong F(G(X)\cong G(*)\oplus  \text{cr}_1^{\mathcal B, \mathcal A}G(X)) \cong F( G(*)) \oplus F(\text{cr}_1^{\mathcal B, \mathcal A}G(X))
    $$
    Applying the definition of the cross-effect to this we obtain that
    $$\text{cr}_n(F \circ G)(X) \cong F(\text{cr}_1^{\mathcal B, \mathcal A}G(X)).$$
    As $\text{cr}_n(F \circ G)$ send a morphisms $f$ of $\mathcal B$ to the unique induced map into the limit that is the cross-effect and $F(\text{cr}_1^{\mathcal B, \mathcal A}G(f)$ gives one such map, $\text{cr}_n^{\mathcal B, \mathcal C} \circ F_*(G) \cong F_* \circ \text{cr}_n^{\mathcal B, \mathcal A}(G)$ as functors $\mathcal B \to \mathcal C$.

    For the base case on morphisms, let $\varphi: G \Rightarrow G'$ be a natural transformation
    between functors $G,G': \mathcal B \to \mathcal A$. Then the component $\varphi_X$ corresponds to $\varphi_* \oplus \text{cr}_1^{\mathcal B, \mathcal A}\varphi_X$ in the sense that 
% https://q.uiver.app/#q=WzAsNCxbMCwwLCJHKFgpIl0sWzAsMSwiRygqKSBcXG9wbHVzIFxcdGV4dHtjcn1fMV57XFxtYXRoY2FsIEIsIFxcbWF0aGNhbCBBfUcoWCkiXSxbMiwxLCJHJygqKSBcXG9wbHVzIFxcdGV4dHtjcn1fMV57XFxtYXRoY2FsIEIsIFxcbWF0aGNhbCBBfUcnKFgpIl0sWzIsMCwiRycoWCkiXSxbMCwzLCJcXHZhcnBoaV9YIl0sWzEsMiwiXFx2YXJwaGlfKiBcXG9wbHVzIChcXHRleHR7Y3J9XzFee1xcbWF0aGNhbCBCLCBcXG1hdGhjYWwgQX1cXHZhcnBoaSlfWCJdLFswLDEsIlxcY29uZyIsMl0sWzMsMiwiXFxjb25nIl1d
\[\begin{tikzcd}
	{G(X)} && {G'(X)} \\
	{G(*) \oplus \text{cr}_1^{\mathcal B, \mathcal A}G(X)} && {G'(*) \oplus \text{cr}_1^{\mathcal B, \mathcal A}G'(X)}
	\arrow["{\varphi_X}", from=1-1, to=1-3]
	\arrow["{\varphi_* \oplus (\text{cr}_1^{\mathcal B, \mathcal A}\varphi)_X}", from=2-1, to=2-3]
	\arrow["\cong"', from=1-1, to=2-1]
	\arrow["\cong", from=1-3, to=2-3]
\end{tikzcd}\]
    commutes (this is how the cross-effect is defined on morphisms). Applying $F$ to this we can read off that $F(\varphi)_X$ corresponds to $F(\varphi_*) \oplus F(\text{cr}_1^{\mathcal B, \mathcal A}(\varphi)_X)$, so by the definition of the cross-effect $\text{cr}_1^{\mathcal B, \mathcal C}(F \circ \varphi) \cong F(\text{cr}_1^{\mathcal B, \mathcal A}(\varphi))$.
    
    For the inductive step, let the statement be true for $\text{cr}_n-1$ (as it will be analogous to the base case we will only sketch this part). Then the definition of the cross-effect tells us 
    $$
    \text{cr}^{\mathcal B, \mathcal A}_{n-1}G(X_1 \vee X_2 , X_3 ,...) = 
    \text{cr}^{\mathcal B, \mathcal A}_{n-1}G(X_1 , X_3 ,...) \oplus \text{cr}^{\mathcal B, \mathcal A}_{n-1}G(X_1 , X_3 ,...) \oplus 
    \text{cr}^{\mathcal B, \mathcal A}_{n}G(X_1, X_2 , X_3 ,...).
    $$
    Applying $F$ and the inductive hypothesis, we obtain
    $$
    \text{cr}^{\mathcal B, \mathcal C}_{n-1}(F \circ G)(X_1 \vee X_2 , X_3 ,...) = 
    \text{cr}^{\mathcal B, \mathcal C}_{n-1}(F \circ G)(X_1 , X_3 ,...) \oplus \text{cr}^{\mathcal B, \mathcal C}_{n-1}(F \circ G)(X_1 , X_3 ,...) \oplus 
    F(\text{cr}^{\mathcal B, \mathcal A}_{n}G(X_1, X_2 , X_3 ,...)).
    $$    
    from which we can see (by the definition of the cross-effect) that
    $$
    \text{cr}_n^{\mathcal B, \mathcal C}(F \circ G)(X_1, X_2 , X_3 ,...) =
    F(\text{cr}^{\mathcal B, \mathcal A}_{n}G(X_1, X_2 , X_3 ,...))$$
    Again the uniqueness of the induced map into a limit gives us that 
    $\text{cr}_n^{\mathcal B, \mathcal C}(F \circ G)=
    F \circ \text{cr}^{\mathcal B, \mathcal A}_{n}G$. 
    In order to do the induction step on morphisms, let $\varphi: G \to G'$ be a natural transformation. Then, by the definition of the cross-effect on morphisms, $    (\text{cr}^{\mathcal B, \mathcal A}_{n-1} \varphi)_{X_1 \vee X_2 , ...} $ corresponds to $(\text{cr}^{\mathcal B, \mathcal A}_{n-1} \varphi)_{X_1 , X_3 ...} \oplus (\text{cr}^{\mathcal B, \mathcal A}_{n-1} \varphi)_{X_2 , X_3 ...} \oplus (\text{cr}^{\mathcal B, \mathcal A}_{n} \varphi)_{X_1, X_2 , ...}$. Applying $F$ and using the inductive hypothesis, we can read off that 
    $$
  (\text{cr}^{\mathcal B, \mathcal A}_{n} F(\varphi))_{X_1, X_2 , ...} =   F (\text{cr}^{\mathcal B, \mathcal A}_{n} \varphi)_{X_1, X_2 , ...}
    $$
    which proves the inductive step for morphisms.
\end{proof}

The basic properties of this approximation are given in the following proposition.

\begin{prop}[label=prop:4.5]
    For $F:\mathcal{B}\rightarrow \cat{Ch}(\mathcal{A})$,
    %%
    \begin{itemize}
        \item[(i)] The functor $P_n(F)$ is degree $n$
        \item[(ii)] If $F$ is degree $n$, then the map $p_n:F\rightarrow P_n(F)$ is a chain homotopy equivalence (\rd{natural})
        \item[(iii)] The pair $(P_n(F),p_n:F\rightarrow P_n(F))$ is universal up to chain homotopy equivalence with respect to degree $n$ functors receiving natural transformations from $F$.
    \end{itemize}
\end{prop}

In order to prove part (i) of Proposition \ref{prop:4.5} we require a certain compatibility of $\text{cr}_n^{\mathcal{B},\cat{Ch}(\mathcal{A})}$ with $\text{Fun}^\cat{Ch}$. First we need to upgrade functors to functors on chains in a naive manner.

\begin{lem}[label=lem:funcActChain]
    Let $\text{Fun}_{Add}(\mathcal{A},\mathcal{C})$ be the category of additive functors between abelian categories with all natural transformations. Then we have a functor
    %%
    \begin{equation*}
        \cat{Ch}:\text{Fun}_{Add}(\mathcal{A},\mathcal{C})\rightarrow \text{Fun}_{Add}(\cat{Ch}(\mathcal{A}),\cat{Ch}(\mathcal{C}))
    \end{equation*}
    given by sending functors to their action componentwise.
\end{lem}
\begin{proof}
    Let $\mathcal{F} \in \text{Fun}_{Add}(\mathcal{A},\mathcal{C})$. Since $\mathcal{F}$ is additive it preserves $0$'s and hence sends chain complexes to chain complexes. Then let $f_\bullet:A_\bullet\rightarrow A_\bullet'$ be a map of chain complexes. Then $\cat{Ch}(\mathcal{F})(f_\bullet)_n := \mathcal{F}(f_n)$, and since $\mathcal{F}$ is additive
    %%
    \begin{equation*}
        \mathcal{F}(f_n)\mathcal{F}(\partial_{n+1}^A)-\mathcal{F}(\partial_n^{A'})\mathcal{F}(f_{n+1}) = \mathcal{F}(f_n\partial_{n+1}^A-\partial_n^{A'}f_{n+1}) = \mathcal{F}(0) = 0
    \end{equation*}
    %%
    so $\cat{Ch}(\mathcal{F})(f_\bullet)$ is a chain map. Further, since $\cat{Ch}(\mathcal{F})$ is defined componentwise and $\mathcal{F}$ is a functor and additive, $\cat{Ch}(\mathcal{F})$ is a functor and additive. 


    Next, let $\eta:\mathcal{F}\rightarrow \mathcal{G}$ be a natural transformation between additive functors. Then define $\cat{Ch}(\eta)_{A_\bullet}:\cat{Ch}(\mathcal{F})(A_\bullet)\rightarrow \cat{Ch}(\mathcal{G})(A_\bullet)$ by $(\cat{Ch}(\eta)_{A_\bullet})_n := \eta_{A_n}$. Then $\cat{Ch}(\eta)_{A_\bullet}$ is a chain map by naturality of $\eta$. Further, $\cat{Ch}(\eta)$ is natural again by naturality of $\eta$, which makes the following diagram commute for $f_\bullet:A_\bullet\rightarrow A_\bullet$:
    \[\begin{tikzcd}
    	{\mathcal{F}(A_n)} & {\mathcal{F}(B_n)} \\
    	{\mathcal{G}(A_n)} & {\mathcal{G}(B_n)}
    	\arrow["{\eta_{A_n}}"', from=1-1, to=2-1]
    	\arrow["{\mathcal{G}(f_n)}"', from=2-1, to=2-2]
    	\arrow["{\eta_{B_n}}", from=1-2, to=2-2]
    	\arrow["{\mathcal{F}(f_n)}", from=1-1, to=1-2]
    \end{tikzcd}\]
    Since $\cat{Ch}(\eta)$ is defined componentwise it preserves composites and identities.
\end{proof}

Next we show compatibility of this functor with the isomorphism $\text{Fun}^\cat{Ch}$.

\begin{lem}[label=lem:ChFuncCommute]
    For any pointed category $\mathcal{B}$ and abelian category $\mathcal{A}$, we have a natural isomorphism
    %%
    \begin{equation*}
        \text{cr}_n^{\mathcal{B},\cat{Ch}(A)}\circ \text{Fun}^\cat{Ch} \cong \text{Fun}^\cat{Ch}\circ \cat{Ch}(\text{cr}_n^{\mathcal{B},\mathcal{A}})
    \end{equation*}
    %%
\end{lem}
\begin{proof}
    Since the cross effect is additive we can apply $\cat{Ch}$, so the claim is well-posed. Let $F\in\cat{Ch}(\text{Fun}(\mathcal{B},\mathcal{A}))$ be a chain complex of functors. Since finite limits in functor categories between abelian categories are computed pointwise, up to natural isomorphism, we have that 
    %%
    \begin{equation*}
        \text{cr}_n^{\mathcal{B},\cat{Ch}(A)}\circ \text{Fun}^\cat{Ch} \cong \text{Fun}^\cat{Ch}\circ \cat{Ch}(\text{cr}_n^{\mathcal{B},\mathcal{A}})
    \end{equation*}
\end{proof}

Finally, one last result we will require that is used in the proof of (iii) is the following computation for a functor $F:\mathcal{B}\rightarrow \cat{Ch}(\mathcal{A})$.

\begin{rmk}
    By construction $p_{n,P_n(F)}$ is the inclusion of $P_n(F)$ into $P_n(P_n(F))$ via the totalization after inclusion into the degree zero part of the bicomplex defining $P_n(P_n(F))$. On the other hand, applying $P_n$ to $p_{n,F}$ \textbf{TBC}
\end{rmk}

\begin{proof}[Proof of Proposition \ref{prop:4.5}]
    Let $F_k:\mathcal{B}\rightarrow \mathcal{A}$ be the $k$th degree component of $F:\mathcal{B}\rightarrow \cat{Ch}(\mathcal{A})$ (under the isomorphism $\text{Fun}^\cat{Ch}$). To prove (i) we show $\text{cr}_{n+1}(P_n(F))$ is contractible (i.e. \rd{naturally} contractible). By Lemma \ref{lem:crossNatTrans} 
    %%
    \begin{equation*}
        \text{cr}_{n+1}^{\mathcal{B},\cat{Ch}(\mathcal{A})}\circ (\text{Tot}_\mathcal{A})_*\circ \text{Fun}^\cat{Ch}\circ C_{n+1}^\cat{Ch} \cong (\text{Tot}_\mathcal{A})_*\circ \text{cr}_{n+1}^{\mathcal{B},\cat{Ch}^2(\mathcal{A})}\circ \text{Fun}^\cat{Ch}\circ C_{n+1}^\cat{Ch}
    \end{equation*}
    %%
    Since $\text{Tot}$ preserves natural homotopies it is sufficient to show that the cross effect for the bicomplex $\text{cr}_{n+1}^{\mathcal{B},\cat{Ch}^2(\mathcal{A})}\circ \text{Fun}^\cat{Ch}\circ C_{n+1}^\cat{Ch}(F)$ defining $P_n(F)$ is contractible. By Lemma \ref{lem:ChFuncCommute} this is equivalent to showing $\text{Fun}^\cat{Ch}\circ\cat{Ch}(\text{cr}_{n+1}^{\mathcal{B},\cat{Ch}(\mathcal{A})})\circ C_{n+1}^\cat{Ch}(F)$ is contractible. Then the $k$th row of this bicomplex is given by
    %%
    \begin{equation*}
        \cdots \rightarrow\text{cr}_{n+1}^{\mathcal{B},\mathcal{A}}C_{n+1}^{\times2}(F_k)\xrightarrow{\text{cr}_{n+1}^{\mathcal{B},\mathcal{A}}(\epsilon_{C_{n+1}}-C_{n+1}\epsilon)}\text{cr}_{n+1}^{\mathcal{B},\mathcal{A}}C_{n+1}(F_k)\xrightarrow{\text{cr}_{n+1}^{\mathcal{B},\mathcal{A}}\epsilon}\text{cr}_{n+1}^{\mathcal{B},\mathcal{A}}(F_k)
    \end{equation*}
    %%
    By Lemma \ref{lem:contractHomotop} we have a family of horizontal contractions for each row after applying $\text{cr}_{n+1}$, denoted $s^{k,h}$. Setting the vertical contractions, $s^v$, to be zero, we obtain a natural contraction for the bicomplex, so under the totalization we obtain a natural contraction for the chain complex $\text{cr}_{n+1}(P_n(F))$, as desired.

    \vspace{10pt}

    For (ii) let $F$ be of degree $n$, so $\text{cr}_{n+1}(F)$ is naturally contractible. Recall the $k$th column of the bicomplex defining $P_n(F)$ is $C_{n+1}^{\times k}(F) = (\Delta^*\text{cr}_{n+1})^k(F)$. The map $p_n:F\rightarrow P_n(F)$ is the natural inclusion of the $0$th column into the totalization. Note that $C_{n+1}^{\times k}(F)$ is contractible for each $k$ since $C_{n+1}$ is exact and $F$ is of degree $n$. Then by Corollary A.7 in \cite{BJORT} the map $p_n$ is a \rd{natural} chain homotopy equivalence. (\textbf{DETAILS TO BE ADDED IN SEPARATE SECTION})

    \vspace{10pt}

    To show (iii) let $\tau:F\rightarrow G$ be a natural transformation transformation where $G$ is a functor of degree $n$. By naturality of $p_n$ in the functor $F$ we have a commutative diagram
    %%
    \[\begin{tikzcd}
    	F & G \\
    	{P_n(F)} & {P_n(G)}
    	\arrow["\tau", from=1-1, to=1-2]
    	\arrow["{P_n(\tau)}"', from=2-1, to=2-2]
    	\arrow["{p_{n,F}}"', from=1-1, to=2-1]
    	\arrow["{p_{n,G}}", from=1-2, to=2-2]
    \end{tikzcd}\]
    %%
    where the right hand $p_{n,G}$ is a natural chain homotopy equivalence by (ii). Let $s_{n,G}$ denote a natural chain homotopy inverse of $p_{n,G}$. Setting $\tau^\# = s_{n,G}\circ P_n(\tau)$ we have that
    %%
    \begin{equation*}
        \tau^\#\circ p_{n,F} = s_{n,G}\circ P_n(\tau)\circ p_{n,F} = s_{n,G}\circ p_{n,G}\circ \tau \simeq_{\cat{Ch},Nat} \tau
    \end{equation*}
    %%
    This shows $\tau$ factors through $p_n:F\rightarrow P_n(F)$ up to \rd{natural} chain homotopy equivalence. 
    
    
    To show uniqueness suppose $\sigma:P_n(F)\rightarrow G$ is another map such that $\tau$ is naturally chain homotopy equivalent to $\sigma\circ p_{n,F}$. Then by naturality of the $p_n$, we have a commuting diagram
    %%
    \[\begin{tikzcd}
    	F & {P_n(F)} & G \\
    	{P_n(F)} & {P_n(P_n(F))} & {P_n(G)}
    	\arrow["{p_{n,F}}"', from=1-1, to=2-1]
    	\arrow["{P_np_{n,F}}"', from=2-1, to=2-2]
    	\arrow["{p_{n,F}}", from=1-1, to=1-2]
    	\arrow["{p_{n,P_n(F)}}", from=1-2, to=2-2]
    	\arrow["\sigma", from=1-2, to=1-3]
    	\arrow["{p_{n,G}}", from=1-3, to=2-3]
    	\arrow["{P_n(\sigma)}"', from=2-2, to=2-3]
    \end{tikzcd}\]
    %%
    
    
    The above factorization applied to $p_n$ and $\sigma$ give factorizations $P_n((p_n)_F)\circ (p_n)_F = (p_n)_F\circ (p_n)_{P_n(F)}$ and $P_n(\sigma)\circ (p_n)_{P_n(F)}=(p_n)_G\circ \sigma$, where $(p_n)_{P_n(F)}$ and $(p_n)_G$ are natural chain homotopy equivalences. 
\end{proof}

Note that the universal property in bullet (iii) of Proposition \ref{prop:4.5}, along with the fact that degrees of a functor are characterized by their minimal value, we obtain a factorization (up to homotopy in general)
%%
\[\begin{tikzcd}
	&& F \\
	\cdots & {P_{n+1}(F)} & {P_n(F)} & {P_{n-1}(F)} & \cdots & {P_0(F)}
	\arrow["{p_0}", from=1-3, to=2-6]
	\arrow["{q_1}"', from=2-5, to=2-6]
	\arrow["{p_n}"', from=1-3, to=2-3]
	\arrow["{p_{n-1}}"{description}, from=1-3, to=2-4]
	\arrow[from=2-4, to=2-5]
	\arrow["{q_n}"', from=2-3, to=2-4]
	\arrow["{p_{n+1}}"{description}, from=1-3, to=2-2]
	\arrow["{q_{n+1}}"', from=2-2, to=2-3]
	\arrow[from=1-3, to=2-1]
	\arrow[from=2-1, to=2-2]
\end{tikzcd}\]
%%
which is known as the \textbf{algebraic Taylor tower} of the functor $F$, where the $q_n$ are determined uniquely by the universal property of the $P_n(F)$. We can also realize the $q_n:P_n(F)\rightarrow P_{n-1}(F)$ as being induced by a natural transformation $\rho_n:C_{n+1}\Rightarrow C_n:\text{Fun}(\mathcal{B},\mathcal{A})\rightarrow \text{Fun}(\mathcal{B},\mathcal{A})$, with components given by
%%
\[\begin{tikzcd}
	{C_{n+1}(F)(X)=\text{cr}_{n+1}(\Delta^*F)(X)} & {\text{cr}_n(F)(X\oplus X,X,...,X)} & {\text{cr}_n(\Delta^*F)(X)=C_n(F)(X)}
	\arrow[tail, from=1-1, to=1-2]
	\arrow["{\text{cr}_n(F)(+,1_X,...,1_X)}", outer sep = 5pt, from=1-2, to=1-3]
\end{tikzcd}\]
%%
\begin{proof}[Proof of construction of $q_n$]
    Let $\rho_n:C_{n+1}\Rightarrow C_n$ be the natural transformation described above. Then we have natural transformations $\rho_n([k]):C_{n+1}^{\times (k+1)}\Rightarrow C_n^{\times (k+1)}$ given by
    %%
    \begin{equation*}
        \rho_n([k]) = C_n^{\times k}\rho_n \circ C_n^{\times (k-1)}(\rho_n)_{C_{n+1}}\circ \cdots \circ C_n(\rho_n)_{C_{n+1}^{\times (k-1)}}\circ(\rho_n)_{C_{n+1}^{\times k}}
    \end{equation*}
    %%
    These natural transformations define a natural map of the bicomplexes defining $P_n$ and $P_{n-1}$. Applying totalization we obtain the natural transformation $q_n:P_n\Rightarrow P_{n-1}$ described. Since the map of bicomplexes defining $q_n$ is the identity on the zeroth column given by the identity, it follows that the desired triangle commutes.
\end{proof}
%%


\begin{eg}{}
    Note that for $\deg_0^\mathcal{A}$, since $\text{cr}_1(\deg_0^\mathcal{A}) = \deg_0^\mathcal{A}$ is the identity concentrated in degree zero, by the inductive definition $\text{cr}_n(\deg_0^\mathcal{A})\cong 0$ for $n \geq 2$. Then the bicomplex defining $P_n(\deg_0^\mathcal{A})$ has $\deg_0^\mathcal{A}$ as the $0$th column, with all other columns zero. This implies that $P_n(\deg_0^\mathcal{A}) = \deg_0^\mathcal{A}$ for $n \geq 1$.

    The map $q_n(\deg_0^\mathcal{A}):P_n(\deg_0^\mathcal{A})\rightarrow P_{n-1}(\deg_0^\mathcal{A})$ for $n \geq 2$ is the identity. For $n = 1$ the $\rho_1$ defining $q_1$ has $0$th component the identity and $k$th component the zero map for $k \geq 1$. It follows that $q_1(\deg_0^\mathcal{A}):P_1(\deg_0^\mathcal{A})\rightarrow P_0(\deg_0^\mathcal{A})$ is the natural inclusion.
\end{eg}
