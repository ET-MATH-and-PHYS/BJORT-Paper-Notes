\documentclass[12pt]{book}

%%%%%% PACKAGES %%%%%%%



%% References
\usepackage{hyperref}
\hypersetup{
    colorlinks,
    citecolor=black,
    filecolor=black,
    linkcolor=black,
    urlcolor=black
}
\usepackage{nameref}
\usepackage{url}
%\usepackage{apacite} % helps you apa cite
\usepackage[numbers]{natbib} % implements both author-year and numbered references


%% Math packages
\usepackage{amsmath} % Math display options
\usepackage{amsthm}
\usepackage{mathtools} % General math tools
\usepackage{array, longtable} % Allows you to write arrays
\usepackage{bbold}
\usepackage{empheq} % For boxing equations
%\usepackage{physics} % useful math symbols and macros (VERY IMPORTANT)
\usepackage{cancel}
% Math fonts and symbols
\usepackage{amsfonts} 
%\usepackage{mathabx}
%\usepackage{mathrsfs}
%\usepackage{txfonts} % defines times new roman as default text font and provides supporting math symbols
\usepackage{braket} % allows the use of detailed Dirac braket notation
\usepackage{amssymb}
\usepackage{mathdots}
\usepackage{siunitx} % allows you to quickly define units within math mode (IMPORTANT)
\usepackage{latexsym}
\usepackage{bbm}
%% Figure and Table Packages
\usepackage{wrapfig}
\usepackage{enumitem}
\usepackage{float}
\usepackage{booktabs} % allows weird lines within tables
\usepackage{multirow}
\usepackage{multicol} % allows for multiple columns and rows that collapse into single columns
\usepackage{longtable} % allows a table to run over multiple pages
\usepackage{colortbl} % colourful tables!
\usepackage{mathrsfs}  

\usepackage{adjustbox}

\usepackage[normalem]{ulem}

\usepackage[toc, page]{appendix}

%% Heading packages
\usepackage{titletoc,tocloft}
\usepackage{titlesec}


%% Numbering commands
\titleformat{\section}
  {\normalfont\fontsize{25}{15}\bfseries}{\thesection}{1em}{}
\titleformat{\section}
  {\normalfont\fontsize{20}{15}\bfseries}{\thesubsection}{1em}{}
\setcounter{secnumdepth}{3}  
\newcommand\numberthis{\refstepcounter{equation}\tag{\theequation}} % For equation labelling


%% Tikz packages
\usepackage[framemethod=tikz]{mdframed}
\usepackage{tikz} % For drawing commutative diagrams
\usetikzlibrary{cd}
\usetikzlibrary{calc}
\tikzset{every picture/.style={line width=0.75pt}} %set default line width to 0.75p


%% Page formatting packages
\usepackage{datetime}
\usepackage[utf8]{inputenc} % the typesetting rules
\usepackage[margin=1in]{geometry} % flexible and easy interface to change page dimensions
\setlength{\parskip}{1em} % Sets the default paragraph skip to 1em
\setlength{\parindent}{4mm} % Sets default paragraph indent
\usepackage{graphicx} % provides additional options for figures
\usepackage{changepage} % allows to change page formatting in the middle of a document (rather than the same throughout) 
\usepackage{float}
\usepackage[labelfont=bf]{caption} % altering options for captions
\usepackage{sidecap} % allows sideways captions for figures and tables
\usepackage{soul,xcolor} % can enhance striking, underlining, and highlighting (in non-math mode)
\usepackage{fancyhdr}
\setlength{\headheight}{15pt} 
\pagestyle{fancy}
\lhead[\leftmark]{}
\rhead[]{\leftmark}
\allowdisplaybreaks
\newdateformat{monthdayyeardate}{%
    \monthname[\THEMONTH]~\THEDAY, \THEYEAR}



%% Custom 
\usepackage{color} % provides coloring capabilities to everything
\definecolor{purp}{rgb}{0.29, 0, 0.51}
\definecolor{bloo}{rgb}{0, 0.13, 0.80}



%%\newtheoremstyle{note}% hnamei
%{3pt}% hSpace above
%{3pt}% hSpace belowi
%{}% hBody fonti
%{}% hIndent amounti
%{\itshape}% hTheorem head fonti
%{:}% hPunctuation after theorem headi
%{.5em}% hSpace after theorem headi
%{}% hTheorem head spec (can be left empty, meaning ‘normal’)

%%%%%%%%%%%%% THEOREM STYLES

\newtheoremstyle{BigTheorem}
{20pt}
{20pt}
{\slshape}
{}
{\Large\bfseries}
{.}
{\newline}
{\thmname{#1}\thmnumber{ #2}\thmnote{ (#3)}}



\newtheoremstyle{TheoremClassic}
{15pt}
{15pt}
{\slshape}
{}
{\bfseries}
{.}
{.5em}
{}

\newtheoremstyle{Definitions}
{15pt}
{15pt}
{\slshape}
{}
{\bfseries}
{.}
{.5em}
{\thmname{#1}\thmnumber{ #2}\thmnote{ (#3)}}


\newtheoremstyle{Custom}
{15pt}
{15pt}
{\slshape}
{}
{\bfseries}
{.}
{.5em}
{\thmname{#1}\thmnumber{ #2}\thmnote{ #3}}

\newtheoremstyle{Remarks}
{10pt}
{10pt}
{\upshape}
{}
{\bfseries}
{.}
{.5em}
{}

\newtheoremstyle{Examples}
{10pt}
{10pt}
{\upshape}
{}
{\bfseries}
{.}
{.5em}
{}


%%%%%%%%%%%%% THEOREM DEFINITIONS

\usepackage[most]{tcolorbox}
%%%%%%%%%%%%%% ENVIRONMENTS
%%% tcolorbox environments

%%% \tcolorboxenvironment: An existing environment is redefined to be allowed to be boxed inside another tcolorbox during use
\tcolorboxenvironment{proof}{% `proof' from `amsthm'  
blanker,breakable=true,left=0.3em,
borderline west={3pt}{-3pt}{black!35}}


%%% titled environments
\newtcolorbox[auto counter, number within=section]{nthm}[2][]{
colback=red!5!white,
colframe=red!5!white,
attach title to upper={},
title={\textbf{Theorem \thetcbcounter\  (#2)\ }},
coltitle={black},
sharp corners,
breakable=true,
enhanced jigsaw,
borderline west = {3pt}{-3pt}{red!75!black},
#1
}

\newtcolorbox[use counter from=nthm, number within=section]{ndefn}[2][]{
colback=blue!5!white,
colframe=blue!5!white,
attach title to upper={},
title={\textbf{Definition \thetcbcounter\  (#2)\ }},
coltitle={black},
sharp corners,
breakable=true,
enhanced jigsaw,
borderline west = {3pt}{-3pt}{blue!75!black},
#1
}

\newtcolorbox[auto counter, number within=section]{eg}[2][]{
colback=white,
attach title to upper={},
title={\underline{Example \thetcbcounter\  (#2)\ :}\smallbreak},
coltitle={green!50!black},
sharp corners,
breakable=true,
enhanced jigsaw,
frame hidden, 
borderline west = {3pt}{-3pt}{green!50!black},
#1
}

\newtcolorbox[auto counter, number within=section]{qst}[1][]{
colback=white,
attach title to upper={},
title={\underline{Question?}\smallbreak},
coltitle={blue!50!black},
sharp corners,
breakable=true,
enhanced jigsaw,
frame hidden, 
borderline west = {3pt}{-3pt}{blue!50!black},
#1
}

\newtcolorbox[auto counter, number within=section]{rmk}[1][]{
colback=white,
attach title to upper={},
title={\underline{Remark:}\smallbreak},
coltitle={red!50!black},
sharp corners,
breakable=true,
enhanced jigsaw,
frame hidden, 
borderline west = {3pt}{-3pt}{red!50!black},
#1
}


\newtcolorbox[auto counter, number within=section]{nte}[1][]{
colback=white,
attach title to upper={},
title={\underline{Note:}\smallbreak},
coltitle={red!50!white},
sharp corners,
breakable=true,
enhanced jigsaw,
frame hidden, 
borderline west = {3pt}{-3pt}{red!50!white},
#1
}


%%% untitled environments
\newtcolorbox[use counter from=nthm, number within=section]{thm}[1][]{
colback=red!5!white,
colframe=red!5!white,
attach title to upper={},
title={\textbf{Theorem \thetcbcounter\ }},
coltitle={black},
sharp corners,
breakable=true,
enhanced jigsaw,
borderline west = {3pt}{-3pt}{red!75!black},
#1
} 

\newtcolorbox[use counter from=nthm, number within=section]{defn}[1][]{
colback=blue!5!white,
colframe=blue!5!white,
attach title to upper={},
title={\textbf{Definition \thetcbcounter\ }},
coltitle={black},
sharp corners,
breakable=true,
enhanced jigsaw,
borderline west = {3pt}{-3pt}{blue!75!black},
#1
}

\newtcolorbox[use counter from=nthm, number within=section]{cor}[1][]{
colback=red!5!white,
colframe=red!5!white,
attach title to upper={},
title={\textbf{Corollary \thetcbcounter\ }},
coltitle={black},
sharp corners,
breakable=true,
enhanced jigsaw,
borderline west = {3pt}{-3pt}{red!50!black},
#1
}

\newtcolorbox[use counter from=nthm, number within=section]{prop}[1][]{
colback=orange!5!white,
colframe=orange!5!white,
attach title to upper={},
title={\textbf{Proposition \thetcbcounter} \ },
coltitle={black},
sharp corners,
breakable=true,
enhanced jigsaw,
borderline west = {3pt}{-3pt}{orange!75!black},
#1
}

\newtcolorbox[use counter from=nthm, number within=section]{lem}[1][]{
colback=orange!5!white,
colframe=orange!5!white,
attach title to upper={},
title={\textbf{Lemma \thetcbcounter} \ },
coltitle={black},
sharp corners,
breakable=true,
enhanced jigsaw,
frame hidden, 
borderline west = {3pt}{-3pt}{orange!75!black},
#1
}

\newtcolorbox[auto counter, number within=section]{home}[1][]{
colback=white,
attach title to upper={},
title={\textbf{Exercise:}\ \ },
coltitle={black},
sharp corners,
breakable=true,
enhanced jigsaw,
frame hidden, 
borderline west = {3pt}{-3pt}{black!10!lime},
#1
}

%% For proofs
\renewcommand{\qedsymbol}{$\blacksquare$}




%%%%%% MACROS %%%%%%%%%

%% Fields
\newcommand\C{\mathbb C} %%% Complex numbers
\newcommand\R{\mathbb R} %%% Real numbers
\newcommand\Z{\mathbb Z} %%% Integers
\newcommand\Q{\mathbb Q} %%% Rationals
\newcommand\N{\mathbb N} %%% Naturals
\newcommand\F{\mathbb F} %%% An arbitrary field

%% Categories
\newcommand{\cat}[1]{\textup{\textsf{#1}}}% for categories
\newcommand{\Cat}{\mathbb{C}\cat{at}}
\newcommand{\Bicat}{\mathbb{B}\cat{icat}}
\newcommand{\sSet}{\cat{s}\mathbb{S}\cat{et}}
\newcommand{\SET}{\mathbb{S}\cat{et}}
\newcommand{\Grp}{\cat{Grp}} %%% Category group
\newcommand{\Rmod}{\cat{R-Mod}} %%% Category r-modules
\newcommand{\Mon}{\cat{Mon}} %%% Category monoid
\newcommand{\Ring}{\cat{Ring}} %%% Category ring
\newcommand{\Topp}{\cat{Top}} %%% Category Topological spaces
\newcommand{\Vect}{\cat{Vect}_{k}} %%% category vector spaces'
\newcommand\Hom{\cat{Hom}} %%% Arrows
\DeclareSymbolFont{bbold}{U}{bbold}{m}{n}
\DeclareSymbolFontAlphabet{\mathbbb}{bbold}
\newcommand{\catone}{\ensuremath{\mathbbb{1}}}
\newcommand{\cattwo}{\ensuremath{\mathbbb{2}}}
\newcommand{\catthree}{\ensuremath{\mathbbb{3}}}
\newcommand{\catn}{\ensuremath{\mathbbb{n}}}
\newcommand{\adjnt}[2]{
\begin{tikzcd}
{#1} & {#2}
\arrow[""{name=0, anchor=center, inner sep=0}, "R"', shift right=2, from=1-1, to=1-2]
\arrow[""{name=1, anchor=center, inner sep=0}, "L"', shift right=2, from=1-2, to=1-1]
\arrow["\dashv"{anchor=center, rotate=-90}, draw=none, from=1, to=0]
\end{tikzcd}
}



%% Convenient brackets
\newcommand{\IP}[1]{\left\langle#1\right\rangle} %%% Inner product
\newcommand{\ABS}[1]{\lvert#1\rvert} %%% Modulus
\newcommand{\COIP}[2]{\left\langle #1\left|#2\right.\right\rangle}


%% Linear Algebra
\newcommand\GL{\text{GL}} %%% General Linear group
\newcommand\SL{\text{SL}} %%% Special linear group

\newcommand\gl{\mathfrak{gl}} %%% General linear algebra
\newcommand\g{\mathfrak{g}} %%% Lie algebra of G

%% Aligned maps
\newcommand{\map}[2]{\begin{array}{c} #1 \\ #2 \end{array}}

%% Text emphasis
\newcommand{\Emph}[1]{\textbf{\ul{\emph{#1}}}}

%% reverse mapsto
\newcommand{\mapsfrom}{\mathrel{\reflectbox{\ensuremath{\mapsto}}}}


%% Math operators
\DeclareMathOperator{\ran}{Im} %%% image
\DeclareMathOperator{\aut}{Aut} %%% Automorphisms
\DeclareMathOperator{\spn}{span} %%% span
\DeclareMathOperator{\ann}{Ann} %%% annihilator
\DeclareMathOperator{\ch}{char} %%% characteristic
\DeclareMathOperator{\Ev}{\bf{ev}} %%% evaluation
\DeclareMathOperator{\sgn}{sign} %%% sign
\DeclareMathOperator{\id}{Id} %%% identity
\DeclareMathOperator{\inn}{Inn} %%% Inner aut
\DeclareMathOperator{\en}{End} %%% Endomorphisms
\DeclareMathOperator{\sym}{Sym} %%% Group of symmetries
\newcommand{\Sob}{^{\Delta^{op}}} %%% Simplicial object category

\newcommand{\rd}[1]{\textcolor{red}{#1}}

%% Diagram Environments
\iffalse
\begin{center}
    \begin{tikzpicture}[baseline= (a).base]
        \node[scale=1] (a) at (0,0){
          \begin{tikzcd}
           
          \end{tikzcd}
        };
    \end{tikzpicture}
\end{center}
\fi





%%%%%%%%%%%%%%%%%%%%%%%
\def\curved{\tikz[baseline=.1ex]{
\fill (0,0) circle (1.5pt) coordinate (A);
\fill (0,1.5ex) circle (1.5pt)
\draw (A)--(B);}
}
\usepackage{quiver}

%%% Specific Macros %%%
\newcommand\lcm{\text{lcm}}
\newcommand{\norm}[1]{\left|\left|#1\right|\right|}

%% Heading
\pagestyle{fancy}
\lhead{BJORT Paper Analysis}
\rhead{Name: E/Ea}
\chead{}
%\lfoot{Author's Name}
%\cfoot{}
%\rfoot{Page \thepage}

%%%%%% BEGIN %%%%%%%%%%


\begin{document}

%%%%%% TITLE PAGE %%%%%


\title{BJORT Paper Analysis and Extension}
\author{E/Ea}
\date{\today}


%%%%%%%%%%%%%%%%%%%%%%%

\maketitle

\tableofcontents

% \chapter{Original Work}

% \section{Project Proposal}

% \input{Proposal}


% \section{Definitions and Notation}

% \input{MainWork/DefinitionsandNotation}

% \section{Examples and Computations}

% \input{MainWork/Examples}


% \section{General Results}

% \input{MainWork/Results}


\chapter{BJORT}

\newcommand{\crn}{\text{cr}}
\newcommand{\invbn}{\text{!`}}

In this chapter we carefully go through the paper Directional derivatives and higher order chain rules for abelian
functor calculus \cite{BJORT}. Unless stated otherwise, compsition using juxtapoosition is done diagrammatically, while composition using $\circ$ follows function notation.

\section{Cross Effects for Functors}

Throughout let $\mathcal{B}$ be a category with a basepoint (i.e. an initial and terminal object), and finite coproducts $\lor$. Let $\mathcal{A}$ denote an abelian category with zero object $0$ and biproducts $\oplus$.

\begin{defn}[label=defn:crossEffect]{(Cross Effects)}
    We define the \textbf{nth cross effect} of a functor $F:\mathcal{B}\rightarrow \mathcal{A}$ implicitly, and recursively, as the $n$-variable functor $\crn_n(F):\mathcal{B}^n\rightarrow \mathcal{A}$ such that:
    %%
    \begin{equation*}
        F(X) \cong F(\star)\oplus \crn_1(F)(X)
    \end{equation*}
    %%
    \begin{equation*}
        \crn_1(F)(X_1\lor X_2) \cong \crn_1(F)(X_1)\oplus \crn_1(F)(X_2)\oplus \crn_2F(X_1,X_2)
    \end{equation*}
    %%
    and in general
    %%
    \begin{align*}
        \crn_{n-1}(F)(X_1\lor X_2,X_3,...,X_n)\cong \crn_{n-1}(F)(X_1,X_3,...,X_n)&\oplus \crn_{n-1}(F)(X_2,...,X_n) \\
        &\oplus \crn_n(F)(X_1,...,X_n)
    \end{align*}
\end{defn}

We prove that this defines a family of functors which are symmetric through a series of lemmas:

\begin{lem}[label=lem:coprodMono]
    For all $X,Y \in \mathcal{B}_0$, the inclusion $X\hookrightarrow X\lor Y$ is a split monomorphism.
\end{lem}
\begin{proof}
    Let $\iota_X$ and $\iota_Y$ denote the coproduct inclusions. Let $\hat{!}:Y\xrightarrow{!} \star\xrightarrow{\text{!`}} X$ denote the unique map $\text{!`}\circ !$. Then by the unniversal property of the coproduct we obtain a map
    %%
    \begin{equation*}
        X\lor Y\xrightarrow{\COIP{1_X}{\hat{!}}}X
    \end{equation*}
    %%
    such that $\COIP{1_X}{\hat{!}} \circ \iota_X= 1_X$, so $\iota_X$ is a split monomorphism, or in other words a section.
\end{proof}

Throughout these notes $\hat{!}$ will denote the unique map which factors through the basepoint (note that by the universal property this is independent of the choice of basepoint). Additionally, we let $\COIP{1_X}{\hat{!}}$ denote the splitting for the inclusion $\iota_X:X\hookrightarrow X\lor Y$ in $\mathcal{B}$.


\begin{lem}[label=lem:biprod]
    Let $A,B,C \in \mathcal{A}_0$. Then $A\oplus B\cong A\oplus C$ if and only if $B \cong C$.
\end{lem}
\begin{proof}
    The reverse direction follows by functoriality of $A\oplus-$ in $\mathcal{A}$. For the forward direction suppose $A\oplus B\cong A\oplus C$ with isomorphism $\psi$. Then the claim is that the composite map $\pi_C\circ \psi\circ \iota_B$ is an isomorphism with inverse $\pi_B\circ \psi^{-1}\circ \iota_C$. Indeed, from the axioms of a biproduct we have 
    %%
    \begin{equation*}
        \pi_B\circ \psi^{-1}\circ \iota_C\circ \pi_C\circ \psi\circ \iota_B = 1_B,\;\;\pi_C\circ \psi\circ \iota_B\circ \pi_B\circ \psi^{-1}\circ \iota_C = 1_C
    \end{equation*}
    %%
    as desired.
\end{proof}

We now aim to make explicit this definition of the cross-effect functor, proceeding inductively. We can realize $\crn_1(F)(X)$ as the kernel of $F(!)$ in the following split short exact sequence.
%%
\[\begin{tikzcd}
	{\text{cr}_1(F)(X)} & {F(X)} & {F(\star)}
	\arrow["{F(!)}", from=1-2, to=1-3]
	\arrow["ker", tail, from=1-1, to=1-2]
	\arrow["{F(\text{!`})}", curve={height=-6pt}, from=1-3, to=1-2]
\end{tikzcd}\]
%%
We choose a representative kernel for each such $X \in \mathcal{B}_0$. In particular, if $F(\star) = 0$ (i.e. $F$ is reduced), we choose $\text{cr}_1(F)(X) := F(X)$. Note that by $F(\text{!`})$ this map splits, with left splitting given by the universal property of the kernel in the diagram
\[\begin{tikzcd}
	{\text{cr}_1(F)(X)} & {F(X)} & {F(\star)} \\
	{F(X)}
	\arrow["ker", from=1-1, to=1-2]
	\arrow["{F(!)}", curve={height=-6pt}, from=1-2, to=1-3]
	\arrow["{F(\text{!`})}", curve={height=-6pt}, from=1-3, to=1-2]
	\arrow["{1-F(\hat{!})}"', from=2-1, to=1-2]
	\arrow["{r_{F,1}}", dashed, from=2-1, to=1-1]
\end{tikzcd}\]
where $1 = 1_{F(X)}$ in the diagram. For simplicity of notation we write $1$ for all identities moving forward, with the object of the identity given by context.


Then, given $X\xrightarrow{f}Y \in \mathcal{B}_1$, we obtain a unique map $\crn_1(F)(f)$ making the following diagram commute:
%%
\[\begin{tikzcd}
	{\text{cr}_1(F)(X)} & {F(X)} & {F(\star)} \\
	{\text{cr}_1(F)(Y)} & {F(Y)} & {F(\star)}
	\arrow["{F(!)}", from=1-2, to=1-3]
	\arrow["ker", tail, from=1-1, to=1-2]
	\arrow["{F(!)}", from=2-2, to=2-3]
	\arrow["ker"', tail, from=2-1, to=2-2]
	\arrow[Rightarrow, no head, from=1-3, to=2-3]
	\arrow["{F(f)}"{description}, from=1-2, to=2-2]
	\arrow["{\text{cr}_1(F)(f)}"', dashed, from=1-1, to=2-1]
\end{tikzcd}\]
%%
by the universal property of the kernel, where uniqueness ensures that this defines a functor. Additionally, observe that
%%
\begin{equation*}
    ker\circ r_{F,1}\circ F(f)\circ ker =(1-F(\hat{!}))\circ F(f)\circ ker = F(f)\circ ker 
\end{equation*}
%%
using the definition of $\hat{!} = \text{!`}\circ !$ and the kernel. Uniqueness implies 
%%
\begin{equation}\label{eq:projFormula1}
    \text{cr}_1(F)(f) = r_{F,1}\circ F(f)\circ s_{F,1}
\end{equation}
%%
where we denote the kernel, which is also the inclusion into $F$ of its first cross-effect, by $s_{F,1}$.



We show functoriality of the remaining $\crn_n(F)$ by induction. Suppose $\crn_{n-1}(F)$ is functorial and symmetric in each component, and we show that so is $\crn_n(F)$. We define $\crn_n(F)(X_1,...,X_n)$ for $X_1,...,X_n \in \mathcal{B}_0$ as the kernel
%%
\[\begin{tikzcd}
	{\text{cr}_n(F)(X_1,...,X_n)} \\
	{\text{cr}_{n-1}(F)(X_1\lor X_2,X_3,...,X_n)} \\
	{\text{cr}_{n-1}(F)(X_1,X_3,...,X_n)\oplus\text{cr}_{n-1}(F)(X_2,X_3,...,X_n)}
	\arrow["{\left\langle\text{cr}_{n-1}(F)(\langle1_{X_1}|\hat{!}\rangle,1_{X_3},...,1_{X_n}),\text{cr}_{n-1}(F)(\langle \hat{!}|1_{X_2}\rangle,1_{X_3},...,1_{X_n})\right\rangle}", curve={height=-12pt}, from=2-1, to=3-1]
	\arrow["{\left\langle\text{cr}_{n-1}(F)(\iota_{X_1},1_{X_3},...,1_{X_n})|\text{cr}_{n-1}(F)(\iota_{X_2},1_{X_3},...,1_{X_n})\right\rangle}", curve={height=-12pt}, from=3-1, to=2-1]
	\arrow["ker", tail, from=1-1, to=2-1]
\end{tikzcd}\]
%%
Again we choose representatives for the kernel. Let $\iota_i = \text{cr}_{n-1}(F)(\iota_{X_i},1_{X_3},...,1_{X_n})$ and $\pi_i = \text{cr}_{n-1}(F)(\langle1_{X_1}|\hat{!}\rangle,1_{X_3},...,1_{X_n})$. This SES is split. It is sufficient to show that $\langle\iota_1|\iota_2\rangle\langle\pi_1,\pi_2\rangle = 1$. Using the universal property of the coproduct it is sufficient to show $\iota_L\langle\iota_1|\iota_2\rangle\langle\pi_1,\pi_2\rangle = \iota_L$ and $\iota_R\langle\iota_1|\iota_2\rangle\langle\pi_1,\pi_2\rangle = \iota_R$ for the left and right inclusions. By symmetry it is sufficient to show this for just the left inclusion. First, using the definition of a map out of a coproduct
%%
\begin{equation*}
    \iota_L\langle\iota_1|\iota_2\rangle\langle\pi_1,\pi_2\rangle = \iota_1\langle\pi_1,\pi_2\rangle
\end{equation*}
%%
In order to show that this is equal to $\iota_L$ it is sufficient to show that post-composition with $\pi_L$ yields the identity while post-composition with $\pi_R$ yields zero by uniqueness of the map into a product and the structure of a biproduct. Observe that 
%%
\begin{align*}
    \iota_1\langle\pi_1,\pi_2\rangle\pi_L &= \iota_1\pi_1 = \text{cr}_{n-1}(F)(\iota_{X_1}\langle 1_{X_1}|!\rangle,1_{X_{3}},...,1_{X_n}) \\
    &= \text{cr}_{n-1}(F)(1_{X_1},1_{X_{3}},...,1_{X_n}) = 1_{\text{cr}_{n-1}(F)(X_1,X_3,...,X_n)}
\end{align*}
%%
and
%%
\begin{equation*}
    \iota_1\langle\pi_1,\pi_2\rangle\pi_R = \iota_1\pi_2 = \text{cr}_{n-1}(F)(\iota_{X_1}\langle \hat{!}|1_{X_2}\rangle,1_{X_{3}},...,1_{X_n}) = \text{cr}_{n-1}(F)(\hat{!},1_{X_{3}},...,1_{X_n})
\end{equation*}
using functoriality of $\text{cr}_{n-1}(F)$ from the inductive hypothesis. To show this second map is zero, it is sufficient to show that $\text{cr}_{n-1}(F)(\star,X_3,...,X_n) \cong 0$ for all $n \geq 2$.

\begin{lem}[label=lem:zeroCross]
    Let $n \in \N$ and $X_2,...,X_n \in \mathcal{B}_0$. Then $\text{cr}_n(F)(\star,X_2,...,X_n) \cong 0$.
\end{lem}
\begin{proof}
    We proceed by induction on $n$ using the implicit definition. If $n = 1$ then we have
    %%
    \begin{equation*}
        F(\star) \cong F(\star)\oplus\text{cr}_1(F)(\star)
    \end{equation*}
    %%
    By Lemma \ref{lem:biprod} $\text{cr}_1(F)(\star)\cong 0$.


    Now, suppose the claim holds for some $n-1 \geq 1$. We have the direct sum decomposition:
    %%
    \begin{align*}
        \text{cr}_{n-1}(F)(\star\lor X_2,X_3,...,X_n) &\cong \text{cr}_{n-1}(F)(\star,X_3,...,X_n)\oplus\text{cr}_{n-1}(F)(X_2,X_3,...,X_n) \\
        &\oplus \text{cr}_n(F)(\star,X_2,...,X_n) \\
        &\cong 
        \text{cr}_{n-1}(F)(X_2,X_3,...,X_n) \oplus \text{cr}_n(F)(\star,X_2,...,X_n)
    \end{align*}
    where the last isomorphism follows by the induction hypothesis and the fact that the $0$ object is the monoidal unit for the biproduct. Then, since $\text{cr}_{n-1}(F)(\star\lor X_2,X_3,...,X_n) \cong \text{cr}_{n-1}(F)(X_2,X_3,...,X_n)$, we are reduced to the base case, so again by Lemma \ref{lem:biprod} $\text{cr}_n(F)(\star,X_2,...,X_n) \cong 0$.
\end{proof}

Using Lemma \ref{lem:zeroCross} we then obtain the desired $\iota_1\pi_2 = 0$, so by uniqueness $\iota_L\langle\iota_1|\iota_2\rangle\langle\pi_1,\pi_2\rangle = \iota_L$, and by a symmetric argument for $\iota_R$ we obtain by uniqueness that the composite of the maps is the identity, so the SES splits. Additionally, the left splitting is given by
\[\begin{tikzcd}
	{\text{cr}_n(F)(X_1,...,X_n)} & {\text{cr}_{n-1}(F)(X_1\lor X_2,\overline{X})} & {\text{cr}_{n-1}(F)(X_1,\overline{X})\oplus\text{cr}_{n-1}(F)(X_2,\overline{X})} \\
	{\text{cr}_{n-1}(F)(X_1\lor X_2,\overline{X})}
	\arrow["ker", from=1-1, to=1-2]
	\arrow["{\langle\text{cr}_{n-1}(F)(\langle 1|\hat{!}\rangle,1),\text{cr}_{n-1}(F)(\langle \hat{!}| 1\rangle,1)\rangle}", curve={height=-18pt}, from=1-2, to=1-3]
	\arrow["{\langle\text{cr}_{n-1}(F)(\iota_{X_1},1)|\text{cr}_{n-1}(F)(\iota_{X_2},1)\rangle}", curve={height=-12pt}, from=1-3, to=1-2]
	\arrow["{1-\Delta(\text{cr}_{n-1}(F)(1\lor\hat{!},1)\oplus\text{cr}_{n-1}(F)(\hat{!}\lor 1,1))\nabla}"', from=2-1, to=1-2]
	\arrow["{r_{F,n}}", dashed, from=2-1, to=1-1]
\end{tikzcd}\]
where $\Delta:B\rightarrow B\oplus B$ is the diagonal map, $\nabla:B\oplus B\rightarrow B$ is the codiagonal map, $\overline{X} = (X_3,...,X_n)$, and $\langle1|\hat{!}\rangle\iota_{X_1} = 1\lor \hat{!}$ while $\langle\hat{!}|1\rangle\iota_{X_2} = \hat{!}\lor 1$.

It remains to show $\text{cr}_n(F)$ is functorial in each component. We define $\text{cr}_n(F)$ on a collection of $f_1,...,f_n:X_i\rightarrow Y_i$ maps as the unique map making the diagram commute:
%%
\[\begin{tikzcd}
	{\text{cr}_n(F)(X_1,...,X_n)} & {\text{cr}_{n-1}(F)(X_1\lor X_2,\overline{X})} & {\text{cr}_{n-1}(F)(X_1,\overline{X})\oplus\text{cr}_{n-1}(F)(X_2,\overline{X})} \\
	{\text{cr}_n(F)(Y_1,...,Y_n)} & {\text{cr}_{n-1}(F)(Y_1\lor Y_2,\overline{Y})} & {\text{cr}_{n-1}(F)(Y_1,\overline{Y})\oplus\text{cr}_{n-1}(F)(Y_2,\overline{Y})}
	\arrow["ker", from=1-1, to=1-2]
	\arrow["{\text{cr}_n(F)(f_1,...,f_n)}"', dashed, from=1-1, to=2-1]
	\arrow["ker"', from=2-1, to=2-2]
	\arrow["{\text{cr}_{n-1}(F)(f_1\lor f_2,\overline{f})}"', from=1-2, to=2-2]
	\arrow["{\text{cr}_{n-1}(F)(f_1,\overline{f})\oplus\text{cr}_{n-1}(F)(f_2,\overline{f})}", from=1-3, to=2-3]
	\arrow["{\langle\pi_1,\pi_2\rangle}", shift left, from=1-2, to=1-3]
	\arrow["{\langle\iota_1|\iota_2\rangle}", shift left, from=1-3, to=1-2]
	\arrow["{\langle\pi_1,\pi_2\rangle}", shift left, from=2-2, to=2-3]
	\arrow["{\langle\iota_1|\iota_2\rangle}", shift left, from=2-3, to=2-2]
\end{tikzcd}\]
%%
Functoriality follows from the inductive hypothesis and the uniqueness of the map between the kernels, which also implies identities are sent to identities. Additionally, as in the case of $n = 1$, by uniqueness the map is equal to
%%
\begin{equation}\label{eq:projFormulaN}
    \text{cr}_n(F)(f_1,...,f_n) = r_{F,n}\circ \text{cr}_{n-1}(F)(f_1\lor f_2,f_3,...,f_n)\circ s_{F,n}
\end{equation}
%%


To show that $\text{cr}_n(F)(X_1,...,X_n)$ is symmetric in each argument we proceed by induction. Let $\sigma \in \Sigma_n$ be a permutation on $n$ letters. By the inductive hypothesis and the implicit definition we obtain
%%
\begin{align*}
    \text{cr}_n(F)(X_{\sigma(1)},...,X_{\sigma(n)})\oplus A \cong \text{cr}_n(F)(X_1,...,X_n)\oplus A
\end{align*}
%%
for $A \cong \text{cr}_{n-1}(F)(X_1,X_3,...,X_n)\oplus\text{cr}_{n-1}(F)(X_2,...,X_n)$.  Thus 
%%
\begin{equation}
    \text{cr}_n(F)(X_{\sigma(1)},...,X_{\sigma(n)})\cong \text{cr}_n(F)(X_1,...,X_n)
\end{equation}
%%
by Lemma \ref{lem:biprod}. These isomorphisms can be realized as unique maps given by the universal property of kernels. In particular, as we will soon show, these maps are the components of the natural transformation $\alpha:F\circ \sigma\Rightarrow F$ under $\text{cr}_n$.


\begin{defn}{}
    We write $\text{Fun}_*(\mathcal{B}^n,\mathcal{A})$ for the category of \textbf{strictly multi-reduced} functors from $\mathcal{B}^n$ to $\mathcal{A}$, i.e. those $F$ such that for $F(X_1,...,X_n)\cong 0$ if $X_i \cong \star$ for some $i$.
\end{defn}


We have shown now that the cross-effect gives an object map $\text{cr}_n:\text{Fun}(\mathcal{B},\mathcal{A})\rightarrow \text{Fun}_*(\mathcal{B}^n,\mathcal{A})$. It remains to show that this assignment is functorial. To this end let $\alpha:F\Rightarrow G$ be a natural transformation. We define $\text{cr}_n(\alpha)$ inductively. For the case of $n = 1$ we let $\text{cr}_1(\alpha)_X$ be the unique map making

\[\begin{tikzcd}
	{\text{cr}_1F(X)} & {F(X)} & {F(\star)} \\
	{\text{cr}_1G(X)} & {G(X)} & {G(\star)}
	\arrow["{F(!)}", from=1-2, to=1-3]
	\arrow["{G(!)}"', from=2-2, to=2-3]
	\arrow["{\alpha_\star}", from=1-3, to=2-3]
	\arrow["{\alpha_X}"', from=1-2, to=2-2]
	\arrow[tail, from=1-1, to=1-2]
	\arrow[tail, from=2-1, to=2-2]
	\arrow["{\text{cr}_1(\alpha)_X}"', dashed, from=1-1, to=2-1]
\end{tikzcd}\]

\noindent commute. As in the case of the maps themselves, we observe that by uniqueness we have the formula 
%%
\begin{equation}\label{eq:natProjFormula1}
    \text{cr}_1(\alpha)_X = r_{G,1}\circ \alpha_X\circ s_{F,1}
\end{equation}
%%
Hence, to show naturality of $\text{cr}_1(\alpha)$ it is sufficient to show naturality of $r_{G,1}$ and $s_{F,1}$. All of these naturalities follow from a general result on limits in Section~\ref{sec:colimFuncs}. Since limits in a functor category are computed componentwise, it follows that the $s_{F,n}$ and $r_{F,n}$ bundle to form natural transformations. Additionally, $\text{cr}_1(\alpha)$ is precisely the map induced by the limit for the map of diagrams
%%
\[\begin{tikzcd}
	F & {\text{ev}_\star\circ F} \\
	G & {\text{ev}_\star\circ G}
	\arrow[shift left, from=1-1, to=1-2]
	\arrow[shift right, from=1-1, to=1-2]
	\arrow["\alpha"', from=1-1, to=2-1]
	\arrow[shift left, from=2-1, to=2-2]
	\arrow[shift right, from=2-1, to=2-2]
	\arrow["{\text{ev}_\star(\alpha)}", from=1-2, to=2-2]
\end{tikzcd}\]



% We may express naturality of this definition through the commutativity of all subdiagrams in the following extended rectangles:

% %%
% \[\begin{tikzcd}
% 	{\text{cr}_1F(Y)} & {F(Y)} & {F(\star)} \\
% 	{\text{cr}_1F(X)} & {F(X)} & {F(\star)} \\
% 	{\text{cr}_1G(X)} & {G(X)} & {G(\star)} \\
% 	{\text{cr}_1G(Y)} & {G(Y)} & {G(\star)}
% 	\arrow["{F(!)}", from=2-2, to=2-3]
% 	\arrow["{G(!)}"', from=3-2, to=3-3]
% 	\arrow["{\alpha_\star}", from=2-3, to=3-3]
% 	\arrow["{\alpha_X}"', from=2-2, to=3-2]
% 	\arrow[tail, from=2-1, to=2-2]
% 	\arrow[tail, from=3-1, to=3-2]
% 	\arrow["{\text{cr}_1\alpha_X}"', dashed, from=2-1, to=3-1]
% 	\arrow[Rightarrow, no head, from=1-3, to=2-3]
% 	\arrow["{F(!)}", from=1-2, to=1-3]
% 	\arrow["{F(f)}", from=2-2, to=1-2]
% 	\arrow[tail, from=1-1, to=1-2]
% 	\arrow["{\text{cr}_1F(f)}"{description}, dashed, from=2-1, to=1-1]
% 	\arrow[Rightarrow, no head, from=3-3, to=4-3]
% 	\arrow["{G(!)}"', from=4-2, to=4-3]
% 	\arrow["{G(f)}"', from=3-2, to=4-2]
% 	\arrow[tail, from=4-1, to=4-2]
% 	\arrow["{\text{cr}_1G(f)}"{description}, dashed, from=3-1, to=4-1]
% 	\arrow["{\alpha_Y}", curve={height=-25pt}, dashed, from=1-2, to=4-2]
% 	\arrow["{\text{cr}_1\alpha_Y}"{description}, curve={height=50pt}, dashed, from=1-1, to=4-1]
% \end{tikzcd}\]
% %%

% By the universal property of the kernel there is a unique arrow from $\text{cr}_1F(X)$ to $\text{cr}_1G(Y)$ such that composing with the kernel inclusion yields $G(f)\circ \alpha_X\circ ker$. We observe that 
% %%
% \begin{equation*}
%     ker\circ \text{cr}_1(G)(f)\circ \text{cr}_1(\alpha_X)= G(f)\circ ker \circ \text{cr}_1(\alpha_X) = G(f)\circ \alpha_X\circ ker
% \end{equation*}
% %%
% while
% %%
% \begin{equation*}
%     ker\circ\text{cr}_1(\alpha_Y)\circ \text{cr}_1(F)(f)= \alpha_Y\circ ker \circ\text{cr}_1(F)(f)= \alpha_Y\circ F(f)\circ ker
% \end{equation*}
% %%
% which are equal by naturality of $\alpha$. Thus $\text{cr}_1(\alpha)$ is a natural transformation. Further, by uniqueness of the construction, $\text{cr}_1$ is also functorial.


Inductively, suppose $\text{cr}_{n-1}$ is functorial. Then by Lemma \ref{lem:limFunctor} and the inductive hypothesis, for $\alpha:F\Rightarrow G$, $\text{cr}_n(\alpha)$ is the limit map induced by the map of diagrams
%%
\[\begin{tikzcd}
	{(\lor_{i=1}^2\times 1)\circ \text{cr}_{n-1}(F)} & {((\hat{\pi}_2)^*\circ \text{cr}_{n-1}(F))\oplus((\hat{\pi}_1)^*\circ \text{cr}_{n-1}(F))} \\
	{(\lor_{i=1}^2\times 1)\circ \text{cr}_{n-1}(G)} & {((\hat{\pi}_2)^*\circ \text{cr}_{n-1}(G))\oplus((\hat{\pi}_1)^*\circ \text{cr}_{n-1}(G))}
	\arrow[shift left, from=1-1, to=1-2]
	\arrow[shift right, from=1-1, to=1-2]
	\arrow["{(\lor_{i=1}^2\times 1)\circ\text{cr}_{n-1}(\alpha)}"', from=1-1, to=2-1]
	\arrow[shift left, from=2-1, to=2-2]
	\arrow[shift right, from=2-1, to=2-2]
	\arrow["{((\hat{\pi}_2)^*\circ \text{cr}_{n-1}(\alpha))\oplus((\hat{\pi}_1)^*\circ \text{cr}_{n-1}(\alpha))}", from=1-2, to=2-2]
\end{tikzcd}\]
%%
where $\hat{\pi}_i:\mathcal{B}^n\rightarrow \mathcal{B}^{n-1}$ is the functor which skips the $i$th argument, and $\lor_{i=1}^n$ is the functor given in Lemma \label{lem:lorFunc} below.
%%
\begin{lem}[label=lem:lorFunc]
    We have a functor $\lor_{i=1}^n:\text{Fun}(\mathcal{B},\mathcal{A})\rightarrow \text{Fun}(\mathcal{B}^n,\mathcal{A})$ given by $\lor_{i=1}^n(G)(X_1,...,X_n) = G(\lor_{i=1}^nX_i)$ and $\lor_{i=1}^n(G)(f_1,...,f_n) = G(\lor_{i=1}^nf_i)$ on objects and by $\lor_{i=1}^n(\eta)_{X_1,...,X_n} = \eta_{\lor_{i=1}^nX_i}$ on arrows.
\end{lem}
\begin{proof}
    To prove $\lor_{i=1}^n$ is a functor we first show it is well-defined. Let $\eta:F\Rightarrow G$ be a natural transformation between single variable functors, and let $(f_i)_{i=1}^n:(X_1,...,X_n)\rightarrow (Y_1,...,Y_n)$ and $(g_i)_{i=1}^n:(Y_1,...,Y_n)\rightarrow (Z_1,...,Z_n)$ be maps in $\mathcal{B}^n$. 

    First, by uniqueness in the definition of $\lor_{i=1}^nf_i$ note that $\lor_{i=1}^ng_i\circ \lor_{i=1}^nf_i = \lor_{i=1}^n(g_i\circ f_i)$. Additionally, $\lor_{i=1}^n1_{X_i} = 1_{\lor_{i=1}^nX_i}$. Combined with functoriality of $F$, we have that $\lor_{i=1}^n(F)$ is indeed a functor.

    Next, naturality of $\lor_{i=1}^n(\eta)$ equates to the following diagram commuting
    %%
    \[\begin{tikzcd}
    	{F(\lor_{i=1}^nX_i)} & {F(\lor_{i=1}^nY_i)} \\
    	{G(\lor_{i=1}^nX_i)} & {G(\lor_{i=1}^nY_i)}
    	\arrow["{F(\lor_{i=1}^nf_i)}", from=1-1, to=1-2]
    	\arrow[from=1-1, to=1-2]
    	\arrow["{G(\lor_{i=1}^nf_i)}"', from=2-1, to=2-2]
    	\arrow["{\eta_{\lor_{i=1}^nX_i}}"', from=1-1, to=2-1]
    	\arrow["{\eta_{\lor_{i=1}^nY_i}}", from=1-2, to=2-2]
    \end{tikzcd}\]
    %%
    which follows from the naturality of $\eta$. Finally, if $\gamma:G\Rightarrow H$ is another natural transformation,
    %%
    \begin{equation*}
        \lor_{i=1}^n(\gamma)_{X_1,...,X_n}\circ \lor_{i=1}^n(\eta)_{X_1,...,X_n} = \gamma(\lor_{i=1}^nX_i)\circ \eta(\lor_{i=1}^nX_i) = (\gamma\circ\eta)(\lor_{i=1}^nX_i) = \lor_{i=1}^n(\gamma\circ \eta)_{X_1,...,X_n}
    \end{equation*}
    %%
    by definition of composition of natural transformations, and
    %%
    \begin{equation*}
        \lor_{i=1}^n(1_F)_{X_1,...,X_n} = 1_F(\lor_{i=1}^nX_i) = 1_{F(\lor_{i=1}^nX_i)} = 1_{\lor_{i=1}^n(F)(X_1,...,X_n)}
    \end{equation*}
    %%
    This finishes the proof that $\lor_{i=1}^n$ is a functor.
\end{proof}
%%
Additionally, as in the previous cases, using the uniqueness of the components of $\text{cr}_n(\alpha)$ we can give the formula
%%
\begin{equation}\label{eq:natProjFormulaN}
    \text{cr}_n(\alpha) = r_{G,n}\circ ((\lor_{i=1}^2\times 1)(\text{cr}_{n-1}(\alpha)))\circ s_{F,n}
\end{equation}
%%

% Then with restricted notation we aim to show that the following diagram commutes, where our maps are those unique ones making the inner rectangles commute:

% \[\begin{tikzcd}
% 	{\text{cr}_n(F)} & {\text{cr}_{n-1}(F)(Y_1\lor Y_2)} & {\text{cr}_{n-1}(F)(Y_1)\oplus\text{cr}_{n-1}(F)(Y_2)} \\
% 	{\text{cr}_n(F)} & {\text{cr}_{n-1}(F)(X_1\lor X_2)} & {\text{cr}_{n-1}(F)(X_1)\oplus\text{cr}_{n-1}(F)(X_2)} \\
% 	{\text{cr}_n(G)} & {\text{cr}_{n-1}(G)(X_1\lor X_2)} & {\text{cr}_{n-1}(G)(X_1)\oplus\text{cr}_{n-1}(G)(X_2)} \\
% 	{\text{cr}_n(G)} & {\text{cr}_{n-1}(G)(Y_1\lor Y_2)} & {\text{cr}_{n-1}(G)(Y_1)\oplus\text{cr}_{n-1}(G)(Y_2)}
% 	\arrow["{\text{cr}_{n-1}\alpha_{X_1\lor X_2}}"{description}, from=2-2, to=3-2]
% 	\arrow["{\text{cr}_{n-1}\alpha_{X_1}\oplus \text{cr}_{n-1}\alpha_{X_2}}"{description}, from=2-3, to=3-3]
% 	\arrow[tail, from=2-1, to=2-2]
% 	\arrow[from=2-2, to=2-3]
% 	\arrow[from=3-2, to=3-3]
% 	\arrow[tail, from=3-1, to=3-2]
% 	\arrow["{\text{cr}_{n-1}(F)(f_1)\oplus \text{cr}_{n-1}(F)(f_2)}"{description}, from=2-3, to=1-3]
% 	\arrow[from=1-2, to=1-3]
% 	\arrow[tail, from=1-1, to=1-2]
% 	\arrow[tail, from=4-1, to=4-2]
% 	\arrow[from=4-2, to=4-3]
% 	\arrow["{\text{cr}_{n-1}(G)(f_1)\oplus \text{cr}_{n-1}(G)(f_2)}"{description}, from=3-3, to=4-3]
% 	\arrow["{\text{cr}_{n-1}(F)(f_1\lor f_2)}"{description}, from=2-2, to=1-2]
% 	\arrow[dashed, from=2-1, to=1-1]
% 	\arrow["{\text{cr}_n\alpha}"{description}, dashed, from=2-1, to=3-1]
% 	\arrow[dashed, from=3-1, to=4-1]
% 	\arrow["{\text{cr}_{n-1}(G)(f_1\lor f_2)}"{description}, from=3-2, to=4-2]
% 	\arrow["{\text{cr}_n\alpha}"{description}, curve={height=30pt}, from=1-1, to=4-1]
% 	\arrow["{\text{cr}_{n-1}\alpha_{Y_1\lor Y_2}}"{description}, curve={height=-80pt}, from=1-2, to=4-2]
% 	\arrow["{\text{cr}_{n-1}\alpha_{Y_1}\oplus\text{cr}_{n-1}\alpha_{Y_2}}", curve={height=-120pt}, from=1-3, to=4-3]
% \end{tikzcd}\]

% \noindent In particular, $\text{cr}_n\alpha$ above denotes $\text{cr}_n(\alpha)_{X_1,...,X_n}$, $\text{cr}_{n-1}\alpha_{X_1\lor X_2} := \text{cr}_{n-1}(\alpha)_{X_1\lor X_2,X_3,...,X_n}$, $\text{cr}_{n-1}\alpha_{X_1} := \text{cr}_{n-1}(\alpha)_{X_1,X_3,...,X_n}$, $\text{cr}_{n-1}\alpha_{X_2} := \text{cr}_{n-1}(\alpha)_{X_2,X_3,...,X_n}$, $\text{cr}_n(F)(f_i) := \text{cr}_n(F)(f_1,...,f_n)$, $\text{cr}_{n-1}(F)(f_1\lor f_2) := \text{cr}_{n-1}(F)(f_1\lor f_2,f_3,...,f_n)$, $\text{cr}_{n-1}(F)(f_1) := \text{cr}_{n-1}(F)(f_1,f_3,...,f_n)$, and finally $\text{cr}_{n-1}(F)(f_2) := \text{cr}_{n-1}(F)(f_2,f_3,...,f_n)$. Then we observe that:
%%
% \begin{equation*}
%     ker \circ \text{cr}_n(G)(f_i)\circ \text{cr}_n\alpha = \text{cr}_{n-1}(G)(f_1\lor f_2)\circ ker \circ \text{cr}_n\alpha = \text{cr}_{n-1}(G)(f_1\lor f_2)\circ \text{cr}_{n-1}\alpha_{X_1\lor X_2}\circ ker
% \end{equation*}
% %%
% while
% %%
% \begin{equation*}
%     ker \circ \text{cr}_n\alpha \circ\text{cr}_n(F)(f_i) = \text{cr}_{n-1}\alpha_{Y_1\lor Y_2}\circ ker \circ \text{cr}_n(F)(f_i) =\text{cr}_{n-1}\alpha_{Y_1\lor Y_2}\circ \text{cr}_{n-1}(F)(f_1\lor f_2)\circ ker
% \end{equation*}
% %%
% which are equal by naturally of $\text{cr}_{n-1}\alpha$ in the induction hypothesis. Once again, as the components are defined using the universal property of the kernel, $\text{cr}_n$ is functorial since the composite of two images makes the diagram commute as well, and identities are sent to identities. 
Therefore
%%
\begin{equation}
    \text{cr}_n:\text{Fun}(\mathcal{B},\mathcal{A})\rightarrow \text{Fun}_*(\mathcal{B}^n,\mathcal{A})
\end{equation}
%%
is indeed a functor between functor categories.


\begin{rmk}
    We can define the cross-effect functors for $\mathcal{A}$ non-abelian, requiring simply that $\mathcal{A}$ has pullbacks and equilizers. To do this let $\mathcal{A}$ be such a category, and let $F:\mathcal{B}\rightarrow \mathcal{A}$. We consider the diagram
    %%
    \[\begin{tikzcd}
    	{F(X\lor Y)} & {F(X)} \\
    	{F(Y)} & {F(\star)}
    	\arrow["{F(!)}", from=1-2, to=2-2]
    	\arrow["{F(1_X\lor!)}", from=1-1, to=1-2]
    	\arrow["{F(!\lor1_Y)}"', from=1-1, to=2-1]
    	\arrow["{F(!)}"', from=2-1, to=2-2]
    \end{tikzcd}\]
    %%
    We remove the first vertex and take a homotopy limit:
    %%
    \[\begin{tikzcd}
    	{holim_{P_0(2)}F(\lor)} & {F(X\lor Y)} & {F(X)} \\
    	& {F(Y)} & {F(\star)}
    	\arrow["{F(!)}", from=1-3, to=2-3]
    	\arrow["{F(!)}"', from=2-2, to=2-3]
    	\arrow["{F(1_X\lor!)}", from=1-2, to=1-3]
    	\arrow["{F(!\lor1_Y)}"', from=1-2, to=2-2]
    	\arrow[curve={height=12pt}, from=1-1, to=2-2]
    	\arrow[curve={height=-24pt}, from=1-1, to=1-3]
    	\arrow["\gamma"', dashed, from=1-2, to=1-1]
    \end{tikzcd}\]
    %%
    We define the second cross effect of $F$ to be
    %%
    \begin{equation*}
        \text{cr}_2(F) := \text{hofib}\gamma
    \end{equation*}
    %%
    In the case of $n = 3$ we obtain a cubical diagram:
    %%
    \[\begin{tikzcd}
    	{F(X_1\lor X_2\lor X_3)} && {F(X_2\lor X_3)} \\
    	& {F(X_1\lor X_3)} && {F(X_3)} \\
    	{F(X_1\lor X_2)} && {F(X_2)} \\
    	& {F(X_1)} && {F(\star)}
    	\arrow[from=1-1, to=3-1]
    	\arrow[from=1-1, to=2-2]
    	\arrow[from=1-1, to=1-3]
    	\arrow[from=1-3, to=2-4]
    	\arrow[dashed, from=1-3, to=3-3]
    	\arrow[from=2-2, to=2-4]
    	\arrow[from=3-1, to=4-2]
    	\arrow[dashed, from=3-1, to=3-3]
    	\arrow[from=2-2, to=4-2]
    	\arrow[from=4-2, to=4-4]
    	\arrow[from=2-4, to=4-4]
    	\arrow[from=3-3, to=4-4]
    \end{tikzcd}\]
    %%
    The diagram can be labeled by $\mathcal{P}(\{1,2,3\})$, where the subset of $\{1,2,3\}$ corresponds to the complement of the indices on a particular node. Let $\chi(S)$ for $S \in \mathcal{P}(\{1,2,3\})$ denote the pullback for the subdiagram consisting on nodes labeled by subsets containing $S$. Then we define
    %%
    \begin{equation*}
        \text{cr}_3F(X_1,X_2,X_3) := \text{hofib}\gamma
    \end{equation*}
    %%
    where fiber indicates the pullback along zero.
\end{rmk}


\begin{lem}[label=lem:idempotCr1]
    For any $n \geq 1$ and any $F:\mathcal{B}\rightarrow \mathcal{A}$, $\text{cr}_n(s_{F,1}):\text{cr}_n(\text{cr}_1(F))\to \text{cr}_n(F)$ is an isomorphism.
\end{lem}
\begin{proof}
    Note that $s_{F,1,\star}:\text{cr}_1(F)(\star)\to F(\star)$ is the kernel of $F(!):F(\star)\to F(\star)$, which is the identity. Thus, by the characterization of limits of Functors in Section~\ref{sec:colimFuncs} we have a natural isomorphism $0_1$ with components $0_{1,F}:\text{cr}_1(F)(\star)\to 0$. Next, we also have $s_{\text{cr}_1(F),1}:\text{cr}_1(\text{cr}_1(F))\to \text{cr}_1(F)$ which is the kernel of $\text{cr}_1(F)(!):\text{cr}_1(F)\to \text{cr}_1(F)(\star)$. Composing with the isomorphism $0_{1,F}$ shows that $s_{\text{cr}_1(F),1}:\text{cr}_1(\text{cr}_1(F))\to \text{cr}_1(F)$ is an isomorphism.
    

    \vspace{10pt}

    We now proceed by induction on $n$. Suppose $\text{cr}_n(s_{F,1}):\text{cr}_n(\text{cr}_1(F))\to \text{cr}_n(F)$ for some $n \geq 1$. Then by definition of the cross-effect we have the commutative diagram of a map between equalizers
    \[\begin{tikzcd}
        {\text{cr}_n(\text{cr}_1(F))} & {(\lor_{i=1}^2\times 1)\circ \text{cr}_{n-1}(\text{cr}_1(F))} & {((\hat{\pi}_2)^*\circ \text{cr}_{n-1}(\text{cr}_1(F))\oplus ((\hat{\pi}_1)^*\circ \text{cr}_{n-1}(\text{cr}_1(F)))} \\
        {\text{cr}_n(F)} & {(\lor_{i=1}^2\times 1)\circ \text{cr}_{n-1}(F)} & {((\hat{\pi}_2)^*\circ \text{cr}_{n-1}(F)\oplus ((\hat{\pi}_1)^*\circ \text{cr}_{n-1}(F))}
        \arrow["{\text{cr}_n(s_{F,1})}"', from=1-1, to=2-1]
        \arrow[from=1-1, to=1-2]
        \arrow[shift left, from=1-2, to=1-3]
        \arrow[from=2-1, to=2-2]
        \arrow[shift left, from=2-2, to=2-3]
        \arrow[shift right, from=2-2, to=2-3]
        \arrow[shift right, from=1-2, to=1-3]
        \arrow["{(\lor_{i=1}^2\times 1)\circ \text{cr}_{n-1}(s_{F,1})}"', from=1-2, to=2-2]
        \arrow["{((\hat{\pi}_2)^*\circ \text{cr}_{n-1}(s_{F,1}))\oplus ((\hat{\pi}_1)^*\circ \text{cr}_{n-1}(s_{F,1}))}"', from=1-3, to=2-3]
    \end{tikzcd}\]
    By the inductive hypothesis the middle and right vertical map are isomorphisms. Since each of the rows are exact and the left maps are monomorphisms, being kernel maps, we can add zeros to the left and use the 5-lemma to conclude that $\text{cr}_n(s_{F,1})$ is an isomorphism, as desired.
    % The isomorphism $F(\star) \cong F(\star)\oplus \text{cr}_1(F)(\star)$ implies by Lemma \ref{lem:biprod} that $\text{cr}_1(F)(\star)\cong 0$. Then 
    % %%
    % \begin{equation*}
    %     \text{cr}_1(F)(X) \cong \text{cr}_1(F)(\star)\oplus \text{cr}_1(\text{cr}_1(F))(X) \cong \text{cr}_1^2(F)(X)
    % \end{equation*}
    % %%
    % By induction we have that $\text{cr}_n(\text{cr}_1(F))(X_1,...,X_n) \cong \text{cr}_n(F)(X_1,...,X_n)$, using Lemma \ref{lem:biprod}. Since these are maps which result from the universal property of the biproduct they amalgamate to form natural isomorphisms between the desired functors by Lemma \ref{lem:limFuncIsLim}.
%     Naturality in the case of $n = 1$ follows from the commutivity of the following diagram for any $f:X\rightarrow Y$:
%     \[\begin{tikzcd}
% 	 && {\text{cr}_1(F)(\star)} \\
% 	{\text{cr}_1^2(F)(X)} & {\text{cr}_1(F)(X)} & {F(X)} & {F(\star)} \\
% 	{\text{cr}_1^2(F)(Y)} & {\text{cr}_1(F)(Y)} & {F(Y)} & {F(\star)} \\
% 	&& {\text{cr}_1(F)(\star)}
% 	\arrow["{F(!)}", from=2-3, to=2-4]
% 	\arrow[tail, from=2-2, to=2-3]
% 	\arrow["{F(!)}"', from=3-3, to=3-4]
% 	\arrow[tail, from=3-2, to=3-3]
% 	\arrow["{F(f)}"{description}, from=2-3, to=3-3]
% 	\arrow[Rightarrow, no head, from=2-4, to=3-4]
% 	\arrow[dashed, from=2-2, to=3-2]
% 	\arrow["\cong", tail, from=2-1, to=2-2]
% 	\arrow["{\text{cr}_1(F)(!)}", from=2-2, to=1-3]
% 	\arrow["\cong"', tail, from=3-1, to=3-2]
% 	\arrow["{\text{cr}_1(F)(!)}"', from=3-2, to=4-3]
% 	\arrow[curve={height=-30pt}, Rightarrow, no head, from=1-3, to=4-3]
% 	\arrow[dashed, from=2-1, to=3-1]
% \end{tikzcd}\]
%     which holds by the definition of $\text{cr}_1^2F$ on maps. The case of $n$ proceeds by induction.
\end{proof}

These properties demonstrate that the inclusion $\text{Fun}_*(\mathcal{B},\mathcal{A})\rightarrow \text{Fun}(\mathcal{B},\mathcal{A})$ admits a right adjoint, namely the first cross-effect functor $\text{cr}_1$. In other words, $\text{Fun}_*(\mathcal{B},\mathcal{A})$ is a coreflective subcategory of $\text{Fun}(\mathcal{B},\mathcal{A})$. Since the left adjoint is full and faithful (being the inclusion of a full subcategory), the unit of this adjunction is an isomorphism $\eta_F:F\Rightarrow \text{cr}_1\iota(F)$, which also re-affirms that $\text{cr}_1F\cong \text{cr}_1^2F$. In fact, by our choice of kernels, $\eta_F$ has identities as components.
\begin{proof}
    To demonstrate the adjunction we show the co-universal property where $\epsilon_F = s_{F,1}$ is the monic inclusion $\crn_1(F)(X)\rightarrowtail F(X)$. Since the $s_{F,1}$ are natural in $X$ and $F$ by Lemma \ref{lem:limFuncIsLim} we need only show the co-universal property.

    To show the co-universal property we take a natural transformation $\alpha:\iota F\rightarrow G$, for $F$ a strictly reduced functor. This generates a commutative diagram
    \[\begin{tikzcd}
	{\text{cr}_1G(X)} & {G(X)} & {G(\star)} \\
	{F(X)} & {F(\star)\cong 0}
	\arrow["{(\epsilon_F)_X}", tail, from=1-1, to=1-2]
	\arrow["{G(!)}", from=1-2, to=1-3]
	\arrow["{\alpha_X}"{description}, from=2-1, to=1-2]
	\arrow["{F(!)}"', from=2-1, to=2-2]
	\arrow["{\alpha_\star}"{description}, from=2-2, to=1-3]
	\arrow["{\hat{\alpha}_X}", dashed, from=2-1, to=1-1]
\end{tikzcd}\]
    where $\hat{\alpha}_X$ is the unique map from the universal property of the kernel. Thus by Lemma \ref{lem:limFuncIsLim} $\hat{\alpha}$ is natural.
%     To show that the $\hat{\alpha}$ map is a natural transformation, let $f:X\rightarrow Y$ be a map in $\mathcal{B}$. We have the following diagram
%     \[\begin{tikzcd}
% 	{F(Y)} & {F(\star)\cong 0} \\
% 	{\text{cr}_1G(Y)} & {G(Y)} & {G(\star)} \\
% 	{\text{cr}_1G(X)} & {G(X)} & {G(\star)} \\
% 	{F(X)} & {F(\star)\cong 0}
% 	\arrow["{(\epsilon_G)_X}", tail, from=3-1, to=3-2]
% 	\arrow["{G(!)}", from=3-2, to=3-3]
% 	\arrow["{\alpha_X}"{description}, from=4-1, to=3-2]
% 	\arrow["{F(!)}"', from=4-1, to=4-2]
% 	\arrow["{\alpha_\star}"{description}, from=4-2, to=3-3]
% 	\arrow["{\hat{\alpha}_X}", dashed, from=4-1, to=3-1]
% 	\arrow["{G(f)}", from=3-2, to=2-2]
% 	\arrow[Rightarrow, no head, from=3-3, to=2-3]
% 	\arrow["{G(!)}", from=2-2, to=2-3]
% 	\arrow["{(\epsilon_G)_Y}", tail, from=2-1, to=2-2]
% 	\arrow["{\text{cr}_1G(f)}", dashed, from=3-1, to=2-1]
% 	\arrow["{\hat{\alpha}_Y}"', dashed, from=1-1, to=2-1]
% 	\arrow["{F(f)}", curve={height=-60pt}, from=4-1, to=1-1]
% 	\arrow["{\alpha_\star}"{description}, from=1-2, to=2-3]
% 	\arrow["{F(!)}", from=1-1, to=1-2]
% 	\arrow["{\alpha_Y}"{description}, from=1-1, to=2-2]
% \end{tikzcd}\]
%     But, using naturality of $\alpha$ we observe:
%     %%
%     \begin{equation*}
%         F(f)\hat{\alpha}_Y(\epsilon_G)_Y = F(f)\alpha_Y = \alpha_XG(f) = \hat{\alpha}_X\text{cr}_1G(f)(\epsilon_G)_Y
%     \end{equation*}
%     %%
%     which implies $F(f)\hat{\alpha}_Y = \hat{\alpha}_X\text{cr}_1G(f)$ since $(\epsilon_G)_Y$ is monic. Hence, $\hat{\alpha}$ is natural, and from this diagram we obtain for free that $\epsilon_G$ is natural since $\text{cr}_1G$ is defined on maps in such a way to make the internal square in the diagram always commute.
    If $\beta$ was another natural transformation making the first diagram commute, then $\beta_X = \hat{\alpha}_X$ for all $X$, by uniqueness of the map to the kernel. In other words we would have $\beta = \hat{\alpha}$, so that $\hat{\alpha}$ is unique, proving the co-universal property.
    % Finally, to show $\epsilon$ is natural, let $\alpha:H\rightarrow K$ be a natural transformation between our functors. Then by definition of how $\text{cr}_1$ acts on natural transformations, $\text{cr}_1\alpha\epsilon_K=\epsilon_H\alpha$, as desired.
\end{proof}


The counit then is the natural inclusion $\epsilon_F:\iota\text{cr}_1(F)\Rightarrow F$.  We can extend this to an adjunction for $\text{cr}_n$. First, consider $G$, a reduced functor, and $X_1,...,X_n \in \mathcal{B}_0$. I claim that $\text{cr}_n(G)(X_1,...,X_n)$ is a direct summand of $G(\lor_{i=1}^nX_i)$. Indeed, we have the inclusion $\iota_G$ given by the composite
%%
\begin{equation*}
    \text{cr}_n(G)(X_1,...,X_n)\xrightarrow{\lor_{i=1}^n(s_{G,1})\circ \cdots \circ s_{G,n}}G(\lor_{i=1}^nX_i)
\end{equation*}
%%
and the projection $\pi_G$ given by the composite
%%
\begin{equation*}
    G(\lor_{i=1}^nX_i)\xrightarrow{r_{G,n}\circ \cdots \circ \lor_{i=1}^n(r_{G,1})}\text{cr}_n(G)(X_1,...,X_n)
\end{equation*}
%%
In particular, from our previous work $\iota_G$ and $\pi_G$ are composites of natural transformations, and hence themselves are natural. In particular, we have the natural transformations
\begin{lem}[label=lem:lorInj]
    The components $\iota_G$ defined above constitute a natural transformation $\iota:\text{cr}_n\Rightarrow \lor_{i=1}^n$.
\end{lem}
\noindent and 
\begin{lem}[label=lem:lorProj]
    The components $\pi_G$ defined above constitute a natural transformation $\pi:\lor_{i=1}^n\Rightarrow \text{cr}_n$.
\end{lem}
Additionally, for $F \in {\text{Fun}_*(\mathcal{B}^n,\mathcal{A})}$ we will let $i$ denote the composite 
%%
\begin{equation*}
    F(X_1,...,X_n)\xrightarrow{F(i_1,...,i_n)}\Delta^*(F)(\lor_{i=1}^nX_i)\xrightarrow{\pi_{\Delta^*(F)}}\text{cr}_n(\Delta^*(F))(X_1,...,X_n)
\end{equation*}
%%
The map $F(i_1,...,i_n)$ is natural in both $F$ and the $X_i$, so that $i$ is also a natural transformation $1_{\text{Fun}_*(\mathcal{B}^n,\mathcal{A})}\Rightarrow \text{cr}_n\circ \Delta^*$.
%%
\begin{lem}[label=lem:compIncNat]
    We have a natural transformation $\overline{i}:1_{\text{Fun}(\mathcal{B}^n,\mathcal{A})}\Rightarrow \lor_{i=1}^n\circ \Delta^*$ which restricts to a natural transformations between functors on $\text{Fun}_*(\mathcal{B}^n,\mathcal{A})$
\end{lem}
%%
\begin{proof}
    Let $\alpha:F\rightarrow G$ be a map of functors and let $f_1:X_1\rightarrow Y_1,...,f_n:X_n\rightarrow Y_n$ be a collection of maps in $\mathcal{B}$. Naturality of $\overline{i}_F$ in the $X_i$ is given by the commutative diagram
    %%
    \[\begin{tikzcd}
    	{F(X_1,...,X_n)} & {\Delta^*(F)(\lor_{i=1}^nX_i)} \\
    	{F(Y_1,...,Y_n)} & {\Delta^*(F)(\lor_{i=1}^nY_i)}
    	\arrow["{F(i_1,...,i_n)}", from=1-1, to=1-2]
    	\arrow["{\Delta^*(F)(\lor_{i=1}^nf_i)}", from=1-2, to=2-2]
    	\arrow["{F(f_1,...,f_n)}"', from=1-1, to=2-1]
    	\arrow["{F(i_1,...,i_n)}"', from=2-1, to=2-2]
    \end{tikzcd}\]
    %%
    which commutes by definition of $\lor_{i=1}^nf_i$ and the functoriality of $F$. On the other hand, naturality of $\overline{i}$ itself is given by the diagram
    \[\begin{tikzcd}
    	{F(X_1,...,X_n)} & {\Delta^*(F)(\lor_{i=1}^nX_i)} \\
    	{G(X_1,...,X_n)} & {\Delta^*(G)(\lor_{i=1}^nX_i)}
    	\arrow["{F(i_1,...,i_n)}", from=1-1, to=1-2]
    	\arrow["{\Delta^*(\alpha)_{\lor_{i=1}^nX_i}}", from=1-2, to=2-2]
    	\arrow["{\alpha_{X_1,...,X_n}}"', from=1-1, to=2-1]
    	\arrow["{G(i_1,...,i_n)}"', from=2-1, to=2-2]
    \end{tikzcd}\]
    which commutes by naturality of $\alpha$.
\end{proof}
%%
Another important natural transformation we require is given by the $+$ operation on disjoint unions which gives the unique map sending a disjoint union of a single object to itself with all inclusions the identity.
%%
\begin{lem}[label=lem:plusNat]
    We have a natural transformation $+:\lor_{i=1}^n\Rightarrow 1_{\mathcal{B}}$.
\end{lem}
\begin{proof}
    Let $f:X\rightarrow Y$ be a map in $\mathcal{B}$. Then naturality equates to the commutivity of
    %%
    \[\begin{tikzcd}
    	{\lor_{i=1}^nX} & {\lor_{i=1}^nY} \\
    	X & Y
    	\arrow["{\lor_{i=1}^nf}", from=1-1, to=1-2]
    	\arrow["{+}"', from=1-1, to=2-1]
    	\arrow["{+}", from=1-2, to=2-2]
    	\arrow["f"', from=2-1, to=2-2]
    \end{tikzcd}\]
    %%
    However, the lower composite is precisely the unique map for each inclusion into $Y$ given by $f$, while the upper composite has as inclusions the composite $X\xrightarrow{f}Y\rightarrowtail \lor_{i=1}^nY\xrightarrow{+}Y$ which by definition also equals $f$. Thus the diagram commutes by uniqueness of the map out of a coproduct.
\end{proof}

% \begin{lem}[label=lem:lorInj]
%     The components $\iota_G$ defined above constitute a natural transformation $\iota:\text{cr}_n\Rightarrow \lor_{i=1}^n$.
% \end{lem}
% \begin{proof}
%     Let $\alpha:F\Rightarrow G$ be a map of one-variable functors $\mathcal{B}\rightarrow \mathcal{A}$. Then naturality of $\iota$ is equivalent to showing that for any $X_1,...,X_n \in \mathcal{B}_0$, the following diagram commutes
%     %%
%     \[\begin{tikzcd}
%     	{\text{cr}_n(F)(X_1,...,X_n)} & {\text{cr}_n(G)(X_1,...,X_n)} \\
%     	{F(\lor_{i=1}^nX_i)} & {G(\lor_{i=1}^nX_i)}
%     	\arrow["{\text{cr}_n(\alpha)_{X_1,...,X_n}}", from=1-1, to=1-2]
%     	\arrow["{\alpha_{\lor_{i=1}^nX_i}}"', from=2-1, to=2-2]
%     	\arrow["{\iota_G}", from=1-2, to=2-2]
%     	\arrow["{\iota_F}"', from=1-1, to=2-1]
%     \end{tikzcd}\]
%     %%

%     We proceed inductively. If $n = 1$, this diagram commutes by construction of $\text{cr}_1(\alpha)$, so suppose the claim holds for some $n$. Then the following diagram commutes by the inductive hypothesis for the right square, and construction of $\text{cr}_{n+1}(\alpha)$ for the left
%     %%
%         \[\begin{tikzcd}
%     	{\text{cr}_{n+1}(F)(X_1,...,X_{n+1})} & {\text{cr}_n(F)(X_1\lor X_2,...,X_{n+1})} & {F(\lor_{i=1}^{n+1}X_i)} \\
%     	{\text{cr}_{n+1}(G)(X_1,...,X_{n+1})} & {\text{cr}_n(G)(X_1\lor X_2,...,X_{n+1})} & {G(\lor_{i=1}^{n+1}X_i)}
%     	\arrow["{\text{cr}_n(\alpha)_{X_1\lor X_2,...,X_{n+1}}}", from=1-2, to=2-2]
%     	\arrow["{\alpha_{\lor_{i=1}^{n+1}X_i}}", from=1-3, to=2-3]
%     	\arrow["{\iota_G}"', from=2-2, to=2-3]
%     	\arrow["{\iota_F}", from=1-2, to=1-3]
%     	\arrow[tail, from=2-1, to=2-2]
%     	\arrow[tail, from=1-1, to=1-2]
%     	\arrow["{\text{cr}_{n+1}(\alpha)_{X_1,...,X_{n+1}}}"', from=1-1, to=2-1]
%     \end{tikzcd}\]
%     %%
%     However, the horizontal maps compose to equal $\iota_F$ and $\iota_G$, respectively, so this is precisely the naturality of $\iota$.

%     A similar inductive argument shows that $\iota_F$ is a natural transformation for each $F$.
% \end{proof}


Next we argue that the isomorphisms in Definition \ref{defn:crossEffect} can be upgraded to natural isomorphisms.

\begin{lem}[label=lem:implNatIso]
    We have natural isomorphisms
    %%
    \begin{equation}\label{eq:nat1}
        1_{\text{Fun}(\mathcal{B},\mathcal{A})} \cong \text{ev}_\star \oplus \text{cr}_1
    \end{equation}
    %%
    \begin{equation}\label{eq:nat2}
        \lor_{i=1}^2\circ\text{cr}_1\cong ((\pi_1)^*\circ\text{cr}_1)\oplus ((\pi_2)^*\circ \text{cr}_1)\oplus \text{cr}_2
    \end{equation}
    %%
    and in general
    %%
    \begin{equation}\label{eq:natN}
        (\lor_{i=1}^2\times1_{\mathcal{B}^{n-2}})\circ \text{cr}_{n-1} \cong ((\hat{\pi}_2)^*\circ\text{cr}_{n-1})\oplus ((\hat{\pi}_1)^*\circ \text{cr}_{n-1})\oplus \text{cr}_n
    \end{equation}
\end{lem}

Indeed, these natural isomorphisms follow from the fact that the inclusion and retractions formed natural transformations, and hence the sequence of functors defining the cross-effect split.


% \noindent We begin by constructing the natural isomorphism for Equation \eqref{eq:nat1}.

% \begin{proof}[Proof of Equation \eqref{eq:nat1}]
%     For the first natural isomorphisms, the components are given by $\langle G(\text{!`})|ker_{G(-)}\rangle$, which themselves are natural isomorphisms with components $\langle G(\text{!`})|ker_{G(X)}\rangle$ for each $X \in \mathcal{B}_0$. To show naturality, let $f : X \rightarrow Y$ be a map in $\mathcal{B}$. Then naturality is equivalent to the commutivity of the square
%     %%
%     \[\begin{tikzcd}
%     	{G(\star)\oplus\text{cr}_1(G)(X)} & {G(\star)\oplus\text{cr}_1(G)(Y)} \\
%     	{G(X)} & {G(Y)}
%     	\arrow["{G(f)}"', from=2-1, to=2-2]
%     	\arrow["{\langle G(\text{!`})|ker_{G(Y)}\rangle}"', tail reversed, no head, from=2-2, to=1-2]
%     	\arrow["{\langle G(\text{!`})|ker_{G(X)}\rangle}", tail reversed, no head, from=2-1, to=1-1]
%     	\arrow["{1_{G(\star)}\oplus \text{cr}_1(G)(f)}", from=1-1, to=1-2]
%     \end{tikzcd}\]
%     %%
%     Showing this diagram commutes is equivalent to show that it commutes after including $G(\star)$ and $\text{cr}_1(G)(X)$ into the upper left node, respectively. However, commutivity for the inclusion of $G(\star)$ is just functoriality of $G$, while commutivity for the inclusion of $\text{cr}_1(G)(X)$ follows from the definition of $\text{cr}_1(G)(f)$. 

%     Next, to show naturality of the full transformation let $\eta:G\Rightarrow F$ be natural a natural transformation in $\text{Fun}(\mathcal{B},\mathcal{A})$. Then for each $X \in \mathcal{B}_0$ we must show commutivity of the diagram
%     %%
%     \[\begin{tikzcd}
%     	{G(\star)\oplus\text{cr}_1(G)(X)} & {F(\star)\oplus\text{cr}_1(F)(X)} \\
%     	{G(X)} & {F(X)}
%     	\arrow["{\eta_X}"', from=2-1, to=2-2]
%     	\arrow["{\langle F(\text{!`})|ker_{F(X)}\rangle}"', tail reversed, no head, from=2-2, to=1-2]
%     	\arrow["{\langle G(\text{!`})|ker_{G(X)}\rangle}", tail reversed, no head, from=2-1, to=1-1]
%     	\arrow["{\eta_\star\oplus\text{cr}_1(\eta)_X}", from=1-1, to=1-2]
%     \end{tikzcd}\]
%     %%
%     We can once again proceed using inclusions. The inclusion for $G(\star)$ provides commutivity through the naturality of $\eta$, while the inclusion for $\text{cr}_1(G)$ provides commutivity through the definition of $\text{cr}_1(\eta)$ and its components. This completes the proof.
% \end{proof}


% We now prove the general case in Equation \eqref{eq:natN} by induction using the base case in the preceding proof.

% \begin{proof}[Proof of Equation \eqref{eq:natN}]
%     Inductively suppose we have natural isomorphisms with components given by $$\langle \text{cr}_{n-1}(F)(\iota_{X_1},1_{X_3},...,1_{X_n})|\text{cr}_{n-1}(F)(\iota_{X_2},1_{X_3},...,1_{X_n})|ker_{X_1,...,X_n}\rangle$$ Inductively we consider natural isomorphisms with $$\langle \text{cr}_{n}(F)(\iota_{X_1},1_{X_3},...,1_{X_{n+1}})|\text{cr}_n(F)(\iota_{X_2},1_{X_3},...,1_{X_{n+1}})|ker_{X_1,...,X_{n+1}}\rangle$$ As in the base case, we must show that this definition is natural in the $X_i$ and in $F$. To show naturality in the $X_i$ let $(f_1,...,f_{n+1}):(X_1,...,X_{n+1})\rightarrow (Y_1,...,Y_{n+1})$ be a map in $\mathcal{B}$ and fix $F:\mathcal{B}\rightarrow \mathcal{A}$. With reduced notation, we wish to show commutivity of the following diagram
%     %%
%     \[\begin{tikzcd}
%     	{\text{cr}_n(F)(\widehat{X_2})\oplus\text{cr}_n(F)(\widehat{X_1})\oplus\text{cr}_{n+1}(F)(X)} & {\text{cr}_n(F)(\widehat{Y_2})\oplus\text{cr}_n(F)(\widehat{Y_1})\oplus\text{cr}_{n+1}(F)(Y)} \\
%     	{\text{cr}_n(F)(X_1\lor X_2)} & {\text{cr}_n(F)(Y_1\lor Y_2)}
%     	\arrow["{\text{cr}_n(F)(\widehat{f_2})\oplus\text{cr}_n(F)(\widehat{f_1})\oplus\text{cr}_{n+1}(F)(f)}", shift left=3, from=1-1, to=1-2]
%     	\arrow["{\text{cr}_n(F)(f_1\lor f_2,f_3,...,f_{n+1})}"', from=2-1, to=2-2]
%     	\arrow["{\langle \iota_2,\iota_1,ker_{F,X}\rangle}"', from=1-1, to=2-1]
%     	\arrow["{\langle \iota_2,\iota_1,ker_{F,Y}\rangle}", from=1-2, to=2-2]
%     \end{tikzcd}\]
%     Commutivity upon including $\text{cr}_n(F)(\widehat{X_i})$ follows from functoriality of $\text{cr}_n(F)$, while commutivity upon including $\text{cr}_{n+1}(F)(X_1,...,X_{n+1})$ follows from the definition of $\text{cr}_{n+1}(F)(f_1,...,f_{n+1})$.


%     It remains to show naturality in $F$, so let $\eta : F\rightarrow G$ by a natural transformations of functors. We aim to show commutivity of 
%     %%
%     \[\begin{tikzcd}
%     	{\text{cr}_n(F)(\widehat{X_2})\oplus\text{cr}_n(F)(\widehat{X_1})\oplus\text{cr}_{n+1}(F)(X)} & {\text{cr}_n(G)(\widehat{X_2})\oplus\text{cr}_n(G)(\widehat{X_1})\oplus\text{cr}_{n+1}(G)(X)} \\
%     	{\text{cr}_n(F)(X_1\lor X_2)} & {\text{cr}_n(G)(X_1\lor X_2)}
%     	\arrow["{\text{cr}_n(\eta)_{\widehat{X_2}}\oplus\text{cr}_n(\eta)_{\widehat{X_1}}\oplus\text{cr}_{n+1}(\eta)_{X}}", shift left, from=1-1, to=1-2]
%     	\arrow["{\text{cr}_n(\eta)_{X_1\lor X_2}}"', from=2-1, to=2-2]
%     	\arrow["{\langle \iota_2,\iota_1,ker_{F,X}\rangle}"', from=1-1, to=2-1]
%     	\arrow["{\langle \iota_2,\iota_1,ker_{G,X}\rangle}", from=1-2, to=2-2]
%     \end{tikzcd}\]
%     %%
%     Commutivity upon including $\text{cr}_n(F)(\widehat{X_i})$ follows from naturality of $\text{cr}_n(\eta)$, while commutivity upon including $\text{cr}_{n+1}(F)(X_1,...,X_{n+1})$ follows from the definition of $\text{cr}_{n+1}(\eta)$. This completes the proof.
% \end{proof}

We can apply these natural isomorphisms inductively to obtain the following isomorphism of functors given in \cite{Johnson2003DerivingCW}.

\begin{thm}[label=thm:dirSumDecomp]
    For any $n \in \N$, we have a natural isomorphism
    %%
    \begin{equation*}
        \lor_{i=1}^n \cong \text{ev}_\star \oplus \left(\bigoplus_{m=1}^n\left(\bigoplus_{j_1<\cdots < j_m =: \overline{j}}\pi_{\overline{j}}^*\circ \text{cr}_m\right)\right)
    \end{equation*}
    %%
    where $\pi_{\overline{j}}:\mathcal{B}^n\rightarrow \mathcal{B}^m$ projects onto the components $j_1 < \cdots < j_m=:\overline{j}$.
\end{thm}
\begin{proof}
    If $n = 1$ this isomorphism is precisely Equation \eqref{eq:nat1}. Inductively suppose this isomorphism exists for some $n-1$, $n \geq 2$. Let $\varphi_{n-1}$ denote this isomorphism. Recall the functor $(\lor_{i=1}^2\times 1_{\mathcal{B}^{n-2}}):\text{Fun}(\mathcal{B}^{n-1},\mathcal{A})\rightarrow \text{Fun}(\mathcal{B}^n,\mathcal{A})$, and observe that $\lor_{i=1}^n \cong (\lor_{i=1}^2\times 1_{\mathcal{B}^{n-1}})\circ \lor_{i=1}^{n-1}$ by the universal property of the coproduct. Then applying $\varphi_{n-1}$, the isomorphisms in Lemma \ref{lem:implNatIso}, as well as isomorphisms associated with re-ordering the direct sum we obtain the desired natural isomorphism.
\end{proof}

To help with future computations we describe the composite $\iota_G\circ \pi_G$ for a functor $G$ a bit more explicitly. For this remark we emphasize the order of the projection and inclusion by writing $\iota_{G,n}$ and $\pi_{G,n}$.

\begin{rmk}{Projection Formulas}
    We construct a formula for the composite $\iota_{G,n}\circ \pi_{G,n}$ by induction on $n$. In the case of $n = 1$ $\iota_{G,1} = s_{G,1}$ and $\pi_{G,1} = r_{G,1}$, so by construction of the retraction
    %%
    \begin{equation}\label{eq:endoForm1}
        \iota_{G,1}\circ \pi_{G,1} = 1-G(\hat{!})
    \end{equation}
    %%
    In the case of $n = 2$, $\iota_{G,2} = \lor_{i=1}^2(s_{G,1})\circ s_{G,2}$ and $\pi_{G,2} = r_{G,2}\circ \lor_{i=1}^2(r_{G,1})$. Then using Equation \ref{eq:projFormula1} the composite is given by
    %%
    \begin{align*}
        \iota_{G,2}\circ \pi_{G,2} &= \lor_{i=1}^2(s_{G,1})\circ s_{G,2}\circ r_{G,2}\circ \lor_{i=1}^2(r_{G,1}) \\
        &= \lor_{i=1}^2(s_{G,1})\circ (1-(\text{cr}_1(G)(1\lor\hat{!})+\text{cr}_1(G)(\hat{!}\lor 1))) \circ \lor_{i=1}^2(r_{G,1}) \\
        &= \lor_{i=1}^2(1-G(\hat{!}))-\lor_{i=1}^2(1-G(\hat{!}))\circ G(1\lor\hat{!})\circ \lor_{i=1}^2(1-G(\hat{!}))\\
        &-\lor_{i=1}^2(1-G(\hat{!}))\circ G(\hat{!}\lor 1)\circ \lor_{i=1}^2(1-G(\hat{!})) \\
        &= \lor_{i=1}^2(1-G(\hat{!}))-(G(1\lor\hat{!})-G(\hat{!}\lor \hat{!}))\circ \lor_{i=1}^2(1-G(\hat{!}))\\
        &-(G(\hat{!}\lor 1)-G(\hat{!}\lor \hat{!}))\circ \lor_{i=1}^2(1-G(\hat{!})) \\
        &= \lor_{i=1}^2(1-G(\hat{!}))-(G(1\lor\hat{!})-G(\hat{!}\lor \hat{!}))-(G(\hat{!}\lor 1)-G(\hat{!}\lor \hat{!})) \\
        &= 1-G(1\lor \hat{!})-G(\hat{!}\lor 1)+G(\hat{!}\lor \hat{!})
    \end{align*}
    Therefore, our formula for $n = 2$ becomes
    %%
    \begin{equation}\label{eq:endoForm2}
        \iota_{G,2}\circ \pi_{G,2} = 1-G(1\lor \hat{!})-G(\hat{!}\lor 1)+G(\hat{!}\lor \hat{!})
    \end{equation}
    %%
    Now, suppose that for some $n \geq 2$ we have the formula
    %%
    \begin{equation*}
        \iota_{G,n}\circ \pi_{G,n} = 1 + \sum_{\overline{i}} k_{\overline{i}}G(\lor_{\overline{i}}\hat{!}) %%\sum_{k=1}^n\sum_{1 \leq i_1 < \cdots < i_k\leq n=:\overline{i}}(-1)^kG(\lor_{\overline{i}} \hat{!}) - general formula for later if useful
    \end{equation*}
    %%
    where the sum is over sequences of distinct integers from $1$ to $n$ of lengths $\geq 1$, the $k_{\overline{i}}$ are integers, and where $\lor_{\overline{i}}\hat{!}$ has $\hat{!}$ in each entry $i_j$, and identities in all other entries. Then our formula for $\iota_{G,n+1}$ for the $n+1$ case can be written as $(\lor_{i=1}^2\times 1_{\mathcal{B}^{n-2}}) (\iota_{G,n})\circ s_{G,n+1}$, while $\pi_{G,n+1} = r_{G,n+1}\circ (\lor_{i=1}^2\times 1_{\mathcal{B}^{n-2}})(\pi_{G,n})$. Note by Equation \ref{eq:projFormulaN} applied inductively, $\text{cr}_n(G)(f_1,...,f_n) = \pi_{G,n}\circ G(\lor_{i=1}^nf_i)\circ \iota_{G,n}$. Then by our inductive hypothesis we can compute
    %%
    \begin{align*}
        \iota_{G,n+1}\circ \pi_{G,n+1} &= (\lor_{i=1}^2\times 1_{\mathcal{B}^{n-2}}) (\iota_{G,n})\circ s_{G,n+1}\circ r_{G,n+1}\circ (\lor_{i=1}^2\times 1_{\mathcal{B}^{n-2}})(\pi_{G,n}) \\
        &=(\lor_{i=1}^2\times 1_{\mathcal{B}^{n-2}}) (\iota_{G,n})\circ (1-(\text{cr}_{n+1}(G)(1\lor \hat{!},1)+\text{cr}_{n+1}(G)(\hat{!}\lor 1,1)))\\
        &\circ (\lor_{i=1}^2\times 1_{\mathcal{B}^{n-2}})(\pi_{G,n}) \\
        &= \left(\lor_{i=1}^2\times 1_{\mathcal{B}^{n-2}}\right)\left(1 + \sum_{\overline{i}} k_{\overline{i}}G(\lor_{\overline{i}}\hat{!})\right) \\
        &-(\lor_{i=1}^2\times 1_{\mathcal{B}^{n-2}}) (\iota_{G,n}\circ \pi_{G,n})\circ G(1\lor \hat{!}\lor 1_{n-1})\circ (\lor_{i=1}^2\times 1_{\mathcal{B}^{n-2}}) (\iota_{G,n}\circ \pi_{G,n}) \\
        &-(\lor_{i=1}^2\times 1_{\mathcal{B}^{n-2}}) (\iota_{G,n}\circ \pi_{G,n})\circ G(\hat{!}\lor 1\lor 1_{n-1})\circ (\lor_{i=1}^2\times 1_{\mathcal{B}^{n-2}}) (\iota_{G,n}\circ \pi_{G,n}) \\
        &= \left(1 + \sum_{\overline{i}} k_{\overline{i}}G(\lor_{\overline{i}'}\hat{!})\right) \\
        &-\left(1 + \sum_{\overline{i}} k_{\overline{i}}G(\lor_{\overline{i}'}\hat{!})\right) \circ G(1\lor \hat{!}\lor 1_{n-1})\circ \left(1 + \sum_{\overline{i}} k_{\overline{i}}G(\lor_{\overline{i}'}\hat{!})\right)  \\
        &-\left(1 + \sum_{\overline{i}} k_{\overline{i}}G(\lor_{\overline{i}'}\hat{!})\right) \circ G(\hat{!}\lor 1\lor 1_{n-1})\circ \left(1 + \sum_{\overline{i}} k_{\overline{i}}G(\lor_{\overline{i}'}\hat{!})\right)  \\
        % &= (\lor_{i=1}^2\times 1_{\mathcal{B}^{n-2}})\left(1 +\sum_{k=1}^n\sum_{1 \leq i_1 < \cdots < i_k\leq n=:\overline{i}}(-1)^kG(\lor_{\overline{i}} \hat{!})\right)\\
        % &-(\lor_{i=1}^2\times 1_{\mathcal{B}^{n-2}}) (\iota_{G,n}\circ \pi_{G,n})\circ G(1\lor \hat{!}\lor 1_{n-1})\circ (\lor_{i=1}^2\times 1_{\mathcal{B}^{n-2}}) (\iota_{G,n}\circ \pi_{G,n}) \\
        % &-(\lor_{i=1}^2\times 1_{\mathcal{B}^{n-2}}) (\iota_{G,n}\circ \pi_{G,n})\circ G(\hat{!}\lor 1\lor 1_{n-1})\circ (\lor_{i=1}^2\times 1_{\mathcal{B}^{n-2}}) (\iota_{G,n}\circ \pi_{G,n}) \\
        % &= 1 +\sum_{k=1}^n\left[\sum_{1 < i_1 < \cdots < i_k\leq n=:\overline{i}}(-1)^kG(\lor_{\overline{i}+1} \hat{!})+\sum_{1 = i_1 < \cdots < i_k\leq n=:\overline{i}}(-1)^kG(\lor_{(1<2<\overline{i}_{n-1}+1)} \hat{!})\right]\\
        % &-\left(G(1\lor \hat{!}\lor 1_{n-1})+\sum_{k=1}^n\left[\sum_{1 < i_1 < \cdots < i_k\leq n=:\overline{i}}(-1)^kG(\lor_{(2,\overline{i}+1)} \hat{!})\right.\right.\\
        % &\left.\left.+\sum_{1 = i_1 < \cdots < i_k\leq n=:\overline{i}}(-1)^kG(\lor_{(1<2<\overline{i}_{n-1}+1)} \hat{!})\right]\right)\\
        % &\circ (\lor_{i=1}^2\times 1_{\mathcal{B}^{n-2}}) (\iota_{G,n}\circ \pi_{G,n}) \\
        % &-\left(G(\hat{!}\lor 1\lor 1_{n-1})+\sum_{k=1}^n\left[\sum_{1 < i_1 < \cdots < i_k\leq n=:\overline{i}}(-1)^kG(\lor_{(1,\overline{i}+1)} \hat{!})\right.\right. \\
        % &\left.\left.+\sum_{1 = i_1 < \cdots < i_k\leq n=:\overline{i}}(-1)^kG(\lor_{(1<2<\overline{i}_{n-1}+1)} \hat{!})\right]\right)\\
        % &\circ (\lor_{i=1}^2\times 1_{\mathcal{B}^{n-2}}) (\iota_{G,n}\circ \pi_{G,n}) \\
    \end{align*}
    %%
    where $\overline{i}'$ is obtained from $\overline{i}$ by either shifting up all degrees by $1$ if $1 \notin \overline{i}$, or by shifting up all degrees by $1$ and adding $1$ to $\overline{i}$ if it does contain $1$. Observe that the second composites will consist of integer combinations of maps of the form $G(\lor_{\overline{i}}\hat{!})$ where $\overline{i}$ is non-empty. Thus, $\iota_{G,n+1}\circ \pi_{G,n+1}$ is of the desired form.
    % the projections $G(\lor_{i=1}^nX_i) \xrightarrow{\pi_G} \text{cr}_n(G)(X_1,...,X_n)$ explicitly through an inductive approach, as this will prove valuable for future computations. For $n = 1$ we observe that we have the diagram
    % %%
    % \[\begin{tikzcd}
    % 	{\text{cr}_1(G)(X)} & {G(X)} & {G(\star)} \\
    % 	{G(X)}
    % 	\arrow[tail, from=1-1, to=1-2]
    % 	\arrow["{G(!)}", from=1-2, to=1-3]
    % 	\arrow["{1_{G(X)}-G(\hat{!})}"', from=2-1, to=1-2]
    % 	\arrow[dashed, from=2-1, to=1-1]
    % \end{tikzcd}\]
    % %%
    % with the unique map on the left given by the universal property of the kernel defining $\text{cr}_1(G)(X)$ (Note: this is the induced left-splitting for any abelian category, so corresponds to the projection). 
    % % Observe that naturality of our isomorphism in Equation \ref{eq:nat1} gives the diagram
    % % \[\begin{tikzcd}
    % % 	{G(X)} & {G(X)} \\
    % % 	{\text{cr}_1(G)(X)\oplus G(\star)} & {\text{cr}_1(G)(X)\oplus G(\star)}
    % % 	\arrow["{1_{G(X)}-G(\hat{!})}", from=1-1, to=1-2]
    % % 	\arrow["{1_{\text{cr}_1(G)(X)}\oplus 1_{G(\star)}-\text{cr}_1(G)(\hat{!})\oplus 1_{G(\star)}}"', outer sep=6pt, from=2-1, to=2-2]
    % % 	\arrow["{\langle ker_{G(X)}|G(\text{!`})\rangle}", from=2-1, to=1-1]
    % % 	\arrow["{\langle ker_{G(X)}|G(\text{!`})\rangle}"', from=2-2, to=1-2]
    % % \end{tikzcd}\]
    % % %%
    % % where the bottom map equals $1_{\text{cr}_1(G)}\oplus 0$ since $\text{cr}_1(G)$ is reduced. This implies that the map induced by $1_{G(X)}-G(\hat{!})$ is in fact the projection, $\pi_G$ (so far this doesn't require $G$ to be reduced).

    % \vspace{10pt}

    % In the case of $n = 2$ we obtain the composite
    % \[\begin{tikzcd}
    % 	{\text{cr}_2(G)(X_1,X_2)} & {\text{cr}_1(G)(X_1\lor X_2)} & {\text{cr}_1(G)(X_1)\oplus \text{cr}_1(G)(X_2)} \\
    % 	{\text{cr}_1(G)(X_1\lor X_2)} \\
    % 	{G(X_1\lor X_2)}
    % 	\arrow[tail, from=1-1, to=1-2]
    % 	\arrow["{\langle\text{cr}_1(G)(\langle 1_{X_1}|\hat{!}\rangle),\text{cr}_1(G)(\langle\hat{!}|1_{X_2}\rangle)\rangle=p_1}"', curve={height=12pt}, from=1-2, to=1-3]
    % 	\arrow["{1-\iota_1\circ p_1}"', from=2-1, to=1-2]
    % 	\arrow["{\langle\text{cr}_1(G)(\iota_{X_1})|\text{cr}_1(G)(\iota_{X_2})\rangle=\iota_1}"', curve={height=12pt}, from=1-3, to=1-2]
    % 	\arrow[dashed, from=2-1, to=1-1]
    % 	\arrow["{\pi_{G,1}}", from=3-1, to=2-1]
    % 	\arrow["{\pi_{G,2}}", curve={height=-30pt}, from=3-1, to=1-1]
    % \end{tikzcd}\]
    % via a similar construction. It follows that \textbf{NEED TO FIGURE OUT NOTATION HERE}
    % %%
    % \begin{align*}
    %     \iota_{G,2}\circ \pi_{G,2} &= (\iota_{G,1}\circ i_2)\circ (p_2\circ \pi_{G,2}) \\
    %     &= \iota_{G,1}\circ (1-\iota_1\circ p_1)\circ \pi_{G,2} \\
    %     &= 1-G(\hat{!})-\iota_{G,1}\iota_1p_1\pi_{G,2} \\
    %     &= 1-G(\hat{!})-G(1_{X_1}\lor\hat{!})-G(\hat{!}\lor 1_{X_2}) 
    % \end{align*}
    % where the last equality is due to Lemma \textbf{REF} (to be typed up shortly)
    % % can consider the diagram
    % % %%
    % % \[\begin{tikzcd}
    % % 	{G(X_1\lor X_2)} && {G(X_1\lor X_2)} & {G(X_1)\oplus G(X_2)} \\
    % % 	{\text{cr}_2(G)(X_1,X_2)} && {\text{cr}_1(G)(X_1\lor X_2)} & {\text{cr}_1(G)(X_1)\oplus \text{cr}_1(G)(X_2)}
    % % 	\arrow["{\langle G(\langle 1_{X_1}|\hat{!}\rangle),G(\langle \hat{!}|1_{X_2}\rangle)\rangle}", from=1-3, to=1-4]
    % % 	\arrow["{1_{G(X_1\lor X_2)}-G(1_{X_1}\lor \hat{!})-G(\hat{!}\lor 1_{X_2})}", from=1-1, to=1-3]
    % % 	\arrow["{\langle\text{cr}_1(G)(\langle 1_{X_1}|\hat{!}\rangle),\text{cr}_1(G)(\langle\hat{!}|1_{X_2}\rangle)\rangle}"', from=2-3, to=2-4]
    % % 	\arrow[tail, from=2-1, to=2-3]
    % % 	\arrow["{\pi_G}", curve={height=-6pt}, two heads, from=1-3, to=2-3]
    % % 	\arrow["{\iota_G}", curve={height=-6pt}, from=2-3, to=1-3]
    % % 	\arrow[dashed, from=1-1, to=2-1]
    % % 	\arrow["{\pi_G\oplus\pi_G}", curve={height=-6pt}, two heads, from=1-4, to=2-4]
    % % 	\arrow["{\iota_G\oplus \iota_G}", curve={height=-6pt}, from=2-4, to=1-4]
    % % \end{tikzcd}\]
    % % where the map induced on the left is due to the universal property of the kernel. To show such a map is induced we require that $\langle\text{cr}_1(G)(\langle 1_{X_1}|\hat{!}\rangle),\text{cr}_1(G)(\langle\hat{!}|1_{X_2}\rangle)\rangle\circ \pi_G\circ (1_{G(X_1\lor X_2)}-G(1_{X_1}\lor \hat{!})-G(\hat{!}\lor 1_{X_2})$ is the zero map. Because $\iota_G$ is monic, this is equivalent to showing that the post-composition with $\iota_G$ is zero. Using the commutivity of the right square, by naturality of the $\iota_G$, this is equivalent to showing $\langle G(\langle 1_{X_1}|\hat{!}\rangle),G(\langle \hat{!}|1_{X_2}\rangle)\rangle\circ \iota_G\circ \pi_G\circ (1_{G(X_1\lor X_2)}-G(1_{X_1}\lor \hat{!})-G(\hat{!}\lor 1_{X_2}))$ is zero. Note that from $n = 1$ case $\iota_G\circ \pi_G = 1_{G(X_1\lor X_2)}-G(\hat{!})$. Then the first part of the composite becomes
    % % %%
    % % \begin{equation*}
    % %     (1_{G(X_1\lor X_2)}-G(\hat{!}))\circ (1_{G(X_1\lor X_2)}-G(1_{X_1}\lor \hat{!})-G(\hat{!}\lor 1_{X_2})) = 1_{G(X_1\lor X_2)}-G(1_{X_1}\lor \hat{!})-G(\hat{!}\lor 1_{X_2})+G(\hat{!})
    % % \end{equation*}
    % % %%
    % % Now, as this is a map to a biproduct, we can test it on its projections, which give the composites
    % % %%
    % % \begin{equation*}
    % %     G(\langle 1_{X_1}|\hat{!}\rangle)\circ (1_{G(X_1\lor X_2)}-G(1_{X_1}\lor \hat{!})-G(\hat{!}\lor 1_{X_2})+G(\hat{!})) = G(\langle 1_{X_1}|\hat{!}\rangle)-G(\langle 1_{X_1}|\hat{!})-G(\langle\hat{!}|\hat{!}\rangle)+G(\hat{!}) = 0
    % % \end{equation*}
    % % %%
    % % and similarly
    % % %%
    % % \begin{equation*}
    % %     G(\langle \hat{!}|1_{X_2}\rangle)\circ (1_{G(X_1\lor X_2)}-G(1_{X_1}\lor \hat{!})-G(\hat{!}\lor 1_{X_2})+G(\hat{!})) = G(\langle \hat{!}|1_{X_2}\rangle)-G(\langle \hat{!}|\hat{!})-G(\langle\hat{!}|1_{X_2}\rangle)+G(\hat{!}) = 0
    % % \end{equation*}
    % % %%
    % % which shows that the composite is indeed zero. It remains to show that the induced map is the projection, $\pi_G$, associated to the direct sum decomposition of $G(X_1\lor X_2)$. 


    % \vspace{10pt}
    
    % The inductive step is identical to the case of $n = 2$ using the inductive hypothesis.
\end{rmk} 


% Next we prove that the projection components can be used to form another natural transformation.

% \begin{lem}[label=lem:lorProj]
%     The components $\pi_G$ defined above constitute a natural transformation $\pi:\lor_{i=1}^n\Rightarrow \text{cr}_n$.
% \end{lem}
% \begin{proof}
%     We first show that for each $G$, $\pi_G:\lor_{i=1}^n(G)\Rightarrow \text{cr}_n(G)$ is a natural transformation. We proceed by induction on $n$. If $n = 1$, $(\pi_G)_X:G(X)\rightarrow \text{cr}_1(G)(X)$. Then for $f:X\rightarrow Y$, from Lemma \ref{lem:implNatIso}
%     %%
%     \[\begin{tikzcd}
%     	{\text{cr}_1(G)(X)} & {G(X)} & {\text{cr}_1(G)(X)} & {\text{cr}_1(G)(X)\oplus G(\star)} & {G(X)} \\
%     	{\text{cr}_1(G)(Y)} & {G(Y)} & {\text{cr}_1(G)(Y)} & {\text{cr}_1(G)(Y)\oplus G(\star)} & {G(Y)}
%     	\arrow["{\text{cr}_1(G)(f)}"', from=1-1, to=2-1]
%     	\arrow[two heads, from=1-2, to=1-1]
%     	\arrow[two heads, from=2-2, to=2-1]
%     	\arrow[""{name=0, anchor=center, inner sep=0}, "{G(f)}"{description}, from=1-2, to=2-2]
%     	\arrow[""{name=1, anchor=center, inner sep=0}, "{\text{cr}_1(G)(f)}"{description}, from=1-3, to=2-3]
%     	\arrow[two heads, from=1-4, to=1-3]
%     	\arrow[two heads, from=2-4, to=2-3]
%     	\arrow["{\text{cr}_1(G)(f)\oplus 1_{G(\star)}}"{description}, from=1-4, to=2-4]
%     	\arrow["\cong"', from=1-5, to=1-4]
%     	\arrow["\cong", from=2-5, to=2-4]
%     	\arrow["{G(f)}", from=1-5, to=2-5]
%     	\arrow[shorten <=17pt, shorten >=23pt, Rightarrow, no head, from=0, to=1]
%     \end{tikzcd}\]
%     %%
%     where both squares on the right commute by naturality of the isomorphism, proving naturality of $\pi_G$. Next, to show naturality in $G$ let $\eta:G\Rightarrow F$ be a natural transformation between functors $G,F:\mathcal{B}\rightarrow \mathcal{A}$. This corresponds to the commutivity of the diagram
%     \[\begin{tikzcd}
%     	{\text{cr}_1(G)(X)} & {G(X)} & {\text{cr}_1(G)(X)} & {\text{cr}_1(G)(X)\oplus G(\star)} & {G(X)} \\
%     	{\text{cr}_1(F)(X)} & {F(X)} & {\text{cr}_1(F)(X)} & {\text{cr}_1(F)(X)\oplus F(\star)} & {F(X)}
%     	\arrow["{\pi_{G,X}}"', from=1-2, to=1-1]
%     	\arrow[""{name=0, anchor=center, inner sep=0}, "{\eta_X}", from=1-2, to=2-2]
%     	\arrow["{\pi_{F,X}}", from=2-2, to=2-1]
%     	\arrow["{\text{cr}_1(\eta)_X}"', from=1-1, to=2-1]
%     	\arrow["\cong"', from=1-5, to=1-4]
%     	\arrow[two heads, from=1-4, to=1-3]
%     	\arrow[""{name=1, anchor=center, inner sep=0}, "{\text{cr}_1(\eta)_X}"{description}, from=1-3, to=2-3]
%     	\arrow["{\text{cr}_1(\eta)_X\oplus \eta_\star}"{description}, from=1-4, to=2-4]
%     	\arrow[two heads, from=2-4, to=2-3]
%     	\arrow["{\eta_X}", from=1-5, to=2-5]
%     	\arrow["\cong", from=2-5, to=2-4]
%     	\arrow[shorten <=17pt, shorten >=22pt, Rightarrow, no head, from=0, to=1]
%     \end{tikzcd}\]
%     where the squares in the rectangle on the right commute by naturality of the isomorphism and definition of the projections onto components.

%     \vspace{10pt}

%     Now, suppose the claim holds for some $n-1$. We aim to show naturality in $G$ and $X_i$. For $G$ fixed we have the following equality
%     %%
%     \[\begin{tikzcd}
%     	{\text{cr}_n(G)(X)} & {G(\lor_{i=1}^nX_i)} &[-20pt] {\text{cr}_n(G)(X)} &[-20pt] {\text{cr}_{n-1}(G)(X_1\lor X_2)} & {G(\lor_{i=1}^nX_i)} \\
%     	{\text{cr}_n(G)(Y)} & {G(\lor_{i=1}^nY_i)} &[-20pt] {\text{cr}_n(G)(Y)} &[-20pt] {\text{cr}_{n-1}(G)(Y_1\lor Y_2)} & {G(\lor_{i=1}^nY_i)}
%     	\arrow["{\pi_{G,X}}"', from=1-2, to=1-1]
%     	\arrow[""{name=0, anchor=center, inner sep=0}, "{G(\lor_{i=1}^nf_i)}"{description}, from=1-2, to=2-2]
%     	\arrow["{\pi_{G,Y}}", from=2-2, to=2-1]
%     	\arrow["{\text{cr}_n(G)(f_1,...,f_n)}"', from=1-1, to=2-1]
%     	\arrow["{G(\lor_{i=1}^nf_i)}"{description}, from=1-5, to=2-5]
%     	\arrow[""{name=1, anchor=center, inner sep=0}, "{\text{cr}_n(G)(f_1,...,f_n)}"{description}, from=1-3, to=2-3]
%     	\arrow["{\pi_{G,(X_1\lor X_2,...)}}"', from=1-5, to=1-4]
%     	\arrow["{\pi_{G,(X_1\lor X_2,...)}}", from=2-5, to=2-4]
%     	\arrow[two heads, from=1-4, to=1-3]
%     	\arrow[two heads, from=2-4, to=2-3]
%     	\arrow["{\text{cr}_{n-1}(G)(f_1\lor f_2,...)}"{description}, from=1-4, to=2-4]
%     	\arrow[shorten <=24pt, shorten >=32pt, Rightarrow, no head, from=0, to=1]
%     \end{tikzcd}\]
%     %%
%     The rightmost square commutes by the inductive hypothesis. On the other hand, decomposing the left square on the right side using the natural isomorphism in Equation \eqref{eq:natN} produces naturality of that square, and hence the whole rectangle. Finally, to show naturality in $G$ let $\eta:G\Rightarrow F$ be natural. Then similarly to above,
%     %%
%     \[\begin{tikzcd}
%     	{\text{cr}_n(G)(X)} & {G(\lor_{i=1}^nX_i)} & {\text{cr}_n(G)(X)} & {\text{cr}_{n-1}(G)(X_1\lor X_2)} & {G(\lor_{i=1}^nX_i)} \\
%     	{\text{cr}_n(F)(X)} & {F(\lor_{i=1}^nX_i)} & {\text{cr}_n(F)(X)} & {\text{cr}_{n-1}(F)(X_1\lor X_2)} & {F(\lor_{i=1}^nX_i)}
%     	\arrow["{\pi_{G,X}}"', from=1-2, to=1-1]
%     	\arrow[""{name=0, anchor=center, inner sep=0}, "{\eta_{\lor_{i=1}^nf_i}}", from=1-2, to=2-2]
%     	\arrow["{\pi_{F,X}}", from=2-2, to=2-1]
%     	\arrow["{\text{cr}_n(\eta)_X}"', from=1-1, to=2-1]
%     	\arrow["{\eta_{\lor_{i=1}^nX_i}}", from=1-5, to=2-5]
%     	\arrow[""{name=1, anchor=center, inner sep=0}, "{\text{cr}_n(\eta)_X}"{description}, from=1-3, to=2-3]
%     	\arrow["{\pi_{G,(X_1\lor X_2,...)}}"', from=1-5, to=1-4]
%     	\arrow["{\pi_{F,(X_1\lor X_2,...)}}", from=2-5, to=2-4]
%     	\arrow[two heads, from=1-4, to=1-3]
%     	\arrow[two heads, from=2-4, to=2-3]
%     	\arrow["{\text{cr}_{n-1}(\eta)_{X_1\lor X_2,X_3,...,X_n}}"{description}, from=1-4, to=2-4]
%     	\arrow[shorten <=35pt, shorten >=32pt, Rightarrow, no head, from=0, to=1]
%     \end{tikzcd}\]
%     %% 
%     commutes by naturality of $\eta$ and expanding the left square on the right using the natural isomorphism in Equation \eqref{eq:natN}.
% \end{proof}

These natural isomorphisms will prove valuable for proving that $\text{cr}_n$ is the right adjoint in an adjunction between categories of reduced functors.

\begin{prop}
    The $n$-th cross effect is a right adjoint to the diagonal functor $\Delta^*:\text{Fun}_*(\mathcal{B}^n,\mathcal{A})\rightarrow \text{Fun}_*(\mathcal{B},\mathcal{A})$.
\end{prop}
\begin{proof}
    We demonstrate the adjunction by showing the co-universal property. Pictorially this can be represented by:
    %%
    \[\begin{tikzcd}
    	{\text{cr}_n(G)} & {\Delta^*(\text{cr}_n(G))} & G \\
    	F & {\Delta^*(F)}
    	\arrow["{\epsilon_G}", from=1-2, to=1-3]
    	\arrow["\alpha"', from=2-2, to=1-3]
    	\arrow[dashed, from=2-2, to=1-2]
    	\arrow["{\hat{\alpha}}", dashed, from=2-1, to=1-1]
    \end{tikzcd}\]
    %%
    Here $\epsilon_G = G(+)\circ \iota_G$. Given such an $\alpha$, we let $\hat{\alpha}$ be given by the composite 
    \begin{equation*}
        F(X_1,...,X_n)\xrightarrow{i}\text{cr}_n(\Delta^*(F))(X_1,...,X_n)\xrightarrow{\text{cr}_n(\alpha)_{X_1,...,X_n}}\text{cr}_n(G)(X_1,...,X_n)
    \end{equation*}
    First we show the components of this proposed map make the diagram commute. Using naturality of $\iota$ we can re-write this composite as
    %%
    \begin{align*}
        F(X,...,X)&\xrightarrow{i}\text{cr}_n(\Delta^*(F))(X,...,X)\xrightarrow{\iota_{\Delta^*(F)}}\Delta^*(F)(\lor_{i=1}^nX) \xrightarrow{\alpha_{\lor_{i=1}^nX}}G(\lor_{i=1}^nX)\xrightarrow{G(+)}G(X)
    \end{align*}
    %%
    Then using naturality of $\alpha$ we obtain
    %%
    \begin{align*}
        F(X,...,X)&\xrightarrow{i}\text{cr}_n\Delta^*(F)(X,...,X)\xrightarrow{\iota_{\Delta^*(F)}}\Delta^*(F)(\lor_{i=1}^nX) \xrightarrow{\Delta^*(F)(+)}\Delta^*(F)(X)\xrightarrow{\alpha_X}G(X)
    \end{align*}
    %%
    It remains to show $\Delta^*(F)(+)\circ \iota_{\Delta^*(F)}\circ i = 1_{\Delta^*(F)(X)}$. However, $i = \pi_{\Delta^*(F)}\circ F(i_1,...,i_n)$, and from our previous remark $\iota_{\Delta^*(F)}\circ \pi_{\Delta^*(F)}$ is equal to $1_{\Delta^*(F)(\lor_{i=1}^nX)}$ plus terms which involve at least one $\hat{!}$. Composing any term which involves $\hat{!}$ with $\Delta^*(F)(+)$ will result in the zero map since $F$ is reduced. Thus, the composite becomes
    %%
    \begin{align*}
        \Delta^*(F)(+)\circ F(i_1,...,i_n) = 1_{\Delta^*(F)(X)}
    \end{align*}
    %%

    \vspace{10pt}

    Next we show that $\hat{\alpha}$ is natural. However, this follows immediately from Lemma \ref{lem:lorProj} and Lemma \ref{lem:compIncNat}, so $\hat{\alpha}$ is a composite of natural transformations.

    \vspace{10pt}

    Finally, it remains to show uniqueness of $\hat{\alpha}$. It is sufficient to show that if $\beta:F\Rightarrow \text{cr}_n(G)$, then $\widehat{\epsilon_G\circ \Delta^*(\beta)} = \beta$, or in other words the composite
    %%
    \begin{align*}
        F(X_1,...,X_n)\xrightarrow{i}\text{cr}_n(\Delta^*(F))(X_1,...,X_n)&\xrightarrow{\text{cr}_n(\Delta^*(\beta))}\text{cr}_n(\Delta^*(\text{cr}_n(G)))(X_1,...,X_n)\\
        %%
        &\xrightarrow{\text{cr}_n(\Delta^*(\iota_G))}\text{cr}_n(\Delta^*(\lor_{i=1}^n(G)))(X_1,...,X_n)\\
        %%
        &\xrightarrow{\text{cr}_n(G(+))}\text{cr}_n(G)(X_1,...,X_n)
    \end{align*}
    %%
    equals $\beta$. Using naturality of the projection and the inclusions into $\lor_{i=1}^nX_i$, this composite can be written as 
    %%
    \begin{align*}
        F(X_1,...,X_n)\xrightarrow{\beta}\text{cr}_n(G)(X_1,...,X_n)&\xrightarrow{\text{cr}_n(G)(i_1,...,i_n)}\Delta^*(\text{cr}_n(G))(\lor_{i=1}^nX_i) \\
        &\xrightarrow{\pi_{\Delta^*(\text{cr}_n(G))}}\text{cr}_n(\Delta^*(\text{cr}_n(G)))(X_1,...,X_n) \\
        &\xrightarrow{\text{cr}_n(\Delta^*(\iota_G))}\text{cr}_n(\Delta^*(\lor_{i=1}^n(G)))(X_1,...,X_n) \\
        &\xrightarrow{\text{cr}_n(G(+))}\text{cr}_n(G)(X_1,...,X_n)
    \end{align*}
    %%
    It remains to show that the composite after $\beta$ is the identity. By naturality of the projection twice this composite becomes
    %%
    \begin{align*}
        \text{cr}_n(G)(X_1,...,X_n)\xrightarrow{\text{cr}_n(G)(i_1,...,i_n)}\Delta^*(\text{cr}_n(G))(\lor_{i=1}^nX_i) &\xrightarrow{\Delta^*(\iota_G)}\Delta^*(\lor_{i=1}^n(G))(\lor_{i=1}^nX_i) \\
        &\xrightarrow{G(+)}G(\lor_{i=1}^nX_i) \\
        &\xrightarrow{\pi_G}\text{cr}_n(G)(X_1,...,X_n)
    \end{align*}
    %%
    Next, using the naturality of $\iota$ we obtain
    %%
    \begin{align*}
        \text{cr}_n(G)(X_1,...,X_n)\xrightarrow{\iota_G}\lor_{i=1}^n(G)(X_1,...,X_n)&\xrightarrow{\lor_{i=1}^n(G)(i_1,...,i_n)}\Delta^*(\lor_{i=1}^n(G))(\lor_{i=1}^nX_i) \\
        &\xrightarrow{G(+)}G(\lor_{i=1}^nX_i) \\
        &\xrightarrow{\pi_G}\text{cr}_n(G)(X_1,...,X_n)
    \end{align*}
    %%
    Now, the middle two arrows compose to give the identity, while we also have that $\pi_G\circ \iota_G$ is the identity, completing the proof.
    % We argue uniqueness by induction on $n$ using the decomposition in the proof of Lemma \ref{lem:lorInj}. First, if $n = 1$, for each $X$ we would obtain the diagram
    % %%
    % \[\begin{tikzcd}
    % 	{\text{cr}_1(G)(X)} & {G(X)} & {G(X)} & {G(\star)} \\
    % 	{F(X)} & {F(\star)\cong 0}
    % 	\arrow["{\iota_G}", tail, from=1-1, to=1-2]
    % 	\arrow["{G(+)=1_{G(X)}}", from=1-2, to=1-3]
    % 	\arrow["{\alpha_X}"{description}, from=2-1, to=1-3]
    % 	\arrow["{F(!)}"', from=2-1, to=2-2]
    % 	\arrow["{\alpha_\star}"', from=2-2, to=1-4]
    % 	\arrow["{G(!)}", from=1-3, to=1-4]
    % 	\arrow[dashed, from=2-1, to=1-1]
    % \end{tikzcd}\]
    % %%
    % where the component along the vertical is uniquely determined by the universal property of the kernel, and equals $\hat{\alpha}_X$, so $\hat{\alpha}$ is also uniquely determined. Note that $F(X) \cong 0$ since $F$ is reduced.

    % Next, suppose uniqueness holds for some $n-1$, $n \geq 2$. The inductive hypothesis says we have a unique \textbf{TBC}
\end{proof}

Composing with the adjunction $\text{inc}\dashv \text{cr}_1$ and using Lemma \ref{lem:idempotCr1} we obtain an adjunction
\[\begin{tikzcd}
	{\text{Fun}(\mathcal{B},\mathcal{A})} & {\text{Fun}_*(\mathcal{B}^n,\mathcal{A})}
	\arrow[""{name=0, anchor=center, inner sep=0}, "{\text{cr}_n\circ \text{cr}_1}"', shift right=2, from=1-1, to=1-2]
	\arrow[""{name=1, anchor=center, inner sep=0}, "{\Delta^*}"', shift right=2, from=1-2, to=1-1]
	\arrow["\dashv"{anchor=center, rotate=-90}, draw=none, from=1, to=0]
\end{tikzcd}\]
We use this adjunction to define a family of comonads.

\begin{defn}[label=defn:crossEffectComonad]
    For each $n \in \N$, we have a comonad $C_n:\text{Fun}(\mathcal{B},\mathcal{A})\rightarrow \text{Fun}(\mathcal{B},\mathcal{A})$ given by $C_n := \Delta^*\circ \text{cr}_n\circ \text{cr}_1$. The counit of the comonad is given by the composite 
    %%
    \begin{equation*}
        C_n(G)(X) = \Delta^*(\text{cr}_n(\text{cr}_1(G)))(X)\xrightarrow{\Delta^*(\iota_{\text{cr}_1(G)})}\text{cr}_1(G)(\lor_{i=1}^nX)\xrightarrow{\text{cr}_1(G)(+)}\text{cr}_1(G)(X)\xrightarrow{s_{G,1,X}}G(X)
    \end{equation*}
    %%
    while the comultiplication is given by the composite 
    %%
    \begin{equation*}
        \Delta^*(\text{cr}_n(\text{cr}_1(G)))(X)\xrightarrow{\text{cr}_n(\text{cr}_1(G))(i_1,...,i_n)}\Delta^*(\text{cr}_n(\text{cr}_1(G)))(\lor_{i=1}^nX)\xrightarrow{\pi_{\Delta^*(\text{cr}_n(\text{cr}_1(G)))}}C_n(C_n(G))(X)
    \end{equation*}
    %%
\end{defn}


\subsubsection{Contracting Homotopies}


In this section we aim to show that the contracting homotopy in the following lemma is natural in $A$.

\begin{lem}[label=lem:contractHomotop]
    Let $\begin{tikzcd}
{\mathcal{A}} & {\mathcal{B}}
\arrow[""{name=0, anchor=center, inner sep=0}, "R"', shift right=2, from=1-1, to=1-2]
\arrow[""{name=1, anchor=center, inner sep=0}, "L"', shift right=2, from=1-2, to=1-1]
\arrow["\dashv"{anchor=center, rotate=-90}, draw=none, from=1, to=0]
\end{tikzcd}$ define an adjunction between abelian categories inducing a comonad $C = LR$ on $\mathcal{A}$ with counit $\epsilon:LR\Rightarrow \text{id}$. Then for each $A \in \mathcal{A}_0$ the chain complex in $\mathcal{B}$ with differentials defined to be the alternating sums $\sum_{i\geq 0}^k(-1)^iR(LR)^i\epsilon$ admits a contracting homotopy.
\end{lem}
\begin{proof}[Contracting Homotopy Proof]
    Define $s_k = \eta_{R(LR)^kA}$ using the unit $\eta:\text{id}\Rightarrow RL$ of the adjunction.

    We first show that the described data defines a chain complex. Observe for $n \in \N\cup\{0\}$,
    %%
    \begin{align*}
        &\left(\sum_{i\geq 0}^{n}(-1)^iR(LR)^i\epsilon_{(LR)^{n-i}A}\right)\circ\left(\sum_{i\geq 0}^{n+1}(-1)^iR(LR)^i\epsilon_{(LR)^{n+1-i}A}\right) \\
        &= \sum_{i=0}^n\sum_{j=0}^{n+1}(-1)^{i+j}R(LR)^i\epsilon_{(LR)^{n-i}A}\circ R(LR)^j\epsilon_{(LR)^{n+1-j}A} \\
        &= \sum_{i=0}^n\sum_{i < j}^{n+1}(-1)^{i+j}R(LR)^i\epsilon_{(LR)^{n-i}A}\circ R(LR)^j\epsilon_{(LR)^{n+1-j}A} \\
        &+\sum_{i=0}^n\sum_{j \leq i}(-1)^{i+j}R(LR)^i\epsilon_{(LR)^{n-i}A}\circ R(LR)^j\epsilon_{(LR)^{n+1-j}A} \\
        &= \sum_{i=0}^n\sum_{i < j}^{n+1}(-1)^{i+j}R(LR)^i(\epsilon_{(LR)^{n-i}A}\circ (LR)^{j-i}\epsilon_{(LR)^{n+1-j}A}) \\
        &+\sum_{i=0}^n\sum_{j \leq i}(-1)^{i+j}R(LR)^j((LR)^{i-j}\epsilon_{(LR)^{n-i}A}\circ \epsilon_{(LR)^{n+1-j}A}) \\
        &= \sum_{i=0}^n\sum_{i \leq k}^{n}(-1)^{i+k+1}R(LR)^i(\epsilon_{(LR)^{n-i}A}\circ (LR)^{k+1-i}\epsilon_{(LR)^{n-k}A}) \tag{substituting $k = j-1$} \\
        &+\sum_{i=0}^n\sum_{j \leq i}(-1)^{i+j}R(LR)^j((LR)^{i-j}\epsilon_{(LR)^{n-i}A}\circ \epsilon_{(LR)^{n+1-j}A}) \\
        &= \sum_{i=0}^n\sum_{i \leq k}^{n}(-1)^{i+k+1}R(LR)^i((LR)^{k-i}\epsilon_{(LR)^{n-k}A}\circ \epsilon_{(LR)^{n+1-i}A}) \tag{by naturality of $\epsilon$} \\
        &+\sum_{i=0}^n\sum_{j \leq i}(-1)^{i+j}R(LR)^j((LR)^{i-j}\epsilon_{(LR)^{n-i}A}\circ \epsilon_{(LR)^{n+1-j}A}) \\
        &= -\sum_{i=0}^n\sum_{i \leq k}^{n}(-1)^{i+k}R(LR)^i((LR)^{k-i}\epsilon_{(LR)^{n-k}A}\circ \epsilon_{(LR)^{n+1-i}A}) \\
        &+\sum_{i=0}^n\sum_{j \leq i}(-1)^{i+j}R(LR)^j((LR)^{i-j}\epsilon_{(LR)^{n-i}A}\circ \epsilon_{(LR)^{n+1-j}A}) \\
        &= -\sum_{k=0}^n\sum_{i \leq k}(-1)^{i+k}R(LR)^i((LR)^{k-i}\epsilon_{(LR)^{n-k}A}\circ \epsilon_{(LR)^{n+1-i}A}) \tag{switching the order of summation} \\
        &+\sum_{i=0}^n\sum_{j \leq i}(-1)^{i+j}R(LR)^j((LR)^{i-j}\epsilon_{(LR)^{n-i}A}\circ \epsilon_{(LR)^{n+1-j}A}) \\
        &= 0
    \end{align*}
    so the maps are differentials of a complex.

    Next we show that $s_k$, as defined, is a contracting homotopy for our chain complex. This is equivalent to saying that $s_{k-1}\circ \partial_{k-1}+\partial_{k}\circ s_{k} = 1_{R(LR)^kA}$, where $\partial_k:R(LR)^{k+1}A\rightarrow R(LR)^kA$ is our differential defined above. Then observe that
    %%
    \begin{align*}
        s_{k-1}\circ \partial_{k-1}+\partial_{k}\circ s_{k} &= \sum_{i=0}^{k-1}(-1)^i\eta_{R(LR)^{k-1}A}\circ R(LR)^i\epsilon_{(LR)^{k-1-i}A} \\
        &+ \sum_{i=0}^k(-1)^iR(LR)^i\epsilon_{(LR)^{k-i}A}\circ \eta_{R(LR)^kA} \\
        &= 1_{R(LR)^kA}+\sum_{i=0}^{k-1}(-1)^i\eta_{R(LR)^{k-1}A}\circ R(LR)^i\epsilon_{(LR)^{k-1-i}A} \\
        &+ \sum_{i=1}^k(-1)^iR(LR)^i\epsilon_{(LR)^{k-i}A}\circ \eta_{R(LR)^kA} \tag{using the triangle identities} 
    \end{align*}
    %%
    It remains to show the extra sum is zero. After re-indexing the first sum it becomes:
    %%
    \begin{equation*}
        -\sum_{i=1}^{k}(-1)^i\eta_{R(LR)^{k-1}A}\circ R(LR)^{i-1}\epsilon_{(LR)^{k-i}A} + \sum_{i=1}^k(-1)^iR(LR)^i\epsilon_{(LR)^{k-i}A}\circ \eta_{R(LR)^kA}
    \end{equation*}
    %%
    which is zero by naturality of $\eta$.
\end{proof}

Additional to the result of this lemma, we claim that the contracting chain homotopy yields a natural transformation $s_k:R(LR)^k\Rightarrow R(LR)^{k+1}$, as $\eta_{R(LR)^k}$ is natural.

Finally, we have the following proposition.

\begin{prop}[label=prop:exactCross]
    For each $n \geq 1$, the functors $\text{cr}_n:\text{Fun}(\mathcal{B},\mathcal{A})\rightarrow \text{Fun}_*(\mathcal{B}^n,\mathcal{A})$ and $C_n:\text{Fun}(\mathcal{B},\mathcal{A})\rightarrow \text{Fun}(\mathcal{B},\mathcal{A})$ are exact.
\end{prop}
\begin{proof}
    Since the functor categories are abelian, showing exactness is equivalent to showing that the functors preserve short exact sequences. The proof for $\text{cr}_n$ follows by the $3\times 3$ lemma and induction on $n$. After the proof for $\text{cr}_n$ the result for $C_n$ follows immediately.
\end{proof}



\subsubsection{Properties of the Cross-Effect}



\clearpage

\section{A categorical context for abelian functor calculus}


Classically, abelian functor calculus deals with functors to abelian categories. In order to discuss universal properties ``up to homotopy" we replace abelian categories by some type of homotopical categories where weak universal properties replace strict ones. 


We denote the category of chain complexes of an abelian category $\mathcal{A}$ concentrated in non-negative degrees by $\cat{Ch}\mathcal{A}$. In this section we will construct a category which has as arrows maps from abelian categories to categories of chains. An important theorem for this construction is the Dold-Kan Equivalence, which is reviewed in Appendix \ref{sec:doldKan}.



\subsection{Pointwise versus Natural Equivalences}\label{sec:ptwiseNat}

\begin{rmk}
    If we choose pointwise isomorphisms for the functors in the definition of $\cat{AbCat}$, then composition in $\cat{AbCat}_{\cat{Ch}}$ will not be well-defined since we require that for any equivalent functors, $G$ and $H$, and any simplicial object $\hat{A}$ with codomain equal to the domain of $G$ and $H$, $G\circ \hat{A}\cong H\circ \hat{A}$ as simplicial objects.
\end{rmk}


Since the polynomial and linearization functors of \cite{JohnsonB.2004Dcwc} are only defined up to quasi-isomorphism in certain viewpoints, in order for them to be well-defined we must pass to the homotopy category $\cat{HoAbCat}_{\cat{Ch}}$. First, in this section we let composition in $\cat{AbCat}_{\cat{Ch}}$ be defined for $G:\mathcal{B}\rightarrow \cat{Ch}(\mathcal{C})$ and $F:\mathcal{A}\rightarrow \cat{Ch}(\mathcal{B})$ by
%%
\begin{equation*}
    G\lhd F := N_\mathcal{C}\Delta_\mathcal{C}(\Gamma_\mathcal{C})_* G_*\Gamma_\mathcal{B} F
\end{equation*}
%%
where $\Delta_\mathcal{C}:(\mathcal{C}\Sob)\Sob\rightarrow \mathcal{C}\Sob$ is the diagonal functor. It remains to show that this does define a categorical structure on $\cat{AbCat}_{\cat{Ch}}$, which we check through the following list of conditions:
%%
\begin{enumerate}
    \item Let $F:\mathcal{A}\rightarrow \cat{Ch}(\mathcal{B})$ be a functor. The identity is given by $\deg_0$. Indeed observe that
    %%
    \begin{equation*}
        [\deg_0^\mathcal{B}\lhd F] = [N_\mathcal{B}\Delta_\mathcal{B}(\Gamma_\mathcal{B})_*(\deg_0^\mathcal{B})_*\Gamma_\mathcal{B} F]
    \end{equation*}
    %%
    However, by Lemma \ref{lem:gammaDeg} $\Gamma_\mathcal{B}\circ \deg_0^\mathcal{B} \cong \iota_\mathcal{B}:\mathcal{B}\rightarrow \mathcal{B}\Sob$ is the constant functor. It follows that $\Delta_\mathcal{B}(\Gamma_{\mathcal{B}}\circ \deg_0^\mathcal{B}\circ \Gamma_\mathcal{B}F) = \Gamma_\mathcal{B}F$, so
    %%
    \begin{equation*}
        [\deg_0^\mathcal{B}\lhd F] = [N_\mathcal{B}\Delta_\mathcal{B}(\Gamma_\mathcal{B})_*(\deg_0^\mathcal{B})_*\Gamma_\mathcal{B} F] = [N_\mathcal{B}\Gamma_\mathcal{B}F] = [F]
    \end{equation*}
    %%
    On the other hand,
    %%
    \begin{equation*}
        [F\lhd \deg_0^\mathcal{A}] = [N_\mathcal{B}\Delta_\mathcal{B}(\Gamma_\mathcal{B})_*F_*\Gamma_\mathcal{A}\deg_0^\mathcal{A}] = [N_\mathcal{B}\Delta_\mathcal{B}(\Gamma_\mathcal{B})_*F_*\iota_\mathcal{A}]
    \end{equation*}
    %%
    Observe that $F_*\iota_\mathcal{A} = \iota_{\cat{Ch}(\mathcal{B})}\circ F$. Similarly, $(\Gamma_\mathcal{B})_*\iota_{\cat{Ch}(\mathcal{B})} = \iota_{\mathcal{B}\Sob}\circ\Gamma_\mathcal{B}$. Finally,
    %%
    \begin{equation*}
        \Delta_\mathcal{B}(\iota_{\mathcal{B}\Sob}\circ\Gamma_\mathcal{B}\circ F) = \Gamma_\mathcal{B}\circ F
    \end{equation*}
    %%
    so
    %%
    \begin{equation*}
        [F\lhd \deg_0^\mathcal{A}] = [N_\mathcal{B}\Delta_\mathcal{B}(\Gamma_\mathcal{B})_*F_*\Gamma_\mathcal{A}\deg_0^\mathcal{A}] = [N_\mathcal{B}\Gamma_\mathcal{B}F] = [F]
    \end{equation*}
    %%
    \item It remains to show composition is associative, so consider $F:\mathcal{A}\rightarrow \cat{Ch}(\mathcal{B}),H:\mathcal{B}\rightarrow \cat{Ch}(\mathcal{C}),G:\mathcal{C}\rightarrow \cat{Ch}(\mathcal{D})$. Then we compute:
    %%
    \begin{align*}
        [(G\lhd H)\lhd F] &= [N_\mathcal{D}\Delta_\mathcal{D}(\Gamma_\mathcal{D})_*(G\lhd H)_*\Gamma_\mathcal{B}F] \\
        &= [N_\mathcal{D}\Delta_\mathcal{D}(\Gamma_\mathcal{D})_*(N_\mathcal{D}\Delta_\mathcal{D}(\Gamma_\mathcal{D})_*G_*\Gamma_\mathcal{C}H)_*\Gamma_\mathcal{B}F] \\
        &= [N_\mathcal{D}\Delta_\mathcal{D}(\Delta_\mathcal{D}(\Gamma_\mathcal{D})_*G_*)_*(\Gamma_\mathcal{C})_*H_*\Gamma_\mathcal{B}F] \\
    \end{align*}
    %%
    while
    %%
    \begin{align*}
        [G\lhd(H\lhd F)] &= [G\lhd(N_\mathcal{C}\Delta_\mathcal{C}(\Gamma_\mathcal{C})_*H_*\Gamma_\mathcal{B}F)] \\
        &= [N_\mathcal{D}\Delta_\mathcal{D}(\Gamma_\mathcal{D})_*G_*\Gamma_\mathcal{C}(N_\mathcal{C}\Delta_\mathcal{C}(\Gamma_\mathcal{C})_*H_*\Gamma_\mathcal{B}F)] \\
        &= [N_\mathcal{D}\Delta_\mathcal{D}(\Gamma_\mathcal{D})_*G_*\Delta_\mathcal{C}(\Gamma_\mathcal{C})_*H_*\Gamma_\mathcal{B}F] \\
    \end{align*}
    Hence, it is sufficient to show that $[(\Gamma_\mathcal{D})_*G_*\Delta_\mathcal{C}] = [(\Delta_\mathcal{D}(\Gamma_\mathcal{D})_*G_*)_*]$. Recall that $(\Gamma_\mathcal{D})_*G_* : \mathcal{C}\Sob \rightarrow \cat{Ch}(\mathcal{D})\Sob\rightarrow (\mathcal{D}\Sob)\Sob$ and $((\Gamma_\mathcal{D})_*G_*)_* : (\mathcal{C}\Sob)\Sob\rightarrow ((\mathcal{D}\Sob)\Sob)\Sob$. Additionally, $(\Delta_\mathcal{D})_* = (-)\Sob\Delta_\mathcal{D} = \Delta_{\mathcal{D}\Sob}$, so this equality is exactly naturality of $\Delta_{(-)}$, which is shown in Section \ref{sec:simpMon}.
\end{enumerate}
Therefore, this composition provides the structure of a 1-category for $\cat{AbCat}_{\cat{Ch}}$.



Next we start describing the homotopy structure on this category. 
\begin{defn}[label=defn:ChEquiv]{}
    Two functors $H,G:\mathcal{B}\rightarrow \cat{Ch}(\mathcal{A})$ are said to be \textbf{pointwise chain homotopy equivalent} if the chain complexes $H(X)$ and $G(X)$ are chain homotopy equivalent in $\cat{Ch}(\mathcal{A})$ for each $X \in \mathcal{B}_0$, and \textbf{naturally chain homotopy equivalent} if the chain homotopies are natural in $X$.

    Explicitly, naturally chain homotopy equivalent means that for each $X \in \mathcal{B}_0$ we have natural transformations $h:H\Rightarrow G, g:G\Rightarrow H$, together with homotopies $s_X:h_X\circ g_X\simeq 1_{G(X)}$ and $r_X:g_X\circ h_X\simeq 1_{H(X)}$ which are natural in the sense that for each $n$, $s_n:(-)_n\circ G\Rightarrow (-)_{n+1}\circ G$ and $r_n:(-)_n\circ H\Rightarrow (-)_{n+1}\circ H$ are natural transformations.
\end{defn}

Both of these notions define equivalence relations on the category $\cat{AbCat}_{\cat{Ch}}$. Indeed, any functor is naturally chain homotopy equivalent to itself through identity natural transformations and zero homotopies, and a natural chain homotopy equivalence from $H$ to $G$ is precisely the same as a natural chain homotopy equivalence from $G$ to $H$. It remains to show that if $H\simeq_{ChN}G\simeq_{ChN}F$, then $H\simeq_{ChN}F$. Let $(h,g,s,r)$ and $(g',f,s',r')$ be the quadruples witnessing the natural chain homotopy equivalence. Then we define a new quadruple by $(g'\circ h, g\circ f, g'\circ s\circ f+s', g\circ r'\circ h+r)$ which has all natural components since the composition of natural transformations is natural, and $+$ is functorial in an abelian category. To see that this does indeed define a chain homotopy equivalence observe that denoting the chain maps by $\partial$, and suppressing subscripts, we compute
%%
\begin{align*}
    \partial(g'\circ s\circ f+s')+(g'\circ s\circ f+s')\partial &= \partial g'sf+\partial s'+g'sf\partial + s'\partial \\
    &= g'\partial sf+g's\partial f+\partial s'+s'\partial \\
    &= g'(\partial s+s\partial)f+\partial s'+s'\partial \tag{using the fact $g'$ and $f$ are chain maps} \\
    &= g'(hg-1_G)f + (g'f-1_F) \tag{by definition of the homotopies $s$ and $s'$} \\
    &= g'hgf-1_F
\end{align*}
%%


Finally, it remains to show that this equivalence relation is well-defined for isomorphism classes of functors. It is sufficient to show that if $[H] = [H']$ and $H$ is (naturally) chain homotopy equivalent to $G$, then so is $H'$. Let $(h,g,s,r)$ witness the (natural) chain homotopy, and let $\alpha:H\rightarrow H'$ be a natural isomorphism witnessing the equivalence. Then I claim that the quadruple $(h\alpha^{-1},\alpha g, s, \alpha[+1]r\alpha^{-1})$ is a (natural) chain homotopy, where $\alpha[+1]$ is the induced natural isomorphism between $H[+1]$ and $H'[+1]$. 

First, observe that this definition preserves naturality, since $\alpha$ is natural, so it is sufficient in both cases to demonstrate that the components define a chain homotopy. Then for $X \in \mathcal{B}_0$ we can compute
%%
\begin{equation*}
    \partial s_X+s_X\partial = h_Xg_X-1_{G(X)} = (h_X\alpha_X^{-1})(\alpha_X g_X)-1_{G(X)}
\end{equation*}
%%
while for a given $n$ (to make the computation more tractable)
%%
\begin{align*}
    \partial_{n+1} ((\alpha_X[+1])_n(r_X)_n(\alpha_X^{-1})_n)&+((\alpha_X[+1])_{n-1}(r_X)_{n-1}(\alpha_X^{-1})_{n-1})\partial_n \\
    &= \partial_{n+1} ((\alpha_X)_{n+1}(r_X)_n(\alpha_X^{-1})_n)+((\alpha_X)_{n}(r_X)_{n-1}(\alpha_X^{-1})_{n-1})\partial_n \\
    &= (\alpha_X)_n\partial_{n+1}(r_X)_n(\alpha_X^{-1})_n+(\alpha_X)_n(r_X)_{n-1}\partial_n(\alpha_X^{-1})_{n} \\
    &= (\alpha_X)_n(\partial_{n+1}(r_X)_n+(r_X)_{n-1}\partial_n)(\alpha_X^{-1})_{n} \\
    &= (\alpha_X)_n((g_X)_n(h_X)_n-(1_{H(X)})_n)(\alpha_X^{-1})_{n} \\
    &= (\alpha_X)_n(g_X)_n(h_X)_n(\alpha_X^{-1})_n-(1_{H(X)})_n
\end{align*}
%%
so that the homotopies indeed hold. Note that these proofs thus far are independent of the composition we have chosen for $\cat{AbCat}_{\cat{Ch}}$. In particular, in the case of natural chain homotopies the results follow from the fact that they correspond to chain homotopies in a category of chain complexes by the work in Section~\ref{subsec:chainHomotop}. In order to define a homotopy category from these equivalence relations we must now show that they are in fact congruence relations. We proceed only for the natural chain homotopies.

% We begin by following \cite{BJORT} in showing this for the original definition of composition on $\cat{AbCat}_{\cat{Ch}}$.

% \begin{lem}[label=lem:3.4]{(Lemma 3.4 \cite{BJORT})}
%     If $G,H:\mathcal{C}\rightarrow \cat{Ch}(\mathcal{B})$ are (naturally) chain homotopy equivalent functors, then for any pair of functors $F:\mathcal{B}\rightarrow \cat{Ch}(\mathcal{A})$ and $K:\mathcal{D}\rightarrow \cat{Ch}(\mathcal{C})$, the composites
%     %%
%     \begin{equation*}
%         F\circ G\circ K,F\circ H\circ K:\mathcal{D}\rightarrow \cat{Ch}(\mathcal{A})
%     \end{equation*}
%     %%
%     are (naturally) chain homotopy equivalent.
% \end{lem}
% \begin{proof}
%     Let $G,H$ be (naturally) chain homotopy equivalent, and let $F$ and $K$ be functors as in the question. We begin by showing that $G\circ K$ is (naturally) chain homotopy equivalent to $H\circ K$, where composition is in the ``Kleisli" category. 

%     \vspace{10pt}

%     % Observe that $[G\circ K] = [\text{Tot}_\mathcal{B}N_{\cat{Ch}(\mathcal{B})}G_*\Gamma_\mathcal{C}K]$. Since $N_{\cat{Ch}}$ preserves homotopy equivalences, it is sufficient to show that $[G_*\Gamma_\mathcal{C}K]$ is (naturally) homotopy equivalent to $[H_*\Gamma_\mathcal{C}K]$, in terms of simplicial homotopies. Finally, it is sufficient to show $[G_*\Gamma_\mathcal{C}]:\cat{Ch}(\mathcal{C})\rightarrow \mathcal{C}\Sob\rightarrow \cat{Ch}(\mathcal{B})\Sob$ and $[H_*\Gamma_\mathcal{C}]:\cat{Ch}(\mathcal{C})\rightarrow \mathcal{C}\Sob\rightarrow \cat{Ch}(\mathcal{B})\Sob$ are (naturally) homotopy equivalent.


%     Let $(h,g,s,r):H\cong_{\cat{Ch}}G$ be a (natural) chain homotopy between our original functors. We prove preservation under pre-composition and post-composition separately
%     %%
%     \begin{itemize}
%         \setlength{\itemindent}{2em}
%         \item[[$-\circ K$]] We recall the compositions $G\circ K$ and $H\circ K$ in $\cat{AbCat}_{\cat{Ch}}$ are given by
%         %%
%         \begin{equation*}
%             G\circ K = \text{Tot}_{\mathcal{B}}N_{\cat{Ch}(\mathcal{B})}G_*\Gamma_\mathcal{C}K\;\;\;H\circ K = \text{Tot}_{\mathcal{B}}N_{\cat{Ch}(\mathcal{B})}H_*\Gamma_\mathcal{C}K
%         \end{equation*}
%         %%
%         \textbf{NOT DONE}
%         \item[[$F\circ -$]] The compositions $F\circ G$ and $F \circ H$ in $\cat{AbCat}_{\cat{ch}}$ are given by
%         %%
%         \begin{equation*}
%             F\circ G = \text{Tot}_{\mathcal{A}}N_{\cat{Ch}(\mathcal{A})}F_*\Gamma_\mathcal{B}G\;\;\;F\circ H = \text{Tot}_{\mathcal{A}}N_{\cat{Ch}(\mathcal{A})}F_*\Gamma_\mathcal{B}H
%         \end{equation*}
%         %%
%         Using the fact that $\text{Tot}_{\cat{Ch}}$ preserves chain homotopy equivalences \cite[Remark 12.18.6]{StacksProject} and the results in Sections \ref{subsec:simpHomotop} and \ref{sec:doldKan} we obtain the following chain of equivalences:
%         %%
%         \begin{align*}
%             H\simeq_{\cat{Ch}}G &\implies \Gamma_\mathcal{B}H\simeq_{Ho}\Gamma_\mathcal{B}G \\
%             &\implies F_*\Gamma_\mathcal{B}H\simeq_{Ho}F_*\Gamma_\mathcal{B}G \\
%             &\implies N_\cat{Ch}(\mathcal{A})F_*\Gamma_\mathcal{B}H\simeq_{\cat{Ch}}N_\cat{Ch}(\mathcal{A})F_*\Gamma_\mathcal{B}G \\
%             &\implies \text{Tot}_\mathcal{A}N_\cat{Ch}(\mathcal{A})F_*\Gamma_\mathcal{B}H\simeq_{\cat{Ch}}\text{Tot}_\mathcal{A}N_\cat{Ch}(\mathcal{A})F_*\Gamma_\mathcal{B}G 
%         \end{align*}
%         %%
%         where each implication preserves naturality as well.
%     \end{itemize}
% \end{proof}


\begin{lem}[label=lem:3.4NewComp]
    If $G,H:\mathcal{C}\rightarrow \cat{Ch}(\mathcal{B})$ are naturally chain homotopy equivalent functors, then for any pair of functors $F:\mathcal{B}\rightarrow \cat{Ch}(\mathcal{A})$ and $K:\mathcal{D}\rightarrow \cat{Ch}(\mathcal{C})$, the composites
    %%
    \begin{equation*}
        F\lhd G\lhd K,F\lhd H\lhd K:\mathcal{D}\rightarrow \cat{Ch}(\mathcal{A})
    \end{equation*}
    %%
    are naturally chain homotopy equivalent.
\end{lem}
\begin{proof}
    We separate this proof into two parts. Let $G,H$ be naturally chain homotopy equivalent, and let $F$ and $K$ be functors as in the question. Let $(h,g,s,r):H\simeq_{\cat{Ch}}G$ be a natural chain homotopy between our original functors. We begin the construction in the case of pre-composition by $K$:
    \begin{itemize}
        \setlength{\itemindent}{2em}
        \item[[$-\lhd K$]] From the definition of $\lhd$, the composites $H\lhd K$ and $G\lhd K$ are given by
        %%
        \begin{equation*}
            H\lhd K = N_\mathcal{B}\Delta_\mathcal{B}(\Gamma_\mathcal{B})_*H_*\Gamma_\mathcal{C}K\;\;\text{ and }\;\;G\lhd K = N_\mathcal{B}\Delta_\mathcal{B}(\Gamma_\mathcal{B})_*G_*\Gamma_\mathcal{C}K
        \end{equation*}
        %%
        First, note that since the homotopy is natural $H\lhd K\simeq G\lhd K$ by the results in Sections~\ref{subsec:simpHomotop} and~\ref{sec:doldKan}. 
        % Otherwise, if the homotopy is pointwise, showing $H\lhd K\simeq_{\cat{Ch}}G\lhd K$ for $K$ arbitrary is equivalent to showing $N_\mathcal{B}\Delta_\mathcal{B}(\Gamma_\mathcal{B}H)_* \simeq_{\cat{Ch}}N_\mathcal{B}\Delta_\mathcal{B}(\Gamma_\mathcal{B}G)_*$. Additionally, since $N_\mathcal{B}$ both reflects and preserves homotopies, this is equivalent to showing $\Delta_\mathcal{B}(\Gamma_\mathcal{B}H)*\simeq_{Ho}\Delta_\mathcal{B}(\Gamma_\mathcal{B}G)*$. Since $\Gamma$ also preserves homotopies, it is sufficient to prove Lemma \ref{lem:PrecompDiag} \textbf{YET TO BE SHOWN FOR POINTWISE}.
        \item[[$F\lhd -$]] The composites $F \lhd H$ and $F\lhd G$ are given by
        %%
        \begin{equation*}
            F\lhd H = N_\mathcal{A}\Delta_\mathcal{A}(\Gamma_\mathcal{A})_*F_*\Gamma_\mathcal{B}H\;\;\text{ and }\;\;F\lhd G = N_\mathcal{A}\Delta_\mathcal{A}(\Gamma_\mathcal{A})_*F_*\Gamma_\mathcal{B}G
        \end{equation*}
        %%
        Using the results in Sections~\ref{sec:doldKan} and~\ref{subsec:simpHomotop} we have the chain of implications
        %%
        \begin{align*}
            H \simeq_{\cat{Ch}}G &\implies \Gamma_\mathcal{B}H \simeq_{Ho} \Gamma_\mathcal{B}G \\
            &\implies F_*\Gamma_\mathcal{B}H \simeq_{Ho} F_*\Gamma_\mathcal{B}G \\
            &\implies (\Gamma_\mathcal{A})_*F_*\Gamma_\mathcal{B}H \simeq_{Ho} (\Gamma_\mathcal{A})_*F_*\Gamma_\mathcal{B}G \\
            &\implies \Delta_\mathcal{A}(\Gamma_\mathcal{A})_*F_*\Gamma_\mathcal{B}H \simeq_{Ho} \Delta_\mathcal{A}(\Gamma_\mathcal{A})_*F_*\Gamma_\mathcal{B}G \\
            &\implies N_\mathcal{A}\Delta_\mathcal{A}(\Gamma_\mathcal{A})_*F_*\Gamma_\mathcal{B}H \simeq_{\cat{Ch}} N_\mathcal{A}\Delta_\mathcal{A}(\Gamma_\mathcal{A})_*F_*\Gamma_\mathcal{B}G \\
        \end{align*}
        where each implication preserves naturality as well.
    \end{itemize}
\end{proof}


We now seek to upgrade the category $\cat{HoAbCat}_{\cat{Ch}}$ in \cite{BJORT} to natural chain homotopy equivalences.

\begin{defn}[label=defn:HoAbCat]{}
    There exists a (large) category $\cat{HoAbCat}_{\cat{Ch}}$ consisting of the following data:
    \begin{itemize}
        \item Objects are abelian categories
        \item Morphisms $\mathcal{B}\rightsquigarrow \mathcal{A}$ are natural chain homotopy equivalence classes of functors $\mathcal{B}\rightarrow \cat{Ch}(\mathcal{A})$
        \item Composition of maps $\mathcal{C}\rightsquigarrow \mathcal{B}$ and $\mathcal{B}\rightsquigarrow \mathcal{A}$, corresponding to functors $G:\mathcal{C}\rightarrow \cat{Ch}(\mathcal{B})$ and $F:\mathcal{B}\rightarrow \cat{Ch}(\mathcal{A})$ is defined by the equivalence class of the composite $F\lhd G$
    \end{itemize}
\end{defn}

This definition exists since the natural chain homotopy equivalences form a congruence relation on $\cat{AbCat}_{\cat{Ch}}$.



\clearpage

\section{The Taylor tower in abelian functor calculus}

With the appropriate categorical language built up, we can begin constructing Taylor towers, as in \cite{JohnsonB.2004Dcwc}. This process involves defining certain polynomial functors of \textbf{degree $n$} with natural transformations which go down by one in degree. In \cite{BJORT} many definitions are given in terms of pointwise chain homotopies. In order to introduce two-dimensional structures on the cat $\cat{HoAbCat}_{\cat{Ch}}$ we aim to upgrade these to natural homotopies. We distinguish where these changes are made by changing the text color to red for occurrences of natural which aren't originally in the text.

\begin{defn}[label=defn:4.1]{}
    A functor $F:\mathcal{B}\rightarrow \cat{Ch}(\mathcal{A})$ is \textbf{degree $n$} if $\text{cr}_{n+1}(F):\mathcal{B}^{n+1}\rightarrow \cat{Ch}(\mathcal{A})$ is \textbf{contractible}, i.e., \rd{natural} chain homotopy equivalent to zero. 
\end{defn}

Note that $\cat{cr}_k(F) \simeq_{\cat{Ch}} 0$ implies $\cat{cr}_\ell(F) \simeq_\cat{Ch} 0$ for any $\ell > k$. Indeed, since $G \oplus F \simeq_{\cat{Ch}} 0$ implies $F \simeq_{\cat{Ch}} 0$, this follows from the inductive definition. Consequently, functors of degree $k$ are also of degree $\ell$ for $\ell > k$.

From \cite[Defn 2.4]{JohnsonB.2004Dcwc}, chain complexes can be constructed from a pair of a comonad and an object in the category on which the comonad acts.
\begin{lem}[label=lem:comonadChain]
    Let $C:\text{Fun}(\mathcal{B},\mathcal{A})\rightarrow \text{Fun}(\mathcal{B},\mathcal{A})$ be a comonad on a functor category where $\mathcal{A}$ is an abelian category (so in particular $\text{Fun}(\mathcal{B},\mathcal{A})$ is abelian). Then $C$ induces a functor $$C^\cat{Ch}:\text{Fun}(\mathcal{B},\mathcal{A})\rightarrow \cat{Ch}(\text{Fun}(\mathcal{B},\mathcal{A}))$$
\end{lem}
\begin{proof}
    Let $\epsilon:C\rightarrow 1$ be the counit of the comonad. Then for $F :\mathcal{B}\rightarrow \mathcal{A}$, define $C^\cat{Ch}(F)$ to be the chain
    %%
    \begin{equation*}
        \cdots \rightarrow C^3(F)\xrightarrow{\epsilon_{C^2(F)}-C\epsilon_{C(F)}+C^2\epsilon_F}C^2(F)\xrightarrow{\epsilon_{C(F)}-C\epsilon_F}C(F)\xrightarrow{\epsilon_F} F
    \end{equation*}
    %%
    where the $k$th differential is defined by the alternating sum $\sum_{i=0}^{k-1}(-1)^iC^i\epsilon_{C^{(k-i)}}$. Note these differentials are indeed natural, and hence this defines a sequence of functors and natural transformations. To see that this sequence forms a chain complex observe that
    %%
    \begin{align*}
        \left(\sum_{i=0}^{n}(-1)^iC^i\epsilon_{C^{n-i}(F)}\right)&\circ \left(\sum_{i=0}^{n+1}(-1)^iC^i\epsilon_{C^{n+1-i}(F)}\right) \\
        &= \sum_{i=0}^n\sum_{j=0}^{n+1}(-1)^{i+j}C^i\epsilon_{C^{n-i}(F)}\circ C^j\epsilon_{C^{n+1-j}(F)} \\
        &= \sum_{i=0}^n\sum_{i < j}^{n+1}(-1)^{i+j}C^i\epsilon_{C^{n-i}(F)}\circ C^j\epsilon_{C^{n+1-j}(F)} \\
        &+ \sum_{i=0}^n\sum_{i\geq j}(-1)^{i+j}C^i\epsilon_{C^{n-i}(F)}\circ C^j\epsilon_{C^{n+1-j}(F)} \\
        &= \sum_{i=0}^n\sum_{i < j}^{n+1}(-1)^{i+j}C^i\left(\epsilon_{C^{n-i}(F)}\circ C^{j-i}\epsilon_{C^{n+1-j}(F)} \right)\\
        &+ \sum_{i=0}^n\sum_{i\geq j}(-1)^{i+j}C^j\left(C^{i-j}\epsilon_{C^{n-i}(F)}\circ \epsilon_{C^{n+1-j}(F)} \right)\\ 
        &= \sum_{i=0}^n\sum_{i\leq k}^n(-1)^{i+k+1}C^{i}\left(\epsilon_{C^{n-i}(F)}\circ C^{k+1-i}\epsilon_{C^{n-k}(F)} \right) \tag{Substituting $k = j-1$}\\
        &+ \sum_{i=0}^n\sum_{i\geq j}(-1)^{i+j}C^j\left(C^{i-j}\epsilon_{C^{n-i}(F)}\circ \epsilon_{C^{n+1-j}(F)} \right)\\ 
        &= \sum_{i=0}^n\sum_{i\leq k}^n(-1)^{i+k+1}C^{i}\left(C^{k-i}\epsilon_{C^{n-k}(F)}\circ \epsilon_{C^{n+1-i}(F)}\right) \tag{Naturality of $\epsilon$} \\
        &+ \sum_{i=0}^n\sum_{i\geq j}(-1)^{i+j}C^j\left(C^{i-j}\epsilon_{C^{n-i}(F)}\circ \epsilon_{C^{n+1-j}(F)} \right)\\ 
        &= -\sum_{k=0}^n\sum_{k\geq i}(-1)^{i+k}C^{i}\left(C^{k-i}\epsilon_{C^{n-k}(F)}\circ \epsilon_{C^{n+1-i}(F)}\right) \tag{Re-ordering the sum} \\
        &+ \sum_{i=0}^n\sum_{i\geq j}(-1)^{i+j}C^j\left(C^{i-j}\epsilon_{C^{n-i}(F)}\circ \epsilon_{C^{n+1-j}(F)} \right) \\
        &= 0
    \end{align*}
    Next, let $\alpha:F\Rightarrow G$ be a natural transformation. Then $C^\cat{Ch}(\alpha):C^\cat{Ch}(F)\rightarrow C^\cat{Ch}(G)$ is defined by $C^\cat{Ch}(\alpha)_n:= C^n\alpha:C^nF\rightarrow C^nG$. To see that this is a chain map observe that for $0 \leq i \leq n$ we have the commutative diagram
    \[\begin{tikzcd}
    	{C^{n+1}(F)} & {C^n(F)} \\
    	{C^{n+1}(G)} & {C^n(G)}
    	\arrow["{C^i\epsilon_{C^{n-i}(F)}}", from=1-1, to=1-2]
    	\arrow["{C^n(\alpha)}", from=1-2, to=2-2]
    	\arrow["{C^{n+1}(\alpha)}"', from=1-1, to=2-1]
    	\arrow["{C^i\epsilon_{C^{n-i}(G)}}"', from=2-1, to=2-2]
    \end{tikzcd}\]
    which commutes by naturality of $\epsilon$. Since composition is bilinear with respect to the group operation on hom sets in an abelian category we have that the $C^n(\alpha)$ form a chain map in $\cat{Ch}(\text{Fun}(\mathcal{B},\mathcal{A}))$. Further, since $C$ is a functor so is $C^\cat{Ch}$, completing the proof.
\end{proof}

Note that we can realize this functor as going to $\text{Fun}(\mathcal{B},\cat{Ch}(\mathcal{A}))$. Indeed, $\cat{Ch}(\text{Fun}(\mathcal{B},\mathcal{A}))\cong \text{Fun}(\mathcal{B},\cat{Ch}(\mathcal{A}))$, as we show in Lemma \ref{lem:funcChain}

\begin{lem}[label=lem:funcChain]
    For $\mathcal{A}$ an abelian category, we have an isomorphism of categories
    \begin{equation}\label{eq:ChainFunc}
        \cat{Ch}(\text{Fun}(\mathcal{B},\mathcal{A}))\cong \text{Fun}(\mathcal{B},\cat{Ch}(\mathcal{A}))
    \end{equation}
\end{lem}
\begin{proof}
    Define a functor $\gamma:\cat{Ch}(\text{Fun}(\mathcal{B},\mathcal{A}))\rightarrow \text{Fun}(\mathcal{B},\cat{Ch}(\mathcal{A}))$ given on a chain complex of functors $F_\bullet$ by
    %%
    \begin{equation*}
        \gamma(F_\bullet)(B)_n := F_n(B),\;\forall B \in \mathcal{B}
    \end{equation*}
    %%
    where the differentials are given by the natural transformation differentials in $F_\bullet$ evaluated at $B$. Given a map of chain complexes $\alpha_\bullet:F_\bullet\rightarrow G_\bullet$ we set
    %%
    \begin{equation*}
        (\gamma(\alpha_\bullet)_B)_n := (\alpha_n)_B
    \end{equation*}
    %%
    This defines a chain map $\gamma(F_\bullet)(B)\rightarrow \gamma(G_\bullet)(B)$ since $\alpha_\bullet$ is a chain map of natural transformations, so all squares with differentials commute. Further, $\gamma(\alpha_\bullet)$ is natural in $B$ since if $f:B\rightarrow B'$ is a map in $\mathcal{B}$, then in
    \[\begin{tikzcd}
    	& {F_{n+1}(B')} &&& {F_n(B')} \\
    	{F_{n+1}(B)} && {F_n(B)} \\
    	& {G_{n+1}(B')} &&& {G_n(B')} \\
    	{G_{n+1}(B)} && {G_n(B)}
    	\arrow["{\partial_{n+1}}"{pos=0.7}, from=2-1, to=2-3]
    	\arrow["{(\alpha_{n+1})_B}"', from=2-1, to=4-1]
    	\arrow["{\partial_{n+1}}"', from=4-1, to=4-3]
    	\arrow["{(\alpha_n)_B}"{pos=0.3}, from=2-3, to=4-3]
    	\arrow["{F_{n+1}(f)}", from=2-1, to=1-2]
    	\arrow["{F_n(f)}"', from=2-3, to=1-5]
    	\arrow["{\partial_{n+1}}", from=1-2, to=1-5]
    	\arrow["{(\alpha_n)_{B'}}", from=1-5, to=3-5]
    	\arrow["{G_n(f)}"', from=4-3, to=3-5]
    	\arrow["{G_{n+1}(f)}", from=4-1, to=3-2]
    	\arrow["{(\alpha_{n+1})_{B'}}"'{pos=0.6}, from=1-2, to=3-2]
    	\arrow["{\partial_{n+1}}"', from=3-2, to=3-5]
    \end{tikzcd}\]
    the front and back faces commute since $\alpha_\bullet$ is a chain map, the top and bottom faces commute by naturality of the boundary maps, and the side faces commute by naturality of the $\alpha_n$. Since this definition is in terms of the components of $\alpha_\bullet$ it is inherently functorial. 

    Next we must witness an inverse $\rho:\text{Fun}(\mathcal{B},\cat{Ch}(\mathcal{A}))\rightarrow \cat{Ch}(\text{Fun}(\mathcal{B},\mathcal{A}))$ functor. Given $F:\mathcal{B}\rightarrow \cat{Ch}(\mathcal{A})$ we set $\rho(F)$ to have $n$th component $(-)_n\circ F$ and differential $\partial_n$ given by components the $n$th differential of $F$ evaluated at $B  \in \mathcal{B}$. Naturality of the differential equates to the commutivity of 
    \[\begin{tikzcd}
    	{F(B)_n} & {F(B)_{n-1}} \\
    	{F(B')_n} & {F(B')_{n-1}}
    	\arrow["{F(f)_n}"', from=1-1, to=2-1]
    	\arrow["{\partial_n(B)}", from=1-1, to=1-2]
    	\arrow["{\partial_n(B')}"', from=2-1, to=2-2]
    	\arrow["{F(f)_{n-1}}", from=1-2, to=2-2]
    \end{tikzcd}\]
    for any $f:B\rightarrow B'$, which follows since $F(f)$ is a chain map. Next, if $\alpha:F\rightarrow G$ is a natural transformation between two such functors we set $\rho(\alpha)$ such that $\rho(\alpha)_n$ is the natural transformation defined by $(\rho(\alpha)_n)_B := (\alpha_B)_n$. Naturality and the chain condition follow by the commutivity of 
    \[\begin{tikzcd}
    	& {F(B')_{n+1}} &&& {F(B')_n} \\
    	{F(B)_{n+1}} && {F(B)_n} \\
    	& {G(B')_{n+1}} &&& {G(B')_n} \\
    	{G(B)_{n+1}} && {G(B)_n}
    	\arrow["{\partial_{n+1}(B)}"{pos=0.7}, from=2-1, to=2-3]
    	\arrow["{(\alpha_B)_{n+1}}"', from=2-1, to=4-1]
    	\arrow["{\partial_{n+1}(B)}"', from=4-1, to=4-3]
    	\arrow["{(\alpha_B)_n}"{pos=0.2}, from=2-3, to=4-3]
    	\arrow["{F(f)_{n+1}}", from=2-1, to=1-2]
    	\arrow["{F(f)_n}"', from=2-3, to=1-5]
    	\arrow["{\partial_{n+1}(B')}", from=1-2, to=1-5]
    	\arrow["{(\alpha_{B'})_n}", from=1-5, to=3-5]
    	\arrow["{G(f)_n}"', from=4-3, to=3-5]
    	\arrow["{G(f)_{n+1}}", from=4-1, to=3-2]
    	\arrow["{(\alpha_{B'})_{n+1}}"'{pos=0.6}, from=1-2, to=3-2]
    	\arrow["{\partial_{n+1}(B')}"', from=3-2, to=3-5]
    \end{tikzcd}\]
    where the bottom and top faces are the fact $G(f)$ and $F(f)$ are chain maps, the front and back faces are the fact $\alpha_B$ is a chain map, and finally the side faces are naturality of $\alpha$. Once again, since $\rho(\alpha)$ is defined in terms of the components of $\alpha$ the assignment is inherently functorial. Further, these operations are exactly inverse of each other as they correspond to swapping the element and natural number indices (in particular, on the other side of the Dold-Kan Equivalence this is simply the swap natural isomorphism on functors of two variables).
\end{proof}


Moving forward we write $\text{Fun}^\cat{Ch}$ for the isomorphism $\gamma$ in the proof. We can use this technique to define the polynomial approximations in the Taylor tower for a functor.

\begin{defn}[label=defn:4.2]{}
    The \textbf{$n$th polynomial approximation} is the composite functor $P_n := (\text{Tot}_\mathcal{A})_*\circ \text{Fun}^\cat{ch}\circ C_{n+1}^{\cat{Ch}}:\text{Fun}(\mathcal{B},\cat{Ch}(\mathcal{A}))\rightarrow \text{Fun}(\mathcal{B},\cat{Ch}(\mathcal{A}))$. 
\end{defn}


Recall by Lemma \ref{lem:idempotCr1} if $n = 0$, $C_1(F) = \text{cr}_1(\text{cr}_1(F))=\text{cr}_1(F)$ (by our choice in defining $\text{cr}_1$), and so $C_1^{\times k}(F) = \text{cr}_1(F)$ for each $k \geq 1$. Further, the co-unit $\epsilon:\Delta^*(\text{cr}_1(\text{cr}_1(F)))(X) = \text{cr}_1(F)(X)\rightarrow F(X)$ is the kernel map in $\text{cr}_1(F)\rightarrow F(X)\rightarrow F(\star)$. In the case of $\epsilon_{C_1}$ the kernel map is the identity from our definition of $\text{cr}_1$, while by the definition of $\text{cr}_1$ on maps we have
\[\begin{tikzcd}
	{\text{cr}_1(F)(X)} & {\text{cr}_1(F)(X)} & 0 \\
	{\text{cr}_1(F)(X)} & {F(X)} & {F(\star)}
	\arrow[Rightarrow, no head, from=1-1, to=1-2]
	\arrow["{!}", from=1-2, to=1-3]
	\arrow["{\text{cr}_1(\epsilon)}"', from=1-1, to=2-1]
	\arrow["\epsilon"', from=1-2, to=2-2]
	\arrow[from=2-2, to=2-3]
	\arrow[from=1-3, to=2-3]
	\arrow["\epsilon"', tail, from=2-1, to=2-2]
\end{tikzcd}\]
so $C_1(\epsilon)$ is the identity as well since $\epsilon$ is monic. Hence, we obtain the chain complex of functors
%%
\begin{equation*}
    \cdots \xrightarrow{0}\text{cr}_1(F)\xrightarrow{\id}\text{cr}_1(F)\xrightarrow{0}\text{cr}_1(F)\xrightarrow{\epsilon}F
\end{equation*}
%%
Since $F \cong \text{cr}_1(F)\oplus F(\star)$, the chain complex defining $P_0(F)$ can be written as a direct sum of the two chain complexes
%%
\begin{equation*}
    \cdots \xrightarrow{0}\text{cr}_1(F)\xrightarrow{\id}\text{cr}_1(F)\xrightarrow{0}\text{cr}_1(F)\xrightarrow{\id}\text{cr}_1(F)
\end{equation*}
%%
and
%%
\begin{equation*}
    \cdots\rightarrow 0\rightarrow 0 \rightarrow 0 \rightarrow F(\star)
\end{equation*}
%%
Note that the chain complex in the top line is contractible. Recall the totalization commutes with direct sums. The totalization of the second complex is isomorphic to $F(\star)$ itself. On the other hand, the totalization of the first complex, $C_\bullet$, has $C_0 = \text{cr}_1(F)_0$, $C_1 = \text{cr}_1(F)_1\oplus \text{cr}_1(F)_0$, and in general
%%
\begin{equation*}
    C_n = \bigoplus_{i=0}^n\text{cr}_1(F)_i
\end{equation*}
%%
with differential $\partial_n:C_n\rightarrow C_{n-1}$ given by 
%%
\begin{equation*}
    \partial_n = (\delta_{n-i,even}\pi_{C_{i-1}}-\partial_i^{\text{cr}_1(F)}\circ \pi_{C_i})_{1\leq i \leq n}
\end{equation*}
%%
where $\delta_{n-i,even}$ is $1$ when $2\mid n-i$ and $0$ else. Then $P_0(F)$ is the direct sum of these two sequences. However, since the first complex before totalization is contractible, we can model $P_0(F)(X) \cong F(\star)$.

We test the above computation, and the results to follow, using the example of $\deg_0^\mathcal{A}:\mathcal{A}\rightarrow \cat{Ch}(\mathcal{A})$

\begin{eg}{}
    First, note that the chain complex $\deg_0^\mathcal{A}(0)$ is the zero complex. Since $\deg_0^\mathcal{A}$ is reduced we also have that $\text{cr}_1(\deg_0^\mathcal{A}) = \deg_0^\mathcal{A}$. It follows that $P_0(\deg_0^\mathcal{A})_n = 1_\mathcal{A}$ for each $n \geq 0$, and $\partial_n = \delta_{n-1,even}1_{\mathcal{A}}$, or in other words
    %%
    \begin{equation*}
        P_0(\deg_0^\mathcal{A}) := \cdots \xrightarrow{1_{1_\mathcal{A}}} 1_\mathcal{A}\xrightarrow{0} 1_\mathcal{A}\xrightarrow{1_{1_\mathcal{A}}} 1_\mathcal{A}
    \end{equation*}
    %%
    Note that this complex is contractible.
\end{eg}

We now prove a preliminary result on exactness of totalization.

\begin{lem}[label=lem:TotExact]
    The totalization functor $\text{Tot}:\cat{Ch}^2(\mathcal{A})\rightarrow \cat{Ch}(\mathcal{A})$ is exact.
\end{lem}
\begin{proof}
    Let 
    %%
    \begin{equation*}
        0\rightarrow A_1\xrightarrow{f_1} A_2\xrightarrow{f_2} A_3\rightarrow 0
    \end{equation*}
    %%
    be a short exact sequence of bicomplexes in $\mathcal{A}$. This becomes a sequence of complexes 
    %%
    \begin{equation*}
        \text{Tot}(A_1)\xrightarrow{\text{Tot}(f_1)}\text{Tot}(A_2)\xrightarrow{\text{Tot}(f_2)}\text{Tot}(A_3)
    \end{equation*}
    %%
    where at a given $n$,
    %%
    \begin{equation*}
        \text{Tot}(A_i)_n = \bigoplus_{j=0}^n(A_i)_{j,n-j}
    \end{equation*}
    %%
    and
    %%
    \begin{equation*}
        \text{Tot}(f_i)_n = \bigoplus_{j=0}^n (f_i)_{j,n-j}
    \end{equation*}
    %%
    Note that the sequence of bicomplexes being exact means that each component sequence in $\mathcal{A}$ is exact. Then the component sequence of the totalization at $n$ is a finite direct sum of exact sequences, and hence exact.
\end{proof}

\begin{lem}[label=pushForward]
    Let $F:\mathcal{B}\rightarrow \mathcal{C}$ be an exact functor between abelian categories. Then for a category $\mathcal{A}$, $F_*:\text{Fun}(\mathcal{A},\mathcal{B})\rightarrow \text{Fun}(\mathcal{A},\mathcal{C})$ is exact.
\end{lem}
\begin{proof}
    Let $0\rightarrow G_1\xrightarrow{\eta_1} G_2\xrightarrow{\eta_2} G_3\rightarrow 0$ be a short exact sequence of functors from $\mathcal{A}$ to $\mathcal{B}$. Since abelian categories are finitely complete and cocomplete, finite limits and colimits in $\text{Fun}(\mathcal{A},\mathcal{B})$ and $\text{Fun}(\mathcal{A},\mathcal{C})$ are computed pointwise, so it is sufficient to prove the lemma at a given $A \in \mathcal{A}$. This follows by exactness of $F$.
\end{proof}

\begin{lem}[label=lem:exactPresHomotop]
    Let $F:\cat{Ch}(\mathcal{A})\to \cat{Ch}(\mathcal{B})$ be an exact functor. Then $F$ preserves chain homotopies.
\end{lem}
\begin{proof}
    Let $f,g:A_\bullet\to A'_\bullet$ be chain maps and let $s_n:A_n\to A'_{n+1}$ be the components of a chain homotopy from $f$ to $g$, so $\partial_{n+1}^{A'}\circ s_n + s_{n-1}\circ \partial_n^A = f_n-g_n$ for each $n$. 
\end{proof}

Due to Proposition \ref{prop:exactCross} we obtain nice properties for the functors $P_n:\text{Fun}(\mathcal{B},\mathcal{A})\rightarrow \text{Fun}(\mathcal{B},\mathcal{A})$.

\begin{prop}[label=prop:exactPol]
    For any $n \geq 0$,
    \begin{itemize}
        \item[(i)] $P_n:\text{Fun}(\mathcal{B},\cat{Ch}(\mathcal{A}))\rightarrow \text{Fun}(\mathcal{B},\cat{Ch}(\mathcal{A}))$ is exact
        \item[(ii)] $P_n$ preserves chain homotopies, chain homotopy equivalences, and contractibility. \textbf{(only pointwise or also natural?)}
    \end{itemize}
\end{prop}
\begin{proof}
    It is sufficient to prove (i) for short exact sequences. Let $0 \rightarrow F\rightarrow G\rightarrow H\rightarrow 0$ be a SES of functors in $\cat{AbCat}_{\cat{Ch}}$. By Proposition \ref{prop:exactCross} we obtain a SES of bicomplexes in the definition of the $n$th polynomial approximation. Since totalization is exact we obtain a SES $0 \rightarrow P_n(F)\rightarrow P_n(G)\rightarrow P_n(H)\rightarrow 0$. (ii) follows from (i) since exact functors preserve chain homotopies and zero chain complexes.
\end{proof}

For each $F:\mathcal{B}\rightarrow \cat{Ch}(\mathcal{A})$, the functor $P_n(F)$ comes equipped with a natural transformation $p_n:F\rightarrow P_n(F)$ defined by inclusion into the degree zero part of the chain complex $P_n(F)$. Explicitly, can define $p_n:\mathbb{1}\Rightarrow P_n$ as done by Jason Parker:

\begin{rmk}[label=defn:littlepN]
    First we define a natural transformation $i:(\deg^{\cat{Ch}(\mathcal{A})})_*\Rightarrow \text{Fun}^{\cat{Ch}}\circ C_{n+1}^\cat{Ch}:\text{Fun}(\mathcal{B},\cat{Ch}(\mathcal{A}))\rightarrow \text{Fun}(\mathcal{B},\cat{Ch}^2(\mathcal{A}))$, and then we define $p_n := (\text{Tot}_\mathcal{A})_*\circ i$ along with Lemma \ref{lem:compIdTot}. 


    For $F:\mathcal{B}\rightarrow \cat{Ch}(\mathcal{A})$ we define $i_F:\deg^{\cat{Ch}(\mathcal{A})}\circ F\Rightarrow \text{Fun}^{\cat{Ch}}\circ C_{n+1}^\cat{Ch}(F):\mathcal{B}\Rightarrow \cat{Ch}^2(\mathcal{A})$ where for each $B \in \mathcal{B}_0$, we define $i_{F,B}:\deg^{\cat{Ch}(\mathcal{A})}(FB)\rightarrow (\text{Fun}^{\cat{Ch}}(C_{n+1}^\cat{Ch}(F)))B$ in $\cat{Ch}^2(\mathcal{A})$ by saying for all $m \geq 0$,
    %%
    \begin{equation*}
        (i_{F,B})_m = \left\{\begin{array}{cc} 0 & m > 0 \\
        1_{FB}:FB\rightarrow FB & m = 0 \end{array}\right.
    \end{equation*}
    %%
    since $\text{Fun}^{\cat{Ch}}(C_{n+1}^\cat{Ch}(F)))(B)_0 = C_{n+1}^0(F)(B) = F(B)$. These form appropriate chain maps which are natural in $B$ and $F$ (\textbf{Maybe expand on this when have time?}). Then explicitly
\end{rmk}


\begin{lem}[label=lem:compIdTot]
    We have a natural isomorphism
    \begin{equation*}
        (\text{Tot}_\mathcal{A})_*\circ \deg^{\cat{Ch}(\mathcal{A})} \cong \mathbb{1}_{\cat{Ch}(\mathcal{A})}
    \end{equation*}
\end{lem}
\begin{proof}
    Let $A \in \cat{Ch}(\mathcal{A})$. Then for $n \geq 0$
    %%
    \begin{equation*}
        (\text{Tot}_\mathcal{A})_*\circ \deg^{\cat{Ch}(\mathcal{A})}(A)_n = \bigoplus_{i+j=n}\deg^{\cat{Ch}((\mathcal{A})}(A)_i)_j \cong A_n
    \end{equation*}
    %%
    since all other terms are zero. This isomorphism is given uniquely by the universal property of the biproduct and zero map, so induces the desired isomorphism in the statement of the Lemma.
\end{proof}

In order to show some basic properties of the approximation functor we first prove the following Lemma related to the behaviour of the cross effect functor.

\begin{lem}[label=lem:crossNatTrans]
    Let $\mathcal{B}$ be a pointed category and let $\mathcal{A}$, $\mathcal{C}$ be abelian categories. Then if $F:\mathcal{A}\rightarrow \mathcal{C}$ is an exact functor we have a natural isomorphism
    %%
    \begin{equation*}
        \text{cr}_n^{\mathcal{B},\mathcal{C}}\circ F_*\cong F_*\circ \text{cr}_n^{\mathcal{B},\mathcal{A}}
    \end{equation*}
    where $\text{cr}_n^{\mathcal{B},-}:\text{Fun}(\mathcal{B},-)\rightarrow \text{Fun}_*(\mathcal{B}^n,-)$ specifies the codomain category.
\end{lem}
Actually, this statement is true whenever $F$ preserves direct sums. However we will only use it for exact functors.
\begin{proof}
    We will prove this by induction. For the base case on objects let $G: \mathcal B \to \mathcal A$ be a functor. Then (by the definition of the cross-effect)  $G(X)\cong G(*) \oplus\text{cr}_1^{\mathcal B, \mathcal A}G(X)$ for every object $X$ of $\mathcal B$. Applying $F$ to this equality and using that $F$ preserves direct sums, we obtain that 
    $$
    F\circ G (X) \cong F(G(X)\cong G(*)\oplus  \text{cr}_1^{\mathcal B, \mathcal A}G(X)) \cong F( G(*)) \oplus F(\text{cr}_1^{\mathcal B, \mathcal A}G(X))
    $$
    Applying the definition of the cross-effect to this we obtain that
    $$\text{cr}_n(F \circ G)(X) \cong F(\text{cr}_1^{\mathcal B, \mathcal A}G(X)).$$
    As $\text{cr}_n(F \circ G)$ send a morphisms $f$ of $\mathcal B$ to the unique induced map into the limit that is the cross-effect and $F(\text{cr}_1^{\mathcal B, \mathcal A}G(f)$ gives one such map, $\text{cr}_n^{\mathcal B, \mathcal C} \circ F_*(G) \cong F_* \circ \text{cr}_n^{\mathcal B, \mathcal A}(G)$ as functors $\mathcal B \to \mathcal C$.

    For the base case on morphisms, let $\varphi: G \Rightarrow G'$ be a natural transformation
    between functors $G,G': \mathcal B \to \mathcal A$. Then the component $\varphi_X$ corresponds to $\varphi_* \oplus \text{cr}_1^{\mathcal B, \mathcal A}\varphi_X$ in the sense that 
% https://q.uiver.app/#q=WzAsNCxbMCwwLCJHKFgpIl0sWzAsMSwiRygqKSBcXG9wbHVzIFxcdGV4dHtjcn1fMV57XFxtYXRoY2FsIEIsIFxcbWF0aGNhbCBBfUcoWCkiXSxbMiwxLCJHJygqKSBcXG9wbHVzIFxcdGV4dHtjcn1fMV57XFxtYXRoY2FsIEIsIFxcbWF0aGNhbCBBfUcnKFgpIl0sWzIsMCwiRycoWCkiXSxbMCwzLCJcXHZhcnBoaV9YIl0sWzEsMiwiXFx2YXJwaGlfKiBcXG9wbHVzIChcXHRleHR7Y3J9XzFee1xcbWF0aGNhbCBCLCBcXG1hdGhjYWwgQX1cXHZhcnBoaSlfWCJdLFswLDEsIlxcY29uZyIsMl0sWzMsMiwiXFxjb25nIl1d
\[\begin{tikzcd}
	{G(X)} && {G'(X)} \\
	{G(*) \oplus \text{cr}_1^{\mathcal B, \mathcal A}G(X)} && {G'(*) \oplus \text{cr}_1^{\mathcal B, \mathcal A}G'(X)}
	\arrow["{\varphi_X}", from=1-1, to=1-3]
	\arrow["{\varphi_* \oplus (\text{cr}_1^{\mathcal B, \mathcal A}\varphi)_X}", from=2-1, to=2-3]
	\arrow["\cong"', from=1-1, to=2-1]
	\arrow["\cong", from=1-3, to=2-3]
\end{tikzcd}\]
    commutes (this is how the cross-effect is defined on morphisms). Applying $F$ to this we can read off that $F(\varphi)_X$ corresponds to $F(\varphi_*) \oplus F(\text{cr}_1^{\mathcal B, \mathcal A}(\varphi)_X)$, so by the definition of the cross-effect $\text{cr}_1^{\mathcal B, \mathcal C}(F \circ \varphi) \cong F(\text{cr}_1^{\mathcal B, \mathcal A}(\varphi))$.
    
    For the inductive step, let the statement be true for $\text{cr}_n-1$ (as it will be analogous to the base case we will only sketch this part). Then the definition of the cross-effect tells us 
    $$
    \text{cr}^{\mathcal B, \mathcal A}_{n-1}G(X_1 \vee X_2 , X_3 ,...) = 
    \text{cr}^{\mathcal B, \mathcal A}_{n-1}G(X_1 , X_3 ,...) \oplus \text{cr}^{\mathcal B, \mathcal A}_{n-1}G(X_1 , X_3 ,...) \oplus 
    \text{cr}^{\mathcal B, \mathcal A}_{n}G(X_1, X_2 , X_3 ,...).
    $$
    Applying $F$ and the inductive hypothesis, we obtain
    $$
    \text{cr}^{\mathcal B, \mathcal C}_{n-1}(F \circ G)(X_1 \vee X_2 , X_3 ,...) = 
    \text{cr}^{\mathcal B, \mathcal C}_{n-1}(F \circ G)(X_1 , X_3 ,...) \oplus \text{cr}^{\mathcal B, \mathcal C}_{n-1}(F \circ G)(X_1 , X_3 ,...) \oplus 
    F(\text{cr}^{\mathcal B, \mathcal A}_{n}G(X_1, X_2 , X_3 ,...)).
    $$    
    from which we can see (by the definition of the cross-effect) that
    $$
    \text{cr}_n^{\mathcal B, \mathcal C}(F \circ G)(X_1, X_2 , X_3 ,...) =
    F(\text{cr}^{\mathcal B, \mathcal A}_{n}G(X_1, X_2 , X_3 ,...))$$
    Again the uniqueness of the induced map into a limit gives us that 
    $\text{cr}_n^{\mathcal B, \mathcal C}(F \circ G)=
    F \circ \text{cr}^{\mathcal B, \mathcal A}_{n}G$. 
    In order to do the induction step on morphisms, let $\varphi: G \to G'$ be a natural transformation. Then, by the definition of the cross-effect on morphisms, $    (\text{cr}^{\mathcal B, \mathcal A}_{n-1} \varphi)_{X_1 \vee X_2 , ...} $ corresponds to $(\text{cr}^{\mathcal B, \mathcal A}_{n-1} \varphi)_{X_1 , X_3 ...} \oplus (\text{cr}^{\mathcal B, \mathcal A}_{n-1} \varphi)_{X_2 , X_3 ...} \oplus (\text{cr}^{\mathcal B, \mathcal A}_{n} \varphi)_{X_1, X_2 , ...}$. Applying $F$ and using the inductive hypothesis, we can read off that 
    $$
  (\text{cr}^{\mathcal B, \mathcal A}_{n} F(\varphi))_{X_1, X_2 , ...} =   F (\text{cr}^{\mathcal B, \mathcal A}_{n} \varphi)_{X_1, X_2 , ...}
    $$
    which proves the inductive step for morphisms.
\end{proof}

The basic properties of this approximation are given in the following proposition.

\begin{prop}[label=prop:4.5]
    For $F:\mathcal{B}\rightarrow \cat{Ch}(\mathcal{A})$,
    %%
    \begin{itemize}
        \item[(i)] The functor $P_n(F)$ is degree $n$
        \item[(ii)] If $F$ is degree $n$, then the map $p_n:F\rightarrow P_n(F)$ is a chain homotopy equivalence (\rd{natural})
        \item[(iii)] The pair $(P_n(F),p_n:F\rightarrow P_n(F))$ is universal up to chain homotopy equivalence with respect to degree $n$ functors receiving natural transformations from $F$.
    \end{itemize}
\end{prop}

In order to prove part (i) of Proposition \ref{prop:4.5} we require a certain compatibility of $\text{cr}_n^{\mathcal{B},\cat{Ch}(\mathcal{A})}$ with $\text{Fun}^\cat{Ch}$. First we need to upgrade functors to functors on chains in a naive manner.

\begin{lem}[label=lem:funcActChain]
    Let $\text{Fun}_{Add}(\mathcal{A},\mathcal{C})$ be the category of additive functors between abelian categories with all natural transformations. Then we have a functor
    %%
    \begin{equation*}
        \cat{Ch}:\text{Fun}_{Add}(\mathcal{A},\mathcal{C})\rightarrow \text{Fun}_{Add}(\cat{Ch}(\mathcal{A}),\cat{Ch}(\mathcal{C}))
    \end{equation*}
    given by sending functors to their action componentwise.
\end{lem}
\begin{proof}
    Let $\mathcal{F} \in \text{Fun}_{Add}(\mathcal{A},\mathcal{C})$. Since $\mathcal{F}$ is additive it preserves $0$'s and hence sends chain complexes to chain complexes. Then let $f_\bullet:A_\bullet\rightarrow A_\bullet'$ be a map of chain complexes. Then $\cat{Ch}(\mathcal{F})(f_\bullet)_n := \mathcal{F}(f_n)$, and since $\mathcal{F}$ is additive
    %%
    \begin{equation*}
        \mathcal{F}(f_n)\mathcal{F}(\partial_{n+1}^A)-\mathcal{F}(\partial_n^{A'})\mathcal{F}(f_{n+1}) = \mathcal{F}(f_n\partial_{n+1}^A-\partial_n^{A'}f_{n+1}) = \mathcal{F}(0) = 0
    \end{equation*}
    %%
    so $\cat{Ch}(\mathcal{F})(f_\bullet)$ is a chain map. Further, since $\cat{Ch}(\mathcal{F})$ is defined componentwise and $\mathcal{F}$ is a functor and additive, $\cat{Ch}(\mathcal{F})$ is a functor and additive. 


    Next, let $\eta:\mathcal{F}\rightarrow \mathcal{G}$ be a natural transformation between additive functors. Then define $\cat{Ch}(\eta)_{A_\bullet}:\cat{Ch}(\mathcal{F})(A_\bullet)\rightarrow \cat{Ch}(\mathcal{G})(A_\bullet)$ by $(\cat{Ch}(\eta)_{A_\bullet})_n := \eta_{A_n}$. Then $\cat{Ch}(\eta)_{A_\bullet}$ is a chain map by naturality of $\eta$. Further, $\cat{Ch}(\eta)$ is natural again by naturality of $\eta$, which makes the following diagram commute for $f_\bullet:A_\bullet\rightarrow A_\bullet$:
    \[\begin{tikzcd}
    	{\mathcal{F}(A_n)} & {\mathcal{F}(B_n)} \\
    	{\mathcal{G}(A_n)} & {\mathcal{G}(B_n)}
    	\arrow["{\eta_{A_n}}"', from=1-1, to=2-1]
    	\arrow["{\mathcal{G}(f_n)}"', from=2-1, to=2-2]
    	\arrow["{\eta_{B_n}}", from=1-2, to=2-2]
    	\arrow["{\mathcal{F}(f_n)}", from=1-1, to=1-2]
    \end{tikzcd}\]
    Since $\cat{Ch}(\eta)$ is defined componentwise it preserves composites and identities.
\end{proof}

Next we show compatibility of this functor with the isomorphism $\text{Fun}^\cat{Ch}$.

\begin{lem}[label=lem:ChFuncCommute]
    For any pointed category $\mathcal{B}$ and abelian category $\mathcal{A}$, we have a natural isomorphism
    %%
    \begin{equation*}
        \text{cr}_n^{\mathcal{B},\cat{Ch}(A)}\circ \text{Fun}^\cat{Ch} \cong \text{Fun}^\cat{Ch}\circ \cat{Ch}(\text{cr}_n^{\mathcal{B},\mathcal{A}})
    \end{equation*}
    %%
\end{lem}
\begin{proof}
    Since the cross effect is additive we can apply $\cat{Ch}$, so the claim is well-posed. Let $F\in\cat{Ch}(\text{Fun}(\mathcal{B},\mathcal{A}))$ be a chain complex of functors. Since finite limits in functor categories between abelian categories are computed pointwise, up to natural isomorphism, we have that 
    %%
    \begin{equation*}
        \text{cr}_n^{\mathcal{B},\cat{Ch}(A)}\circ \text{Fun}^\cat{Ch} \cong \text{Fun}^\cat{Ch}\circ \cat{Ch}(\text{cr}_n^{\mathcal{B},\mathcal{A}})
    \end{equation*}
\end{proof}

Finally, one last result we will require that is used in the proof of (iii) is the following computation for a functor $F:\mathcal{B}\rightarrow \cat{Ch}(\mathcal{A})$.

\begin{rmk}
    By construction $p_{n,P_n(F)}$ is the inclusion of $P_n(F)$ into $P_n(P_n(F))$ via the totalization after inclusion into the degree zero part of the bicomplex defining $P_n(P_n(F))$. On the other hand, applying $P_n$ to $p_{n,F}$ \textbf{TBC}
\end{rmk}

\begin{proof}[Proof of Proposition \ref{prop:4.5}]
    Let $F_k:\mathcal{B}\rightarrow \mathcal{A}$ be the $k$th degree component of $F:\mathcal{B}\rightarrow \cat{Ch}(\mathcal{A})$ (under the isomorphism $\text{Fun}^\cat{Ch}$). To prove (i) we show $\text{cr}_{n+1}(P_n(F))$ is contractible (i.e. \rd{naturally} contractible). By Lemma \ref{lem:crossNatTrans} 
    %%
    \begin{equation*}
        \text{cr}_{n+1}^{\mathcal{B},\cat{Ch}(\mathcal{A})}\circ (\text{Tot}_\mathcal{A})_*\circ \text{Fun}^\cat{Ch}\circ C_{n+1}^\cat{Ch} \cong (\text{Tot}_\mathcal{A})_*\circ \text{cr}_{n+1}^{\mathcal{B},\cat{Ch}^2(\mathcal{A})}\circ \text{Fun}^\cat{Ch}\circ C_{n+1}^\cat{Ch}
    \end{equation*}
    %%
    Since $\text{Tot}$ preserves natural homotopies it is sufficient to show that the cross effect for the bicomplex $\text{cr}_{n+1}^{\mathcal{B},\cat{Ch}^2(\mathcal{A})}\circ \text{Fun}^\cat{Ch}\circ C_{n+1}^\cat{Ch}(F)$ defining $P_n(F)$ is contractible. By Lemma \ref{lem:ChFuncCommute} this is equivalent to showing $\text{Fun}^\cat{Ch}\circ\cat{Ch}(\text{cr}_{n+1}^{\mathcal{B},\cat{Ch}(\mathcal{A})})\circ C_{n+1}^\cat{Ch}(F)$ is contractible. Then the $k$th row of this bicomplex is given by
    %%
    \begin{equation*}
        \cdots \rightarrow\text{cr}_{n+1}^{\mathcal{B},\mathcal{A}}C_{n+1}^{\times2}(F_k)\xrightarrow{\text{cr}_{n+1}^{\mathcal{B},\mathcal{A}}(\epsilon_{C_{n+1}}-C_{n+1}\epsilon)}\text{cr}_{n+1}^{\mathcal{B},\mathcal{A}}C_{n+1}(F_k)\xrightarrow{\text{cr}_{n+1}^{\mathcal{B},\mathcal{A}}\epsilon}\text{cr}_{n+1}^{\mathcal{B},\mathcal{A}}(F_k)
    \end{equation*}
    %%
    By Lemma \ref{lem:contractHomotop} we have a family of horizontal contractions for each row after applying $\text{cr}_{n+1}$, denoted $s^{k,h}$. Setting the vertical contractions, $s^v$, to be zero, we obtain a natural contraction for the bicomplex, so under the totalization we obtain a natural contraction for the chain complex $\text{cr}_{n+1}(P_n(F))$, as desired. 

    \vspace{10pt}

    For (ii) let $F$ be of degree $n$, so $\text{cr}_{n+1}(F)$ is naturally contractible. Recall the $k$th column of the bicomplex defining $P_n(F)$ is $C_{n+1}^{\times k}(F) = (\Delta^*\text{cr}_{n+1})^k(F)$. The map $p_n:F\rightarrow P_n(F)$ is the natural inclusion of the $0$th column into the totalization. Note that $C_{n+1}^{\times k}(F)$ is contractible for each $k$ since $C_{n+1}$ is exact and $F$ is of degree $n$. Then by Corollary A.7 in \cite{BJORT} the map $p_n$ is a \rd{natural} chain homotopy equivalence. (\textbf{DETAILS TO BE ADDED IN SEPARATE SECTION})

    \vspace{10pt}

    To show (iii) let $\tau:F\rightarrow G$ be a natural transformation transformation where $G$ is a functor of degree $n$. By naturality of $p_n$ in the functor $F$ we have a commutative diagram
    %%
    \[\begin{tikzcd}
    	F & G \\
    	{P_n(F)} & {P_n(G)}
    	\arrow["\tau", from=1-1, to=1-2]
    	\arrow["{P_n(\tau)}"', from=2-1, to=2-2]
    	\arrow["{p_{n,F}}"', from=1-1, to=2-1]
    	\arrow["{p_{n,G}}", from=1-2, to=2-2]
    \end{tikzcd}\]
    %%
    where the right hand $p_{n,G}$ is a natural chain homotopy equivalence by (ii). Let $s_{n,G}$ denote a natural chain homotopy inverse of $p_{n,G}$. Setting $\tau^\# = s_{n,G}\circ P_n(\tau)$ we have that
    %%
    \begin{equation*}
        \tau^\#\circ p_{n,F} = s_{n,G}\circ P_n(\tau)\circ p_{n,F} = s_{n,G}\circ p_{n,G}\circ \tau \simeq_{\cat{Ch},Nat} \tau
    \end{equation*}
    %%
    This shows $\tau$ factors through $p_n:F\rightarrow P_n(F)$ up to \rd{natural} chain homotopy equivalence. 
    
    
    To show uniqueness suppose $\sigma:P_n(F)\rightarrow G$ is another map such that $\tau$ is naturally chain homotopy equivalent to $\sigma\circ p_{n,F}$. Then by naturality of the $p_n$, we have a commuting diagram
    %%
    \[\begin{tikzcd}
    	F & {P_n(F)} & G \\
    	{P_n(F)} & {P_n(P_n(F))} & {P_n(G)}
    	\arrow["{p_{n,F}}"', from=1-1, to=2-1]
    	\arrow["{P_np_{n,F}}"', from=2-1, to=2-2]
    	\arrow["{p_{n,F}}", from=1-1, to=1-2]
    	\arrow["{p_{n,P_n(F)}}", from=1-2, to=2-2]
    	\arrow["\sigma", from=1-2, to=1-3]
    	\arrow["{p_{n,G}}", from=1-3, to=2-3]
    	\arrow["{P_n(\sigma)}"', from=2-2, to=2-3]
    \end{tikzcd}\]
    %%
    
    
    The above factorization applied to $p_n$ and $\sigma$ give factorizations $P_n((p_n)_F)\circ (p_n)_F = (p_n)_F\circ (p_n)_{P_n(F)}$ and $P_n(\sigma)\circ (p_n)_{P_n(F)}=(p_n)_G\circ \sigma$, where $(p_n)_{P_n(F)}$ and $(p_n)_G$ are natural chain homotopy equivalences. 
\end{proof}

Note that the universal property in bullet (iii) of Proposition \ref{prop:4.5}, along with the fact that degrees of a functor are characterized by their minimal value, we obtain a factorization (up to homotopy in general)
%%
\[\begin{tikzcd}
	&& F \\
	\cdots & {P_{n+1}(F)} & {P_n(F)} & {P_{n-1}(F)} & \cdots & {P_0(F)}
	\arrow["{p_0}", from=1-3, to=2-6]
	\arrow["{q_1}"', from=2-5, to=2-6]
	\arrow["{p_n}"', from=1-3, to=2-3]
	\arrow["{p_{n-1}}"{description}, from=1-3, to=2-4]
	\arrow[from=2-4, to=2-5]
	\arrow["{q_n}"', from=2-3, to=2-4]
	\arrow["{p_{n+1}}"{description}, from=1-3, to=2-2]
	\arrow["{q_{n+1}}"', from=2-2, to=2-3]
	\arrow[from=1-3, to=2-1]
	\arrow[from=2-1, to=2-2]
\end{tikzcd}\]
%%
which is known as the \textbf{algebraic Taylor tower} of the functor $F$, where the $q_n$ are determined uniquely by the universal property of the $P_n(F)$. We can also realize the $q_n:P_n(F)\rightarrow P_{n-1}(F)$ as being induced by a natural transformation $\rho_n:C_{n+1}\Rightarrow C_n:\text{Fun}(\mathcal{B},\mathcal{A})\rightarrow \text{Fun}(\mathcal{B},\mathcal{A})$, with components given by
%%
\[\begin{tikzcd}
	{C_{n+1}(F)(X)=\text{cr}_{n+1}(\Delta^*F)(X)} & {\text{cr}_n(F)(X\oplus X,X,...,X)} & {\text{cr}_n(\Delta^*F)(X)=C_n(F)(X)}
	\arrow[tail, from=1-1, to=1-2]
	\arrow["{\text{cr}_n(F)(+,1_X,...,1_X)}", outer sep = 5pt, from=1-2, to=1-3]
\end{tikzcd}\]
%%
\begin{proof}[Proof of construction of $q_n$]
    Let $\rho_n:C_{n+1}\Rightarrow C_n$ be the natural transformation described above. Then we have natural transformations $\rho_n([k]):C_{n+1}^{\times (k+1)}\Rightarrow C_n^{\times (k+1)}$ given by
    %%
    \begin{equation*}
        \rho_n([k]) = C_n^{\times k}\rho_n \circ C_n^{\times (k-1)}(\rho_n)_{C_{n+1}}\circ \cdots \circ C_n(\rho_n)_{C_{n+1}^{\times (k-1)}}\circ(\rho_n)_{C_{n+1}^{\times k}}
    \end{equation*}
    %%
    These natural transformations define a natural map of the bicomplexes defining $P_n$ and $P_{n-1}$. Applying totalization we obtain the natural transformation $q_n:P_n\Rightarrow P_{n-1}$ described. Since the map of bicomplexes defining $q_n$ is the identity on the zeroth column given by the identity, it follows that the desired triangle commutes.
\end{proof}
%%


\begin{eg}{}
    Note that for $\deg_0^\mathcal{A}$, since $\text{cr}_1(\deg_0^\mathcal{A}) = \deg_0^\mathcal{A}$ is the identity concentrated in degree zero, by the inductive definition $\text{cr}_n(\deg_0^\mathcal{A})\cong 0$ for $n \geq 2$. Then the bicomplex defining $P_n(\deg_0^\mathcal{A})$ has $\deg_0^\mathcal{A}$ as the $0$th column, with all other columns zero. This implies that $P_n(\deg_0^\mathcal{A}) = \deg_0^\mathcal{A}$ for $n \geq 1$.

    The map $q_n(\deg_0^\mathcal{A}):P_n(\deg_0^\mathcal{A})\rightarrow P_{n-1}(\deg_0^\mathcal{A})$ for $n \geq 2$ is the identity. For $n = 1$ the $\rho_1$ defining $q_1$ has $0$th component the identity and $k$th component the zero map for $k \geq 1$. It follows that $q_1(\deg_0^\mathcal{A}):P_1(\deg_0^\mathcal{A})\rightarrow P_0(\deg_0^\mathcal{A})$ is the natural inclusion.
\end{eg}


\clearpage

\section{Linear Approximations}


In this section we define the linear approximation of a functor $F:\mathcal{B}\rightarrow \cat{Ch}(\mathcal{A})$ as a functor $D_1(F)$ in $\cat{AbCat}_{\cat{Ch}}$. This construction will coincie, up to homotopy, with the homotopy fiber of the map $q_1:P_1(F)\to P_0(F)$. All of the properties we will describe for $D_1(F)$ are developed only up to \rd{natural} chain homotopy equivalence.

\begin{defn}[label=defn:linearization]
    The \textbf{linearization} of $F:\mathcal{B}\rightarrow \cat{Ch}(\mathcal{A})$ is the functor $D_1(F):\mathcal{B}\rightarrow \mathcal{A}$ given as the totalization of the explicit chain complex of chain complexes $(D_1(F)_\bullet,\partial_\bullet)$ where:
    %%
    \begin{equation*}
        D_1(F)_k := \left\{\begin{array}{cc} C_2^{\times k}(F) & k \geq 1 \\ \text{cr}_1(F) & k = 0 \\ 0 & \text{else} \end{array}\right.
    \end{equation*}
    %%
    and the differential $\partial_1:D_1(F)_1\rightarrow D_1(F)_0$ is given by $\rho_1$, while for $k \geq 2$, $\partial_k:D_1(F)_k\rightarrow D_1(F)_{k-1}$ is given by the alternating sum $\sum_{i=0}^{k-1}(-1)^iC_2^{\times i}\epsilon_{C_2^{\times (k-1-i)}}$. Note $\rho_1 = \epsilon$ since $C_2(F) \cong C_2(\text{cr}_1(F))$.
\end{defn}

Recall that $P_0(F)$ is naturally chain homotopy equivalent to $F(0)$. Using this model the induced map $q_1:P_1(F)\to P_0(F)\to F(0)$ is a degree-wise epimorphism, and hence a fibration in the standard model structure on non-negatively graded chain complexes.

\begin{proof}[Proof of Epimorphism]
    Note by our explicit construction of $q_1$, $q_1\circ p_1:F\to F(0)$ is exactly the projection for the isomorphism $F \cong F(0)\oplus \text{cr}_1(F)$. But this is epi as a map of chain complexes in each degree, so $q_1$ must also be epi.
\end{proof}

This implies that the kernel of $q_1$ is a model of the homotopy fiber, when this structure makes sense. Note that our definition of $D_1(F)$ coincides with $P_1(\text{cr}_1(F))$ due to Lemma \ref{lem:idempotCr1}, which is exactly the kernel of $q_1:P_1(F)\to F(0)$. 

To start understanding the linear approximation we consider an example:

\begin{eg}{}
    We consider the affine functor $F(X) = A\oplus X$. Recall $\text{cr}_2(F) \cong 0$ and $\text{cr}_1(F) \cong \id$. This implies that $D_1(F)$ is chain homotopy equivalent to $\id$ in degree $0$.
\end{eg}

We begin by inspecting properties of $D_1$. First we obtain immediate results since $D_1\cong P_1\circ \text{cr}_1$.

\begin{prop}[label=prop:D1Exact]
    \begin{itemize}
        \item[(i)] $D_1:\text{Fun}(\mathcal{B},\cat{Ch}(\mathcal{A}))\to \text{Fun}(\mathcal{B},\cat{Ch}(\mathcal{A}))$ is exact
        \item[(ii)] $D_1$ preserves (natural) chain homotopies, chain homotopy equivalences, and contractibility.
    \end{itemize}
\end{prop}
\begin{proof}
    These results follow from Propositions~\ref{prop:exactPol} and~\ref{prop:exactCross} using the isomorphism $D_1\cong P_1\circ \text{cr}_1$.
\end{proof}

Relaxing some of our previous constraints to the level of \rd{natural} chain homotopy equivalence, we define what it means for a functor to be linear.

\begin{defn}[label=defn:linear]
    A functor $F:\mathcal{B}\to \cat{Ch}(\mathcal{A})$ is said to be \textbf{linear} if it is degree one and weakly reduced, so $F(0)$ is naturally contractible.
\end{defn}

This definition is equivalent to a characterization in terms of finite direct sums. In order to establish this equivalence we must first prove results on how direct sums interact with natural chain homotopy equivalence. These results are a generalization of Lemma~\ref{lem:biprod}.

\begin{lem}[label=lem:oplusPres]
    Let $F,G,H:\mathcal{B}\to \cat{Ch}(\mathcal{A})$ be functors and let $f,g:F\Rightarrow G$ be chain homotopic maps. Then $f\oplus 1_H$ is chain homotopic to $g\oplus 1_H$.
\end{lem}
\begin{proof}
    Note that under the isomorphism $\text{Fun}^{\cat{Ch}}:\cat{Ch}(\text{Fun}(\mathcal{B},\mathcal{A}))\to \text{Fun}(\mathcal{B},\cat{Ch}(\mathcal{A}))$, it is sufficient to show for $F_\bullet,G_\bullet,H_\bullet \in \cat{Ch}(\text{Fun}(\mathcal{B},\mathcal{A}))$, with $f,g:F_\bullet\Rightarrow G_\bullet$, and let for each $n \in \Z$, $s_n:F_n\Rightarrow G_{n+1}$ denote the component natural transformations for the homotopy. Note that $-\oplus H_\bullet:\cat{Ch}(\text{Fun}(\mathcal{B},\mathcal{A}))\to \cat{Ch}(\text{Fun}(\mathcal{B},\mathcal{A}))$ is a functor. We consider $s_n\oplus 0:F_n\oplus H_n\Rightarrow G_{n+1}\oplus H_{n+1}$. Then 
    %%
    \begin{align*}
        (\partial_{n+1}^G\oplus \partial_{n+1}^H)\circ (s_n\oplus 0)+(s_{n-1}\oplus 0)\circ (\partial_n^F\oplus \partial_n^H) &= (\partial_{n+1}^G\circ s_n+s_{n-1}\circ \partial_n^F)\oplus 0  \\
        &= (f_n-g_n)\oplus (1_{H_n}-1_{H_n}) \\
        &= f_n\oplus 1_{H_n}-g_n\oplus 1_{H_n}
    \end{align*}
    %%
    as desired.
\end{proof}

We also have the converse result.

\begin{lem}[label=lem:homotopCancel]
    Let $F,G,H:\mathcal{B}\to \cat{Ch}(\mathcal{A})$ be functors and let $f,g:F\Rightarrow G$ be maps such that $f\oplus 1_H$ and $g\oplus 1_H$ are chain homotopic maps. Then $f$ is chain homotopic to $g$.
\end{lem}
\begin{proof}
    Once again we pass to $F_\bullet,G_\bullet,H_\bullet\in \cat{Ch}(\text{Fun}(\mathcal{B},\mathcal{A}))$ with $f,g:F_\bullet\Rightarrow G_\bullet$. Let 
    %%
    \begin{equation*}
        \begin{pmatrix} s_n^{1,1} & s_n^{1,2} \\ s_n^{2,1} & s_n^{2,2} \end{pmatrix}:F_n\oplus H_n\to G_{n+1}\oplus H_{n+1}
    \end{equation*}
    %%
    be the homotopy witnessing $f\oplus 1_H$ homotopic to $g\oplus 1_H$. Then the homotopy condition takes the form
    %%
    \begin{equation*}
        \begin{pmatrix} \partial_{n+1}^Gs_n^{1,1}+s_{n-1}^{1,1}\partial_n^G & \partial_{n+1}^Gs_n^{1,2}+s_{n-1}^{1,2}\partial_n^H \\ \partial_{n+1}^Hs_n^{2,1}+s_{n-1}^{2,1}\partial_n^G & \partial_{n+1}^Hs_n^{2,2}+s_{n-1}^{2,2}\partial_n^H \end{pmatrix} = \begin{pmatrix} f_n-g_n & 0 \\ 0 & 0 \end{pmatrix}
    \end{equation*}
    %%
    It follows that $s_n^{1,1}:F_n\to G_{n+1}$ forms a homotopy from $f$ to $g$.
\end{proof}

As an immediate corollary of Lemma~\ref{lem:oplusPres} and Lemma~\ref{lem:homotopCancel} we have that chain homotopy equivalences are preserved by direct sum and satisfy the cancellative property.

We can now prove the previously asserted equivalence.

\begin{prop}[label=prop:linearEquiv]
    A functor $F:\mathcal{B}\to \cat{Ch}(\mathcal{A})$ is linear if and only if the natural map
    %%
    \begin{equation*}
        D_1(F)(X)\oplus D_1(F)(Y)\hookrightarrow D_1(F)(X\oplus Y)
    \end{equation*}
    %%
    is a natural chain homotopy equivalence.
\end{prop}
\begin{proof}
    First suppose $F:\mathcal{B}\to \cat{Ch}(\mathcal{A})$ is linear. Then we have that $\text{cr}_2(F)\simeq_{\cat{Ch}}0$ and $F(0)\simeq_{\cat{Ch}}0$. Recall from the inductive definition of the cross effect that 
    %%
    \begin{equation*}
        \text{cr}_1(F)(X\oplus Y) \cong \text{cr}_1(F)(X)\oplus \text{cr}_1(F)(Y)\oplus \text{cr}_2(F)(X,Y)
    \end{equation*}
    %%
    naturally in $X$ and $Y$. Then by Lemma~\ref{lem:oplusPres} we have that \[\text{cr}_1(F)(X)\oplus \text{cr}_1(F)(Y)\oplus \text{cr}_2(F)(X,Y)\simeq_{\cat{Ch}}\text{cr}_1(F)(X)\oplus \text{cr}_1(F)(Y)\oplus 0 \cong \text{cr}_1(F)(X)\oplus \text{cr}_1(F)(Y)\] 
    so $\text{cr}_1(F)$ preserves direct sums up to natural chain homotopy equivalence. Then since $F(0)\simeq_{\cat{Ch}}0$ and $F\cong \text{cr}_1(F)\oplus F(0)$, it follows that $F$ also preserves direct sums up to natural chain homotopy equivalence.

    \vspace{10pt}

    Conversely, if $F$ preserves direct sums up to natural chain homotopy equivalence we have the identity
    %%
    \begin{equation*}
        F(0)\cong F(0\oplus 0)\simeq_{\cat{Ch}}F(0)\oplus F(0)
    \end{equation*}
    %%
    so by Lemma~\ref{lem:homotopCancel} it follows that $F(0) \simeq_{\cat{Ch}} 0$. Similarly, using the inductive formula for $\text{cr}_2(F)$ and Lemma~\ref{lem:homotopCancel} we have that $\text{cr}_2(F)$ is naturally contractible, so $F$ is degree 1.
\end{proof}

Note that since $D_1$ is an exact functor it is linear in the sense that it preserves the zero object and direct sums of functors up to natural isomorphism.


The next main result we wish to show is that $D_1$ preserves composition up to \rd{naturally} chain homotopy, upgrading the original pointwise chain homotopy results in~\cite{BJORT}. The primary work and lemmas for this result are contained in Section~\ref{sec:Lotswork}. In particular, we will prove the following result in Section~\ref{sec:Lotswork}.

\begin{prop}[label=prop:5.7]
    If $F:\mathcal{B}\to \cat{Ch}(\mathcal{A})$ and $G:\mathcal{C}\to \cat{Ch}(\mathcal{B})$ are composible functors in $\cat{AbCat}_{\cat{Ch}}$ such that $G$ is reduced, Then
    %%
    \begin{equation*}
        D_1(F\lhd G) \simeq_{\cat{Ch}} D_1(F)\lhd D_1(G)
    \end{equation*}
    %%
\end{prop}

In this section we will extend this result to non-reduced $G$. First, by Lemma~\ref{lem:idempotCr1} we have that $\text{cr}_1(\text{cr}_1(F))\cong \text{cr}_1(F)$, so as $D_1(F) \cong P_1(\text{cr}_1(F))$ and post-composition preserves natural isomorphisms we obtain that $D_1(F) \cong D_1(\text{cr}_1(F))$. Before extending the proposition we first prove some preliminary results on the relationship between the direct sum operation on functors and our composition, $\lhd$, in $\cat{AbCat}_{\cat{Ch}}$. 

\begin{lem}[label=lem:compDirSum]
    Let $F,G:\mathcal{B}\to \cat{Ch}(\mathcal{A})$ and $H,K:\mathcal{C}\to \cat{Ch}(\mathcal{B})$. Then there is a natural isomorphism
    %%
    \begin{equation*}
        (F\oplus G)\lhd (H\oplus K) \cong (F\lhd H)\oplus (F\lhd K)\oplus (G\lhd H)\oplus (G\lhd K)
    \end{equation*}
    %%
\end{lem}
\begin{proof}
    To prove the claim we proceed in steps. First, we show that
    %%
    \begin{equation*}
        F\lhd (H\oplus K) \cong (F\lhd H)\oplus (F\lhd K)
    \end{equation*}
    %%
    Observe that by definition 
    %%
    \begin{equation*}
        F\lhd (H\oplus K) = N_\mathcal{A}\Delta_\mathcal{A}(\Gamma_\mathcal{A})_*F_*\Gamma_\mathcal{B}(H\oplus K)
    \end{equation*}
    %%
    Since isomorphisms and $\Delta_\mathcal{A}$ preserve limits, and direct sums of functors in abelian categories are given pointwise, it follows that
    %%
    \begin{equation*}
        N_\mathcal{A}\Delta_\mathcal{A}(\Gamma_\mathcal{A})_*F_*\Gamma_\mathcal{B}(H\oplus K) \cong (N_\mathcal{A}\Delta_\mathcal{A}(\Gamma_\mathcal{A})_*F_*\Gamma_\mathcal{B}H)\oplus (N_\mathcal{A}\Delta_\mathcal{A}(\Gamma_\mathcal{A})_*F_*\Gamma_\mathcal{B}K)
    \end{equation*}
    %%
    and so 
    %%
    \begin{equation*}
        F\lhd (H\oplus K) \cong (F\lhd H)\oplus (F\lhd K)
    \end{equation*}
    %%

    \vspace{10pt}

    Next we show 
    %%
    \begin{equation*}
        (F\oplus G)\lhd H \cong (F\lhd H)\oplus (G\lhd H)
    \end{equation*}
    %%
    First, note that since $\Gamma$ preserves direct sums, $\Gamma_\mathcal{A}\circ (F\oplus G)\cong (\Gamma_\mathcal{A}\circ F)\oplus (\Gamma_\mathcal{A}\circ G)$. Then by definition of the direct sum of functors
    %%
    \begin{equation*}
        ((\Gamma_\mathcal{A}\circ F)\oplus (\Gamma_\mathcal{A}\circ G))_*\Gamma_\mathcal{B}H = (\Gamma_\mathcal{A}\circ F)_*\Gamma_\mathcal{B}H\oplus (\Gamma_\mathcal{A}\circ G)_*\Gamma_\mathcal{B}H
    \end{equation*}
    %%
    Since $\Delta_\mathcal{A}$ and $N_\mathcal{A}$ preserve limits, we have that 
    %%
    \begin{equation*}
        (F\oplus G)\lhd H \cong (F\lhd H)\oplus (G\lhd H)
    \end{equation*}
    %%
    as desired.
\end{proof}


In order to extend the proposition we require one additional result.

\begin{lem}[label=lem:compGen]
    If $F:\mathcal{B}\to \cat{Ch}(\mathcal{A})$ and $G:\mathcal{C}\to \cat{Ch}(\mathcal{B})$ are composible functors, then 
    %%
    \begin{equation*}
        \text{cr}_1(F\lhd G)(X) \cong (\text{cr}_1(F)\lhd \text{cr}_1(G))(X)\oplus (\text{cr}_2(F)\lhd (G(0)\oplus \text{cr}_1(G)(X)))
    \end{equation*}
    %%
\end{lem}
\begin{proof}
    First, observe that $\text{cr}_1(F\lhd G)\oplus (F\lhd G)(0) \cong F\lhd G$ by construction of the cross-effect. Next, using the isomorphism again but now on $F$ and $G$ individually,
    %%
    \begin{equation*}
        F\lhd G \cong F\lhd (G(0)\oplus \text{cr}_1(G)) \cong (F(0)\oplus \text{cr}_1(F))\lhd (G(0)\oplus \text{cr}_1(G))
    \end{equation*}
    %%
    
\end{proof}



\clearpage

\section{A General Bicomplex Retraction}\label{sec:bicomplexes}

In order to construct certain explicit chain homotopy equivalences in the text we require a criteria for when total complexes of certain bicomplexes are chain homotopy equivalent. Throughout we consider $A_{\bullet,\bullet}$ to be denote a first-quadrant bicomplex. This is sufficient for our case since all our bicomplexes are constructed from chain complexes concentrated in non-negative degrees. As in \cite{BJORT} we proceed with bicomplexes having anti-commuting squares. To apply this to the work elsewhere all that must be done is the replacement of $d_h:A_{p,q}\to A_{p,q-1}$ by $(-1)^pd_h$.

\begin{defn}[label=defn:RowWiseStrngDefRetr]
    We say a morphism $\iota:A_{\bullet,\bullet}\to B_{\bullet,\bullet}$ \textbf{admits a row-wise strong deformation retraction} if for all $p \geq 0$ there exists a map $f_{p,\bullet}:B_{p,\bullet}\to A_{p,\bullet}$ such that
    %%
    \begin{itemize}
        \item[(i)] $f_{p,\bullet}\circ \iota_{p,\bullet} = 1_{A_{p,\bullet}}$
        \item[(ii)] there exist morphisms $s_h:B_{p,q}\to B_{p,q+1}$ such that $d_hs+sd_h = 1-\iota_{p,q}f_{p,q}$ and $s_h\circ \iota_{p,q} = 0$ (i.e.\ we have a strong chain homotopy equivalence between $A_{p,\bullet}$ and $B_{p,\bullet}$)
    \end{itemize}
\end{defn}

Throughout this section we will denote the horizontal differentials of a bicomplex by $d_h:A_{p,q}\to A_{p,q-1}$ and the vertical differentials by $d_v:B_{p,q}\to B_{p-1,q}$. Although our maps in definition \ref{defn:RowWiseStrngDefRetr} are given only for $p,q\geq 0$, they can easily be extended to all $p,q$ by setting ones with negative indices equal to zero. We record some commutativity equalities for use in the proofs to follow
%%
\begin{align*}
    d_h^2 = 0 &\;\; d_v^2=0 \;\; d_hd_v+d_vd_h = 0\;\; d_hs_h+s_hd_h=1-\iota \circ f\;\; s_h\circ \iota = 0 \\
    & f\circ \iota = 1\;\; f \circ d_h= d_h\circ f\;\; \iota \circ d_h= d_h\circ \iota \;\; \iota \circ d_v = d_v\circ \iota 
\end{align*}

We begin with the following lemma. (\textbf{Note:} Juxtaposition in the following lemma still denotes functional compositional ordering for the sake of preserving space).

%%
\begin{lem}[label=lem:A3]
    For any $k \geq 0$ we have the following equalities:
    %%
    \begin{itemize}
        \item[(i)] $d_v f(-d_vs_h)^k+d_hf (-d_vs_h)^{k+1}=f(-d_vs_h)^kd_v + f (-d_vs_h)^{k+1}d_h$
        \item[(ii)] $d_vs_h(-d_vs_h)^k+d_hs_h(-d_vs_h)^{k+1} = -\iota f(-d_vs_h)^{k+1}-s_hd_h(-d_vs_h)^{k+1}$ 
        \item[(iii)] $s_h(-d_vs_h)^{k+1}d_h+s_h(-d_vs_h)^kd_v = -(-s_hd_v)^{k+1}d_hs_h$ 
        \item[(iv)] $s_hd_h(-d_vs_h)^{k+1} = -(-s_hd_v)^{k+1}d_hs_h$
    \end{itemize}
\end{lem}
\begin{proof}
    We will prove each formula by induction.
    \begin{itemize}
        \item[(i)] If $k = 0$ we want to show 
        %%
        \begin{equation*}
            d_vf+d_hf(-d_vs_h) = fd_v + f(-d_vs_h)d_h
        \end{equation*}
        %%
        Using our relations
        \begin{align*}
            d_vf+d_hf(-d_vs_h) &= d_vf - fd_hd_vs_h \\
            &= d_vf+fd_vd_hs_h \\
            &= d_vf+fd_v(1-\iota f-s_hd_h) \\
            &= d_vf+fd_v-fd_v\iota f - fd_vs_hd_h \\
            &= d_vf+fd_v-d_vf + f(-d_vs_h)d_h \\
            &= fd_v+f(-d_vs_h)d_h
        \end{align*}
        as desired. Suppose now that the claim holds for some $k \geq 0$. Then
        %%
        \begin{align*}
            d_vf(-d_vs_h)^{k+1}+d_hf(-d_vs_h)^{k+2} &= [f(-d_vs_h)^kd_v+f(-d_vs_h)^{k+1}d_h](-d_vs_h) \\
            &= f(-d_vs_h)^kd_v(-d_vs_h)+f(-d_vs_h)^{k+1}d_h(-d_vs_h) \\
            &= f(-d_vs_h)^{k+1}d_v(1-s_hd_h-\iota f) \\
            &= f(-d_vs_h)^{k+1}d_v-f(-d_vs_h)^{k+1}d_vs_hd_h-f(-d_vs_h)^{k+1}d_v\iota f \\
            &= f(-d_vs_h)^{k+1}d_v-f(-d_vs_h)^{k+2}d_h-f(-d_vs_h)^{k+1}\iota d_vf \\
            &= f(-d_vs_h)^{k+1}d_v-f(-d_vs_h)^{k+2}d_h
        \end{align*}
        as desired.
        \item[(ii)] We can immediately compute
        %%
        \begin{align*}
            d_vs_h(-d_vs_h)^k + d_hs_h(-d_vs_h)^{k+1} &= [d_vs_h + d_hs_h(-d_vs_h)](-d_vs_h)^k \\
            &= [d_vs_h + (1-s_hd_h-\iota f)(-d_vs_h)](-d_vs_h)^k \\
            &= [-\iota f(-d_vs_h) - s_hd_h(-d_vs_h)](-d_vs_h)^k \\
            &= -\iota f(-d_vs_h)^{k+1} - s_hd_h(-d_vs_h)^{k+1}
        \end{align*}
        %%
        as desired.
        \item[(iii)] If $k = 0$ we compute
        %%
        \begin{align*}
            s_h(-d_vs_h)d_h+s_hd_v &= -s_hd_v(1-d_hs_h-\iota f)+s_hd_v \\
            &= s_hd_vd_hs_h+s_hd_v\iota f \\
            &= -(-s_hd_v)d_hs_h+s_h\iota d_vf \\
            &= -(-s_hd_v)d_hs_h
        \end{align*}
        %%
        Now if the claim holds for $k \geq 0$ we can compute
        %%
        \begin{align*}
            -(-s_hd_v)^{k+2}d_hs_h &= (-s_hd_v)[s_h(-d_vs_h)^{k+1}d_h+s_h(-d_vs_h)^kd_v] \\
            &= s_h(-d_vs_h)^{k+2}d_h+s_h(-d_vs_h)^{k+1}d_v
        \end{align*}
        %%
        as desired.
        \item[(iv)] If $k = 0$ we observe that
        %%
        \begin{align*}
            s_hd_h(-d_vs_h) &= s_hd_vd_hs_h = -(-s_hd_v)d_hs_h
        \end{align*}
        %%
        If the claim holds for some $k \geq 0$, then we can compute
        %%
        \begin{align*}
            s_hd_h(-d_vs_h)^{k+2} &= -(-s_hd_v)^{k+1}d_hs_h(-d_vs_h) \\
            &= (-s_hd_v)^{k+1}(1-s_hd_h-\iota f)d_vs_h \\
            &= (-s_hd_v)^{k+1}d_vs_h-(-s_hd_v)^{k+1}s_hd_hd_vs_h-(-s_hd_v)^{k+1}\iota fd_vs_h \\
            &= -(-s_hd_v)^{k+2}d_hs_h
        \end{align*}
        as desired since $d_v^2 = 0$, $d_v\iota = \iota d_v$, and $s_h\iota = 0$.
    \end{itemize}
\end{proof}

We recall that for first-quadrant bicomplexes the totalization at each degree is a finite direct sum, so its differentials can be described by finite matrices. Explicitly the $n$th differential $\text{Tot}(A_{\bullet,\bullet})_n\to \text{Tot}(A_{\bullet,\bullet})_{n-1}$ is given by the matrix
%%
\begin{equation*}
    \begin{pmatrix}
        d_v & d_h & 0 & \cdots & \cdots & 0 \\
        0 & d_v & d_h & 0 & \cdots & 0 \\
        \vdots & \ddots & \ddots & \ddots & \ddots & \vdots \\
        0 & \cdots & 0 & d_v & d_h & 0 \\
        0 & \cdots & \cdots & 0 & d_v & d_h
    \end{pmatrix}
\end{equation*}


\begin{prop}[label=prop:A5]
    Let $\iota:A_{\bullet,\bullet}\to B_{\bullet,\bullet}$ be a map of bicomplexes that admits a row-wise strong deformation retraction. Then the induced morphism of total complexes $\text{Tot}(\iota):\text{Tot}(A_{\bullet,\bullet})_\bullet\to\text{Tot}(B_{\bullet,\bullet})_{\bullet}$ admits a retraction $\rho:\text{Tot}(B_{\bullet,\bullet})_{\bullet}\to \text{Tot}(A_{\bullet,\bullet})_{\bullet}$ defined in degree $n$ by the $(n+1)\times (n+1)$ matrix
    %%
    \begin{equation*}
        \begin{pmatrix}
            f & 0 & \cdots & \cdots & 0 & 0 \\
            f(-d_vs_h) & f & 0 & \ddots & \ddots & 0 \\
            f(-d_vs_h)^2 & f(-d_vs_h) & f & 0 & \ddots & \vdots \\
            \vdots & \vdots & \vdots &\vdots & \ddots & \ddots & \vdots \\
            f(-d_vs_h)^n & f(-d_vs_h)^{n-1} & \cdots & \cdots & f(-d_vs_h) & f 
        \end{pmatrix}
    \end{equation*}
\end{prop}
\begin{proof}
    First, to see that $\rho$ is a chain map fix $n \geq 1$. Then $\partial_n\rho_n$ and $\rho_{n-1}\partial_n$ are $n\times (n+1)$ matrices with $i,j$ component given by 
    %%
    \begin{align*}
        (\partial_n\rho_n)_{i,j} = \left\{\begin{array}{cc} 
            0 & i+1 < j \\
            d_hf & i+1 = j \\
            d_hf(-d_vs_h)+d_vf & i = j \\
            d_hf(-d_vs_h)^{i-j+1}+d_vf(-d_vs_h)^{i-j} & i > j 
        \end{array}\right.
    \end{align*} 
    %%
    while
    %%
    \begin{align*}
        (\rho_{n-1}\partial_n)_{i,j} = \left\{\begin{array}{cc} 
            0 & i+1 < j \\
            fd_h & i+1 = j \\ 
            fd_v & i = j = 1 \\
            f(-d_vs_h)d_h+fd_v & i = j \neq 1 \\
            f(-d_vs_h)^{i-j+1}d_v+f(-d_vs_h)^{i-j}d_h & i > j 
        \end{array}\right.
    \end{align*}
    %%
    We have equality for $i > j$ by equation (i) of Lemma \ref{lem:A3}, equality for $i+1 < j$ vacuously, equality for $i+1 = j$ since $f$ is a chain map with respect to the horizontal differentials, equality for $i = j \neq 1$ is also from equation (i) of Lemma \ref*{lem:A3}, and $i = j = 1$ is equation (i) and the fact that $A_{n+1,-1} = 0$ as the bicomplex is concentrated in non-negative degree.

    \vspace{10pt}

    To show $\rho$ is a retraction it remains to show $\rho_n\iota_n = 1$. Observe that for $i,j$,
    %%
    \begin{equation*}
        (\rho_n\iota_n)_{i,j} = \left\{\begin{array}{cc} 
            0 & i > j \\
            f\circ \iota & i = j \\
            f(-d_vs_h)^{j-i}\circ \iota & i < j \\
        \end{array}\right.
    \end{equation*}
    %%
    But since $f$ is part of a row-wise strong deformation retraction, $f\circ \iota = 1$, and $s_h \circ \iota = 0$, so the matrix is the identity.
\end{proof}


It remains to show that $\rho$ is associated with a deformation retraction for $\iota$. In other words, we want to show that $\iota\circ \rho$ is chain homotopic to the identity.

\begin{prop}[label=prop:A6]
    The composite map $\iota\circ \rho:\text{Tot}(B)_\bullet\to \text{Tot}(B)_\bullet$ is chain homotopic to the identity via the chain homotopy $\sigma:\text{Tot}(B)_\bullet\to \text{Tot}(B)_{\bullet+1}$ defined in degree $n$ by the $n+2\times n+1$ matrix
    %%
    \begin{equation*}
        \begin{pmatrix}
                0 & 0 & \cdots & \cdots & 0 & 0 \\
                s_h & 0 & \ddots & \ddots & \ddots & 0 \\
                s_h(-d_vs_h) & s_h & 0 & \ddots & \ddots & \vdots \\
                s_h(-d_vs_h)^2 & s_h(-d_vs_h) & s_h & 0 & \ddots & \vdots \\
                \vdots & \vdots & \vdots & \ddots & \ddots & \vdots \\
                s_h(-d_vs_h)^{n-1} & \cdots & \cdots & s_h(-d_vs_h) & s_h & 0 \\
                s_h(-d_vs_h)^n & s_h(-d_vs_h)^{n-1} & \cdots & \cdots & s_h(-d_vs_h) & s_h
        \end{pmatrix}
    \end{equation*}
    %%
\end{prop}
\begin{proof}
    Let $\partial_n:\text{Tot}(B)_n\to \text{Tot}(B)_{n-1}$ be the total complex differential. Observe that that $1-\iota\circ \rho$ is the matrix 
    %%
    \begin{equation*}
            \begin{pmatrix} 
                1-\iota f & 0 & \cdots & \cdots & 0 \\
                -\iota f(-d_vs_h) & 1-\iota f & 0 & \cdots & 0 \\
                -\iota f(-d_vs_h)^2 & -\iota f(-d_vs_h) & 1-\iota f & \cdots & 0 \\
                \vdots & \vdots & \vdots & \ddots & \vdots \\
                -\iota f(-d_vs_h)^n & \cdots & -\iota f(-d_vs_h)^2 & -\iota f(-d_vs_h) & 1-\iota f \\
            \end{pmatrix}
    \end{equation*}
    %%
    On the other hand, we can compute for $1 \leq i,j\leq n+1$
    %%
    \begin{equation*}
        (\partial_{n+1}\sigma_n)_{i,j} = \left\{
            \begin{array}{cc}
                0 & i < j \\
                d_hs_h & i = j \\
                d_vs_h(-d_vs_h)^{i-j-1}+d_hs_h(-d_vs_h)^{i-j} & i > j 
            \end{array}
        \right.
    \end{equation*}
    %%
    and 
    %%
    \begin{equation*}
        (\sigma_{n-1}\partial_{n})_{i,j} = \left\{
            \begin{array}{cc}
                0 & i < j \\
                s_hd_h & i = j \\
                s_h(-d_vs_h)^{i-j}d_h + s_h(-d_vs_h)^{i-j-1}d_v & i > j 
            \end{array}
        \right.
    \end{equation*}
    %%
    Adding these together we observe that the case of $i=j$ gives equality since $d_hs_h + s_hd_h = 1-\iota f$. On the other hand, for $i > j$ we can use the relations in Lemma \ref{lem:A3}
    %%
    \begin{align*}
        d_vs_h(-d_vs_h)^{i-j-1}+d_hs_h(-d_vs_h)^{i-j}&+s_h(-d_vs_h)^{i-j}d_h + s_h(-d_vs_h)^{i-j-1}d_v \\
        &= -\iota f(-d_vs_h)^{i-j}-s_hd_h(-d_vs_h)^{i-j}\\
        &+s_h(-d_vs_h)^{i-j}d_h + s_h(-d_vs_h)^{i-j-1}d_v \tag{by (ii)} \\
        &= -\iota f(-d_vs_h)^{i-j}-s_hd_h(-d_vs_h)^{i-j}-(-s_hd_v)^{i-j}d_hs_h \tag{by (iii)} \\
        &= -\iota f(-d_vs_h)^{i-j}-s_hd_h(-d_vs_h)^{i-j}+s_hd_h(-d_vs_h)^{i-j} \tag{by (iv)} \\
        &= -\iota f(-d_vs_h)^{i-j}
    \end{align*}
    %%
    which is precisely the $i,j$ entry of $1-\iota \rho$, completing the proof.
\end{proof}


Together these lemmas prove the following result:

\begin{thm}[label=thm:A2]
    Let $\iota:A_{\bullet,\bullet}\to B_{\bullet,\bullet}$ be a morphism of first-quadrant bicomplexes that admit a row-wise strong deformation retraction. Then $\iota$ induces a chain homotopy equivalence of total complexes $\text{Tot}(A_{\bullet,\bullet})\to \text{Tot}(B_{\bullet,\bullet})$.  
\end{thm}

As a quick corollary we obtain the following sufficient condition for a chain homotopy equivalence between the degree $0$ inclusion of a chain complex into a first quadrant bicomplex with contractible rows and its totalization.

\begin{cor}[label=cor:A7]
    Let $A_{\bullet,\bullet}$ be a first-quadrant bicomplex so that every row except the zeroth row $A_{0,\bullet}$ is contractible. Then the natural inclusion $A_{0,\bullet}\hookrightarrow \text{Tot}(A)_\bullet$ is a chain homotopy equivalence.
\end{cor}
\begin{proof}
    Let $\iota:\deg_0(A_{0,\bullet})\to A_{\bullet,\bullet}$ denote the inclusion in degree $0$. Let $f_{p,\bullet}:A_{p,\bullet} \to \deg_0(A_{0,\bullet})_{p,\bullet}$ denote the chain complex map for the contraction for $p > 0$ (which is the zero map), the identity for $p = 0$, and zero for $p < 0$. Let $s_h:A_{p,q} \to A_{p,q+1}$ denote the contraction for $p \geq 1$, and $0$ for $p \leq 0$, so $d_hs_h+s_hd_h = 1-\iota f$ and $s_h\iota = 0$ since $\iota$ is zero for $p > 0$ and $s_h$ is zero for $p \leq 0$. By Theorem \ref{thm:A2} there exists a chain homotopy equivalence $\iota:A_{0,\bullet}\hookrightarrow \text{Tot}(A)_\bullet$ where the retraction $\rho:\text{Tot}(A)_\bullet\to A_{0,\bullet}$ is the $1\times (n+1)$ is given By
    %%
    \begin{equation*}
        \begin{pmatrix}
            f(-d_vs_h)^n & f(-d_vs_h)^{n-1} & \cdots & f(-d_vs_h) & f
        \end{pmatrix}
    \end{equation*}
    %%
    or equivalently
    %%
    \begin{equation*}
        \begin{pmatrix}
            (-d_vs_h)^n & (-d_vs_h)^{n-1} & \cdots & (-d_vs_h) & 1
        \end{pmatrix}
    \end{equation*}
    %%
    since $f$ is the identity on $A_{0,\bullet}$.
\end{proof}

\clearpage

\section{Quasi-isomorphism for Composition}\label{sec:Lotswork}

In this section we will prove Proposition~\ref{prop:5.7} using the work in Section~\ref{sec:bicomplexes}. To this goal let $F:\mathcal{B}\to \cat{Ch}(\mathcal{A})$ and $G:\mathcal{C}\to \cat{Ch}(\mathcal{B})$ be composable functors with $G$ reduced. We first observe the following lemma which will allow us to reduce to the case that $F$ is also reduced.

\begin{lem}[label=lem:B.1]
    Let $F:\mathcal{B}\to \cat{Ch}(\mathcal{A})$ and $G:\mathcal{C}\to \cat{Ch}(\mathcal{B})$ be composable functors with $G$ reduced. Then $\text{cr}_1(F\lhd G)\cong \text{cr}_1(F)\lhd G$.
\end{lem}
\begin{proof}
    Observe that using our construction of the cross-effect we have isomorphisms
    %%
    \begin{equation*}
        F\lhd G \cong (F\lhd G)(0) \oplus \text{cr}_1(F\lhd G)
    \end{equation*}
    %%
    and by Lemma~\ref{lem:compDirSum} and Lemma~\ref{lem:constComp}
    %%
    \begin{equation*}
        F\lhd G\cong (\text{cr}_1(F)\oplus F(0))\lhd G \cong (\text{cr}_1(F)\lhd G)\oplus (F(0)\lhd G) \cong (\text{cr}_1(F)\lhd G)\oplus F(0)
    \end{equation*}
    %%
    Further, $(F\lhd G)(0) \cong F(0)$ since $G$ is reduced and $\Gamma_\mathcal{B}$ preserves zero objects. Thus taking direct sum complements we obtain the desired isomorphism.
\end{proof}

We can use this lemma to begin reducing our goal.

\begin{cor}[label=cor:B.2]
    Let $F:\mathcal{B}\to \cat{Ch}(\mathcal{A})$ and $G:\mathcal{C}\to \cat{Ch}(\mathcal{B})$ be composable functors with $G$ reduced. Then
    %%
    \begin{equation*}
        D_1(F\lhd G)\cong D_1(\text{cr}_1(F)\lhd G)
    \end{equation*}
\end{cor}
\begin{proof}
    Recall that $D_1\cong D_1\circ \text{cr}_1$ since the first cross-effect is idempotent. Then by Lemma~\ref{lem:B.1}
    %%
    \begin{equation*}
        D_1(F\lhd G) \cong D_1(\text{cr}_1(F\lhd G)) \cong D_1(\text{cr}_1(F)\lhd G)
    \end{equation*}
    %%
    as desired.
\end{proof}

Note that since $D_1(F)\lhd D_1(G)\cong D_1(\text{cr}_1(F))\lhd D_1(G)$, Corollary~\ref{cor:B.2} implies that it is sufficient to prove Proposition~\ref{prop:5.7} when both functors are reduced. Note that Lemma~\ref{lem:funcActChain} restricts to a functor $\cat{Ch}:\text{Fun}_*(\mathcal{A},\mathcal{C})\to \text{Fun}_*(\cat{Ch}(\mathcal{A}),\cat{Ch}(\mathcal{C}))$. 


\begin{rmk}
    Let $F:\mathcal{A}\to \cat{Ch}(\mathcal{B})$ be a strictly reduced functor. We define a comparison map $\text{sw}_F:\Gamma_{\cat{Ch}(\mathcal{B})}\circ \cat{Ch}(F)\to F_*\circ \Gamma_\mathcal{A}$ at $A_\bullet \in \cat{Ch}(\mathcal{A})$ and $n \in \N$,
    %%
    \begin{equation*}
        \text{sw}_{F,A_\bullet,n}:\bigoplus_{[n]\twoheadrightarrow [k]}F(A_k)\to F\left(\bigoplus_{[n]\twoheadrightarrow [k]}A_k\right)
    \end{equation*}
    %%
    given by the universal coproduct property of the biproduct. This is evidently natural in $n$, $A_\bullet$, and $F$. Then we have that $(\Gamma_{\mathcal{B}})_*\text{sw}_F$ is given by 
    %%
    \begin{equation*}
        (\Gamma_{\mathcal{B}})_*\text{sw}_{F,A_\bullet,n}:\Gamma_\mathcal{B}\left(\bigoplus_{[n]\twoheadrightarrow [k]}F(A_k)\right)\to \Gamma_\mathcal{B}\circ F\left(\bigoplus_{[n]\twoheadrightarrow [k]}A_k\right)
    \end{equation*} 
    %%
    which at $m$ is
    %%
    \begin{equation*}
        ((\Gamma_{\mathcal{B}})_*\text{sw}_{F,A_\bullet,n})_m:\bigoplus_{[m]\twoheadrightarrow[\ell]}\left(\bigoplus_{[n]\twoheadrightarrow [k]}F(A_k)_\ell\right)\to \bigoplus_{[m]\twoheadrightarrow[\ell]}F\left(\bigoplus_{[n]\twoheadrightarrow [k]}A_k\right)_\ell
    \end{equation*}
    Taking the diagonal we obtain at $A_\bullet$ and $n$ the map
    %%
    \begin{equation*}
        \bigoplus_{[n]\twoheadrightarrow[\ell]}\left(\bigoplus_{[n]\twoheadrightarrow[k]}F(A_k)_\ell\right) \to \bigoplus_{[n]\twoheadrightarrow[\ell]}F\left(\bigoplus_{[n]\twoheadrightarrow[k]}A_k\right)_\ell
    \end{equation*}
    %%
\end{rmk}

We now show some chain homotopy equivalence results for linear functors. First we investigate how linear functors act on sums of maps.

\begin{rmk}
    Let $F:\mathcal{B}\to \cat{Ch}(\mathcal{A})$ be a linear functor, so by Proposition~\ref{prop:linearEquiv} we have a natural chain homotopy equivalence
    %%
    \begin{equation*}
        F\oplus F\simeq_{\cat{Ch}}F(-\oplus -)
    \end{equation*}
    %%
    induced by the natural inclusion $F\oplus F\hookrightarrow F(-\oplus -)$. Now consider maps $f,g:B\to B'$ in $\mathcal{B}$. Then the sum $f+g$ can be represented as the composite
    %%
    \begin{equation*}
        B\xrightarrow{\langle 1_B,1_B\rangle}B\oplus B\xrightarrow{f\oplus g}B'\oplus B'\xrightarrow{\langle 1_{B'}|1_{B'}\rangle}B'
    \end{equation*}
    %%
    Now we can apply $F$ to this  \textbf{TBC}
\end{rmk}



\begin{lem}[label=lem:equivDef]
    Let $F:\mathcal{A}\to \cat{Ch}(\mathcal{B})$ be linear. Then $F_*\Gamma_\mathcal{A}$ and $\Gamma_{\cat{Ch}(\mathcal{B})}\cat{Ch}(F)$ are \rd{naturally} chain homotopy equivalent.
\end{lem}
\begin{proof}
    Note that by assumption $F$ is linear and strictly reduced. Then by Proposition~\ref{prop:linearEquiv}, for each $A_\bullet \in \cat{Ch}(\mathcal{A})$ and each $n$ we have a natural chain homotopy equivalence
    %%
    \begin{equation*}
        F\left(\bigoplus_{[n]\twoheadrightarrow[k]}A_k\right) \simeq_{\cat{Ch}} \bigoplus_{[n]\twoheadrightarrow[k]}F(A_k)
    \end{equation*}
    %%
    which is natural in the $A_k$, and hence in $A_{\bullet}$. We want to enhance these to a natural chain homotopy $F_*\Gamma_\mathcal{A}\simeq_{\cat{Ch}}\Gamma_{\cat{Ch}(\mathcal{B})}\cat{Ch}(F)$. To this end let $\alpha_n,\beta_n,s^n,r^n$ denote such a natural chain homotopy for each $n$. 
\end{proof}

\clearpage

\begin{subappendices}
    \section{Appendices}

    \subsection{Simplicial Object 2-Monad}\label{sec:simpMon}

    
In this section we attempt to construct a (pseudo)monad on $2\cat{Ab}$ corresponding to simplicial objects. The goal is that this (pseudo)monad is easier to construct than the chain complex pseudomonad, and that via conjugation by the Dold-Kan equivalence, we can obtain the chain complex pseudomonad, at least up to a suitable equivalence.

We define $(-)\Sob:2\cat{Ab}\rightarrow2\cat{Ab}$ as a pseudofunctor as follows:
\begin{enumerate}
    \item On 0-cells, $(-)\Sob$ sends an abelian category $\mathcal{A}$ to its category of simplicial objects $\mathcal{A}\Sob$
    \item Given abelian categories $\mathcal{A},\mathcal{B}$, we have a functor $(-)\Sob:[\mathcal{A},\mathcal{B}]\rightarrow [\mathcal{A}\Sob,\mathcal{B}\Sob]$ given as follows:
    \begin{enumerate}
        \item A functor $F:\mathcal{A}\rightarrow \mathcal{B}$ is sent to its push-forward $F_* :\mathcal{A}\Sob\rightarrow \mathcal{B}\Sob$ defined by post-composition
        \item A natural transformation $\gamma:F\Rightarrow G:\mathcal{A}\rightarrow \mathcal{B}$ is sent to a natural transformation $\gamma\Sob:F_*\Rightarrow G_*$ such that for $X \in \mathcal{A}\Sob_0$,
        %%
        \begin{equation*}
            \gamma\Sob_X:F\circ X\Rightarrow G\circ X := \gamma_X
        \end{equation*}
        %%
    \end{enumerate}
    \item We observe $m(F,G) := 1_{(G\circ F)_*}:G_*\circ F_*\Rightarrow (G\circ F)_*$ is our comparison 2-cell
    \item For each abelian category $\mathcal{A}$, an invertible 2-cell $i:= 1_{1_{\mathcal{A}\Sob}}:1_{\mathcal{A}\Sob}\Rightarrow (1_\mathcal{A})_*$ which is an identity.
\end{enumerate}
The psuedofunctor comes with the following monad data:
\begin{enumerate}
    \item A pseudonatural transformation $\eta:1_{2\cat{Ab}}\Rightarrow (-)\Sob$ given by the following data:
    \begin{enumerate}
        \item For each abelian category $\mathcal{A}$, a functor $\eta_\mathcal{A}:\mathcal{A}\Rightarrow \mathcal{A}\Sob$ given by the diagonal functor, sending an object $A$ to the constant functor for $A$ with identities on arrows.
        \item For each functor $F:\mathcal{A}\rightarrow \mathcal{B}$ a natural transformation $\eta_F:\eta_\mathcal{B}\circ F\Rightarrow F_*\circ \eta_\mathcal{A}$
        \[\begin{tikzcd}
        	{\mathcal{A}} & {\mathcal{B}} \\
        	{\mathcal{A}\Sob} & {\mathcal{B}\Sob}
        	\arrow["F", from=1-1, to=1-2]
        	\arrow["{\eta_\mathcal{B}}", from=1-2, to=2-2]
        	\arrow["{\eta_\mathcal{A}}"', from=1-1, to=2-1]
        	\arrow["{F_*}"', from=2-1, to=2-2]
        	\arrow["{\eta_F}"{description}, Rightarrow, from=1-2, to=2-1]
        \end{tikzcd}\]
        which is the identity, since the square commutes
    \end{enumerate}
    \item A pseudonatural transformation $m:(-)\Sob\circ (-)\Sob\Rightarrow(-)\Sob$ given by the following data:
    \begin{enumerate}
        \item For every abelian category $\mathcal{A}$, a functor $m_\mathcal{A}:(\mathcal{A}\Sob)\Sob\rightarrow \mathcal{A}\Sob$. For $A \in (\mathcal{A}\Sob)\Sob$
        %%
        \begin{equation*}
            m_\mathcal{A}(A)([n]) := A([n])([n])
        \end{equation*}
        %%
        and for $\alpha:[n]\rightarrow [m]$ we set
        %%
        \begin{equation*}
            m_\mathcal{A}(A)(\alpha):A([m])([m])\rightarrow A([n])([n]) := A([n])(\alpha) \circ A(\alpha)_{[m]} = A(\alpha)_{[n]}\circ A([m])(\alpha)
        \end{equation*}
        %%
        by naturality of $A(\alpha)$. Given a map of simplicial objects $\beta:A\Rightarrow B$ in $(\mathcal{A}\Sob)\Sob$, we set
        %%
        \begin{equation*}
            m_\mathcal{A}(\beta):m_\mathcal{A}(A)\rightarrow m_\mathcal{A}(B)
        \end{equation*}
        %%
        with $[n]$th component given by
        %%
        \begin{equation*}
            m_\mathcal{A}(\beta)_{[n]}:A([n])([n])\rightarrow B([n])([n]) := (\beta_{[n]})_{[n]}
        \end{equation*}
        \item For each functor $F:\mathcal{A}\rightarrow \mathcal{B}$ between abelian categories, a natural transformation $m_F:m_\mathcal{B}\circ (F_*)_*\Rightarrow F_*\circ m_\mathcal{A}$
        \[\begin{tikzcd}
        	{(\mathcal{A}\Sob)\Sob} & {(\mathcal{B}\Sob)\Sob} \\
        	{\mathcal{A}\Sob} & {\mathcal{B}\Sob}
        	\arrow["{(F_*)_*}", from=1-1, to=1-2]
        	\arrow["{m_\mathcal{B}}", from=1-2, to=2-2]
        	\arrow["{m_\mathcal{A}}"', from=1-1, to=2-1]
        	\arrow["{F_*}"', from=2-1, to=2-2]
        	\arrow["{m_F}"{description}, Rightarrow, from=1-2, to=2-1]
        \end{tikzcd}\]
        which is the identity since the square commutes. Indeed, for each $A \in (\mathcal{A}\Sob)\Sob$, and each $[n] \in\cat{Ob}(\Delta)$
        %%
        \begin{equation*}
            m_\mathcal{B}\circ (F_*)_*(A)([n]) = m_\mathcal{B}(F_*\circ A)([n]) = F(A([n])([n])) = F_*\circ m_\mathcal{A}(A)([n])
        \end{equation*}
        %%
        while for $\alpha:[m]\rightarrow [n]$
        %%
        \begin{align*}
            m_\mathcal{B}\circ (F_*)_*(A)(\alpha) &= m_\mathcal{B}(F_*\circ A)(\alpha) \\
            &= F(A([n])(\alpha)\circ A(\alpha)_{[m]}) \\
            &= F_*\circ m_\mathcal{A}(A)(\alpha)
        \end{align*}
        %%
        Further, for $\alpha:A\rightarrow A'$ in $(\mathcal{A}\Sob)\Sob$, and $[n] \in \cat{Ob}(\Delta)$,
        \begin{align*}
            m_\mathcal{B}\circ (F_*)_*(\alpha)_{[n]} &= m_\mathcal{B}(F_*\alpha)_{[n]} \\
            &= F(\alpha_{[n]})_{[n]}) \\
            &= F(m_\mathcal{A}(\alpha)_{[n]}) \\
            &= F_*\circ m_\mathcal{A}(\alpha)_{[n]}
        \end{align*}
        Thus the functors along each edge are equal, so the comparison cell is the identity.
    \end{enumerate}
    \item An invertible modification $\mu:m\circ (-)\Sob m\Rrightarrow m\circ m_{(-)\Sob}$ given by the following data:
    \begin{enumerate}
        \item For each abelian category $\mathcal{A}$, a natural transformation $\mu_\mathcal{A}:m_\mathcal{A}\circ (-)\Sob m_\mathcal{A}\Rightarrow m_\mathcal{A}\circ m_{\mathcal{A}\Sob}$ which has identity components since for a simplicial object $A \in ((\mathcal{A}\Sob)\Sob)\Sob$
        %%
        \begin{align*}
            m_\mathcal{A}((m_{\mathcal{A}})_*A)([n]) &= (m_\mathcal{A}\circ A)([n])([n]) \\
            &= m_\mathcal{A}(A([n]))([n]) \\
            &= A([n])([n])([n]) \\
            &= m_{\mathcal{A}\Sob}(A)([n])([n]) \\
            &= (m_\mathcal{A}\circ m_{\mathcal{A}\Sob}(A))([n]) 
        \end{align*}
        %%
        and for $\alpha:[m]\rightarrow [n]$,
        %%
        \begin{align*}
            m_\mathcal{A}((m_{\mathcal{A}})_*A)(\alpha) &= (m_\mathcal{A}\circ A)([n])(\alpha)\circ (m_\mathcal{A}\circ A)(\alpha)_{[m]} \\
            &= m_\mathcal{A}(A([n]))(\alpha)\circ m_\mathcal{A}(A(\alpha))_{[m]} \\
            &= A([n])([n])(\alpha)\circ A([n])(\alpha)_{[m]}\circ (A(\alpha)_{[m]})_{[m]} \\
            &= A([n])([n])(\alpha)\circ (A([n])(\alpha)\circ A(\alpha)_{[m]})_{[m]} \\
            &= m_{\mathcal{A}\Sob}(A)([n])(\alpha)\circ m_{\mathcal{A}\Sob}(A)(\alpha)_{[m]} \\
            &= (m_\mathcal{A}(m_{\mathcal{A}\Sob}(A)))(\alpha)
        \end{align*}
        %%
    \end{enumerate}
    \item An invertible modification $\lambda:m\circ \eta_{(-)\Sob} \Rrightarrow 1_{(-)\Sob}$ given by the following data:
    \begin{enumerate}
        \item For each abelian category $\mathcal{A}$, a natural transformation $\lambda_\mathcal{A}:m_\mathcal{A}\circ \eta_{\mathcal{A}\Sob}\Rightarrow 1_{\mathcal{A}\Sob}$ which is given by identities since for a simplicial object $A$
        %%
        \begin{equation*}
            m_\mathcal{A}(\eta_{\mathcal{A}\Sob}(A))([n]) = \eta_{\mathcal{A}\Sob}(A)([n])([n]) = A([n])
        \end{equation*}
        %%
        and for $\alpha:[m]\rightarrow [n]$,
        %%
        \begin{equation*}
            m_\mathcal{A}(\eta_{\mathcal{A}\Sob}(A))(\alpha) = \eta_{\mathcal{A}\Sob}(A)([n])(\alpha)\circ \eta_{\mathcal{A}\Sob}(A)(\alpha)_{[m]} = A(\alpha)\circ (1_A)_{[m]}=A(\alpha)
        \end{equation*}
        %%
    \end{enumerate}
    \item An invertible modification $\rho:m\circ(-)\Sob\eta\Rrightarrow 1_{(-)\Sob}$ given by the following data:
    \begin{enumerate}
        \item For each abelian category $\mathcal{A}$, a natural transformation $\rho_\mathcal{A}:m_\mathcal{A}\circ (-)\Sob\eta_{\mathcal{A}}\Rightarrow 1_{\mathcal{A}\Sob}$ which is also given by identities since for a simplicial object $A$
        %%
        \begin{equation*}
            m_\mathcal{A}((-)\Sob\eta_{\mathcal{A}}(A))([n]) = (\eta_\mathcal{A}\circ A)([n])([n]) = \eta_\mathcal{A}(A([n]))([n]) = A([n])
        \end{equation*}
        %%
        and for $\alpha:[m]\rightarrow [n]$,
        %%
        \begin{equation*}
            m_\mathcal{A}((-)\Sob\eta_{\mathcal{A}}(A))(\alpha) = (\eta_{\mathcal{A}}\circ A)([n])(\alpha)\circ (\eta_{\mathcal{A}}\circ A)(\alpha)_{[m]} = 1_{A([n])}\circ \eta_\mathcal{A}(A(\alpha))_{[m]} = A(\alpha)
        \end{equation*}
        %%
    \end{enumerate}
\end{enumerate}

Since all the higher comparison cells are identities, it follows that all coherence diagrams commute automatically, and in particular, the simplicial objects functor is a strict 2-monad on the (large) 2-category of abelian categories, $2\cat{Ab}$.

\subsubsection{Simplicial Homotopies}\label{subsec:simpHomotop}

Homotopies in categories $\mathcal{C}\Sob$ will be important in our analysis with the Dold-Kan Equivalence. This requires the consideration how to form products with simplicial sets in $\mathcal{C}\Sob$, which we can obtain from \cite[Defn 14.13.1]{StacksProject}.

\begin{defn}
    Let $\mathcal{C}$ be a category with finite coproducts and let $X \in \mathcal{C}\Sob$. If $U \in \cat{Set}\Sob$ is a finite, non-empty, simplicial set, we define the product $X \times U$ to be the simplicial object with $n$th component
    %%
    \begin{equation*}
        (X\times U)_n := \coprod_{u \in U_n}X_n
    \end{equation*}
    %%
    such that for any map $\varphi:[m]\rightarrow [n]$, $(X\times U)(\varphi):\coprod_{u \in U_n}X_n\rightarrow \coprod_{u' \in U_m}X_m$ is defined by
    %%
    \begin{equation*}
        (X\times U)(\varphi)\circ \iota_u = \iota_{U(\varphi)(u)}\circ X(\varphi)
    \end{equation*}
    %%
    Given maps $f:X\Rightarrow Y$ and $g:U\Rightarrow V$ of simplicial objects and simplicial sets, respectively, we obtain a map of simplicial objects $f\times g:X\times U\rightarrow Y\times V$ given on components by 
    %%
    \begin{equation*}
        (f\times g)_n : \coprod_{u \in U_n}X_n\rightarrow \coprod_{v \in V_n}Y_n,\;\; (f\times g)_n\circ \iota_u = \iota_{g_n(u)}\circ f_n
    \end{equation*}
    %%
\end{defn}

We can now define simplicial homotopies. Let $\Delta^n := Hom_{\Delta}(-,[n])$ be the standard $n$-simplex as a simplicial set. Recall that $\Delta^0$ is a singleton in each component, while 
%%
\begin{equation*}
    (\Delta^1)_n = \{\alpha_0^n,...,\alpha_{n+1}^n\},\;\;\alpha_i^n(j) = \left\{\begin{array}{cc} 0 & j < i \\ 1 & j \geq i \end{array}\right.
\end{equation*}
%%
By Yoneda we can identify these maps with natural isomorphisms, so in particular we have $\alpha_0^0:\Delta^0\Rightarrow \Delta^1$ and $\alpha_1^0:\Delta^0\Rightarrow \Delta^1$ corresponding to sending $0$ to $1$ and sending $0$ to $0$, respectively (note the flip). We will write $e_0 := \alpha_1^0$ and $e_1 := \alpha_0^0$. Noting that for any simplicial object $U \in \mathcal{C}\Sob$ $U\times \Delta^0 \cong U$, we obtain $e_0,e_1:U\Rightarrow U\times \Delta^1$. This is sufficient to define simplicial homotopies~\cite[Defn 14.26.1]{StacksProject}.

\begin{defn}{}
    Let $X, Y \in \mathcal{C}\Sob$ be simplicial objects in a category with finite coproducts, and let $f,g:X\Rightarrow Y$ be simplicial maps. Then a \textbf{simplicial homotopy} between $f$ and $g$ is a simplicial map $h:X\times \Delta^1\Rightarrow Y$ making the following diagram commute
    %%
    \[\begin{tikzcd}
    	X \\
    	{X\times \Delta^1} & Y \\
    	X
    	\arrow["h"{description}, from=2-1, to=2-2]
    	\arrow["{e_0}"', from=1-1, to=2-1]
    	\arrow["f", from=1-1, to=2-2]
    	\arrow["{e_1}", from=3-1, to=2-1]
    	\arrow["g"', from=3-1, to=2-2]
    \end{tikzcd}\]
    %%
    When $\mathcal{C}$ is an abelian category this defines an additive equivalence relation on the simplicial maps $X\Rightarrow Y$ \cite{weibel_1994}. Otherwise, we say $f$ and $g$ are simplicially homotopic if there is a sequence $f = f_0,f_1,...,f_n = g$ of maps such that there is a simplicial homotopy from $f_i$ to $f_{i+1}$ or from $f_{i+1}$ to $f_i$ for each $i < n$.
\end{defn}

We can extend this definition to functors valued in simplicial objects.

\begin{defn}{}
    Let $F,G:\mathcal{B}\rightarrow \mathcal{A}\Sob$ be functors valued in simplicial objects. We say $F$ and $G$ are \textbf{pointwise homotopy equivalent} if for each $B \in \mathcal{B}$, we have a simplicial homotopy equivalence $(f_B:F(B)\rightarrow G(B), g_B:G(B)\rightarrow F(B), h_B:F(B)\times \Delta^1\rightarrow F(B), h'_B:G(B)\times \Delta^1\rightarrow G(B))$. We say $F$ and $G$ are \textbf{naturally homotopy equivalent} if we have natural transformations $(f:F\Rightarrow G,g:G\rightarrow F,h:F\times \Delta^1\Rightarrow F,h':G\times \Delta^1\Rightarrow G)$ which comprise homotopy equivalences at each $B \in \mathcal{B}$.
\end{defn}

We also have a completely combinatorial description of simplicial homotopies which is equivalent we working with simplicial objects in finitely cocomplete categories~\cite{weibel_1994}. In particular, a simplicial homotopy between $f,g:X\to Y$ is a family of maps $h_i^n:X_n\to Y_{n+1}$ for $n \in \N$ and $0\leq i \leq n$, such that $Y_{d_0^{n+1}}h^n_0 = f_n$, $Y_{d_{n+1}^{n+1}}h_n^n=g_n$, and 
%%
\begin{equation*}
    Y(d_i^{n+1})h_j^n = \left\{\begin{array}{cc} h_{j-1}^{n-1}X(d_i^n) & i < j \\ Y(d_i^{n+1})h_{i-1}^n & i = j \neq 0 \\ h_j^{n-1}X(d_{i-1}^n) & i > j+1 \end{array}\right.
\end{equation*}
%%
\begin{equation*}
    Y(s_i^{n+1})h_j^n = \left\{\begin{array}{cc} h_{j+1}^{n+1}X(s_i^n) & i \leq j \\ h_j^{n+1}X(s_{i-1}^n) & i > j \end{array}\right.
\end{equation*}

We now show how these equivalences behave under composition as well as some behaviour of the category of functors into simplicial objects.

\begin{lem}[label=lem:isoOfSimpFuncs]
    If $\mathcal{A},\mathcal{B}$ are categories, there exists an isomorphism of categories
    %%
    \begin{equation*}
        \text{Fun}(\mathcal{A},\mathcal{B}\Sob) \cong \text{Fun}(\mathcal{A},\mathcal{B})\Sob
    \end{equation*}
\end{lem}
\begin{proof}
    We define a natural isomorphism $\gamma:\text{Fun}(\mathcal{A},\mathcal{B}\Sob) \to \text{Fun}(\mathcal{A},\mathcal{B})\Sob$ on an object $F:\mathcal{A}\to \mathcal{B}\Sob$ by 
    %%
    \begin{equation*}
        \gamma(F)_n(A) := F(A)_n,
    \end{equation*}
    %%
    For each $n$ $\gamma(F)_n$ is a functor so for $\alpha:[m]\to [n]$ we need to show a natural transformation $\gamma(F)_\alpha:\gamma(F)_n\to \gamma(F)_m$ which we define to be
    %%
    \begin{equation*}
        (\gamma(F)_\alpha)_A := F(A)_\alpha 
    \end{equation*}
    %%
    This is functorial since each $F(A)$ is a functor. If $f:A\to B$, then since $F(f)$ is a natural transformation, we also have the commuting square
    \[\begin{tikzcd}
        {F(A)_n} & {F(B)_n} \\
        {F(A)_m} & {F(B)_m}
        \arrow["{F(A)_\alpha}"', from=1-1, to=2-1]
        \arrow["{F(f)_n}", from=1-1, to=1-2]
        \arrow["{F(f)_m}"', from=2-1, to=2-2]
        \arrow["{F(B)_\alpha}", from=1-2, to=2-2]
    \end{tikzcd}\]
    It follows that $\gamma(F)_\alpha$ is natural. Thus $\gamma$ is a well-defined functor. Similarly, $\gamma$ has an inverse $\tau:\text{Fun}(\mathcal{A},\mathcal{B})\Sob\to \text{Fun}(\mathcal{A},\mathcal{B}\Sob)$ given by $\tau(F)(A)_n := F_n(A)$.
\end{proof}

Note that under this isomorphism, natural simplicial homotopy is simplicial homotopy in $\text{Fun}(\mathcal{A},\mathcal{B})\Sob$. 

\begin{lem}[label=lem:simpFuncCat]
    We have a functor $\text{Fun}(-,-)\Sob:\cat{Cat}^{op}\times \cat{Cat}\to \cat{Cat}$, and for any $\mathcal{A},\mathcal{B},\mathcal{C} \in \cat{Cat}$ where $\mathcal{C}$ has finite coproducts, and any functor $F:\mathcal{A}\to \mathcal{B}$, $\text{Fun}(F,\mathcal{C})\Sob:\text{Fun}(\mathcal{B},\mathcal{C})\Sob\to \text{Fun}(\mathcal{A},\mathcal{C})\Sob$ preserves simplicial homotopies. Similarly, if $\mathcal{B}$ and $\mathcal{C}$ both have finite colimits, and $F:\mathcal{B}\to \mathcal{C}$, then $\text{Fun}(\mathcal{A},F)$ preserves simplicial homotopies.
\end{lem}
\begin{proof}
    The functor $\text{Fun}(-,-)\Sob$ is simply the composite of $\text{Fun}(-,-)$ with the 2-monad $\Sob$. Now let $\mathcal{A},\mathcal{B},\mathcal{C} \in \cat{Cat}$ and let $F:\mathcal{A}\to \mathcal{B}$ be a functor. Then suppose $G,H \in \text{Fun}(\mathcal{B},\mathcal{C})\Sob$ and suppose $g,h:G\to H$ are simplicially homotopic. Then there exists $f:G\times \Delta^1\to H$ such that $f\circ e_0 = g$ and $f\circ e_1 = h$. Then $\text{Fun}(F,\mathcal{C})\Sob$ is defined by sending $K \in \text{Fun}(\mathcal{B},\mathcal{C})\Sob$ $K_F$, where $(K_F)_n := K_n\circ F$. Observe that
    %%
    \begin{equation*}
        ((G\times \Delta_1)_F)_n = (\coprod_{u \in \Delta^1_n}G_n)\circ F = \coprod_{u \in \Delta^1_n}(G_n\circ F) = ((G\circ F)\times \Delta^1)_n
    \end{equation*}
    %%
    so $(G\times \Delta^1)_F = (G\circ F)\times \Delta^1$. Additionally, $((e_i)_F)_n= ((e_i)_n)_F$, which is exactly $(e_i)_n:(G\circ F)_n\to ((G\circ F)\times \Delta^1)_n$. Thus $\text{Fun}(F,\mathcal{C})\Sob$ preserves simplicial homotopies.

    \vspace{10pt}

    Conversely, if $\mathcal{B}$ and $\mathcal{C}$ are finitely cocomplete and $F:\mathcal{B}\to \mathcal{C}$, we can use the combinatorial description of homotopies. In this case, $\text{Fun}(\mathcal{A},F)$ is defined by ${_F}H_n := F\circ H_n$ on objects and ${_F}h_n := Fh_n$ on maps. Let $f,g:H\to K$ be simplicially homotopic maps in $\text{Fun}(\mathcal{A},\mathcal{B})\Sob$, by a simplicial homotopy $h_i^n:H_n\to K_{n+1}$. Then $Fh_i^n:F\circ H_n\to F\circ K_{n+1}$ defines a simplicial homotopy between ${_F}f$ and ${_F}g$, as desired.
\end{proof}

% Post-composition also preserves simplicial homotopies if we are post-composing by functors that preserve finite coproducts.

% \begin{lem}[label=lem:postcompNat]
%     If $\mathcal{A},\mathcal{B},\mathcal{C} \in \cat{Cat}$ where $\mathcal{B},\mathcal{C}$ have finite coproducts, and $F:\mathcal{B}\to \mathcal{C}$ is a functor that preserves finite coproducts, then $\text{Fun}(\mathcal{A},F)\Sob:\text{Fun}(\mathcal{A},\mathcal{B})\Sob\to \text{Fun}(\mathcal{A},\mathcal{C})\Sob$ preserves simplicial homotopies.
% \end{lem}
% \begin{proof}
%     Let $H,K \in \text{Fun}(\mathcal{A},\mathcal{B})\Sob$ and let $h,k:H\to K$ be simplicially homotopic by a map $s:H\times \Delta^1\to K$. Note that $\text{Fun}(\mathcal{A},F)$ is defined by ${_F}H_n := F\circ H_n$ on objects and ${_F}h_n := Fh_n$ on maps. By Corollary~\ref{cor:presFunccoLim}, since $F$ preserves colimits we have that ${_F}(H\times \Delta^1) \cong {_F}H\times \Delta^1$, and under this isomorphism $e_i:H\times \Delta^0\to H\times \Delta^1$ corresponds to $e_i:{_F}H\times \Delta^0\to {_F}H\times \Delta^1$. Thus $\text{Fun}(\mathcal{A},F)$ produces the desired commuting triangles to make ${_F}s$ a simplicial homotopy from ${_F}h$ to ${_F}k$. 
% \end{proof}



\begin{lem}[label=lem:diagHo]
    Let $F,G: \mathcal{C}\rightarrow (\mathcal{B}\Sob)\Sob$ and let $f,g:F\to G$ be naturally simplicially homotopic. Then $\Delta_\mathcal{B}f,\Delta_\mathcal{B}g:\Delta_\mathcal{B}\circ F\to\Delta_\mathcal{B}\circ G$ are naturally simplicially homotopic.
\end{lem}
\begin{proof}
    Let $h:F\times \Delta^1\to G$ be a simplicial homotopy between $f$ and $g$. Then 
    %%
    \begin{equation*}
        \Delta_\mathcal{B}(F\times \Delta^1)(C)_n = \coprod_{u \in \Delta^1_n}F(C)_{n,n} = (\Delta_\mathcal{B}F\times \Delta^1)(C)_n
    \end{equation*}
    %%
    In particular, $\Delta_\mathcal{B}h$ defines a simplicial homotopy from $\Delta_\mathcal{B}f$ to $\Delta_\mathcal{B}g$.
\end{proof}


\begin{lem}[label=lem:Precomp]
    Let $F,G: \mathcal{C}\rightarrow \mathcal{B}\Sob$ and let $f,g:F\to G$ be naturally simplicially homotopic. Then $\Delta^{op}(f),\Delta^{op}(g):\Delta^{op}(F)\to\Delta^{op}(G)$ are naturally simplicially homotopic.
\end{lem}
\begin{proof}
    Let $h:F\times \Delta^1\to G$ be a simplicial homotopy between $f$ and $g$. Then observe that 
    %%
    \begin{equation*}
        \Delta^{op}(F\times \Delta^1)(C)_n = F(C_n)\times \Delta^1 = \Delta^{op}(F)(C)_n\times \Delta^1
    \end{equation*}
    %%
    Then $\Delta^{op}(h):\Delta^{op}(F)\times \Delta^1\to \Delta^{op}(G)$, and since $\Delta^{op}$ is a strict 2-functor it preserves composition so $\Delta^{op}(h)$ is a homotopy from $\Delta^{op}(f)$ to $\Delta^{op}(g)$.
\end{proof}




\subsubsection{Chain Homotopies}\label{subsec:chainHomotop}


In this section we expand on the behaviour of natural chain homotopies and the interaction between natural chain homotopies and direct sums. We begin by proving equivalent formulations of natural chain homotopies.

\begin{lem}[label=lem:funcChain]
    For $\mathcal{A}$ an abelian category, we have an isomorphism of categories
    \begin{equation}\label{eq:ChainFunc}
        \cat{Ch}(\text{Fun}(\mathcal{B},\mathcal{A}))\cong \text{Fun}(\mathcal{B},\cat{Ch}(\mathcal{A}))
    \end{equation}
\end{lem}
\begin{proof}
    Define a functor $\gamma:\cat{Ch}(\text{Fun}(\mathcal{B},\mathcal{A}))\rightarrow \text{Fun}(\mathcal{B},\cat{Ch}(\mathcal{A}))$ given on a chain complex of functors $F_\bullet$ by
    %%
    \begin{equation*}
        \gamma(F_\bullet)(B)_n := F_n(B),\;\forall B \in \mathcal{B}
    \end{equation*}
    %%
    where the differentials are given by the natural transformation differentials in $F_\bullet$ evaluated at $B$. Given a map of chain complexes $\alpha_\bullet:F_\bullet\rightarrow G_\bullet$ we set
    %%
    \begin{equation*}
        (\gamma(\alpha_\bullet)_B)_n := (\alpha_n)_B
    \end{equation*}
    %%
    This defines a chain map $\gamma(F_\bullet)(B)\rightarrow \gamma(G_\bullet)(B)$ since $\alpha_\bullet$ is a chain map of natural transformations, so all squares with differentials commute. Further, $\gamma(\alpha_\bullet)$ is natural in $B$ since if $f:B\rightarrow B'$ is a map in $\mathcal{B}$, then in
    \[\begin{tikzcd}
    	& {F_{n+1}(B')} &&& {F_n(B')} \\
    	{F_{n+1}(B)} && {F_n(B)} \\
    	& {G_{n+1}(B')} &&& {G_n(B')} \\
    	{G_{n+1}(B)} && {G_n(B)}
    	\arrow["{\partial_{n+1}}"{pos=0.7}, from=2-1, to=2-3]
    	\arrow["{(\alpha_{n+1})_B}"', from=2-1, to=4-1]
    	\arrow["{\partial_{n+1}}"', from=4-1, to=4-3]
    	\arrow["{(\alpha_n)_B}"{pos=0.3}, from=2-3, to=4-3]
    	\arrow["{F_{n+1}(f)}", from=2-1, to=1-2]
    	\arrow["{F_n(f)}"', from=2-3, to=1-5]
    	\arrow["{\partial_{n+1}}", from=1-2, to=1-5]
    	\arrow["{(\alpha_n)_{B'}}", from=1-5, to=3-5]
    	\arrow["{G_n(f)}"', from=4-3, to=3-5]
    	\arrow["{G_{n+1}(f)}", from=4-1, to=3-2]
    	\arrow["{(\alpha_{n+1})_{B'}}"'{pos=0.6}, from=1-2, to=3-2]
    	\arrow["{\partial_{n+1}}"', from=3-2, to=3-5]
    \end{tikzcd}\]
    the front and back faces commute since $\alpha_\bullet$ is a chain map, the top and bottom faces commute by naturality of the boundary maps, and the side faces commute by naturality of the $\alpha_n$. Since this definition is in terms of the components of $\alpha_\bullet$ it is inherently functorial. 

    Next we must witness an inverse $\rho:\text{Fun}(\mathcal{B},\cat{Ch}(\mathcal{A}))\rightarrow \cat{Ch}(\text{Fun}(\mathcal{B},\mathcal{A}))$ functor. Given $F:\mathcal{B}\rightarrow \cat{Ch}(\mathcal{A})$ we set $\rho(F)$ to have $n$th component $(-)_n\circ F$ and differential $\partial_n$ given by components the $n$th differential of $F$ evaluated at $B  \in \mathcal{B}$. Naturality of the differential equates to the commutivity of 
    \[\begin{tikzcd}
    	{F(B)_n} & {F(B)_{n-1}} \\
    	{F(B')_n} & {F(B')_{n-1}}
    	\arrow["{F(f)_n}"', from=1-1, to=2-1]
    	\arrow["{\partial_n(B)}", from=1-1, to=1-2]
    	\arrow["{\partial_n(B')}"', from=2-1, to=2-2]
    	\arrow["{F(f)_{n-1}}", from=1-2, to=2-2]
    \end{tikzcd}\]
    for any $f:B\rightarrow B'$, which follows since $F(f)$ is a chain map. Next, if $\alpha:F\rightarrow G$ is a natural transformation between two such functors we set $\rho(\alpha)$ such that $\rho(\alpha)_n$ is the natural transformation defined by $(\rho(\alpha)_n)_B := (\alpha_B)_n$. Naturality and the chain condition follow by the commutivity of 
    \[\begin{tikzcd}
    	& {F(B')_{n+1}} &&& {F(B')_n} \\
    	{F(B)_{n+1}} && {F(B)_n} \\
    	& {G(B')_{n+1}} &&& {G(B')_n} \\
    	{G(B)_{n+1}} && {G(B)_n}
    	\arrow["{\partial_{n+1}(B)}"{pos=0.7}, from=2-1, to=2-3]
    	\arrow["{(\alpha_B)_{n+1}}"', from=2-1, to=4-1]
    	\arrow["{\partial_{n+1}(B)}"', from=4-1, to=4-3]
    	\arrow["{(\alpha_B)_n}"{pos=0.2}, from=2-3, to=4-3]
    	\arrow["{F(f)_{n+1}}", from=2-1, to=1-2]
    	\arrow["{F(f)_n}"', from=2-3, to=1-5]
    	\arrow["{\partial_{n+1}(B')}", from=1-2, to=1-5]
    	\arrow["{(\alpha_{B'})_n}", from=1-5, to=3-5]
    	\arrow["{G(f)_n}"', from=4-3, to=3-5]
    	\arrow["{G(f)_{n+1}}", from=4-1, to=3-2]
    	\arrow["{(\alpha_{B'})_{n+1}}"'{pos=0.6}, from=1-2, to=3-2]
    	\arrow["{\partial_{n+1}(B')}"', from=3-2, to=3-5]
    \end{tikzcd}\]
    where the bottom and top faces are the fact $G(f)$ and $F(f)$ are chain maps, the front and back faces are the fact $\alpha_B$ is a chain map, and finally the side faces are naturality of $\alpha$. Once again, since $\rho(\alpha)$ is defined in terms of the components of $\alpha$ the assignment is inherently functorial. Further, these operations are exactly inverse of each other as they correspond to swapping the element and natural number indices (in particular, on the other side of the Dold-Kan Equivalence this is simply the swap natural isomorphism on functors of two variables).
\end{proof}

Moving forward we write $\text{Fun}^\cat{Ch}$ for the isomorphism $\gamma$ in the proof. We also have another description of this category:

\begin{lem}[label=lem:adjChFuncCat]
    Let $\Z$ denote the category associated with the linear ordered set $(\Z,\geq)$. Let $\text{Fun}_{\cat{Ch}}(\mathcal{B}\times \Z,\mathcal{A})$ be the sub-category such that $F(-,n+2\leq n)$ is the zero map. Then under the adjunction $-\times \Z\dashv \text{Fun}(\Z,\mathcal{A})$ we have the isomorphism
    %%
    \begin{equation*}
        \text{Fun}_{\cat{Ch}}(\mathcal{B}\times \Z,\mathcal{A})\cong \text{Fun}(\mathcal{B},\text{Fun}_{\cat{Ch}}(\Z,\mathcal{A}))
    \end{equation*}
    %%
    and $\text{Fun}_{\cat{Ch}}(\Z,\mathcal{A})\cong \cat{Ch}(\mathcal{A})$.
\end{lem}
\begin{proof}
    It is sufficient to show that the isomorphism in the adjunction restricts to the isomorphism above. But this follows immediately by definition, so all that there is to show is $\text{Fun}_{\cat{Ch}}(\Z,\mathcal{A})\cong \cat{Ch}$. But this is also immediate by sending $A:\Z\to \mathcal{A}$ to $A_\bullet$, where $A_n = A(n)$ and $\partial_n^A = A(n\geq n-1)$, and vice-versa.
\end{proof}

\begin{lem}[label=lem:natHomotopIsChainFunctHomotop]
    Chain homotopies correspond to natural chain homotopies of functors under the isomorphism $\text{Fun}^{\cat{Ch}}:\cat{Ch}(\text{Fun}(\mathcal{B},\mathcal{A}))\to \text{Fun}(\mathcal{B},\cat{Ch}(\mathcal{B}))$.
\end{lem}
\begin{proof}
    First, let $\alpha,\beta:F_\bullet\to G_\bullet$ be a map of chain complexes of functors in $\cat{Ch}(\text{Fun}(\mathcal{B},\mathcal{A}))$. Then a chain homotopy from $\alpha$ to $\beta$ is, for each $n \in \Z$, a natural transformation $s_n:F_n\Rightarrow G_{n+1}$ such that 
    %%
    \begin{equation*}
        \partial_{n+1}^G\circ s_n + s_{n-1}\circ \partial_n^G = \alpha_n-\beta_n
    \end{equation*}
    %%
    On the other hand, under $\text{Fun}^\cat{Ch}$ $\alpha$ and $\beta$ correspond to natural transformations between functors valued in chain complexes, $F,G$. By definition, a natural chain homotopy is then a family of natural transformations $s_n: (-)_n\circ F\to (-)_{n+1}\circ G$, where $(-)_n:\cat{Ch}(\mathcal{A})\to \mathcal{A}$. But this is precisely the same data as the chain homotopy in $\cat{Ch}(\text{Fun}(\mathcal{B},\mathcal{A}))$.
\end{proof}

Next, we also show an equivalent form of chain homotopies.

\begin{lem}[label=lem:cylHomotop]
    Chain homotopies between maps $f,g:A_\bullet\to B_\bullet$ in $\cat{Ch}(\mathcal{A})$ are equivalent to chain maps $H:\text{cyl}(-1_{A_\bullet})\to B_\bullet$ such that the triangle
    %%
    \[\begin{tikzcd}
        {\text{cyl}(-1_{A_\bullet})} \\
        {A_\bullet\oplus A_\bullet} & {B_\bullet}
        \arrow["{f+g}"', from=2-1, to=2-2]
        \arrow["H", from=1-1, to=2-2]
        \arrow["{q_1+q_2}", from=2-1, to=1-1]
    \end{tikzcd}\]
    %%
    commutes where $q_1$ is the inclusion in the top of the cylinder and $q_2$ is the inclusion in the bottom cylinder.
\end{lem}
\begin{proof}
    We begin with a triangle as in the statement of the Lemma where $\text{cyl}(1_{A_\bullet})$ is the chain complex with $n$th degree term given by $A_{n-1}\oplus A_n\oplus A_n$ and chain map given by 
    %%
    \begin{equation*}
        A_n\oplus A_{n+1}\oplus A_{n+1}\xrightarrow{\begin{pmatrix} \partial_n^A & 0 & 0 \\ (-1)^n1_{A_n} & \partial_{n+1}^A & 0 \\ (-1)^{n+1}1_{A_n} & 0 & \partial_{n+1}^A \end{pmatrix}}A_{n-1}\oplus A_n\oplus A_n
    \end{equation*}
    %%
    Additionally, $q_1$ is given by 
    %%
    \begin{equation*}
        A_n\xrightarrow{\begin{pmatrix} 0 \\ 1_{A_n} \\ 0 \end{pmatrix}} A_{n-1}\oplus A_n\oplus A_n
    \end{equation*}
    %%
    and $q_2$ is given by 
    %%
    \begin{equation*}
        A_n\xrightarrow{\begin{pmatrix} 0 \\ 0 \\ 1_{A_n} \end{pmatrix}} A_{n-1}\oplus A_n\oplus A_n
    \end{equation*}
    %%
    Then a map $H:\text{cyl}(1_{A_\bullet})\to B_\bullet$ making the triangle commute is on each degree of the form
    %%
    \begin{equation*}
        A_{n-1}\oplus A_n\oplus A_n\xrightarrow{\begin{pmatrix} (-1)^{n-1}s_{n-1} & f_n & g_n \end{pmatrix}} B_n
    \end{equation*}
    %%
    where the chain map condition reduces to 
    %%
    \begin{equation*}
        \partial_n^B\circ s_{n-1}+s_{n-2}\circ \partial_{n-1}^A = f_{n-1}-g_{n-1}
    \end{equation*}
    %%
    which is exactly the condition for a chain homotopy.
\end{proof}

% \begin{lem}[label=lem:totAdd]
%     Let $F:\cat{Ch}(\mathcal{A})\to \cat{Ch}(\mathcal{B})$ be an additive functor. By Lemma~\ref{lem:funcActChain} we have an additive functor $\cat{Ch}(F):\cat{Ch}^2(\mathcal{A})\to \cat{Ch}^2(\mathcal{B})$ given by $F$ acting component-wise. Then
%     %%
%     \begin{equation*}
%         F\circ \text{Tot}_\mathcal{A} \cong \text{Tot}_\mathcal{B}\circ \cat{Ch}(F)
%     \end{equation*}
%     %%
% \end{lem}
% \begin{proof}
%     Let $A_{\bullet,\bullet} \in \cat{Ch}^2(\mathcal{A})$
% \end{proof}

% \begin{cor}[label=cor:cylinderPres]
%     If $F:\cat{Ch}(\mathcal{A})\to \cat{Ch}(\mathcal{B})$ is additive and $f:A_\bullet \to B_\bullet$ is a map of chain complexes, then $F(\text{cyl}(f)) \cong \text{cyl}(F(f))$.
% \end{cor}
% \begin{proof}
%     By Lemma~\ref{lem:funcActChain} we have an additive functor $\cat{Ch}(F):\cat{Ch}^2(\mathcal{A})\to \cat{Ch}^2(\mathcal{B})$ given by $F$ acting component-wise. 
% \end{proof}

% Using these equivalent characterizations of chain homotopy we prove that exact functors of chain complexes preserve chain homotopies.

% \begin{lem}[label=lem:exactChFunc]
%     Let $F:\cat{Ch}(\mathcal{A})\to \cat{Ch}(\mathcal{B})$ be an exact functor. Then $F$ preserves chain homotopies.
% \end{lem}
% \begin{proof}
%     \textbf{TBD}
% \end{proof}


We now prove some preliminary results on exactness and preserving chain homotopies in order to show the exactness and preservation of limits for this construction.

\begin{lem}[label=lem:TotExact]
    The totalization functor $\text{Tot}:\cat{Ch}^2(\mathcal{A})\rightarrow \cat{Ch}(\mathcal{A})$ is exact.
\end{lem}
\begin{proof}
    Let 
    %%
    \begin{equation*}
        0\rightarrow A_1\xrightarrow{f_1} A_2\xrightarrow{f_2} A_3\rightarrow 0
    \end{equation*}
    %%
    be a short exact sequence of bicomplexes in $\mathcal{A}$. This becomes a sequence of complexes 
    %%
    \begin{equation*}
        \text{Tot}(A_1)\xrightarrow{\text{Tot}(f_1)}\text{Tot}(A_2)\xrightarrow{\text{Tot}(f_2)}\text{Tot}(A_3)
    \end{equation*}
    %%
    where at a given $n$,
    %%
    \begin{equation*}
        \text{Tot}(A_i)_n = \bigoplus_{j=0}^n(A_i)_{j,n-j}
    \end{equation*}
    %%
    and
    %%
    \begin{equation*}
        \text{Tot}(f_i)_n = \bigoplus_{j=0}^n (f_i)_{j,n-j}
    \end{equation*}
    %%
    Note that the sequence of bicomplexes being exact means that each component sequence in $\mathcal{A}$ is exact. Then the component sequence of the totalization at $n$ is a finite direct sum of exact sequences, and hence exact.
\end{proof}

\begin{lem}[label=pushForward]
    Let $F:\mathcal{B}\rightarrow \mathcal{C}$ be an exact functor between abelian categories. Then for a category $\mathcal{A}$, $F_*:\text{Fun}(\mathcal{A},\mathcal{B})\rightarrow \text{Fun}(\mathcal{A},\mathcal{C})$ is exact.
\end{lem}
\begin{proof}
    Let $0\rightarrow G_1\xrightarrow{\eta_1} G_2\xrightarrow{\eta_2} G_3\rightarrow 0$ be a short exact sequence of functors from $\mathcal{A}$ to $\mathcal{B}$. Since abelian categories are finitely complete and cocomplete, finite limits and colimits in $\text{Fun}(\mathcal{A},\mathcal{B})$ and $\text{Fun}(\mathcal{A},\mathcal{C})$ are computed pointwise, so it is sufficient to prove the lemma at a given $A \in \mathcal{A}$. This follows by exactness of $F$.
\end{proof}

In order to prove our desired exactness result we first introduce a naive notion chain complexes of functors for the subcategory of additive functors.

\begin{lem}[label=lem:funcActChain]
    Let $\text{Fun}_{Add}(\mathcal{A},\mathcal{C})$ be the category of additive functors between abelian categories with all natural transformations. Then we have a functor
    %%
    \begin{equation*}
        \cat{Ch}:\text{Fun}_{Add}(\mathcal{A},\mathcal{C})\rightarrow \text{Fun}_{Add}(\cat{Ch}(\mathcal{A}),\cat{Ch}(\mathcal{C}))
    \end{equation*}
    given by sending functors to their action componentwise.
\end{lem}
\begin{proof}
    Let $\mathcal{F} \in \text{Fun}_{Add}(\mathcal{A},\mathcal{C})$. Since $\mathcal{F}$ is additive it preserves $0$'s and hence sends chain complexes to chain complexes. Then let $f_\bullet:A_\bullet\rightarrow A_\bullet'$ be a map of chain complexes. Then $\cat{Ch}(\mathcal{F})(f_\bullet)_n := \mathcal{F}(f_n)$, and since $\mathcal{F}$ is additive
    %%
    \begin{equation*}
        \mathcal{F}(f_n)\mathcal{F}(\partial_{n+1}^A)-\mathcal{F}(\partial_n^{A'})\mathcal{F}(f_{n+1}) = \mathcal{F}(f_n\partial_{n+1}^A-\partial_n^{A'}f_{n+1}) = \mathcal{F}(0) = 0
    \end{equation*}
    %%
    so $\cat{Ch}(\mathcal{F})(f_\bullet)$ is a chain map. Further, since $\cat{Ch}(\mathcal{F})$ is defined componentwise and $\mathcal{F}$ is a functor and additive, $\cat{Ch}(\mathcal{F})$ is a functor and additive. 


    Next, let $\eta:\mathcal{F}\rightarrow \mathcal{G}$ be a natural transformation between additive functors. Then define $\cat{Ch}(\eta)_{A_\bullet}:\cat{Ch}(\mathcal{F})(A_\bullet)\rightarrow \cat{Ch}(\mathcal{G})(A_\bullet)$ by $(\cat{Ch}(\eta)_{A_\bullet})_n := \eta_{A_n}$. Then $\cat{Ch}(\eta)_{A_\bullet}$ is a chain map by naturality of $\eta$. Further, $\cat{Ch}(\eta)$ is natural again by naturality of $\eta$, which makes the following diagram commute for $f_\bullet:A_\bullet\rightarrow A_\bullet$:
    \[\begin{tikzcd}
    	{\mathcal{F}(A_n)} & {\mathcal{F}(B_n)} \\
    	{\mathcal{G}(A_n)} & {\mathcal{G}(B_n)}
    	\arrow["{\eta_{A_n}}"', from=1-1, to=2-1]
    	\arrow["{\mathcal{G}(f_n)}"', from=2-1, to=2-2]
    	\arrow["{\eta_{B_n}}", from=1-2, to=2-2]
    	\arrow["{\mathcal{F}(f_n)}", from=1-1, to=1-2]
    \end{tikzcd}\]
    Since $\cat{Ch}(\eta)$ is defined componentwise it preserves composites and identities.
\end{proof}

\begin{lem}[label=lem:funcActChainHori]
    Let $F,G:\mathcal{B}\to \mathcal{C}$, $H:\mathcal{C}\to \mathcal{D}$, and $K:\mathcal{A}\to \mathcal{B}$ be additive functors. If $\eta:F\Rightarrow G$ is a natural transformation, then
    %%
    \begin{equation*}
        \cat{Ch}(H\eta_K) = \cat{Ch}(H)\cat{Ch}(\eta)_{\cat{Ch}(K)}
    \end{equation*}
\end{lem}
\begin{proof}
    From the proof of Lemma~\ref{lem:funcActChain} we have that for a chain complex $A_\bullet$, 
    %%
    \begin{equation*}
        (\cat{Ch}(H\eta_K)_{A_\bullet})_n = H(\eta_{K(A_n)}) = H(\eta_{\cat{Ch}(K)(A_\bullet)_n}) = H((\cat{Ch}(\eta)_{\cat{Ch}(K)(A_\bullet)})_n) = \cat{Ch}(H)(\cat{Ch}(\eta)_{\cat{Ch}(K)(A_\bullet)})_n
    \end{equation*}
    %%
\end{proof}

Note that Lemma~\ref{lem:funcActChainHori} shows that $\cat{Ch}$ is a strict 2-functor on the category of abelian categories with additive functors between them. We now show how additive functors acting componentwise preserve chain homotopies.

\begin{lem}[label=lem:addFuncPres]
    Let $F:\mathcal{A}\to \mathcal{B}$ be an additive functor. Then $\cat{Ch}(F):\cat{Ch}(\mathcal{A})\to \cat{Ch}(\mathcal{B})$ preserves chain homotopies.
\end{lem}
\begin{proof}
    Let $f,g:A_\bullet\to B_\bullet$ be chain maps with a homotopy $s_n:A_n\to B_{n+1}$ for $n \in \Z$ from $f$ to $g$. Since $F$ is additive we have that
    %%
    \begin{equation*}
        F(\partial_{n+1}^B)\circ F(s_n)+F(s_{n-1})\circ F(\partial_n^A) = F(\partial_{n+1}^B\circ s_n+s_{n-1}\circ \partial_n^A) = F(f_n-g_n) = F(f_n)-F(g_n)
    \end{equation*}
    %%
    Since $\cat{Ch}(F)(f)_n := F(f_n)$ and $\cat{Ch}(F)(g)_n := F(g_n)$, it follows that $F(f)$ and $F(g)$ are homotopic by $F(s_n):F(A_n)\to F(B_{n+1})$.
\end{proof}

We also have an analogous result for the chain construction from comonads.

\begin{lem}[label=lem:ChConsPres]
    Let $(C,\epsilon,\delta)$ be a comonad on $\mathcal{A}$ which is also an additive functor. Then $C^{\cat{Ch}}:\mathcal{A}\to \cat{Ch}(\mathcal{A})$ is additive.
\end{lem}
\begin{proof}
    Let $f,g:A\to B$ in $\mathcal{A}$. Then since $C$ is additive, so is all of its powers, so \[C^{\cat{Ch}}(f+g)_n := C^n(f+g)=C^n(f)+C^n(g) = C^\cat{Ch}(f)_n+C^\cat{Ch}(g)_n\]
    Thus $C^{\cat{Ch}}$ is additive.
\end{proof}

By Lemma~\ref{lem:ChConsPtwise} and Lemma~\ref{ lem:addFuncPres} it follows that $\cat{Ch}(C)^\cat{Ch}$ preserves homotopies. Another important result for this construction is how it is affected by isomorphisms between comonads.

\begin{lem}[label=lem:ChComIso]
    Let $(C,\epsilon,\delta)$ be a comonad on $\mathcal{A}$ and let $(C',\epsilon',\delta')$ be a comonad on $\mathcal{A}'$. Suppose $\gamma:\mathcal{A}\to \mathcal{A}'$ is an additive isomorphism of categories such that $C' = \gamma\circ C\circ \gamma^{-1}$, $\epsilon' = \gamma\epsilon_{\gamma^{-1}}$, and $\delta' = \gamma\delta'_{\gamma^{-1}}$. Then there is an equality
    %%
    \begin{equation*}
        {C'}^{\cat{Ch}} = \cat{Ch}(\gamma)\circ C^{\cat{Ch}}\circ \gamma^{-1}
    \end{equation*}
\end{lem}
\begin{proof}
    This equality is an immediate consequence of the construction of $C^{\cat{Ch}}$ and the specified equalities for the comonad natural transformations. Indeed, observe that for $A' \in \mathcal{A}'$
    %%
    \begin{align*}
        \cat{Ch}(\gamma)\circ C^{\cat{Ch}}\circ \gamma^{-1}(A') &= \cat{Ch}(\gamma)(\cdots \to C^2\gamma^{-1}(A')\xrightarrow{C\epsilon_{\gamma^{-1}(A')}-\epsilon_{C\gamma^{-1}(A')}}C\gamma^{-1}(A')\xrightarrow{\epsilon_{\gamma^{-1}(A')}}\gamma^{-1}(A')) \\
        &= \cdots \to \gamma C^2\gamma^{-1}(A')\xrightarrow{\gamma C\epsilon_{\gamma^{-1}(A')}-\gamma \epsilon_{C\gamma^{-1}(A')}}\gamma C\gamma^{-1}(A')\xrightarrow{\gamma \epsilon_{\gamma^{-1}(A')}}\gamma \gamma^{-1}(A') \\
        &= \cdots \to {C'}^2(A')\xrightarrow{C'\epsilon'_{A'}-\epsilon'_{C'(A')}}C'(A')\xrightarrow{\epsilon'_{A'}}A'
    \end{align*}
    %%
\end{proof}


Next, we have the following result on (co)limits in chain complex categories, which is a analoguous to the result for functor categories.

\begin{lem}[label=lem:computecoLim]
    Let $\mathcal{A}$ be an abelian category with $I$ shaped (co)limits for a small category $I$. Then $\cat{Ch}(\mathcal{A})$ has $I$ shaped (co)limits which are computed pointwise.
\end{lem}
\begin{proof}
    We provide the proof in the case of colimits, while the case of limits is analoguous. Let $D:I\to \cat{Ch}(\mathcal{A})$ be an $I$ shaped diagram in $\cat{Ch}(\mathcal{A})$. 
    
    
    For each $n \in \Z$ we have a functor $(-)_n:\cat{Ch}(\mathcal{A})\to \mathcal{A}$ given by projecting on the $n$th component. By assumption, for each $n \in \Z$ the colimit of $(-)_n\circ D:I\to \mathcal{A}$ exists. Denote this limit by $D_n$, and denote its injections by $\iota_{n,i}$, for $i \in I$. For each $i \in I$ we have an induced map $D(i)_{n+1}\to D(i)_n \to D_n$ obtained by composing with the the boundary map for $D(i)_\bullet \in \cat{Ch}(\mathcal{A})$ and the inclusion. By the universal property of the colimit we obtain a unique map making the square
    \[\begin{tikzcd}
    	{D_{n+1}} & {D_n} \\
    	{D(i)_{n+1}} & {D(i)_n}
    	\arrow["{\partial^{D(i)}_{n+1}}"', from=2-1, to=2-2]
    	\arrow[dashed, from=1-1, to=1-2]
    	\arrow["{\iota_{n+1,i}}", from=2-1, to=1-1]
    	\arrow["{\iota_{n,i}}"', from=2-2, to=1-2]
    \end{tikzcd}\]
    commute. Let $\partial_{n+1}$ denote this map. Then by uniqueness of this map and the fact that pasting two such squares together gives a commuting rectangle
    %%
    \[\begin{tikzcd}
    	{D_{n+1}} & {D_n} & {D_{n-1}} \\
    	{D(i)_{n+1}} & {D(i)_n} & {D(i)_{n-1}}
    	\arrow["{\partial^{D(i)}_{n+1}}"', from=2-1, to=2-2]
    	\arrow[dashed, from=1-1, to=1-2]
    	\arrow["{\iota_{n+1,i}}", from=2-1, to=1-1]
    	\arrow["{\iota_{n,i}}"', from=2-2, to=1-2]
    	\arrow["{\partial_n^{D(i)}}"', from=2-2, to=2-3]
    	\arrow["{\iota_{n-1,i}}"', from=2-3, to=1-3]
    	\arrow[dashed, from=1-2, to=1-3]
    \end{tikzcd}\]
    %%
    we must have that $\partial_n\circ \partial_{n+1}$ is the zero map. This implies that $(D_\bullet,\partial_\bullet)$ defines a chain complex on $\mathcal{A}$. It remains to show that it is the colimit of $D$. But by construction any other such chain complex with maps out of each $D(i)$ induces maps unique pointwise maps which form commuting squares in a chain map by their uniqueness.
\end{proof}


This allows us to also describe how totalization behaves under the isomorphism from functors valued in chain complexes and chain complexes of functors.

\begin{lem}[label=lem:totFuncCh]
    For abelian categories $\mathcal{A},\mathcal{B}$, we have a natural isomorphism
    %%
    \begin{equation*}
        (\text{Tot}_{\mathcal{A}})_*\circ \text{Fun}^{\cat{Ch}}\circ \text{Fun}^{\cat{Ch}} \cong \text{Fun}^{\cat{Ch}}\circ \text{Tot}_{\text{Fun}(\mathcal{B},\mathcal{A})}
    \end{equation*}
    %%
\end{lem}
\begin{proof}
    Let $F_{\bullet,\bullet} \in \cat{Ch}^2(\text{Fun}(\mathcal{B},\mathcal{A}))$. Let $F:\text{Fun}(\mathcal{B},\cat{Ch}^2(\mathcal{A}))$ be the associated functor under two applications of $\text{Fun}^{\cat{Ch}}$. For $B \in \mathcal{B}$, the left functor gives the complex
    %%
    \begin{equation*}
        \text{Tot}_{\mathcal{A}}(F(B))_n := \bigoplus_{p+q=n}F_{p,q}(B)
    \end{equation*}
    %%
    at $B$, while the right functor gives the complex
    %%
    \begin{equation*}
        \text{Fun}^{\cat{Ch}}(\text{Tot}_{\text{Fun}(\mathcal{B},\mathcal{A})}(F_{\bullet,\bullet}))(B)_n := \left(\bigoplus_{p+q=n}F_{p,q}\right)(B) \cong \bigoplus_{p+q=n}F_{p,q}(B)
    \end{equation*}
    %%
    Under the adjunction in Lemma~\ref{lem:adjChFuncCat}, these give functors $\text{Fun}_{\cat{Ch}}(\mathcal{B},\cat{Ch}(\mathcal{A}))\cong \text{Fun}_{\cat{Ch}}(\mathcal{B}\times \Z,\mathcal{A})$. Then we can use the results in Section~\ref{sec:colimFuncs} to conclude that 
    %%
    \begin{equation*}
        \text{Fun}^{\cat{Ch}}(\text{Tot}_{\text{Fun}(\mathcal{B},\mathcal{A})}(F_{\bullet,\bullet})) \cong  \text{Tot}_{\mathcal{A}}(F)
    \end{equation*} 
    %%
    for all $F_{\bullet,\bullet}$, and further these isomorphisms are natural in $F_{\bullet,\bullet}$ \textbf{STILL NEED SOME CLARIFICATION}.
\end{proof}
    
    \subsection{Properties of Dold-Kan and Simplicial Homotopies}\label{sec:doldKan}

    In this section we collect properties related to the Dold-Kan equivalence, and unify notation for use in the remainder of the notes. We first recall the statement of the general Dold-Kan equivalence from \cite[Thm 14.24.3]{StacksProject}.

\begin{thm}[label=thm:DoldKanEquiv]{}
    For $\mathcal{A}$ an abelian category, there is an equivalence of categories $N :\mathcal{A}^{\Delta^{op}}\simeq \cat{Ch}(\mathcal{A}): \Gamma$ \cite{StacksProject}. Let $\eta:1_{\cat{Ch}(\mathcal{A})}\Rightarrow N\circ \Gamma$ and $\varepsilon:\Gamma\circ N\Rightarrow 1_{\mathcal{A}^{\Delta^{op}}}$ be the unit and counit for the equivalence, which can be chosen to satisfy the triangle identities.

    $N$ is given explicitly on a simplicial object $X$ by
    %%
    \begin{equation*}
        N(X)_n := \left\{\begin{array}{cc} \bigcap_{i=0}^{n-1}\ker(d_n^i) & n \geq 1 \\ X_0 & n = 0 \\ 0 & n < 0 \end{array}\right.
    \end{equation*}
    %%
    with differential given by $(-1)^nd_n^n:N(X)_n\rightarrow N(X)_{n-1}$. $N$ is given on arrows by restriction. On the other hand, for a chain complex $A_\bullet$, with boundary maps $d_{A,n}$, $\Gamma(A_\bullet)$ is given on objects by 
    %%
    \begin{equation*}
        \Gamma(A_\bullet)_n = \bigoplus_{\alpha\in I_n}A_{k(\alpha)}
    \end{equation*}
    %%
    where $I_n = \{\alpha:[n]\rightarrow \N\vert \ran(\alpha) = [k(\alpha)]\}$ where $k(\alpha)$ is the maximum element in the image. For a monotonic map $\varphi:[m]\rightarrow [n]$, we define $\Gamma$ using the universal property of the biproduct by
    %%
    \begin{equation*}
        \pi_\beta\circ \Gamma(A_\bullet)(\varphi)\circ \iota_\alpha := \left\{\begin{array}{cc} 0 & \alpha\circ \varphi \notin I_m \\ 0 & \alpha\circ \varphi \in I_m, k(\alpha\circ \varphi) \neq k(\alpha),k(\alpha)-1 \\ 0 & k(\alpha\circ \varphi) \neq \beta \\ 1_{A_{k(\alpha)}} & \alpha\circ \varphi \in I_m, k(\alpha\circ \varphi) = k(\alpha) \\  (-1)^{k(\alpha)}d_{A,k(\alpha)} & \alpha\circ \varphi \in I_m, k(\alpha\circ \varphi) = k(\alpha)-1\end{array}\right.
    \end{equation*}
    %%
\end{thm}

Occasionally we will denote the Dold-Kan equivalence functors for a category $\mathcal{A}$ by $N_\mathcal{A}$ and $\Gamma_\mathcal{A}$ if multiple categories are involved, or if the category isn't clear from the context. We begin by reciting certain properties that the functors in the Dold-Kan equivalence satisfy \cite[Section 14.24]{StacksProject}.

\begin{lem}[label=lem:refl]
    The functor $N$ reflects isomorphisms, injections, and surjections.
\end{lem}

Recall since $N$ and $\Gamma$ are equivalences, in particular this means that they are full, faithful, and essentially surjective. They also satisfy a number of other properties.

%%
\begin{lem}[label=lem:doldKanProps]
    For any abelian category $\mathcal{A}$, the Dold-Kan functors satisfy the following properties:
    %%
    \begin{itemize}
        \item $N$ is exact
        \item $N$ sends simplicial homotopies to chain homotopies, and hence sends simplicially homotopic maps to chain homotopic maps
        \item $N$ reflects chain homotopies
        \item $\Gamma$ sends chain homotopies to simplicial homotopies
    \end{itemize}
\end{lem}

The majority of our results will involve the interaction of $N$ and $\Gamma$ with standard functors and natural transformations, which we collect the definitions of here for simplicity:

\begin{defn}[label=defn:Diag]{}
    Let $(-)\Sob:2\cat{Ab}\rightarrow 2\cat{Ab}$ denote the pseudomonad constructed in Section \ref{sec:simpMon} which sends an abelian category to its category of simplicial objects, a functor to its post composition, and natural transformation to its post composition through whiskering with simplicial objects.

    From Section \ref{sec:simpMon} we also have a natural transformation, $\Delta_{(-)}:(-)\Sob\circ (-)\Sob\Rightarrow (-)\Sob$, which gives the multiplication of the pseudomonad, and on objects sends a bisimplicial complex to its diagonal.


    Another important pseudonatural transformation is given by $\iota_{(-)}:1_{2\cat{Ab}}\rightarrow (-)\Sob$, which on an abelian category $\mathcal{A}$ has component $\iota_\mathcal{A}$ which sends objects to the constant simplicial object for them.
\end{defn}

The first result we give is used in Section \ref{sec:ptwiseNat} during the proof that $\lhd$ gives a well-defined composition on $\cat{AbCat}_{\cat{Ch}}$.

\begin{lem}[label=lem:gammaDeg]
    For any abelian category $\mathcal{A}$, $\Gamma_\mathcal{A}\circ \deg_0^\mathcal{A} = \iota_\mathcal{A}$.
\end{lem}
\begin{proof}
    Let $A \in \mathcal{A}$. Then $\Gamma_\mathcal{A}(\deg_0^\mathcal{A}(A))([n]) = A$ for all $n$, since $\deg_0^\mathcal{A}(A)$ contains $A$ concentrated in degree $0$ and their is a unique $\alpha:[n]\rightarrow [0]$ for each $n$. Next, for each $\alpha:[m]\rightarrow [n]$, $\Gamma(\deg_0^\mathcal{A}(A))(\alpha) = 1_A$ from the piecewise definition of $\Gamma$ on arrows.

    Next, let $f:A\rightarrow B$ be a map in $\mathcal{A}$. Then $\deg_0^\mathcal{A}(f)$ is the map concentrated in degree $0$. Then $\Gamma(\deg_0^\mathcal{A}(f))([n]) = f$ for each $n$. It follows that $\Gamma_\mathcal{A}\circ \deg_0^\mathcal{A} = \iota_\mathcal{A}$, as claimed.
\end{proof}


\end{subappendices}


\chapter{Side Work}

\section{Pseudo-monad Attempts}

\subsection{Pseudomonad (attempt)}

Let $2\cat{Ab}$ denote the (large) 2-category of abelian categories, arbitrary functors between them, and natural transformations. We can consider $2\cat{Ab}$ as an object in the $\cat{Gray}$-category $\cat{Bicat}$ of bicategories, pseudofunctors, pseudonatural transformations, and modifications. Then $\cat{Ch}(-)$ is a pseudomonad on this 2-category. Explicitly, $\cat{Ch}(-)$ is a pseudofunctor in $\cat{Bicat}(2\cat{Ab},2\cat{Ab})$ defined as follows:
%%
\begin{enumerate}
    \item On 0-cells, $\cat{Ch}(-)$ sends an abelian category $\mathcal{A}$ to the category $\cat{Ch}(\mathcal{A})$ of chain complexes in $\mathcal{A}$, concentrated in non-negative degree.
    \item Given abelian categories $\mathcal{A},\mathcal{B}$, we have a functor $\cat{Ch}_{\mathcal{A},\mathcal{B}}:\mathcal{B}^\mathcal{A}\rightarrow \cat{Ch}(\mathcal{B})^{\cat{Ch}(\mathcal{A})}$ given as follows:
    %% 
    \begin{enumerate}
        \item On 0-cells (1-cells of the underlying bicategory) $\cat{Ch}_{\mathcal{A},\mathcal{B}}$ sends a functor $F:\mathcal{A}\rightarrow \mathcal{B}$ to its prolongation $\cat{Ch}(F):\cat{Ch}(\mathcal{A})\rightarrow \cat{Ch}(\mathcal{B})$. The prolongation is defined in terms of the Dold-Kan equivalence \ref{thm:DoldKanEquiv} as follows
        %%
        \begin{equation*}
            \cat{Ch}(F):\cat{Ch}(\mathcal{A})\xrightarrow{\Gamma}\mathcal{A}^{\Delta^{op}}\xrightarrow{F_*}\mathcal{B}^{\Delta^{op}}\xrightarrow{N}\cat{Ch}(\mathcal{A})
        \end{equation*}
        %%
        \item On 1-cells (2-cells of the underlying bicategory), $\cat{Ch}_{\mathcal{A},\mathcal{B}}$ sends a natural transformation $\gamma:F\Rightarrow G$ to a natural transformation $\cat{Ch}(\gamma)$ such that for $A_\bullet \in \cat{Ch}(\mathcal{A})_0$,
        %%
        \begin{equation*}
            \cat{Ch}(\gamma)_{A_\bullet}:\cat{Ch}(F)(A_\bullet)\rightarrow \cat{Ch}(G)(A_\bullet) := N(\gamma_{\Gamma(A_\bullet))}
        \end{equation*}
        %%
    \end{enumerate}
    %%
    \item We define $m(F,G):= NG_*\eta_{F_*\Gamma}: \cat{Ch}(G)\circ \cat{Ch}(F)\Rightarrow \cat{Ch}(G\circ F)$
    \begin{equation}\label{eq:compiso}\begin{tikzcd}
    	&&& {\cat{Ch}(\mathcal{B})} \\
    	{\cat{Ch}(\mathcal{A})} & {\mathcal{A}^{\Delta^{op}}} & {\mathcal{B}^{\Delta^{op}}} && {\mathcal{B}^{\Delta^{op}}} & {\mathcal{C}^{\Delta^{op}}} & {\cat{Ch}(\mathcal{C})}
    	\arrow["{\Gamma}", from=2-1, to=2-2]
    	\arrow["{F_*}", from=2-2, to=2-3]
    	\arrow["{N}", from=2-3, to=1-4]
    	\arrow["{\Gamma}", from=1-4, to=2-5]
    	\arrow["{G_*}", from=2-5, to=2-6]
    	\arrow["{N}", from=2-6, to=2-7]
    	\arrow[""{name=0, anchor=center, inner sep=0}, curve={height=30pt}, Rightarrow, no head, from=2-3, to=2-5]
    	\arrow["{\eta_\mathcal{B}}", shorten <=7pt, shorten >=7pt, Rightarrow, from=1-4, to=0]
    \end{tikzcd}
    \end{equation}
    \item For each $\mathcal{A} \in 2\cat{Ab}$ an invertible 2-cell $i := \varepsilon :1_{\cat{Ch}(\mathcal{A})}\Rightarrow \cat{Ch}(1_{\mathcal{A}})$
    \begin{equation}\label{eq:idiso}\begin{tikzcd}
    	{\cat{Ch}(\mathcal{A})} && {\cat{Ch}(\mathcal{A})} \\
    	& {\mathcal{A}^{\Delta^{op}}}
    	\arrow[""{name=0, anchor=center, inner sep=0}, "{1_{\cat{Ch}(\mathcal{A})}}", from=1-1, to=1-3]
    	\arrow["{\Gamma}"', from=1-1, to=2-2]
    	\arrow["{N}"', from=2-2, to=1-3]
    	\arrow["\varepsilon", shorten <=3pt, Rightarrow, from=0, to=2-2]
    \end{tikzcd}
    \end{equation}
\end{enumerate}
together with the following monad data
\begin{enumerate}
    \item A 2-cell (i.e. pseudonatural transformation) $\eta:1_{2\cat{Ab}}\Rightarrow\cat{Ch}$ given by the following data:
    \begin{enumerate}
        \item For each abelian cat $\mathcal{A}$, a functor $\eta_\mathcal{A} := \deg_0^\mathcal{A}:\mathcal{A}\rightarrow \cat{Ch}(\mathcal{A})$ sending an object to the chain complex
        %%
        \begin{equation*}
            \deg_0^\mathcal{A}(A)_n := \left\{\begin{array}{cc} A & n=0 \\ 0 & n \neq 0 \end{array}\right.
        \end{equation*}
        %%
        and a map to its action on degree zero.
        \item For each functor $F:\mathcal{A}\rightarrow \mathcal{B}$ a natural transformation $\eta_F=\deg_0^F:\deg_\mathcal{B}\circ F\Rightarrow \cat{Ch}(F)\circ \deg_\mathcal{A}$
        \[\begin{tikzcd}
        	{\mathcal{A}} & {\mathcal{B}} \\
        	{\cat{Ch}(\mathcal{A})} & {\cat{Ch}(\mathcal{B})}
        	\arrow["F", from=1-1, to=1-2]
        	\arrow["{\deg_0^\mathcal{B}}", from=1-2, to=2-2]
        	\arrow["{\deg_0^\mathcal{A}}"', from=1-1, to=2-1]
        	\arrow["{\cat{Ch}(F)}"', from=2-1, to=2-2]
        	\arrow["{\deg_0^F}"{description}, Rightarrow, from=1-2, to=2-1]
        \end{tikzcd}\]
        with components given by identities since $\cat{Ch}(F)(\deg_0^{\mathcal{A}}(A)) = \deg_0^\mathcal{A}F(A)$.
    \end{enumerate}
    \item A 2-cell $m:\cat{Ch}\circ\cat{Ch}\Rightarrow \cat{Ch}$ given by the following data:
    \begin{enumerate}
        \item For every abelian category $\mathcal{A}$ a functor $m_\mathcal{A}:=\text{Tot}_\mathcal{A}:\cat{Ch}\cat{Ch}(\mathcal{A})\rightarrow \cat{Ch}(\mathcal{A})$ given by the totalization. Explicitly, for $A_{\bullet,\bullet}\in\cat{Ch}\cat{Ch}(\mathcal{A})$, we set
        %%
        \begin{equation*}
            m_\mathcal{A}(A_{\bullet,\bullet})_n := \bigoplus_{i+j=n}A_{i,j}
        \end{equation*}
        %%
        with differential given by the components:
        %%
        \begin{equation*}
            (d_n)_{r,s} := (-1)^sd^h_{r+1,s}+(-1)^{s+1}d^v_{r,s+1}
        \end{equation*}
        For a map $F:A_{\bullet,\bullet}\rightarrow B_{\bullet,\bullet}$, we set 
        %%
        \begin{equation*}
            m_\mathcal{A}(F):\text{Tot}_\mathcal{A}(A_{\bullet,\bullet})\rightarrow \text{Tot}_\mathcal{A}(B_{\bullet,\bullet})
        \end{equation*}
        %%
        with $n$th component given by $\bigoplus_{i+j=n}F_{i,j}$
        \item For each functor $F:\mathcal{A}\rightarrow \mathcal{B}$ between abelian categories, a natural transformation $m_F:m_\mathcal{B}\circ \cat{Ch}^2(F)\Rightarrow \cat{Ch}(F)\circ m_\mathcal{A}$
        \[\begin{tikzcd}
        	{\cat{Ch}^2(\mathcal{A})} & {\cat{Ch}^2(\mathcal{B})} \\
        	{\cat{Ch}(\mathcal{A})} & {\cat{Ch}(\mathcal{B})}
        	\arrow["{\cat{Ch}^2(F)}", from=1-1, to=1-2]
        	\arrow["{m_\mathcal{B}}", from=1-2, to=2-2]
        	\arrow["{m_\mathcal{A}}"', from=1-1, to=2-1]
        	\arrow["{\cat{Ch}(F)}"', from=2-1, to=2-2]
        \end{tikzcd}\]
        given by \textbf{TBD}
    \end{enumerate}
    \item An invertible 3-cell (i.e. modification) $\mu:m\circ \cat{Ch}m\Rrightarrow m \circ m_{\cat{Ch}}$ given by the following data:
    \begin{enumerate}
        \item For each abelian category $\mathcal{A}$, a natural transformation $\mu_\mathcal{A}:m_\mathcal{A}\circ \cat{Ch}m_\mathcal{A}\Rightarrow m_\mathcal{A}\circ m_{\cat{Ch}(\mathcal{A})}$
    \end{enumerate}
    \item An invertible 3-cell $\lambda:m\circ \text{deg}_0^{\cat{Ch}}\Rrightarrow 1_{\cat{Ch}}$ given by:
    \begin{enumerate}
        \item For each abelian category $\mathcal{A}$, a natural transformation $\lambda_\mathcal{A}:m_\mathcal{A}\circ \text{deg}_0^{\cat{Ch}(\mathcal{A})}\Rightarrow 1_{\cat{Ch}(\mathcal{A})}$ with components given by identities.
    \end{enumerate}
    \item And an invertible 3-cell $\rho:1_{\cat{Ch}}\Rrightarrow m\circ \cat{Ch}\text{deg}_0$ given by:
    \begin{enumerate}
        \item For each abelian category $\mathcal{A}$, a natural transformation $\rho_\mathcal{A}:1_{\cat{Ch}(\mathcal{A})}\Rightarrow m_\mathcal{A}\circ \cat{Ch}(\text{deg}_0^{\mathcal{A}})$. First, we observe that the following diagram commutes up to a unique natural isomorphism specified by the universal property of the kernel
        \[\begin{tikzcd}
        	{\mathcal{A}\Sob} & {\cat{Ch}(\mathcal{A})\Sob} \\
        	{\cat{Ch}(\mathcal{A})} & {\cat{Ch}^2(\mathcal{A})}
        	\arrow["{(\deg_0^\mathcal{A})\Sob}", from=1-1, to=1-2]
        	\arrow["{N_\mathcal{A}}"', from=1-1, to=2-1]
        	\arrow["{N_{\cat{Ch}(\mathcal{A})}}", from=1-2, to=2-2]
        	\arrow["{\deg_0^{\cat{Ch}(\mathcal{A})}}"', from=2-1, to=2-2]
        	\arrow["\simeq"{description}, Rightarrow, from=1-2, to=2-1]
        \end{tikzcd}\]
        Additionally, from the previous 3-cell we have that $m_\mathcal{A}\circ  \deg_0^\mathcal{A}(N\Gamma(A_\bullet)) = N\Gamma(A_\bullet)$,  so the components of $\rho$ are given by the $\eta:1_{\cat{Ch}(\mathcal{A})}\Rightarrow N_\mathcal{A}\Gamma_\mathcal{A}$ from the Dold-Kan equivalence composed with the above natural isomorphism. \textbf{Need to make more explicit and show coherence diagram}
    \end{enumerate}
\end{enumerate}

It remains to show that this data satisfies the necessary coherence diagrams. We shall show these in a sequences of lemmas.






\subsection{Quotient Monad}

Although we have yet to show that $\cat{Ch}(-)$ defines a pseudomonad on $2\cat{Ab}$, we claim that it does define a monad on the 1-category $\cat{AbCat}$ consisting of abelian categories and natural isomorphism classes of functors. Since horizontal composition of natural transformations is functorial, the partition given by natural isomorphism classes of functors is associated with a congruence relation, and hence $\cat{AbCat}$ is a well-defined 1-category. We will denote the isomorphism class of a functor $F$ by $[F]$ throughout.

We show in a sequence of lemmas that $\cat{Ch}(-)$ is a well-defined monad on $\cat{AbCat}$, define on objects as before and defined on natural isomorphism classes of functors by $\cat{Ch}([F]) := [\cat{Ch}(F)]$. In order to show that this is well-defined we first demonstrate $\cat{Ch}$ is strictly functorial on isomorphism classes:

\begin{lem}[label=lem:strictCh]
    Let $F:\mathcal{A}\rightarrow \mathcal{B}$ and $G:\mathcal{B}\rightarrow \mathcal{C}$ be functors. Then $[\cat{Ch}(G)\circ \cat{Ch}(F)] = [\cat{Ch}(G\circ F)]$.
\end{lem}
\begin{proof}
    Using $NG_*\eta_{F_*\Gamma}$, as in Equation \eqref{eq:compiso}, we have that $\cat{Ch}(G)\circ \cat{Ch}(F)$ and $\cat{Ch}(G\circ F)$ are naturally isomorphic.
\end{proof}

With this functoriality result we can show that $\cat{Ch}$ is a well-defined functor on the quotient category.

\begin{lem}[label=lem:functCh]
    $\cat{Ch}$ defines a functor on $\cat{AbCat}$.
\end{lem}
\begin{proof}
    It remains to show that $\cat{Ch}$ is well-defined on arrows, and sends identities to identities, since Lemma \ref{lem:strictCh} provides functoriality. Let $\alpha:F\Rightarrow G:\mathcal{A}\rightarrow \mathcal{B}$ be a natural isomorphism. Then $\cat{Ch}(\alpha) = N\alpha_{\Gamma}:\cat{Ch}(F)\Rightarrow \cat{Ch}(G)$ is a natural transformation, and further, for any $A_\bullet \in \cat{Ch}(\mathcal{A})_0$,
    %%
    \begin{equation*}
        N(\alpha_{\Gamma(A_\bullet)}) \circ N(\alpha^{-1}_{\Gamma(A_\bullet)}) = N(\alpha_{\Gamma(A_\bullet)})\circ \alpha^{-1}_{\Gamma(A_\bullet)} = N(1_{G(\Gamma(A_\bullet))}) = 1_{\cat{Ch}(G)(A_\bullet)} 
    \end{equation*}
    %%
    using functoriality of $N$. The other composition is identical, so $\cat{Ch}(\alpha)$ is a natural isomorphism, implying $[\cat{Ch}(F)] = [\cat{Ch}(G)]$. 

    Finally, consider an identity functor $1_\mathcal{A}$. Then using the invertible 2-cell $\varepsilon$ in Equation \eqref{eq:idiso} we witness that $[1_{\cat{Ch}(\mathcal{A})}] = [\cat{Ch}(1_{\mathcal{A}})]$, so $\cat{Ch}(-)$ is a well-defined functor on $\cat{AbCat}$.
\end{proof}


It remains to show that the functor $\cat{Ch}(-)$ has the structure of a monad. We take the unit and multiplication 2-cells to be defined as in our description of the possible 2-monad $\cat{Ch}(-)$ on $2\cat{Ab}$. As $\deg_0$ is already natural, we need only show that $m$ is natural in $\cat{AbCat}$, and that the appropriate monad laws hold.


\begin{lem}[label=lem:mNat]
    Viewed as a map in $\cat{AbCat}$, $m:\cat{Ch}^2\Rightarrow \cat{Ch}$ is a natural transformation.
\end{lem}
\begin{proof}
    Explicitly, for each abelian category $\mathcal{A}$, $m_\mathcal{A} = [\text{Tot}_\mathcal{A}]$ is an isomorphism class of functors. Showing naturality is then equivalent to showing the following equation for any $F:\mathcal{A}\rightarrow\mathcal{B}$
    %%
    \begin{equation*}
        [\text{Tot}_\mathcal{B}\circ \cat{Ch}^2(F)] = [\cat{Ch}(F)\circ\text{Tot}_\mathcal{A}]
    \end{equation*}
    %%
    Recall $[\cat{Ch}(F)] = [N_\mathcal{B}\circ F_*\circ \Gamma_\mathcal{A}]$ and $[\cat{Ch}^2(F)] = [N_{\cat{Ch}(\mathcal{B})}\circ (N_\mathcal{B})_*\circ (F_*)_*\circ (\Gamma_\mathcal{A})_*\circ \Gamma_{\cat{Ch}(\mathcal{A})}]$. \textbf{TBD}
\end{proof}





\chapter{Project Notes and Future Directions}


\section{Current Questions}


\begin{itemize}
    \item Look up mapping cone and its relation to homotopy limits in an abelian category
    \item Look up homotopy limits
\end{itemize}


\section{Meeting Notes}


\subsection{September 27 Notes}


\begin{itemize}
    \item Began looking through BJORT\cite{BJORT} section 2, and in particular the notion of a cross-effect 
    \item Went over preliminary definitions, such as that of an abelian category
    \item Analyzed the inductive definition of the cross-effect functor, and determined how it is explicitly constructed.
\end{itemize}


\subsection{October 4 Notes}


\begin{itemize}
    \item Went through the proofs of Lemma 2.4 and Proposition 2.5 ourselves and argued for the naturality of the counit.
\end{itemize}



\subsection{October 18 Notes}


\begin{itemize}
    \item In the paper we begin with a functor $F:\mathcal{B}\rightarrow \mathcal{A}$, and produce a functor $D_1F:\mathcal{B}\rightarrow \cat{Ch}\mathcal{A}$, but this results in issues of composition and functoriality if we have another functor $G:\mathcal{C}\rightarrow \mathcal{B}$. Although we can consider $D_1(F\circ G)$, $D_1(F)\circ D_1(G)$ is not well typed as $D_1(F):\mathcal{B}\rightarrow \cat{Ch}\mathcal{A}$ and $D_1(G):\mathcal{C}\rightarrow \cat{Ch}\mathcal{B}$.
    \item Question:
    \begin{quotation}
        \noindent To what degree is $\cat{Ch}:\cat{AbCat}\rightarrow \cat{AbCat}$ a monad on $\cat{AbCat}$? (In fact it is a psuedo-monad, and we must be careful on how maps on 1-cells and 2-cells is defined)
    \end{quotation}
    \item We are not going to work with $\cat{AbCat}$ directly, but rather a quotient of $\cat{AbCat}$. 
    \item In this context we can ask if chain homotopy equivalences are pointwise, or can be promoted to being natural?
    \item 
\end{itemize}


\section{To-do List}

This section lists tasks which are yet to be completed from previous meetings


\textbf{Completed Sec 2:}
\begin{itemize}
    \item[1.] The cross effects operation in Definition 2.1 is explicitly defined to be functorial in the X variables.  Is it also functorial in F?  That is, is $cr_n$ a functor from the category $Fun(B, A) $ to $Fun(B^n, A)$ (these categories have functors as objects and natural transformations as morphisms).  See remarks before Lemma 2.4.  What else needs to be verified?
    \item[2.] Verify that the counit in Remark 2.8 of \cite{BJORT} is natural in $X$. Is it also a natural transformation $\text{cr}_n\Rightarrow \text{id}$?
    \item[3.] Is the contracting chain homotopy in Lemma 2.9 of \cite{BJORT} natural in $A$?
\end{itemize}


\textbf{Completed* Sec 3:}
\begin{itemize}
    \item[1.] In Observation 3.1, we claim that Ch is a pseudomonad \cite{BJORT}.  Is it?  This should be viewed with skepticism.
    \item[2.] In Observation 3.1 \cite{BJORT}, we claim that there is a quotient monad $\cat{Ch}$ acting on the category of abelian categories and isomorphism classes of functors.  Show that there is such a monad.
    \item[3.] While you are at it, please make sure you understand the phrase ``here we are not interested in the 2-dimensional aspects." What are these two dimensional aspects, and what is the consequence of ignoring them?
    \item[4.] In definition 3.2 \cite{BJORT}, we use natural isomorphism classes.  What happens if you use pointwise defined isomorphism classes?
    \item[5.] At the top of page 388 (following the proof of Lemma 3.4) \cite{BJORT}, we establish an equivalence relation on $\cat{AbCat}_\cat{Ch}$.  Do both pointwise defined chain homotopy equivalences and natural chain homotopy equivalences result in an equivalence relation on this category?
    \item[6.] Is it possible to alter Definition 3.5 \cite{BJORT} to use natural chain homotopy equivalence classes instead of pointwise chain homotopy equivalence classes?
\end{itemize}


\textbf{Not completed Sec 4:}
\begin{itemize}
    \item Go through section 4 and try to go through with the example of the identity in mind.
\end{itemize}


\textbf{Separate To-Do:}
\begin{itemize}
    \item Make a list/section of lemmas for $\Gamma$ and $N$ in the Dold-Kan equivalence
\end{itemize}






%%%%%%%%%%%%%%%%%%%%%% - Appendices
% \begin{appendices}

\bibliographystyle{amsalpha}
\bibliography{Refs.bib}
\nocite{*}

% \end{appendices}


\end{document}


%%%%%% END %%%%%%%%%%%%%
