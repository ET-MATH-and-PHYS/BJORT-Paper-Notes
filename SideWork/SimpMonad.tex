
In this section we attempt to construct a (pseudo)monad on $2\cat{Ab}$ corresponding to simplicial objects. The goal is that this (pseudo)monad is easier to construct than the chain complex pseudomonad, and that via conjugation by the Dold-Kan equivalence, we can obtain the chain complex pseudomonad, at least up to a suitable equivalence.

We define $(-)\Sob:2\cat{Ab}\rightarrow2\cat{Ab}$ as a pseudofunctor as follows:
\begin{enumerate}
    \item On 0-cells, $(-)\Sob$ sends an abelian category $\mathcal{A}$ to its category of simplicial objects $\mathcal{A}\Sob$
    \item Given abelian categories $\mathcal{A},\mathcal{B}$, we have a functor $(-)\Sob:[\mathcal{A},\mathcal{B}]\rightarrow [\mathcal{A}\Sob,\mathcal{B}\Sob]$ given as follows:
    \begin{enumerate}
        \item A functor $F:\mathcal{A}\rightarrow \mathcal{B}$ is sent to its push-forward $F_* :\mathcal{A}\Sob\rightarrow \mathcal{B}\Sob$ defined by post-composition
        \item A natural transformation $\gamma:F\Rightarrow G:\mathcal{A}\rightarrow \mathcal{B}$ is sent to a natural transformation $\gamma\Sob:F_*\Rightarrow G_*$ such that for $X \in \mathcal{A}\Sob_0$,
        %%
        \begin{equation*}
            \gamma\Sob_X:F\circ X\Rightarrow G\circ X := \gamma_X
        \end{equation*}
        %%
    \end{enumerate}
    \item We observe $m(F,G) := 1_{(G\circ F)_*}:G_*\circ F_*\Rightarrow (G\circ F)_*$ is our comparison 2-cell
    \item For each abelian category $\mathcal{A}$, an invertible 2-cell $i:= 1_{1_{\mathcal{A}\Sob}}:1_{\mathcal{A}\Sob}\Rightarrow (1_\mathcal{A})_*$ which is an identity.
\end{enumerate}
The psuedofunctor comes with the following monad data:
\begin{enumerate}
    \item A pseudonatural transformation $\eta:1_{2\cat{Ab}}\Rightarrow (-)\Sob$ given by the following data:
    \begin{enumerate}
        \item For each abelian category $\mathcal{A}$, a functor $\eta_\mathcal{A}:\mathcal{A}\Rightarrow \mathcal{A}\Sob$ given by the diagonal functor, sending an object $A$ to the constant functor for $A$ with identities on arrows.
        \item For each functor $F:\mathcal{A}\rightarrow \mathcal{B}$ a natural transformation $\eta_F:\eta_\mathcal{B}\circ F\Rightarrow F_*\circ \eta_\mathcal{A}$
        \[\begin{tikzcd}
        	{\mathcal{A}} & {\mathcal{B}} \\
        	{\mathcal{A}\Sob} & {\mathcal{B}\Sob}
        	\arrow["F", from=1-1, to=1-2]
        	\arrow["{\eta_\mathcal{B}}", from=1-2, to=2-2]
        	\arrow["{\eta_\mathcal{A}}"', from=1-1, to=2-1]
        	\arrow["{F_*}"', from=2-1, to=2-2]
        	\arrow["{\eta_F}"{description}, Rightarrow, from=1-2, to=2-1]
        \end{tikzcd}\]
        which is the identity, since the square commutes
    \end{enumerate}
    \item A pseudonatural transformation $m:(-)\Sob\circ (-)\Sob\Rightarrow(-)\Sob$ given by the following data:
    \begin{enumerate}
        \item For every abelian category $\mathcal{A}$, a functor $m_\mathcal{A}:(\mathcal{A}\Sob)\Sob\rightarrow \mathcal{A}\Sob$. For $A \in (\mathcal{A}\Sob)\Sob$
        %%
        \begin{equation*}
            m_\mathcal{A}(A)([n]) := A([n])([n])
        \end{equation*}
        %%
        and for $\alpha:[n]\rightarrow [m]$ we set
        %%
        \begin{equation*}
            m_\mathcal{A}(A)(\alpha):A([m])([m])\rightarrow A([n])([n]) := A([n])(\alpha) \circ A(\alpha)_{[m]} = A(\alpha)_{[n]}\circ A([m])(\alpha)
        \end{equation*}
        %%
        by naturality of $A(\alpha)$. Given a map of simplicial objects $\beta:A\Rightarrow B$ in $(\mathcal{A}\Sob)\Sob$, we set
        %%
        \begin{equation*}
            m_\mathcal{A}(\beta):m_\mathcal{A}(A)\rightarrow m_\mathcal{A}(B)
        \end{equation*}
        %%
        with $[n]$th component given by
        %%
        \begin{equation*}
            m_\mathcal{A}(\beta)_{[n]}:A([n])([n])\rightarrow B([n])([n]) := (\beta_{[n]})_{[n]}
        \end{equation*}
        \item For each functor $F:\mathcal{A}\rightarrow \mathcal{B}$ between abelian categories, a natural transformation $m_F:m_\mathcal{B}\circ (F_*)_*\Rightarrow F_*\circ m_\mathcal{A}$
        \[\begin{tikzcd}
        	{(\mathcal{A}\Sob)\Sob} & {(\mathcal{B}\Sob)\Sob} \\
        	{\mathcal{A}\Sob} & {\mathcal{B}\Sob}
        	\arrow["{(F_*)_*}", from=1-1, to=1-2]
        	\arrow["{m_\mathcal{B}}", from=1-2, to=2-2]
        	\arrow["{m_\mathcal{A}}"', from=1-1, to=2-1]
        	\arrow["{F_*}"', from=2-1, to=2-2]
        	\arrow["{m_F}"{description}, Rightarrow, from=1-2, to=2-1]
        \end{tikzcd}\]
        which is the identity since the square commutes. Indeed, for each $A \in (\mathcal{A}\Sob)\Sob$, and each $[n] \in\cat{Ob}(\Delta)$
        %%
        \begin{equation*}
            m_\mathcal{B}\circ (F_*)_*(A)([n]) = m_\mathcal{B}(F_*\circ A)([n]) = F(A([n])([n])) = F_*\circ m_\mathcal{A}(A)([n])
        \end{equation*}
        %%
        while for $\alpha:[m]\rightarrow [n]$
        %%
        \begin{align*}
            m_\mathcal{B}\circ (F_*)_*(A)(\alpha) &= m_\mathcal{B}(F_*\circ A)(\alpha) \\
            &= F(A([n])(\alpha)\circ A(\alpha)_{[m]}) \\
            &= F_*\circ m_\mathcal{A}(A)(\alpha)
        \end{align*}
        %%
        Further, for $\alpha:A\rightarrow A'$ in $(\mathcal{A}\Sob)\Sob$, and $[n] \in \cat{Ob}(\Delta)$,
        \begin{align*}
            m_\mathcal{B}\circ (F_*)_*(\alpha)_{[n]} &= m_\mathcal{B}(F_*\alpha)_{[n]} \\
            &= F(\alpha_{[n]})_{[n]}) \\
            &= F(m_\mathcal{A}(\alpha)_{[n]}) \\
            &= F_*\circ m_\mathcal{A}(\alpha)_{[n]}
        \end{align*}
        Thus the functors along each edge are equal, so the comparison cell is the identity.
    \end{enumerate}
    \item An invertible modification $\mu:m\circ (-)\Sob m\Rrightarrow m\circ m_{(-)\Sob}$ given by the following data:
    \begin{enumerate}
        \item For each abelian category $\mathcal{A}$, a natural transformation $\mu_\mathcal{A}:m_\mathcal{A}\circ (-)\Sob m_\mathcal{A}\Rightarrow m_\mathcal{A}\circ m_{\mathcal{A}\Sob}$ which has identity components since for a simplicial object $A \in ((\mathcal{A}\Sob)\Sob)\Sob$
        %%
        \begin{align*}
            m_\mathcal{A}((m_{\mathcal{A}})_*A)([n]) &= (m_\mathcal{A}\circ A)([n])([n]) \\
            &= m_\mathcal{A}(A([n]))([n]) \\
            &= A([n])([n])([n]) \\
            &= m_{\mathcal{A}\Sob}(A)([n])([n]) \\
            &= (m_\mathcal{A}\circ m_{\mathcal{A}\Sob}(A))([n]) 
        \end{align*}
        %%
        and for $\alpha:[m]\rightarrow [n]$,
        %%
        \begin{align*}
            m_\mathcal{A}((m_{\mathcal{A}})_*A)(\alpha) &= (m_\mathcal{A}\circ A)([n])(\alpha)\circ (m_\mathcal{A}\circ A)(\alpha)_{[m]} \\
            &= m_\mathcal{A}(A([n]))(\alpha)\circ m_\mathcal{A}(A(\alpha))_{[m]} \\
            &= A([n])([n])(\alpha)\circ A([n])(\alpha)_{[m]}\circ (A(\alpha)_{[m]})_{[m]} \\
            &= A([n])([n])(\alpha)\circ (A([n])(\alpha)\circ A(\alpha)_{[m]})_{[m]} \\
            &= m_{\mathcal{A}\Sob}(A)([n])(\alpha)\circ m_{\mathcal{A}\Sob}(A)(\alpha)_{[m]} \\
            &= (m_\mathcal{A}(m_{\mathcal{A}\Sob}(A)))(\alpha)
        \end{align*}
        %%
    \end{enumerate}
    \item An invertible modification $\lambda:m\circ \eta_{(-)\Sob} \Rrightarrow 1_{(-)\Sob}$ given by the following data:
    \begin{enumerate}
        \item For each abelian category $\mathcal{A}$, a natural transformation $\lambda_\mathcal{A}:m_\mathcal{A}\circ \eta_{\mathcal{A}\Sob}\Rightarrow 1_{\mathcal{A}\Sob}$ which is given by identities since for a simplicial object $A$
        %%
        \begin{equation*}
            m_\mathcal{A}(\eta_{\mathcal{A}\Sob}(A))([n]) = \eta_{\mathcal{A}\Sob}(A)([n])([n]) = A([n])
        \end{equation*}
        %%
        and for $\alpha:[m]\rightarrow [n]$,
        %%
        \begin{equation*}
            m_\mathcal{A}(\eta_{\mathcal{A}\Sob}(A))(\alpha) = \eta_{\mathcal{A}\Sob}(A)([n])(\alpha)\circ \eta_{\mathcal{A}\Sob}(A)(\alpha)_{[m]} = A(\alpha)\circ (1_A)_{[m]}=A(\alpha)
        \end{equation*}
        %%
    \end{enumerate}
    \item An invertible modification $\rho:m\circ(-)\Sob\eta\Rrightarrow 1_{(-)\Sob}$ given by the following data:
    \begin{enumerate}
        \item For each abelian category $\mathcal{A}$, a natural transformation $\rho_\mathcal{A}:m_\mathcal{A}\circ (-)\Sob\eta_{\mathcal{A}}\Rightarrow 1_{\mathcal{A}\Sob}$ which is also given by identities since for a simplicial object $A$
        %%
        \begin{equation*}
            m_\mathcal{A}((-)\Sob\eta_{\mathcal{A}}(A))([n]) = (\eta_\mathcal{A}\circ A)([n])([n]) = \eta_\mathcal{A}(A([n]))([n]) = A([n])
        \end{equation*}
        %%
        and for $\alpha:[m]\rightarrow [n]$,
        %%
        \begin{equation*}
            m_\mathcal{A}((-)\Sob\eta_{\mathcal{A}}(A))(\alpha) = (\eta_{\mathcal{A}}\circ A)([n])(\alpha)\circ (\eta_{\mathcal{A}}\circ A)(\alpha)_{[m]} = 1_{A([n])}\circ \eta_\mathcal{A}(A(\alpha))_{[m]} = A(\alpha)
        \end{equation*}
        %%
    \end{enumerate}
\end{enumerate}

Since all the higher comparison cells are identities, it follows that all coherence diagrams commute automatically, and in particular, the simplicial objects functor is a strict 2-monad on the (large) 2-category of abelian categories, $2\cat{Ab}$.

\subsubsection{Simplicial Homotopies}\label{subsec:simpHomotop}

Homotopies in categories $\mathcal{C}\Sob$ will be important in our analysis with the Dold-Kan Equivalence. This requires the consideration how to form products with simplicial sets in $\mathcal{C}\Sob$, which we can obtain from \cite[Defn 14.13.1]{StacksProject}.

\begin{defn}
    Let $\mathcal{C}$ be a category with finite coproducts and let $X \in \mathcal{C}\Sob$. If $U \in \cat{Set}\Sob$ is a finite, non-empty, simplicial set, we define the product $X \times U$ to be the simplicial object with $n$th component
    %%
    \begin{equation*}
        (X\times U)_n := \coprod_{u \in U_n}X_n
    \end{equation*}
    %%
    such that for any map $\varphi:[m]\rightarrow [n]$, $(X\times U)(\varphi):\coprod_{u \in U_n}X_n\rightarrow \coprod_{u' \in U_m}X_m$ is defined by
    %%
    \begin{equation*}
        (X\times U)(\varphi)\circ \iota_u = \iota_{U(\varphi)(u)}\circ X(\varphi)
    \end{equation*}
    %%
    Given maps $f:X\Rightarrow Y$ and $g:U\Rightarrow V$ of simplicial objects and simplicial sets, respectively, we obtain a map of simplicial objects $f\times g:X\times U\rightarrow Y\times V$ given on components by 
    %%
    \begin{equation*}
        (f\times g)_n : \coprod_{u \in U_n}X_n\rightarrow \coprod_{v \in V_n}Y_n,\;\; (f\times g)_n\circ \iota_u = \iota_{g_n(u)}\circ f_n
    \end{equation*}
    %%
\end{defn}

We can now define simplicial homotopies. Let $\Delta^n := Hom_{\Delta}(-,[n])$ be the standard $n$-simplex as a simplicial set. Recall that $\Delta^0$ is a singleton in each component, while 
%%
\begin{equation*}
    (\Delta^1)_n = \{\alpha_0^n,...,\alpha_{n+1}^n\},\;\;\alpha_i^n(j) = \left\{\begin{array}{cc} 0 & j < i \\ 1 & j \geq i \end{array}\right.
\end{equation*}
%%
By Yoneda we can identify these maps with natural isomorphisms, so in particular we have $\alpha_0^0:\Delta^0\Rightarrow \Delta^1$ and $\alpha_1^0:\Delta^0\Rightarrow \Delta^1$ corresponding to sending $0$ to $1$ and sending $0$ to $0$, respectively (note the flip). We will write $e_0 := \alpha_1^0$ and $e_1 := \alpha_0^0$. Noting that for any simplicial object $U \in \mathcal{C}\Sob$ $U\times \Delta^0 \cong U$, we obtain $e_0,e_1:U\Rightarrow U\times \Delta^1$. This is sufficient to define simplicial homotopies \cite[Defn 14.26.1]{StacksProject}.

\begin{defn}{}
    Let $X, Y \in \mathcal{C}\Sob$ be simplicial objects in a category with finite coproducts, and let $f,g:X\Rightarrow Y$ be simplicial maps. Then a \textbf{simplicial homotopy} between $f$ and $g$ is a simplicial map $h:X\times \Delta^1\Rightarrow Y$ making the following diagram commute
    %%
    \[\begin{tikzcd}
    	X \\
    	{X\times \Delta^1} & Y \\
    	X
    	\arrow["h"{description}, from=2-1, to=2-2]
    	\arrow["{e_0}"', from=1-1, to=2-1]
    	\arrow["f", from=1-1, to=2-2]
    	\arrow["{e_1}", from=3-1, to=2-1]
    	\arrow["g"', from=3-1, to=2-2]
    \end{tikzcd}\]
    %%
    When $\mathcal{C}$ is an abelian category this defines an additive equivalence relation on the simplicial maps $X\Rightarrow Y$ \cite{weibel_1994}. Otherwise, we say $f$ and $g$ are simplicially homotopic if there is a sequence $f = f_0,f_1,...,f_n = g$ of maps such that there is a simplicial homotopy from $f_i$ to $f_{i+1}$ or from $f_{i+1}$ to $f_i$ for each $i < n$.
\end{defn}

We can extend this definition to functors valued in simplicial objects.

\begin{defn}{}
    Let $F,G:\mathcal{B}\rightarrow \mathcal{A}\Sob$ be functors valued in simplicial objects. We say $F$ and $G$ are \textbf{pointwise homotopy equivalent} if for each $B \in \mathcal{B}$, we have a simplicial homotopy equivalence $(f_B:F(B)\rightarrow G(B), g_B:G(B)\rightarrow F(B), h_B:F(B)\times \Delta^1\rightarrow F(B), h'_B:G(B)\times \Delta^1\rightarrow G(B))$. We say $F$ and $G$ are \textbf{naturally homotopy equivalent} if we have natural transformations $(f:F\Rightarrow G,g:G\rightarrow F,h:F\times \Delta^1\Rightarrow F,h':G\times \Delta^1\Rightarrow G)$ which comprise homotopy equivalences at each $B \in \mathcal{B}$.
\end{defn}

We now show how these equivalences behave under composition.

\begin{lem}[label=lem:postcompNat]
    Let $F,G: \mathcal{C}\rightarrow \mathcal{B}\Sob$ be (naturally) simplicially homotopy equivalent. If $H:\mathcal{B}\rightarrow \mathcal{A}$ is any other functor into an abelian category, then $H_*\circ F$ and $H_*\circ G$ are (naturally) simplicially homotopy equivalent.
\end{lem}
\begin{proof}
    Let $(f,g,h,h')$ be a (natural) simplicial homotopy equivalence between $F$ and $G$. We define maps $(H_*f,H_*g,H_*h,H_*h')$. This definition evidently preserves naturality, so it is sufficient to show that it gives a simplicial homotopy. However this follows by functoriality of $H$.
\end{proof}

\begin{lem}[label=lem:precompNat]
    Let $F,G: \mathcal{C}\rightarrow \mathcal{B}\Sob$ be (naturally) simplicially homotopy equivalent. If $H:\mathcal{D}\rightarrow \mathcal{C}$ is any other functor into an abelian category, then $F\circ H$ and $G\circ H$ are (naturally) simplicially homotopy equivalent.
\end{lem}
\begin{proof}
    Is immediate from restriction of maps.
\end{proof}

\begin{lem}[label=lem:diagHo]
    Let $F,G: \mathcal{C}\rightarrow (\mathcal{B}\Sob)\Sob$ be (naturally) simplicially homotopy equivalent. Then $\Delta_\mathcal{B}\circ F$ and $\Delta_\mathcal{B}\circ G$ are (naturally) simplicially homotopy equivalent.
\end{lem}
\begin{proof}
    Let $(f,g,h,h')$ be the maps witnessing the (natural) simplicial homotopy equivalence between $F$ and $G$. Consider the collection of maps obtained by whiskering $(\Delta_\mathcal{B}f,\Delta_\mathcal{B}g,\Delta_\mathcal{B}h,\Delta_\mathcal{B}h')$. Again naturality is automatically preserved, so it is sufficient to show that this defines a pointwise simplicial homotopy between the functors. But this is simply restriction to the diagonal of the bicomplex, so the commutivity of the defining commutative triangles is preserved.
\end{proof}


\begin{lem}[label=lem:Precomp]
    Let $F,G: \mathcal{C}\rightarrow \mathcal{B}\Sob$ be naturally simplicially homotopy equivalent. Then $\Delta^{op}(F),\Delta^{op}(G): \mathcal{C}\Sob\rightarrow (\mathcal{B}\Sob)\Sob$ are naturally simplicially homotopy equivalent.
\end{lem}
\begin{proof}
    Let $(f,g,h,h')$ be a (natural) simplicial homotopy equivalence between $F$ and $G$. If the equivalence is natural, we can define $(\Delta^{op}(f),\Delta^{op}(g),\Delta^{op}(h),\Delta^{op}(h'))$ which is again a (natural) simplicial homotopy equivalence since $\Delta^{op}$ is a strict 2-functor, and so preserves composition.
\end{proof}


\begin{lem}[label=lem:PrecompDiag]
    Let $F,G: \mathcal{C}\rightarrow \mathcal{B}\Sob$ be naturally simplicially homotopy equivalent. Then $\Delta_\mathcal{B}F_*,\Delta_\mathcal{B}G_*: \mathcal{C}\Sob\rightarrow (\mathcal{B}\Sob)\Sob\rightarrow \mathcal{B}\Sob$ are naturally simplicially homotopy equivalent.
\end{lem}
\begin{proof}
    Since the homotopy equivalences are natural this follows from Lemmas~\ref{lem:Precomp} and~\ref{lem:diagHo}.
\end{proof}



\subsubsection{Chain Homotopies}\label{subsec:chainHomotop}


In this section we expand on the behaviour of natural chain homotopies and the interaction between natural chain homotopies and direct sums. We begin by proving equivalent formulations of natural chain homotopies.

\begin{lem}[label=lem:natHomotopIsChainFunctHomotop]
    Chain homotopies correspond to natural chain homotopies of functors under the isomorphism $\text{Fun}^{\cat{Ch}}:\cat{Ch}(\text{Fun}(\mathcal{B},\mathcal{A}))\to \text{Fun}(\mathcal{B},\cat{Ch}(\mathcal{B}))$.
\end{lem}
\begin{proof}
    First, let $\alpha,\beta:F_\bullet\to G_\bullet$ be a map of chain complexes of functors in $\cat{Ch}(\text{Fun}(\mathcal{B},\mathcal{A}))$. Then a chain homotopy from $\alpha$ to $\beta$ is, for each $n \in \Z$, a natural transformation $s_n:F_n\Rightarrow G_{n+1}$ such that 
    %%
    \begin{equation*}
        \partial_{n+1}^G\circ s_n + s_{n-1}\circ \partial_n^G = \alpha_n-\beta_n
    \end{equation*}
    %%
    On the other hand, under $\text{Fun}^\cat{Ch}$ $\alpha$ and $\beta$ correspond to natural transformations between functors valued in chain complexes, $F,G$. By definition, a natural chain homotopy is then a family of natural transformations $s_n: (-)_n\circ F\to (-)_{n+1}\circ G$, where $(-)_n:\cat{Ch}(\mathcal{A})\to \mathcal{A}$. But this is precisely the same data as the chain homotopy in $\cat{Ch}(\text{Fun}(\mathcal{B},\mathcal{A}))$.
\end{proof}

Next, we also show an equivalent form of chain homotopies.

\begin{lem}[label=lem:cylHomotop]
    Chain homotopies between maps $f,g:A_\bullet\to B_\bullet$ in $\cat{Ch}(\mathcal{A})$ are equivalent to chain maps $H:\text{cyl}(-1_{A_\bullet})\to B_\bullet$ such that the triangle
    %%
    \[\begin{tikzcd}
        {\text{cyl}(-1_{A_\bullet})} \\
        {A_\bullet\oplus A_\bullet} & {B_\bullet}
        \arrow["{f+g}"', from=2-1, to=2-2]
        \arrow["H", from=1-1, to=2-2]
        \arrow["{q_1+q_2}", from=2-1, to=1-1]
    \end{tikzcd}\]
    %%
    commutes where $q_1$ is the inclusion in the top of the cylinder and $q_2$ is the inclusion in the bottom cylinder.
\end{lem}
\begin{proof}
    We begin with a triangle as in the statement of the Lemma where $\text{cyl}(1_{A_\bullet})$ is the chain complex with $n$th degree term given by $A_{n-1}\oplus A_n\oplus A_n$ and chain map given by 
    %%
    \begin{equation*}
        A_n\oplus A_{n+1}\oplus A_{n+1}\xrightarrow{\begin{pmatrix} \partial_n^A & 0 & 0 \\ (-1)^n1_{A_n} & \partial_{n+1}^A & 0 \\ (-1)^{n+1}1_{A_n} & 0 & \partial_{n+1}^A \end{pmatrix}}A_{n-1}\oplus A_n\oplus A_n
    \end{equation*}
    %%
    Additionally, $q_1$ is given by 
    %%
    \begin{equation*}
        A_n\xrightarrow{\begin{pmatrix} 0 \\ 1_{A_n} \\ 0 \end{pmatrix}} A_{n-1}\oplus A_n\oplus A_n
    \end{equation*}
    %%
    and $q_2$ is given by 
    %%
    \begin{equation*}
        A_n\xrightarrow{\begin{pmatrix} 0 \\ 0 \\ 1_{A_n} \end{pmatrix}} A_{n-1}\oplus A_n\oplus A_n
    \end{equation*}
    %%
    Then a map $H:\text{cyl}(1_{A_\bullet})\to B_\bullet$ making the triangle commute is on each degree of the form
    %%
    \begin{equation*}
        A_{n-1}\oplus A_n\oplus A_n\xrightarrow{\begin{pmatrix} (-1)^{n-1}s_{n-1} & f_n & g_n \end{pmatrix}} B_n
    \end{equation*}
    %%
    where the chain map condition reduces to 
    %%
    \begin{equation*}
        \partial_n^B\circ s_{n-1}+s_{n-2}\circ \partial_{n-1}^A = f_{n-1}-g_{n-1}
    \end{equation*}
    %%
    which is exactly the condition for a chain homotopy.
\end{proof}

% \begin{lem}[label=lem:totAdd]
%     Let $F:\cat{Ch}(\mathcal{A})\to \cat{Ch}(\mathcal{B})$ be an additive functor. By Lemma~\ref{lem:funcActChain} we have an additive functor $\cat{Ch}(F):\cat{Ch}^2(\mathcal{A})\to \cat{Ch}^2(\mathcal{B})$ given by $F$ acting component-wise. Then
%     %%
%     \begin{equation*}
%         F\circ \text{Tot}_\mathcal{A} \cong \text{Tot}_\mathcal{B}\circ \cat{Ch}(F)
%     \end{equation*}
%     %%
% \end{lem}
% \begin{proof}
%     Let $A_{\bullet,\bullet} \in \cat{Ch}^2(\mathcal{A})$
% \end{proof}

% \begin{cor}[label=cor:cylinderPres]
%     If $F:\cat{Ch}(\mathcal{A})\to \cat{Ch}(\mathcal{B})$ is additive and $f:A_\bullet \to B_\bullet$ is a map of chain complexes, then $F(\text{cyl}(f)) \cong \text{cyl}(F(f))$.
% \end{cor}
% \begin{proof}
%     By Lemma~\ref{lem:funcActChain} we have an additive functor $\cat{Ch}(F):\cat{Ch}^2(\mathcal{A})\to \cat{Ch}^2(\mathcal{B})$ given by $F$ acting component-wise. 
% \end{proof}

% Using these equivalent characterizations of chain homotopy we prove that exact functors of chain complexes preserve chain homotopies.

% \begin{lem}[label=lem:exactChFunc]
%     Let $F:\cat{Ch}(\mathcal{A})\to \cat{Ch}(\mathcal{B})$ be an exact functor. Then $F$ preserves chain homotopies.
% \end{lem}
% \begin{proof}
%     \textbf{TBD}
% \end{proof}

We now show how additive functors acting componentwise preserve chain homotopies.

\begin{lem}[label=lem:addFuncPres]
    Let $F:\mathcal{A}\to \mathcal{B}$ be an additive functor. Then $\cat{Ch}(F):\cat{Ch}(\mathcal{A})\to \cat{Ch}(\mathcal{B})$ preserves chain homotopies.
\end{lem}
\begin{proof}
    Let $f,g:A_\bullet\to B_\bullet$ be chain maps with a homotopy $s_n:A_n\to B_{n+1}$ for $n \in \Z$ from $f$ to $g$. Since $F$ is additive we have that
    %%
    \begin{equation*}
        F(\partial_{n+1}^B)\circ F(s_n)+F(s_{n-1})\circ F(\partial_n^A) = F(\partial_{n+1}^B\circ s_n+s_{n-1}\circ \partial_n^A) = F(f_n-g_n) = F(f_n)-F(g_n)
    \end{equation*}
    %%
    Since $\cat{Ch}(F)(f)_n := F(f_n)$ and $\cat{Ch}(F)(g)_n := F(g_n)$, it follows that $F(f)$ and $F(g)$ are homotopic by $F(s_n):F(A_n)\to F(B_{n+1})$.
\end{proof}

We also have an analogous result for the chain construction from comonads.

\begin{lem}[label=lem:ChConsPres]
    Let $(C,\epsilon,\delta)$ be a comonad on $\mathcal{A}$ which is also an additive functor. Then $C^{\cat{Ch}}:\mathcal{A}\to \cat{Ch}(\mathcal{A})$ is additive.
\end{lem}
\begin{proof}
    Let $f,g:A\to B$ in $\mathcal{A}$. Then since $C$ is additive, so is all of its powers, so \[C^{\cat{Ch}}(f+g)_n := C^n(f+g)=C^n(f)+C^n(g) = C^\cat{Ch}(f)_n+C^\cat{Ch}(g)_n\]
    Thus $C^{\cat{Ch}}$ is additive.
\end{proof}

By Lemma~\ref{lem:ChConsPtwise} and Lemma~\ref{ lem:addFuncPres} it follows that $\cat{Ch}(C)^\cat{Ch}$ preserves homotopies. Another important result for this construction is how it is affected by isomorphisms between comonads.

\begin{lem}[label=lem:ChComIso]
    Let $(C,\epsilon,\delta)$ be a comonad on $\mathcal{A}$ and let $(C',\epsilon',\delta')$ be a comonad on $\mathcal{A}'$. Suppose $\gamma:\mathcal{A}\to \mathcal{A}'$ is an additive isomorphism of categories such that $C' = \gamma\circ C\circ \gamma^{-1}$, $\epsilon' = \gamma\epsilon_{\gamma^{-1}}$, and $\delta' = \gamma\delta'_{\gamma^{-1}}$. Then there is an equality
    %%
    \begin{equation*}
        {C'}^{\cat{Ch}} = \cat{Ch}(\gamma)\circ C^{\cat{Ch}}\circ \gamma^{-1}
    \end{equation*}
\end{lem}
\begin{proof}
    This equality is an immediate consequence of the construction of $C^{\cat{Ch}}$ and the specified equalities for the comonad natural transformations.
\end{proof}