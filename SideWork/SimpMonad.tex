
In this section we attempt to construct a (pseudo)monad on $2\cat{Ab}$ corresponding to simplicial objects. The goal is that this (pseudo)monad is easier to construct than the chain complex pseudomonad, and that via conjugation by the Dold-Kan equivalence, we can obtain the chain complex pseudomonad, at least up to a suitable equivalence.

We define $(-)\Sob:2\cat{Ab}\rightarrow2\cat{Ab}$ as a pseudofunctor as follows:
\begin{enumerate}
    \item On 0-cells, $(-)\Sob$ sends an abelian category $\mathcal{A}$ to its category of simplicial objects $\mathcal{A}\Sob$
    \item Given abelian categories $\mathcal{A},\mathcal{B}$, we have a functor $(-)\Sob:[\mathcal{A},\mathcal{B}]\rightarrow [\mathcal{A}\Sob,\mathcal{B}\Sob]$ given as follows:
    \begin{enumerate}
        \item A functor $F:\mathcal{A}\rightarrow \mathcal{B}$ is sent to its push-forward $F_* :\mathcal{A}\Sob\rightarrow \mathcal{B}\Sob$ defined by post-composition
        \item A natural transformation $\gamma:F\Rightarrow G:\mathcal{A}\rightarrow \mathcal{B}$ is sent to a natural transformation $\gamma\Sob:F_*\Rightarrow G_*$ such that for $X \in \mathcal{A}\Sob_0$,
        %%
        \begin{equation*}
            \gamma\Sob_X:F\circ X\Rightarrow G\circ X := \gamma_X
        \end{equation*}
        %%
    \end{enumerate}
    \item We observe $m(F,G) := 1_{(G\circ F)_*}:G_*\circ F_*\Rightarrow (G\circ F)_*$ is our comparison 2-cell
    \item For each abelian category $\mathcal{A}$, an invertible 2-cell $i:= 1_{1_{\mathcal{A}\Sob}}:1_{\mathcal{A}\Sob}\Rightarrow (1_\mathcal{A})_*$ which is an identity.
\end{enumerate}
The psuedofunctor comes with the following monad data:
\begin{enumerate}
    \item A pseudonatural transformation $\eta:1_{2\cat{Ab}}\Rightarrow (-)\Sob$ given by the following data:
    \begin{enumerate}
        \item For each abelian category $\mathcal{A}$, a functor $\eta_\mathcal{A}:\mathcal{A}\Rightarrow \mathcal{A}\Sob$ given by the diagonal functor, sending an object $A$ to the constant functor for $A$ with identities on arrows.
        \item For each functor $F:\mathcal{A}\rightarrow \mathcal{B}$ a natural transformation $\eta_F:\eta_\mathcal{B}\circ F\Rightarrow F_*\circ \eta_\mathcal{A}$
        \[\begin{tikzcd}
        	{\mathcal{A}} & {\mathcal{B}} \\
        	{\mathcal{A}\Sob} & {\mathcal{B}\Sob}
        	\arrow["F", from=1-1, to=1-2]
        	\arrow["{\eta_\mathcal{B}}", from=1-2, to=2-2]
        	\arrow["{\eta_\mathcal{A}}"', from=1-1, to=2-1]
        	\arrow["{F_*}"', from=2-1, to=2-2]
        	\arrow["{\eta_F}"{description}, Rightarrow, from=1-2, to=2-1]
        \end{tikzcd}\]
        which is the identity, since the square commutes
    \end{enumerate}
    \item A pseudonatural transformation $m:(-)\Sob\circ (-)\Sob\Rightarrow(-)\Sob$ given by the following data:
    \begin{enumerate}
        \item For every abelian category $\mathcal{A}$, a functor $m_\mathcal{A}:(\mathcal{A}\Sob)\Sob\rightarrow \mathcal{A}\Sob$. For $A \in (\mathcal{A}\Sob)\Sob$
        %%
        \begin{equation*}
            m_\mathcal{A}(A)([n]) := A([n])([n])
        \end{equation*}
        %%
        and for $\alpha:[n]\rightarrow [m]$ we set
        %%
        \begin{equation*}
            m_\mathcal{A}(A)(\alpha):A([m])([m])\rightarrow A([n])([n]) := A([n])(\alpha) \circ A(\alpha)_{[m]} = A(\alpha)_{[n]}\circ A([m])(\alpha)
        \end{equation*}
        %%
        by naturality of $A(\alpha)$. Given a map of simplicial objects $\beta:A\Rightarrow B$ in $(\mathcal{A}\Sob)\Sob$, we set
        %%
        \begin{equation*}
            m_\mathcal{A}(\beta):m_\mathcal{A}(A)\rightarrow m_\mathcal{A}(B)
        \end{equation*}
        %%
        with $[n]$th component given by
        %%
        \begin{equation*}
            m_\mathcal{A}(\beta)_{[n]}:A([n])([n])\rightarrow B([n])([n]) := (\beta_{[n]})_{[n]}
        \end{equation*}
        \item For each functor $F:\mathcal{A}\rightarrow \mathcal{B}$ between abelian categories, a natural transformation $m_F:m_\mathcal{B}\circ (F_*)_*\Rightarrow F_*\circ m_\mathcal{A}$
        \[\begin{tikzcd}
        	{(\mathcal{A}\Sob)\Sob} & {(\mathcal{B}\Sob)\Sob} \\
        	{\mathcal{A}\Sob} & {\mathcal{B}\Sob}
        	\arrow["{(F_*)_*}", from=1-1, to=1-2]
        	\arrow["{m_\mathcal{B}}", from=1-2, to=2-2]
        	\arrow["{m_\mathcal{A}}"', from=1-1, to=2-1]
        	\arrow["{F_*}"', from=2-1, to=2-2]
        	\arrow["{m_F}"{description}, Rightarrow, from=1-2, to=2-1]
        \end{tikzcd}\]
        which is the identity since the square commutes. Indeed, for each $A \in (\mathcal{A}\Sob)\Sob$, and each $[n] \in\cat{Ob}(\Delta)$
        %%
        \begin{equation*}
            m_\mathcal{B}\circ (F_*)_*(A)([n]) = m_\mathcal{B}(F_*\circ A)([n]) = F(A([n])([n])) = F_*\circ m_\mathcal{A}(A)([n])
        \end{equation*}
        %%
        while for $\alpha:[m]\rightarrow [n]$
        %%
        \begin{align*}
            m_\mathcal{B}\circ (F_*)_*(A)(\alpha) &= m_\mathcal{B}(F_*\circ A)(\alpha) \\
            &= F(A([n])(\alpha)\circ A(\alpha)_{[m]}) \\
            &= F_*\circ m_\mathcal{A}(A)(\alpha)
        \end{align*}
        %%
        Further, for $\alpha:A\rightarrow A'$ in $(\mathcal{A}\Sob)\Sob$, and $[n] \in \cat{Ob}(\Delta)$,
        \begin{align*}
            m_\mathcal{B}\circ (F_*)_*(\alpha)_{[n]} &= m_\mathcal{B}(F_*\alpha)_{[n]} \\
            &= F(\alpha_{[n]})_{[n]}) \\
            &= F(m_\mathcal{A}(\alpha)_{[n]}) \\
            &= F_*\circ m_\mathcal{A}(\alpha)_{[n]}
        \end{align*}
        Thus the functors along each edge are equal, so the comparison cell is the identity.
    \end{enumerate}
    \item An invertible modification $\mu:m\circ (-)\Sob m\Rrightarrow m\circ m_{(-)\Sob}$ given by the following data:
    \begin{enumerate}
        \item For each abelian category $\mathcal{A}$, a natural transformation $\mu_\mathcal{A}:m_\mathcal{A}\circ (-)\Sob m_\mathcal{A}\Rightarrow m_\mathcal{A}\circ m_{\mathcal{A}\Sob}$ which has identity components since for a simplicial object $A \in ((\mathcal{A}\Sob)\Sob)\Sob$
        %%
        \begin{align*}
            m_\mathcal{A}((m_{\mathcal{A}})_*A)([n]) &= (m_\mathcal{A}\circ A)([n])([n]) \\
            &= m_\mathcal{A}(A([n]))([n]) \\
            &= A([n])([n])([n]) \\
            &= m_{\mathcal{A}\Sob}(A)([n])([n]) \\
            &= (m_\mathcal{A}\circ m_{\mathcal{A}\Sob}(A))([n]) 
        \end{align*}
        %%
        and for $\alpha:[m]\rightarrow [n]$,
        %%
        \begin{align*}
            m_\mathcal{A}((m_{\mathcal{A}})_*A)(\alpha) &= (m_\mathcal{A}\circ A)([n])(\alpha)\circ (m_\mathcal{A}\circ A)(\alpha)_{[m]} \\
            &= m_\mathcal{A}(A([n]))(\alpha)\circ m_\mathcal{A}(A(\alpha))_{[m]} \\
            &= A([n])([n])(\alpha)\circ A([n])(\alpha)_{[m]}\circ (A(\alpha)_{[m]})_{[m]} \\
            &= A([n])([n])(\alpha)\circ (A([n])(\alpha)\circ A(\alpha)_{[m]})_{[m]} \\
            &= m_{\mathcal{A}\Sob}(A)([n])(\alpha)\circ m_{\mathcal{A}\Sob}(A)(\alpha)_{[m]} \\
            &= (m_\mathcal{A}(m_{\mathcal{A}\Sob}(A)))(\alpha)
        \end{align*}
        %%
    \end{enumerate}
    \item An invertible modification $\lambda:m\circ \eta_{(-)\Sob} \Rrightarrow 1_{(-)\Sob}$ given by the following data:
    \begin{enumerate}
        \item For each abelian category $\mathcal{A}$, a natural transformation $\lambda_\mathcal{A}:m_\mathcal{A}\circ \eta_{\mathcal{A}\Sob}\Rightarrow 1_{\mathcal{A}\Sob}$ which is given by identities since for a simplicial object $A$
        %%
        \begin{equation*}
            m_\mathcal{A}(\eta_{\mathcal{A}\Sob}(A))([n]) = \eta_{\mathcal{A}\Sob}(A)([n])([n]) = A([n])
        \end{equation*}
        %%
        and for $\alpha:[m]\rightarrow [n]$,
        %%
        \begin{equation*}
            m_\mathcal{A}(\eta_{\mathcal{A}\Sob}(A))(\alpha) = \eta_{\mathcal{A}\Sob}(A)([n])(\alpha)\circ \eta_{\mathcal{A}\Sob}(A)(\alpha)_{[m]} = A(\alpha)\circ (1_A)_{[m]}=A(\alpha)
        \end{equation*}
        %%
    \end{enumerate}
    \item An invertible modification $\rho:m\circ(-)\Sob\eta\Rrightarrow 1_{(-)\Sob}$ given by the following data:
    \begin{enumerate}
        \item For each abelian category $\mathcal{A}$, a natural transformation $\rho_\mathcal{A}:m_\mathcal{A}\circ (-)\Sob\eta_{\mathcal{A}}\Rightarrow 1_{\mathcal{A}\Sob}$ which is also given by identities since for a simplicial object $A$
        %%
        \begin{equation*}
            m_\mathcal{A}((-)\Sob\eta_{\mathcal{A}}(A))([n]) = (\eta_\mathcal{A}\circ A)([n])([n]) = \eta_\mathcal{A}(A([n]))([n]) = A([n])
        \end{equation*}
        %%
        and for $\alpha:[m]\rightarrow [n]$,
        %%
        \begin{equation*}
            m_\mathcal{A}((-)\Sob\eta_{\mathcal{A}}(A))(\alpha) = (\eta_{\mathcal{A}}\circ A)([n])(\alpha)\circ (\eta_{\mathcal{A}}\circ A)(\alpha)_{[m]} = 1_{A([n])}\circ \eta_\mathcal{A}(A(\alpha))_{[m]} = A(\alpha)
        \end{equation*}
        %%
    \end{enumerate}
\end{enumerate}

Since all the higher comparison cells are identities, it follows that all coherence diagrams commute automatically, and in particular, the simplicial objects functor is a strict 2-monad on the (large) 2-category of abelian categories, $2\cat{Ab}$.

\subsubsection{Simplicial Homotopies}\label{subsec:simpHomotop}

Homotopies in categories $\mathcal{C}\Sob$ will be important in our analysis with the Dold-Kan Equivalence. This requires the consideration how to form products with simplicial sets in $\mathcal{C}\Sob$, which we can obtain from \cite[Defn 14.13.1]{StacksProject}.

\begin{defn}
    Let $\mathcal{C}$ be a category with finite coproducts and let $X \in \mathcal{C}\Sob$. If $U \in \cat{Set}\Sob$ is a finite, non-empty, simplicial set, we define the product $X \times U$ to be the simplicial object with $n$th component
    %%
    \begin{equation*}
        (X\times U)_n := \coprod_{u \in U_n}X_n
    \end{equation*}
    %%
    such that for any map $\varphi:[m]\rightarrow [n]$, $(X\times U)(\varphi):\coprod_{u \in U_n}X_n\rightarrow \coprod_{u' \in U_m}X_m$ is defined by
    %%
    \begin{equation*}
        (X\times U)(\varphi)\circ \iota_u = \iota_{U(\varphi)(u)}\circ X(\varphi)
    \end{equation*}
    %%
    Given maps $f:X\Rightarrow Y$ and $g:U\Rightarrow V$ of simplicial objects and simplicial sets, respectively, we obtain a map of simplicial objects $f\times g:X\times U\rightarrow Y\times V$ given on components by 
    %%
    \begin{equation*}
        (f\times g)_n : \coprod_{u \in U_n}X_n\rightarrow \coprod_{v \in V_n}Y_n,\;\; (f\times g)_n\circ \iota_u = \iota_{g_n(u)}\circ f_n
    \end{equation*}
    %%
\end{defn}

We can now define simplicial homotopies. Let $\Delta^n := Hom_{\Delta}(-,[n])$ be the standard $n$-simplex as a simplicial set. Recall that $\Delta^0$ is a singleton in each component, while 
%%
\begin{equation*}
    (\Delta^1)_n = \{\alpha_0^n,...,\alpha_{n+1}^n\},\;\;\alpha_i^n(j) = \left\{\begin{array}{cc} 0 & j < i \\ 1 & j \geq i \end{array}\right.
\end{equation*}
%%
By Yoneda we can identify these maps with natural isomorphisms, so in particular we have $\alpha_0^0:\Delta^0\Rightarrow \Delta^1$ and $\alpha_1^0:\Delta^0\Rightarrow \Delta^1$ corresponding to sending $0$ to $1$ and sending $0$ to $0$, respectively (note the flip). We will write $e_0 := \alpha_1^0$ and $e_1 := \alpha_0^0$. Noting that for any simplicial object $U \in \mathcal{C}\Sob$ $U\times \Delta^0 \cong U$, we obtain $e_0,e_1:U\Rightarrow U\times \Delta^1$. This is sufficient to define simplicial homotopies~\cite[Defn 14.26.1]{StacksProject}.

\begin{defn}{}
    Let $X, Y \in \mathcal{C}\Sob$ be simplicial objects in a category with finite coproducts, and let $f,g:X\Rightarrow Y$ be simplicial maps. Then a \textbf{simplicial homotopy} between $f$ and $g$ is a simplicial map $h:X\times \Delta^1\Rightarrow Y$ making the following diagram commute
    %%
    \[\begin{tikzcd}
    	X \\
    	{X\times \Delta^1} & Y \\
    	X
    	\arrow["h"{description}, from=2-1, to=2-2]
    	\arrow["{e_0}"', from=1-1, to=2-1]
    	\arrow["f", from=1-1, to=2-2]
    	\arrow["{e_1}", from=3-1, to=2-1]
    	\arrow["g"', from=3-1, to=2-2]
    \end{tikzcd}\]
    %%
    When $\mathcal{C}$ is an abelian category this defines an additive equivalence relation on the simplicial maps $X\Rightarrow Y$ \cite{weibel_1994}. Otherwise, we say $f$ and $g$ are simplicially homotopic if there is a sequence $f = f_0,f_1,...,f_n = g$ of maps such that there is a simplicial homotopy from $f_i$ to $f_{i+1}$ or from $f_{i+1}$ to $f_i$ for each $i < n$.
\end{defn}

We can extend this definition to functors valued in simplicial objects.

\begin{defn}{}
    Let $F,G:\mathcal{B}\rightarrow \mathcal{A}\Sob$ be functors valued in simplicial objects. We say $F$ and $G$ are \textbf{pointwise homotopy equivalent} if for each $B \in \mathcal{B}$, we have a simplicial homotopy equivalence $(f_B:F(B)\rightarrow G(B), g_B:G(B)\rightarrow F(B), h_B:F(B)\times \Delta^1\rightarrow F(B), h'_B:G(B)\times \Delta^1\rightarrow G(B))$. We say $F$ and $G$ are \textbf{naturally homotopy equivalent} if we have natural transformations $(f:F\Rightarrow G,g:G\rightarrow F,h:F\times \Delta^1\Rightarrow F,h':G\times \Delta^1\Rightarrow G)$ which comprise homotopy equivalences at each $B \in \mathcal{B}$.
\end{defn}

We also have a completely combinatorial description of simplicial homotopies which is equivalent when working with simplicial objects in finitely cocomplete categories~\cite{weibel_1994}. In particular, a simplicial homotopy between $f,g:X\to Y$ is a family of maps $h_i^n:X_n\to Y_{n+1}$ for $n \in \N$ and $0\leq i \leq n$, such that $Y_{d_0^{n+1}}h^n_0 = f_n$, $Y_{d_{n+1}^{n+1}}h_n^n=g_n$, and 
%%
\begin{equation*}
    Y(d_i^{n+1})h_j^n = \left\{\begin{array}{cc} h_{j-1}^{n-1}X(d_i^n) & i < j \\ Y(d_i^{n+1})h_{i-1}^n & i = j \neq 0 \\ h_j^{n-1}X(d_{i-1}^n) & i > j+1 \end{array}\right.
\end{equation*}
%%
\begin{equation*}
    Y(s_i^{n+1})h_j^n = \left\{\begin{array}{cc} h_{j+1}^{n+1}X(s_i^n) & i \leq j \\ h_j^{n+1}X(s_{i-1}^n) & i > j \end{array}\right.
\end{equation*}


First we show that these homotopies are preserved by biproducts in abelian categories.


\begin{lem}[label=lem:coprodPresSimpHomotop]
    Let $A,B,C,D \in \mathcal{C}\Sob$ for $\mathcal{C}$ an abelian category, and let $f,g:A\to B$ and $h:C\to D$ be simplicial maps. Then $f$ and $g$ are simplicially homotopic if and only if $f\oplus h$ is simplicially homotopic to $g\oplus h$.
\end{lem}
\begin{proof}
    First, suppose that $f$ and $g$ are simplicially homotopic by maps $h_i^n:A_n\to B_{n+1}$ for $n \in \N$ and $0 \leq i \leq n$. Then I claim $h_i^n\oplus 0:A_n\oplus C_n\to B_{n+1}\oplus D_{n+1}$ gives a homotopy from $f\oplus h$ to $g\oplus h$. Indeed since $\oplus$ is functorial the simplicial identities still hold in the first variable, and they also hold in the second variable vacuously since the composite with zero by any map is zero.
    
    \vspace{10pt}

    Conversely, suppose we have a homotopy 
    %%
    \begin{equation*}
        A_n\oplus C_n\xrightarrow{\begin{pmatrix} h_{n,i}^{1,1} & h_{n,i}^{1,2} \\ h_{n,i}^{2,1} & h_{n,i}^{2,2} \end{pmatrix}} B_{n+1}\oplus D_{n+1}
    \end{equation*}
    %%
    from $f\oplus h$ to $g \oplus h$. The homotopy conditions then become the equalities
    %%
    \begin{align*}
        \begin{pmatrix} B(d_i^{n+1})h_{n,j}^{1,1} & B(d_i^{n+1})h_{n,j}^{1,2} \\
        D(d_i^{n+1})h_{n,j}^{2,1} & D(d_i^{n+1})h_{n,j}^{2,2} \end{pmatrix} &= \begin{pmatrix} h_{n-1,j-1}^{1,1}A(d_i^{n}) & h_{n-1,j-1}^{1,2}C(d_i^{n}) \\
        h_{n-1,j-1}^{2,1}A(d_i^{n}) & h_{n-1,j-1}^{2,2}C(d_i^{n}) \end{pmatrix} \tag{$i < j$} \\
        %%
        \begin{pmatrix} B(d_i^{n+1})h_{n,j}^{1,1} & B(d_i^{n+1})h_{n,j}^{1,2} \\
        D(d_i^{n+1})h_{n,j}^{2,1} & D(d_i^{n+1})h_{n,j}^{2,2} \end{pmatrix} &= \begin{pmatrix} B(d_i^{n+1})h_{n,i-1}^{1,1} & B(d_i^{n+1})h_{n,i-1}^{1,2} \\
        D(d_i^{n+1})h_{n,i-1}^{2,1} & D(d_i^{n+1})h_{n,i-1}^{2,2} \end{pmatrix} \tag{$i = j \neq 0$} \\
        %%
        \begin{pmatrix} B(d_i^{n+1})h_{n,j}^{1,1} & B(d_i^{n+1})h_{n,j}^{1,2} \\
        D(d_i^{n+1})h_{n,j}^{2,1} & D(d_i^{n+1})h_{n,j}^{2,2} \end{pmatrix} &= \begin{pmatrix} h_{n-1,j}^{1,1}A(d_{i-1}^{n}) & h_{n-1,j}^{1,2}C(d_{i-1}^{n}) \\
        h_{n-1,j}^{2,1}A(d_{i-1}^{n}) & h_{n-1,j}^{2,2}C(d_{i-1}^{n}) \end{pmatrix} \tag{$i > j+1$} \\
        %%
        \begin{pmatrix} B(s_i^{n+1})h_{n,j}^{1,1} & B(s_i^{n+1})h_{n,j}^{1,2} \\
        D(s_i^{n+1})h_{n,j}^{2,1} & D(s_i^{n+1})h_{n,j}^{2,2} \end{pmatrix} &= \begin{pmatrix} h_{n+1,j+1}^{1,1}A(s_i^{n}) & h_{n+1,j+1}^{1,2}C(s_i^{n}) \\
        h_{n+1,j+1}^{2,1}A(s_i^{n}) & h_{n+1,j+1}^{2,2}C(s_i^{n}) \end{pmatrix} \tag{$i \leq j$} \\
        %%
        \begin{pmatrix} B(s_i^{n+1})h_{n,j}^{1,1} & B(s_i^{n+1})h_{n,j}^{1,2} \\
        D(s_i^{n+1})h_{n,j}^{2,1} & D(s_i^{n+1})h_{n,j}^{2,2} \end{pmatrix} &= \begin{pmatrix} h_{n+1,j}^{1,1}A(s_{i-1}^{n}) & h_{n+1,j}^{1,2}C(s_{i-1}^{n}) \\
        h_{n+1,j}^{2,1}A(s_{i-1}^{n}) & h_{n+1,j}^{2,2}C(s_{i-1}^{n}) \end{pmatrix} \tag{$i > j$} \\
    \end{align*}
    It follows that $h_{n,i}^{1,1}$ give a simplicial homotopy from $f$ to $g$, as desired.
\end{proof}

We now show how these equivalences behave under composition as well as some behaviour of the category of functors into simplicial objects.

\begin{lem}[label=lem:isoOfSimpFuncs]
    If $\mathcal{A},\mathcal{B}$ are categories, there exists an isomorphism of categories
    %%
    \begin{equation*}
        \text{Fun}(\mathcal{A},\mathcal{B}\Sob) \cong \text{Fun}(\mathcal{A},\mathcal{B})\Sob
    \end{equation*}
\end{lem}
\begin{proof}
    We define a natural isomorphism $\gamma:\text{Fun}(\mathcal{A},\mathcal{B}\Sob) \to \text{Fun}(\mathcal{A},\mathcal{B})\Sob$ on an object $F:\mathcal{A}\to \mathcal{B}\Sob$ by 
    %%
    \begin{equation*}
        \gamma(F)_n(A) := F(A)_n,
    \end{equation*}
    %%
    For each $n$ $\gamma(F)_n$ is a functor so for $\alpha:[m]\to [n]$ we need to show a natural transformation $\gamma(F)_\alpha:\gamma(F)_n\to \gamma(F)_m$ which we define to be
    %%
    \begin{equation*}
        (\gamma(F)_\alpha)_A := F(A)_\alpha 
    \end{equation*}
    %%
    This is functorial since each $F(A)$ is a functor. If $f:A\to B$, then since $F(f)$ is a natural transformation, we also have the commuting square
    \[\begin{tikzcd}
        {F(A)_n} & {F(B)_n} \\
        {F(A)_m} & {F(B)_m}
        \arrow["{F(A)_\alpha}"', from=1-1, to=2-1]
        \arrow["{F(f)_n}", from=1-1, to=1-2]
        \arrow["{F(f)_m}"', from=2-1, to=2-2]
        \arrow["{F(B)_\alpha}", from=1-2, to=2-2]
    \end{tikzcd}\]
    It follows that $\gamma(F)_\alpha$ is natural. Thus $\gamma$ is a well-defined functor. Similarly, $\gamma$ has an inverse $\tau:\text{Fun}(\mathcal{A},\mathcal{B})\Sob\to \text{Fun}(\mathcal{A},\mathcal{B}\Sob)$ given by $\tau(F)(A)_n := F_n(A)$.
\end{proof}

Note that under this isomorphism, natural simplicial homotopy is simplicial homotopy in $\text{Fun}(\mathcal{A},\mathcal{B})\Sob$. 

\begin{lem}[label=lem:simpFuncCat]
    We have a functor $\text{Fun}(-,-)\Sob:\cat{Cat}^{op}\times \cat{Cat}\to \cat{Cat}$, and for any $\mathcal{A},\mathcal{B},\mathcal{C} \in \cat{Cat}$ where $\mathcal{C}$ has finite coproducts, and any functor $F:\mathcal{A}\to \mathcal{B}$, $\text{Fun}(F,\mathcal{C})\Sob:\text{Fun}(\mathcal{B},\mathcal{C})\Sob\to \text{Fun}(\mathcal{A},\mathcal{C})\Sob$ preserves simplicial homotopies. Similarly, if $\mathcal{B}$ and $\mathcal{C}$ both have finite colimits, and $F:\mathcal{B}\to \mathcal{C}$, then $\text{Fun}(\mathcal{A},F)$ preserves simplicial homotopies.
\end{lem}
\begin{proof}
    The functor $\text{Fun}(-,-)\Sob$ is simply the composite of $\text{Fun}(-,-)$ with the 2-monad $\Sob$. Now let $\mathcal{A},\mathcal{B},\mathcal{C} \in \cat{Cat}$ and let $F:\mathcal{A}\to \mathcal{B}$ be a functor. Then suppose $G,H \in \text{Fun}(\mathcal{B},\mathcal{C})\Sob$ and suppose $g,h:G\to H$ are simplicially homotopic. Then there exists $f:G\times \Delta^1\to H$ such that $f\circ e_0 = g$ and $f\circ e_1 = h$. Then $\text{Fun}(F,\mathcal{C})\Sob$ is defined by sending $K \in \text{Fun}(\mathcal{B},\mathcal{C})\Sob$ $K_F$, where $(K_F)_n := K_n\circ F$. Observe that
    %%
    \begin{equation*}
        ((G\times \Delta_1)_F)_n = (\coprod_{u \in \Delta^1_n}G_n)\circ F = \coprod_{u \in \Delta^1_n}(G_n\circ F) = ((G\circ F)\times \Delta^1)_n
    \end{equation*}
    %%
    so $(G\times \Delta^1)_F = (G\circ F)\times \Delta^1$. Additionally, $((e_i)_F)_n= ((e_i)_n)_F$, which is exactly $(e_i)_n:(G\circ F)_n\to ((G\circ F)\times \Delta^1)_n$. Thus $\text{Fun}(F,\mathcal{C})\Sob$ preserves simplicial homotopies.

    \vspace{10pt}

    Conversely, if $\mathcal{B}$ and $\mathcal{C}$ are finitely cocomplete and $F:\mathcal{B}\to \mathcal{C}$, we can use the combinatorial description of homotopies. In this case, $\text{Fun}(\mathcal{A},F)$ is defined by ${_F}H_n := F\circ H_n$ on objects and ${_F}h_n := Fh_n$ on maps. Let $f,g:H\to K$ be simplicially homotopic maps in $\text{Fun}(\mathcal{A},\mathcal{B})\Sob$, by a simplicial homotopy $h_i^n:H_n\to K_{n+1}$. Then $Fh_i^n:F\circ H_n\to F\circ K_{n+1}$ defines a simplicial homotopy between ${_F}f$ and ${_F}g$, as desired.
\end{proof}

% Post-composition also preserves simplicial homotopies if we are post-composing by functors that preserve finite coproducts.

% \begin{lem}[label=lem:postcompNat]
%     If $\mathcal{A},\mathcal{B},\mathcal{C} \in \cat{Cat}$ where $\mathcal{B},\mathcal{C}$ have finite coproducts, and $F:\mathcal{B}\to \mathcal{C}$ is a functor that preserves finite coproducts, then $\text{Fun}(\mathcal{A},F)\Sob:\text{Fun}(\mathcal{A},\mathcal{B})\Sob\to \text{Fun}(\mathcal{A},\mathcal{C})\Sob$ preserves simplicial homotopies.
% \end{lem}
% \begin{proof}
%     Let $H,K \in \text{Fun}(\mathcal{A},\mathcal{B})\Sob$ and let $h,k:H\to K$ be simplicially homotopic by a map $s:H\times \Delta^1\to K$. Note that $\text{Fun}(\mathcal{A},F)$ is defined by ${_F}H_n := F\circ H_n$ on objects and ${_F}h_n := Fh_n$ on maps. By Corollary~\ref{cor:presFunccoLim}, since $F$ preserves colimits we have that ${_F}(H\times \Delta^1) \cong {_F}H\times \Delta^1$, and under this isomorphism $e_i:H\times \Delta^0\to H\times \Delta^1$ corresponds to $e_i:{_F}H\times \Delta^0\to {_F}H\times \Delta^1$. Thus $\text{Fun}(\mathcal{A},F)$ produces the desired commuting triangles to make ${_F}s$ a simplicial homotopy from ${_F}h$ to ${_F}k$. 
% \end{proof}



\begin{lem}[label=lem:diagHo]
    Let $F,G: \mathcal{C}\rightarrow (\mathcal{B}\Sob)\Sob$ and let $f,g:F\to G$ be naturally simplicially homotopic. Then $\Delta_\mathcal{B}f,\Delta_\mathcal{B}g:\Delta_\mathcal{B}\circ F\to\Delta_\mathcal{B}\circ G$ are naturally simplicially homotopic.
\end{lem}
\begin{proof}
    Let $h:F\times \Delta^1\to G$ be a simplicial homotopy between $f$ and $g$. Then 
    %%
    \begin{equation*}
        \Delta_\mathcal{B}(F\times \Delta^1)(C)_n = \coprod_{u \in \Delta^1_n}F(C)_{n,n} = (\Delta_\mathcal{B}F\times \Delta^1)(C)_n
    \end{equation*}
    %%
    In particular, $\Delta_\mathcal{B}h$ defines a simplicial homotopy from $\Delta_\mathcal{B}f$ to $\Delta_\mathcal{B}g$.
\end{proof}


\begin{lem}[label=lem:Precomp]
    Let $F,G: \mathcal{C}\rightarrow \mathcal{B}\Sob$ and let $f,g:F\to G$ be naturally simplicially homotopic. Then $\Delta^{op}(f),\Delta^{op}(g):\Delta^{op}(F)\to\Delta^{op}(G)$ are naturally simplicially homotopic.
\end{lem}
\begin{proof}
    Let $h:F\times \Delta^1\to G$ be a simplicial homotopy between $f$ and $g$. Then observe that 
    %%
    \begin{equation*}
        \Delta^{op}(F\times \Delta^1)(C)_n = F(C_n)\times \Delta^1 = \Delta^{op}(F)(C)_n\times \Delta^1
    \end{equation*}
    %%
    Then $\Delta^{op}(h):\Delta^{op}(F)\times \Delta^1\to \Delta^{op}(G)$, and since $\Delta^{op}$ is a strict 2-functor it preserves composition so $\Delta^{op}(h)$ is a homotopy from $\Delta^{op}(f)$ to $\Delta^{op}(g)$.
\end{proof}

Next, we construct simplicial objects out comonads on a category.

\begin{prop}[label=prop:comonadSimp]
    Let $(C,\epsilon,\delta)$ be a comonad on a category $\mathcal{C}$. Then there exists a functor
    %%
    \begin{equation*}
        C^{\bullet+1}:\mathcal{C}\to \mathcal{C}\Sob
    \end{equation*}
\end{prop}
\begin{proof}
    Let $A \in \mathcal{C}_0$. Then we define the simplicial object $C^{\bullet+1}(A)$ by 
    %%
    \begin{equation*}
        C^{\bullet+1}(A)_n := C^{n+1}(A)
    \end{equation*}
    %%
    while for the face operators $d_i^n:[n-1]\to [n]$ and degeneracy operators $s_i^n:[n+1]\to [n]$ we define 
    %%
    \begin{equation*}
        C^{\bullet+1}(A)_{d_i^n} := C^i\epsilon_{C^{n-i}(A)}:C^{n+1}(A)\to C^n(A),\;\; C^{\bullet+1}(A)_{s_i^n} := C^i\delta_{C^{n-i}(A)}:C^n(A)\to C^{n+1}(A)
    \end{equation*}
    %%
    It remains to show the simplicial identities for these maps:
    \begin{align*}
        C^i\epsilon_{C^{n-i}(A)}\circ C^j\epsilon_{C^{n+1-j}(A)} &= C^i(\epsilon_{C^{n-i}(A)}\circ C^{j-i}\epsilon_{C^{n+1-j}(A)}) \\
        &= C^i(C^{j-1-i}\epsilon_{C^{n+1-j}(A)}\circ \epsilon_{C^{n+1-i}(A)}) = C^{j-1}\epsilon_{C^{n-(j-1)}(A)}\circ C^{i}\epsilon_{C^{n+1-i}(A)} \tag{$i < j$} \\
        C^i\delta_{C^{n+1-i}(A)}\circ C^j\delta_{C^{n-j}(A)} &= C^j(C^{i-j}\delta_{C^{n+1-i}(A)}\circ \delta_{C^{n-j}(A)}) \\
        &= C^j(\delta_{C^{n+1-j}(A)}\circ C^{i-1-j}\delta_{C^{n+1-i}(A)}) = C^j\delta_{C^{n+1-j}(A)}\circ C^{i-1}\delta_{C^{n-(i-1)}(A)} \tag{$i > j$} \\
        %%
        C^i\epsilon_{C^{n+1-i}(A)}\circ C^j\delta_{C^{n-j}(A)} &= 1_{A_n} \tag{$i \in \{j,j+1\}$} \\
        C^i\epsilon_{C^{n+1-i}(A)}\circ C^j\delta_{C^{n-j}(A)} &= C^i(\epsilon_{C^{n+1-i}(A)}\circ C^{j-i}\delta_{C^{n-j}(A)}) \\
        &= C^i(C^{j-i-1}\delta_{C^{n-j}(A)}\circ \epsilon_{C^{n-i}(A)}) = C^{j-1}\delta_{C^{n-1-(j-1)}(A)}\circ C^i\epsilon_{C^{n-i}(A)} \tag{$i < j$} \\
        %%
        C^i\epsilon_{C^{n+1-i}(A)}\circ C^j\delta_{C^{n-j}(A)} &= C^j(C^{i-j}\epsilon_{C^{n+1-i}(A)}\circ \delta_{C^{n-j}(A)}) \\
        &= C^j(\delta_{C^{C^{n-1-j}(A)}}\circ C^{i-1-j}\epsilon_{C^{n+1-i}(A)}) = C^{j}\delta_{C^{n-1-j}(A)}\circ C^{i-1}\epsilon_{C^{n-(i-1)}(A)} \tag{$i > j$}
    \end{align*}
    where we have used the naturality of $\epsilon$ and $\delta$, as well as the monad identities. Finally, if $f:A\to A'$ in $\mathcal{C}$ we define $C^{\bullet+1}(f)_n := C^{n+1}(f)$. This gives a map of simplicial sets since the face and degeneracy operators are natural. Since $C$ is a functor this assignment is functorial.
\end{proof}


For any category we have the trivial comonad given by identities. All comonads then given simplicial objects which are augmented over the simplicial object for this comonad.

\begin{prop}[label=prop:simpMonMapToId]
    Let $(C,\epsilon,\delta)$ be a comonad on $\mathcal{C}$. Then there exists a natural transformation $C^{\bullet+1}\to 1_{\mathcal{C}}^{\bullet+1}$
\end{prop}
\begin{proof}
    We define $\rho:C^{\bullet+1}\to 1_{\mathcal{C}}^{\bullet+1}$ at $A \in \mathcal{C}_0$ and $n \in \N$ by
    %%
    \begin{equation*}
        C^{n+1}(A)\xrightarrow{\epsilon_A\circ \epsilon_{C(A)}\circ \cdots \circ \epsilon_{C^n(A)}}A
    \end{equation*}
    %%
    Then for the image of the face operators we obtain the commuting rectangle
    %%
    \[\begin{tikzcd}
        {C^{n+1}(A)} & {C^n(A)} & \cdots & {C(A)} & A \\
        {C^n(A)} & {C^{n-1}(A)} & \cdots & {C(A)} & A
        \arrow["{\epsilon_{C^n(A)}}", from=1-1, to=1-2]
        \arrow[from=1-2, to=1-3]
        \arrow[from=1-3, to=1-4]
        \arrow["{\epsilon_A}", from=1-4, to=1-5]
        \arrow[Rightarrow, no head, from=1-5, to=2-5]
        \arrow["{C^i\epsilon_{C^{n-i}(A)}}"', from=1-1, to=2-1]
        \arrow["{\epsilon_{C^{n-1}(A)}}"', from=2-1, to=2-2]
        \arrow[from=2-2, to=2-3]
        \arrow[from=2-3, to=2-4]
        \arrow[from=2-4, to=2-5]
        \arrow["{C^{i-1}\epsilon_{C^{n-i}(A)}}", from=1-2, to=2-2]
    \end{tikzcd}\]
    by applying naturality $i$ times. Similarly, applying naturality of $\epsilon$ $i$ times to 
    \[\begin{tikzcd}
        {C^{n+1}(A)} & {C^n(A)} & \cdots & {C(A)} & A \\
        {C^{n+2}(A)} & {C^{n+1}(A)} & \cdots & {C(A)} & A
        \arrow["{\epsilon_{C^n(A)}}", from=1-1, to=1-2]
        \arrow[from=1-2, to=1-3]
        \arrow[from=1-3, to=1-4]
        \arrow["{\epsilon_A}", from=1-4, to=1-5]
        \arrow[Rightarrow, no head, from=1-5, to=2-5]
        \arrow["{C^i\delta_{C^{n-i}(A)}}"', from=1-1, to=2-1]
        \arrow["{\epsilon_{C^{n+1}(A)}}"', from=2-1, to=2-2]
        \arrow[from=2-2, to=2-3]
        \arrow[from=2-3, to=2-4]
        \arrow[from=2-4, to=2-5]
        \arrow["{C^{i-1}\delta_{C^{n-i}(A)}}", from=1-2, to=2-2]
    \end{tikzcd}\]
    we obtain the diagram 
    \[\begin{tikzcd}
        {C^{n-i+1}(A)} & {C^{n-i}(A)} & \cdots & A \\
        {C^{n-i+2}(A)} & {C^{n-i+1}(A)} & \cdots & A
        \arrow["{\delta_{C^{n-i}(A)}}"', from=1-1, to=2-1]
        \arrow["{\epsilon_{C^{n-i+1}(A)}}"', from=2-1, to=2-2]
        \arrow["{\epsilon_{C^{n-i}(A)}}", from=1-1, to=1-2]
        \arrow[from=1-2, to=1-3]
        \arrow[from=2-2, to=2-3]
        \arrow[from=2-3, to=2-4]
        \arrow[from=1-3, to=1-4]
        \arrow[Rightarrow, no head, from=1-4, to=2-4]
        \arrow[Rightarrow, no head, from=1-1, to=2-2]
    \end{tikzcd}\]
    which commutes. Thus $\rho_A$ is a map of simplicial sets. For $f:A\to A'$, we have the commuting diagram
    \[\begin{tikzcd}
        {C^{n+1}(A)} & {C^n(A)} & \cdots & {C(A)} & A \\
        {C^{n+1}(A')} & {C^n(A')} & \cdots & {C(A')} & {A'}
        \arrow["{\epsilon_{C^n(A)}}", from=1-1, to=1-2]
        \arrow[from=1-2, to=1-3]
        \arrow[from=1-3, to=1-4]
        \arrow["{\epsilon_A}", from=1-4, to=1-5]
        \arrow["{\epsilon_{A'}}"', from=2-4, to=2-5]
        \arrow[from=2-3, to=2-4]
        \arrow[from=2-2, to=2-3]
        \arrow["{\epsilon_{C^n(A')}}"', from=2-1, to=2-2]
        \arrow["{C^{n+1}(f)}"', from=1-1, to=2-1]
        \arrow["{C^n(f)}", from=1-2, to=2-2]
        \arrow["{C(f)}", from=1-4, to=2-4]
        \arrow["f", from=1-5, to=2-5]
    \end{tikzcd}\]
    so $\rho$ is natural.
\end{proof}



\subsubsection{Chain Homotopies}\label{subsec:chainHomotop}


In this section we expand on the behaviour of natural chain homotopies and the interaction between natural chain homotopies and direct sums. We begin by proving equivalent formulations of natural chain homotopies.

\begin{lem}[label=lem:funcChain]
    For $\mathcal{A}$ an abelian category, we have an isomorphism of categories
    \begin{equation}\label{eq:ChainFunc}
        \cat{Ch}(\text{Fun}(\mathcal{B},\mathcal{A}))\cong \text{Fun}(\mathcal{B},\cat{Ch}(\mathcal{A}))
    \end{equation}
\end{lem}
\begin{proof}
    Define a functor $\gamma:\cat{Ch}(\text{Fun}(\mathcal{B},\mathcal{A}))\rightarrow \text{Fun}(\mathcal{B},\cat{Ch}(\mathcal{A}))$ given on a chain complex of functors $F_\bullet$ by
    %%
    \begin{equation*}
        \gamma(F_\bullet)(B)_n := F_n(B),\;\forall B \in \mathcal{B}
    \end{equation*}
    %%
    where the differentials are given by the natural transformation differentials in $F_\bullet$ evaluated at $B$. Given a map of chain complexes $\alpha_\bullet:F_\bullet\rightarrow G_\bullet$ we set
    %%
    \begin{equation*}
        (\gamma(\alpha_\bullet)_B)_n := (\alpha_n)_B
    \end{equation*}
    %%
    This defines a chain map $\gamma(F_\bullet)(B)\rightarrow \gamma(G_\bullet)(B)$ since $\alpha_\bullet$ is a chain map of natural transformations, so all squares with differentials commute. Further, $\gamma(\alpha_\bullet)$ is natural in $B$ since if $f:B\rightarrow B'$ is a map in $\mathcal{B}$, then in
    \[\begin{tikzcd}
    	& {F_{n+1}(B')} &&& {F_n(B')} \\
    	{F_{n+1}(B)} && {F_n(B)} \\
    	& {G_{n+1}(B')} &&& {G_n(B')} \\
    	{G_{n+1}(B)} && {G_n(B)}
    	\arrow["{\partial_{n+1}}"{pos=0.7}, from=2-1, to=2-3]
    	\arrow["{(\alpha_{n+1})_B}"', from=2-1, to=4-1]
    	\arrow["{\partial_{n+1}}"', from=4-1, to=4-3]
    	\arrow["{(\alpha_n)_B}"{pos=0.3}, from=2-3, to=4-3]
    	\arrow["{F_{n+1}(f)}", from=2-1, to=1-2]
    	\arrow["{F_n(f)}"', from=2-3, to=1-5]
    	\arrow["{\partial_{n+1}}", from=1-2, to=1-5]
    	\arrow["{(\alpha_n)_{B'}}", from=1-5, to=3-5]
    	\arrow["{G_n(f)}"', from=4-3, to=3-5]
    	\arrow["{G_{n+1}(f)}", from=4-1, to=3-2]
    	\arrow["{(\alpha_{n+1})_{B'}}"'{pos=0.6}, from=1-2, to=3-2]
    	\arrow["{\partial_{n+1}}"', from=3-2, to=3-5]
    \end{tikzcd}\]
    the front and back faces commute since $\alpha_\bullet$ is a chain map, the top and bottom faces commute by naturality of the boundary maps, and the side faces commute by naturality of the $\alpha_n$. Since this definition is in terms of the components of $\alpha_\bullet$ it is inherently functorial. 

    Next we must witness an inverse $\rho:\text{Fun}(\mathcal{B},\cat{Ch}(\mathcal{A}))\rightarrow \cat{Ch}(\text{Fun}(\mathcal{B},\mathcal{A}))$ functor. Given $F:\mathcal{B}\rightarrow \cat{Ch}(\mathcal{A})$ we set $\rho(F)$ to have $n$th component $(-)_n\circ F$ and differential $\partial_n$ given by components the $n$th differential of $F$ evaluated at $B  \in \mathcal{B}$. Naturality of the differential equates to the commutivity of 
    \[\begin{tikzcd}
    	{F(B)_n} & {F(B)_{n-1}} \\
    	{F(B')_n} & {F(B')_{n-1}}
    	\arrow["{F(f)_n}"', from=1-1, to=2-1]
    	\arrow["{\partial_n(B)}", from=1-1, to=1-2]
    	\arrow["{\partial_n(B')}"', from=2-1, to=2-2]
    	\arrow["{F(f)_{n-1}}", from=1-2, to=2-2]
    \end{tikzcd}\]
    for any $f:B\rightarrow B'$, which follows since $F(f)$ is a chain map. Next, if $\alpha:F\rightarrow G$ is a natural transformation between two such functors we set $\rho(\alpha)$ such that $\rho(\alpha)_n$ is the natural transformation defined by $(\rho(\alpha)_n)_B := (\alpha_B)_n$. Naturality and the chain condition follow by the commutivity of 
    \[\begin{tikzcd}
    	& {F(B')_{n+1}} &&& {F(B')_n} \\
    	{F(B)_{n+1}} && {F(B)_n} \\
    	& {G(B')_{n+1}} &&& {G(B')_n} \\
    	{G(B)_{n+1}} && {G(B)_n}
    	\arrow["{\partial_{n+1}(B)}"{pos=0.7}, from=2-1, to=2-3]
    	\arrow["{(\alpha_B)_{n+1}}"', from=2-1, to=4-1]
    	\arrow["{\partial_{n+1}(B)}"', from=4-1, to=4-3]
    	\arrow["{(\alpha_B)_n}"{pos=0.2}, from=2-3, to=4-3]
    	\arrow["{F(f)_{n+1}}", from=2-1, to=1-2]
    	\arrow["{F(f)_n}"', from=2-3, to=1-5]
    	\arrow["{\partial_{n+1}(B')}", from=1-2, to=1-5]
    	\arrow["{(\alpha_{B'})_n}", from=1-5, to=3-5]
    	\arrow["{G(f)_n}"', from=4-3, to=3-5]
    	\arrow["{G(f)_{n+1}}", from=4-1, to=3-2]
    	\arrow["{(\alpha_{B'})_{n+1}}"'{pos=0.6}, from=1-2, to=3-2]
    	\arrow["{\partial_{n+1}(B')}"', from=3-2, to=3-5]
    \end{tikzcd}\]
    where the bottom and top faces are the fact $G(f)$ and $F(f)$ are chain maps, the front and back faces are the fact $\alpha_B$ is a chain map, and finally the side faces are naturality of $\alpha$. Once again, since $\rho(\alpha)$ is defined in terms of the components of $\alpha$ the assignment is inherently functorial. Further, these operations are exactly inverse of each other as they correspond to swapping the element and natural number indices (in particular, on the other side of the Dold-Kan Equivalence this is simply the swap natural isomorphism on functors of two variables).
\end{proof}

Moving forward we write $\text{Fun}^\cat{Ch}$ for the isomorphism $\gamma$ in the proof. We also have another description of this category:

\begin{lem}[label=lem:adjChFuncCat]
    Let $\Z$ denote the category associated with the linear ordered set $(\Z,\geq)$. Let $\text{Fun}_{\cat{Ch}}(\mathcal{B}\times \Z,\mathcal{A})$ be the sub-category such that $F(-,n+2\leq n)$ is the zero map. Then under the adjunction $-\times \Z\dashv \text{Fun}(\Z,\mathcal{A})$ we have the isomorphism
    %%
    \begin{equation*}
        \text{Fun}_{\cat{Ch}}(\mathcal{B}\times \Z,\mathcal{A})\cong \text{Fun}(\mathcal{B},\text{Fun}_{\cat{Ch}}(\Z,\mathcal{A}))
    \end{equation*}
    %%
    and $\text{Fun}_{\cat{Ch}}(\Z,\mathcal{A})\cong \cat{Ch}(\mathcal{A})$.
\end{lem}
\begin{proof}
    It is sufficient to show that the isomorphism in the adjunction restricts to the isomorphism above. But this follows immediately by definition, so all that there is to show is $\text{Fun}_{\cat{Ch}}(\Z,\mathcal{A})\cong \cat{Ch}$. But this is also immediate by sending $A:\Z\to \mathcal{A}$ to $A_\bullet$, where $A_n = A(n)$ and $\partial_n^A = A(n\geq n-1)$, and vice-versa.
\end{proof}

\begin{lem}[label=lem:natHomotopIsChainFunctHomotop]
    Chain homotopies correspond to natural chain homotopies of functors under the isomorphism $\text{Fun}^{\cat{Ch}}:\cat{Ch}(\text{Fun}(\mathcal{B},\mathcal{A}))\to \text{Fun}(\mathcal{B},\cat{Ch}(\mathcal{B}))$.
\end{lem}
\begin{proof}
    First, let $\alpha,\beta:F_\bullet\to G_\bullet$ be a map of chain complexes of functors in $\cat{Ch}(\text{Fun}(\mathcal{B},\mathcal{A}))$. Then a chain homotopy from $\alpha$ to $\beta$ is, for each $n \in \Z$, a natural transformation $s_n:F_n\Rightarrow G_{n+1}$ such that 
    %%
    \begin{equation*}
        \partial_{n+1}^G\circ s_n + s_{n-1}\circ \partial_n^G = \alpha_n-\beta_n
    \end{equation*}
    %%
    On the other hand, under $\text{Fun}^\cat{Ch}$ $\alpha$ and $\beta$ correspond to natural transformations between functors valued in chain complexes, $F,G$. By definition, a natural chain homotopy is then a family of natural transformations $s_n: (-)_n\circ F\to (-)_{n+1}\circ G$, where $(-)_n:\cat{Ch}(\mathcal{A})\to \mathcal{A}$. But this is precisely the same data as the chain homotopy in $\cat{Ch}(\text{Fun}(\mathcal{B},\mathcal{A}))$.
\end{proof}

Next, we also show an equivalent form of chain homotopies.

\begin{lem}[label=lem:cylHomotop]
    Chain homotopies between maps $f,g:A_\bullet\to B_\bullet$ in $\cat{Ch}(\mathcal{A})$ are equivalent to chain maps $H:\text{cyl}(-1_{A_\bullet})\to B_\bullet$ such that the triangle
    %%
    \[\begin{tikzcd}
        {\text{cyl}(-1_{A_\bullet})} \\
        {A_\bullet\oplus A_\bullet} & {B_\bullet}
        \arrow["{f+g}"', from=2-1, to=2-2]
        \arrow["H", from=1-1, to=2-2]
        \arrow["{q_1+q_2}", from=2-1, to=1-1]
    \end{tikzcd}\]
    %%
    commutes where $q_1$ is the inclusion in the top of the cylinder and $q_2$ is the inclusion in the bottom cylinder.
\end{lem}
\begin{proof}
    We begin with a triangle as in the statement of the Lemma where $\text{cyl}(1_{A_\bullet})$ is the chain complex with $n$th degree term given by $A_{n-1}\oplus A_n\oplus A_n$ and chain map given by 
    %%
    \begin{equation*}
        A_n\oplus A_{n+1}\oplus A_{n+1}\xrightarrow{\begin{pmatrix} \partial_n^A & 0 & 0 \\ (-1)^n1_{A_n} & \partial_{n+1}^A & 0 \\ (-1)^{n+1}1_{A_n} & 0 & \partial_{n+1}^A \end{pmatrix}}A_{n-1}\oplus A_n\oplus A_n
    \end{equation*}
    %%
    Additionally, $q_1$ is given by 
    %%
    \begin{equation*}
        A_n\xrightarrow{\begin{pmatrix} 0 \\ 1_{A_n} \\ 0 \end{pmatrix}} A_{n-1}\oplus A_n\oplus A_n
    \end{equation*}
    %%
    and $q_2$ is given by 
    %%
    \begin{equation*}
        A_n\xrightarrow{\begin{pmatrix} 0 \\ 0 \\ 1_{A_n} \end{pmatrix}} A_{n-1}\oplus A_n\oplus A_n
    \end{equation*}
    %%
    Then a map $H:\text{cyl}(1_{A_\bullet})\to B_\bullet$ making the triangle commute is on each degree of the form
    %%
    \begin{equation*}
        A_{n-1}\oplus A_n\oplus A_n\xrightarrow{\begin{pmatrix} (-1)^{n-1}s_{n-1} & f_n & g_n \end{pmatrix}} B_n
    \end{equation*}
    %%
    where the chain map condition reduces to 
    %%
    \begin{equation*}
        \partial_n^B\circ s_{n-1}+s_{n-2}\circ \partial_{n-1}^A = f_{n-1}-g_{n-1}
    \end{equation*}
    %%
    which is exactly the condition for a chain homotopy.
\end{proof}

% \begin{lem}[label=lem:totAdd]
%     Let $F:\cat{Ch}(\mathcal{A})\to \cat{Ch}(\mathcal{B})$ be an additive functor. By Lemma~\ref{lem:funcActChain} we have an additive functor $\cat{Ch}(F):\cat{Ch}^2(\mathcal{A})\to \cat{Ch}^2(\mathcal{B})$ given by $F$ acting component-wise. Then
%     %%
%     \begin{equation*}
%         F\circ \text{Tot}_\mathcal{A} \cong \text{Tot}_\mathcal{B}\circ \cat{Ch}(F)
%     \end{equation*}
%     %%
% \end{lem}
% \begin{proof}
%     Let $A_{\bullet,\bullet} \in \cat{Ch}^2(\mathcal{A})$
% \end{proof}

% \begin{cor}[label=cor:cylinderPres]
%     If $F:\cat{Ch}(\mathcal{A})\to \cat{Ch}(\mathcal{B})$ is additive and $f:A_\bullet \to B_\bullet$ is a map of chain complexes, then $F(\text{cyl}(f)) \cong \text{cyl}(F(f))$.
% \end{cor}
% \begin{proof}
%     By Lemma~\ref{lem:funcActChain} we have an additive functor $\cat{Ch}(F):\cat{Ch}^2(\mathcal{A})\to \cat{Ch}^2(\mathcal{B})$ given by $F$ acting component-wise. 
% \end{proof}

% Using these equivalent characterizations of chain homotopy we prove that exact functors of chain complexes preserve chain homotopies.

% \begin{lem}[label=lem:exactChFunc]
%     Let $F:\cat{Ch}(\mathcal{A})\to \cat{Ch}(\mathcal{B})$ be an exact functor. Then $F$ preserves chain homotopies.
% \end{lem}
% \begin{proof}
%     \textbf{TBD}
% \end{proof}


We now prove some preliminary results on exactness and preserving chain homotopies in order to show the exactness and preservation of limits for this construction.

\begin{lem}[label=lem:TotExact]
    The totalization functor $\text{Tot}:\cat{Ch}^2(\mathcal{A})\rightarrow \cat{Ch}(\mathcal{A})$ is exact.
\end{lem}
\begin{proof}
    Let 
    %%
    \begin{equation*}
        0\rightarrow A_1\xrightarrow{f_1} A_2\xrightarrow{f_2} A_3\rightarrow 0
    \end{equation*}
    %%
    be a short exact sequence of bicomplexes in $\mathcal{A}$. This becomes a sequence of complexes 
    %%
    \begin{equation*}
        \text{Tot}(A_1)\xrightarrow{\text{Tot}(f_1)}\text{Tot}(A_2)\xrightarrow{\text{Tot}(f_2)}\text{Tot}(A_3)
    \end{equation*}
    %%
    where at a given $n$,
    %%
    \begin{equation*}
        \text{Tot}(A_i)_n = \bigoplus_{j=0}^n(A_i)_{j,n-j}
    \end{equation*}
    %%
    and
    %%
    \begin{equation*}
        \text{Tot}(f_i)_n = \bigoplus_{j=0}^n (f_i)_{j,n-j}
    \end{equation*}
    %%
    Note that the sequence of bicomplexes being exact means that each component sequence in $\mathcal{A}$ is exact. Then the component sequence of the totalization at $n$ is a finite direct sum of exact sequences, and hence exact.
\end{proof}

\begin{lem}[label=pushForward]
    Let $F:\mathcal{B}\rightarrow \mathcal{C}$ be an exact functor between abelian categories. Then for a category $\mathcal{A}$, $F_*:\text{Fun}(\mathcal{A},\mathcal{B})\rightarrow \text{Fun}(\mathcal{A},\mathcal{C})$ is exact.
\end{lem}
\begin{proof}
    Let $0\rightarrow G_1\xrightarrow{\eta_1} G_2\xrightarrow{\eta_2} G_3\rightarrow 0$ be a short exact sequence of functors from $\mathcal{A}$ to $\mathcal{B}$. Since abelian categories are finitely complete and cocomplete, finite limits and colimits in $\text{Fun}(\mathcal{A},\mathcal{B})$ and $\text{Fun}(\mathcal{A},\mathcal{C})$ are computed pointwise, so it is sufficient to prove the lemma at a given $A \in \mathcal{A}$. This follows by exactness of $F$.
\end{proof}

In order to prove our desired exactness result we first introduce a naive notion chain complexes of functors for the subcategory of additive functors.

\begin{lem}[label=lem:funcActChain]
    Let $\text{Fun}_{Add}(\mathcal{A},\mathcal{C})$ be the category of additive functors between abelian categories with all natural transformations. Then we have a functor
    %%
    \begin{equation*}
        \cat{Ch}:\text{Fun}_{Add}(\mathcal{A},\mathcal{C})\rightarrow \text{Fun}_{Add}(\cat{Ch}(\mathcal{A}),\cat{Ch}(\mathcal{C}))
    \end{equation*}
    given by sending functors to their action componentwise.
\end{lem}
\begin{proof}
    Let $\mathcal{F} \in \text{Fun}_{Add}(\mathcal{A},\mathcal{C})$. Since $\mathcal{F}$ is additive it preserves $0$'s and hence sends chain complexes to chain complexes. Then let $f_\bullet:A_\bullet\rightarrow A_\bullet'$ be a map of chain complexes. Then $\cat{Ch}(\mathcal{F})(f_\bullet)_n := \mathcal{F}(f_n)$, and since $\mathcal{F}$ is additive
    %%
    \begin{equation*}
        \mathcal{F}(f_n)\mathcal{F}(\partial_{n+1}^A)-\mathcal{F}(\partial_n^{A'})\mathcal{F}(f_{n+1}) = \mathcal{F}(f_n\partial_{n+1}^A-\partial_n^{A'}f_{n+1}) = \mathcal{F}(0) = 0
    \end{equation*}
    %%
    so $\cat{Ch}(\mathcal{F})(f_\bullet)$ is a chain map. Further, since $\cat{Ch}(\mathcal{F})$ is defined componentwise and $\mathcal{F}$ is a functor and additive, $\cat{Ch}(\mathcal{F})$ is a functor and additive. 


    Next, let $\eta:\mathcal{F}\rightarrow \mathcal{G}$ be a natural transformation between additive functors. Then define $\cat{Ch}(\eta)_{A_\bullet}:\cat{Ch}(\mathcal{F})(A_\bullet)\rightarrow \cat{Ch}(\mathcal{G})(A_\bullet)$ by $(\cat{Ch}(\eta)_{A_\bullet})_n := \eta_{A_n}$. Then $\cat{Ch}(\eta)_{A_\bullet}$ is a chain map by naturality of $\eta$. Further, $\cat{Ch}(\eta)$ is natural again by naturality of $\eta$, which makes the following diagram commute for $f_\bullet:A_\bullet\rightarrow A_\bullet$:
    \[\begin{tikzcd}
    	{\mathcal{F}(A_n)} & {\mathcal{F}(B_n)} \\
    	{\mathcal{G}(A_n)} & {\mathcal{G}(B_n)}
    	\arrow["{\eta_{A_n}}"', from=1-1, to=2-1]
    	\arrow["{\mathcal{G}(f_n)}"', from=2-1, to=2-2]
    	\arrow["{\eta_{B_n}}", from=1-2, to=2-2]
    	\arrow["{\mathcal{F}(f_n)}", from=1-1, to=1-2]
    \end{tikzcd}\]
    Since $\cat{Ch}(\eta)$ is defined componentwise it preserves composites and identities.
\end{proof}

\begin{lem}[label=lem:funcActChainHori]
    Let $F,G:\mathcal{B}\to \mathcal{C}$, $H:\mathcal{C}\to \mathcal{D}$, and $K:\mathcal{A}\to \mathcal{B}$ be additive functors. If $\eta:F\Rightarrow G$ is a natural transformation, then
    %%
    \begin{equation*}
        \cat{Ch}(H\eta_K) = \cat{Ch}(H)\cat{Ch}(\eta)_{\cat{Ch}(K)}
    \end{equation*}
\end{lem}
\begin{proof}
    From the proof of Lemma~\ref{lem:funcActChain} we have that for a chain complex $A_\bullet$, 
    %%
    \begin{equation*}
        (\cat{Ch}(H\eta_K)_{A_\bullet})_n = H(\eta_{K(A_n)}) = H(\eta_{\cat{Ch}(K)(A_\bullet)_n}) = H((\cat{Ch}(\eta)_{\cat{Ch}(K)(A_\bullet)})_n) = \cat{Ch}(H)(\cat{Ch}(\eta)_{\cat{Ch}(K)(A_\bullet)})_n
    \end{equation*}
    %%
\end{proof}

Note that Lemma~\ref{lem:funcActChainHori} shows that $\cat{Ch}$ is a strict 2-functor on the category of abelian categories with additive functors between them. We now show how additive functors acting componentwise preserve chain homotopies.

\begin{lem}[label=lem:addFuncPres]
    Let $F:\mathcal{A}\to \mathcal{B}$ be an additive functor. Then $\cat{Ch}(F):\cat{Ch}(\mathcal{A})\to \cat{Ch}(\mathcal{B})$ preserves chain homotopies.
\end{lem}
\begin{proof}
    Let $f,g:A_\bullet\to B_\bullet$ be chain maps with a homotopy $s_n:A_n\to B_{n+1}$ for $n \in \Z$ from $f$ to $g$. Since $F$ is additive we have that
    %%
    \begin{equation*}
        F(\partial_{n+1}^B)\circ F(s_n)+F(s_{n-1})\circ F(\partial_n^A) = F(\partial_{n+1}^B\circ s_n+s_{n-1}\circ \partial_n^A) = F(f_n-g_n) = F(f_n)-F(g_n)
    \end{equation*}
    %%
    Since $\cat{Ch}(F)(f)_n := F(f_n)$ and $\cat{Ch}(F)(g)_n := F(g_n)$, it follows that $F(f)$ and $F(g)$ are homotopic by $F(s_n):F(A_n)\to F(B_{n+1})$.
\end{proof}

We also have an analogous result for the chain construction from comonads.

\begin{lem}[label=lem:ChConsPres]
    Let $(C,\epsilon,\delta)$ be a comonad on $\mathcal{A}$ which is also an additive functor. Then $C^{\cat{Ch}}:\mathcal{A}\to \cat{Ch}(\mathcal{A})$ is additive.
\end{lem}
\begin{proof}
    Let $f,g:A\to B$ in $\mathcal{A}$. Then since $C$ is additive, so is all of its powers, so \[C^{\cat{Ch}}(f+g)_n := C^n(f+g)=C^n(f)+C^n(g) = C^\cat{Ch}(f)_n+C^\cat{Ch}(g)_n\]
    Thus $C^{\cat{Ch}}$ is additive.
\end{proof}

By Lemma~\ref{lem:ChConsPtwise} and Lemma~\ref{ lem:addFuncPres} it follows that $\cat{Ch}(C)^\cat{Ch}$ preserves homotopies. Another important result for this construction is how it is affected by isomorphisms between comonads.

\begin{lem}[label=lem:ChComIso]
    Let $(C,\epsilon,\delta)$ be a comonad on $\mathcal{A}$ and let $(C',\epsilon',\delta')$ be a comonad on $\mathcal{A}'$. Suppose $\gamma:\mathcal{A}\to \mathcal{A}'$ is an additive isomorphism of categories such that $C' = \gamma\circ C\circ \gamma^{-1}$, $\epsilon' = \gamma\epsilon_{\gamma^{-1}}$, and $\delta' = \gamma\delta'_{\gamma^{-1}}$. Then there is an equality
    %%
    \begin{equation*}
        {C'}^{\cat{Ch}} = \cat{Ch}(\gamma)\circ C^{\cat{Ch}}\circ \gamma^{-1}
    \end{equation*}
\end{lem}
\begin{proof}
    This equality is an immediate consequence of the construction of $C^{\cat{Ch}}$ and the specified equalities for the comonad natural transformations. Indeed, observe that for $A' \in \mathcal{A}'$
    %%
    \begin{align*}
        \cat{Ch}(\gamma)\circ C^{\cat{Ch}}\circ \gamma^{-1}(A') &= \cat{Ch}(\gamma)(\cdots \to C^2\gamma^{-1}(A')\xrightarrow{C\epsilon_{\gamma^{-1}(A')}-\epsilon_{C\gamma^{-1}(A')}}C\gamma^{-1}(A')\xrightarrow{\epsilon_{\gamma^{-1}(A')}}\gamma^{-1}(A')) \\
        &= \cdots \to \gamma C^2\gamma^{-1}(A')\xrightarrow{\gamma C\epsilon_{\gamma^{-1}(A')}-\gamma \epsilon_{C\gamma^{-1}(A')}}\gamma C\gamma^{-1}(A')\xrightarrow{\gamma \epsilon_{\gamma^{-1}(A')}}\gamma \gamma^{-1}(A') \\
        &= \cdots \to {C'}^2(A')\xrightarrow{C'\epsilon'_{A'}-\epsilon'_{C'(A')}}C'(A')\xrightarrow{\epsilon'_{A'}}A'
    \end{align*}
    %%
\end{proof}


Next, we have the following result on (co)limits in chain complex categories, which is a analoguous to the result for functor categories.

\begin{lem}[label=lem:computecoLim]
    Let $\mathcal{A}$ be an abelian category with $I$ shaped (co)limits for a small category $I$. Then $\cat{Ch}(\mathcal{A})$ has $I$ shaped (co)limits which are computed pointwise.
\end{lem}
\begin{proof}
    We provide the proof in the case of colimits, while the case of limits is analoguous. Let $D:I\to \cat{Ch}(\mathcal{A})$ be an $I$ shaped diagram in $\cat{Ch}(\mathcal{A})$. 
    
    
    For each $n \in \Z$ we have a functor $(-)_n:\cat{Ch}(\mathcal{A})\to \mathcal{A}$ given by projecting on the $n$th component. By assumption, for each $n \in \Z$ the colimit of $(-)_n\circ D:I\to \mathcal{A}$ exists. Denote this limit by $D_n$, and denote its injections by $\iota_{n,i}$, for $i \in I$. For each $i \in I$ we have an induced map $D(i)_{n+1}\to D(i)_n \to D_n$ obtained by composing with the the boundary map for $D(i)_\bullet \in \cat{Ch}(\mathcal{A})$ and the inclusion. By the universal property of the colimit we obtain a unique map making the square
    \[\begin{tikzcd}
    	{D_{n+1}} & {D_n} \\
    	{D(i)_{n+1}} & {D(i)_n}
    	\arrow["{\partial^{D(i)}_{n+1}}"', from=2-1, to=2-2]
    	\arrow[dashed, from=1-1, to=1-2]
    	\arrow["{\iota_{n+1,i}}", from=2-1, to=1-1]
    	\arrow["{\iota_{n,i}}"', from=2-2, to=1-2]
    \end{tikzcd}\]
    commute. Let $\partial_{n+1}$ denote this map. Then by uniqueness of this map and the fact that pasting two such squares together gives a commuting rectangle
    %%
    \[\begin{tikzcd}
    	{D_{n+1}} & {D_n} & {D_{n-1}} \\
    	{D(i)_{n+1}} & {D(i)_n} & {D(i)_{n-1}}
    	\arrow["{\partial^{D(i)}_{n+1}}"', from=2-1, to=2-2]
    	\arrow[dashed, from=1-1, to=1-2]
    	\arrow["{\iota_{n+1,i}}", from=2-1, to=1-1]
    	\arrow["{\iota_{n,i}}"', from=2-2, to=1-2]
    	\arrow["{\partial_n^{D(i)}}"', from=2-2, to=2-3]
    	\arrow["{\iota_{n-1,i}}"', from=2-3, to=1-3]
    	\arrow[dashed, from=1-2, to=1-3]
    \end{tikzcd}\]
    %%
    we must have that $\partial_n\circ \partial_{n+1}$ is the zero map. This implies that $(D_\bullet,\partial_\bullet)$ defines a chain complex on $\mathcal{A}$. It remains to show that it is the colimit of $D$. But by construction any other such chain complex with maps out of each $D(i)$ induces maps unique pointwise maps which form commuting squares in a chain map by their uniqueness.
\end{proof}


This allows us to also describe how totalization behaves under the isomorphism from functors valued in chain complexes and chain complexes of functors.

\begin{lem}[label=lem:totFuncCh]
    For abelian categories $\mathcal{A},\mathcal{B}$, we have a natural isomorphism
    %%
    \begin{equation*}
        (\text{Tot}_{\mathcal{A}})_*\circ \text{Fun}^{\cat{Ch}}\circ \text{Fun}^{\cat{Ch}} \cong \text{Fun}^{\cat{Ch}}\circ \text{Tot}_{\text{Fun}(\mathcal{B},\mathcal{A})}
    \end{equation*}
    %%
\end{lem}
\begin{proof}
    Let $F_{\bullet,\bullet} \in \cat{Ch}^2(\text{Fun}(\mathcal{B},\mathcal{A}))$. Let $F:\text{Fun}(\mathcal{B},\cat{Ch}^2(\mathcal{A}))$ be the associated functor under two applications of $\text{Fun}^{\cat{Ch}}$. For $B \in \mathcal{B}$, the left functor gives the complex
    %%
    \begin{equation*}
        \text{Tot}_{\mathcal{A}}(F(B))_n := \bigoplus_{p+q=n}F_{p,q}(B)
    \end{equation*}
    %%
    at $B$, while the right functor gives the complex
    %%
    \begin{equation*}
        \text{Fun}^{\cat{Ch}}(\text{Tot}_{\text{Fun}(\mathcal{B},\mathcal{A})}(F_{\bullet,\bullet}))(B)_n := \left(\bigoplus_{p+q=n}F_{p,q}\right)(B) \cong \bigoplus_{p+q=n}F_{p,q}(B)
    \end{equation*}
    %%
    Under the adjunction in Lemma~\ref{lem:adjChFuncCat}, these give functors $\text{Fun}_{\cat{Ch}}(\mathcal{B},\cat{Ch}(\mathcal{A}))\cong \text{Fun}_{\cat{Ch}}(\mathcal{B}\times \Z,\mathcal{A})$. Then we can use the results in Section~\ref{sec:colimFuncs} to conclude that 
    %%
    \begin{equation*}
        \text{Fun}^{\cat{Ch}}(\text{Tot}_{\text{Fun}(\mathcal{B},\mathcal{A})}(F_{\bullet,\bullet})) \cong  \text{Tot}_{\mathcal{A}}(F)
    \end{equation*} 
    %%
    for all $F_{\bullet,\bullet}$, and further these isomorphisms are natural in $F_{\bullet,\bullet}$ \textbf{STILL NEED SOME CLARIFICATION}.
\end{proof}