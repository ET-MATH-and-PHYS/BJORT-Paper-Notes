This section lists tasks which are yet to be completed from previous meetings


\textbf{Completed Sec 2:}
\begin{itemize}
    \item[1.] The cross effects operation in Definition 2.1 is explicitly defined to be functorial in the X variables.  Is it also functorial in F?  That is, is $cr_n$ a functor from the category $Fun(B, A) $ to $Fun(B^n, A)$ (these categories have functors as objects and natural transformations as morphisms).  See remarks before Lemma 2.4.  What else needs to be verified?
    \item[2.] Verify that the counit in Remark 2.8 of \cite{BJORT} is natural in $X$. Is it also a natural transformation $\text{cr}_n\Rightarrow \text{id}$?
    \item[3.] Is the contracting chain homotopy in Lemma 2.9 of \cite{BJORT} natural in $A$?
\end{itemize}


\textbf{Completed* Sec 3:}
\begin{itemize}
    \item[1.] In Observation 3.1, we claim that Ch is a pseudomonad \cite{BJORT}.  Is it?  This should be viewed with skepticism.
    \item[2.] In Observation 3.1 \cite{BJORT}, we claim that there is a quotient monad $\cat{Ch}$ acting on the category of abelian categories and isomorphism classes of functors.  Show that there is such a monad.
    \item[3.] While you are at it, please make sure you understand the phrase ``here we are not interested in the 2-dimensional aspects." What are these two dimensional aspects, and what is the consequence of ignoring them?
    \item[4.] In definition 3.2 \cite{BJORT}, we use natural isomorphism classes.  What happens if you use pointwise defined isomorphism classes?
    \item[5.] At the top of page 388 (following the proof of Lemma 3.4) \cite{BJORT}, we establish an equivalence relation on $\cat{AbCat}_\cat{Ch}$.  Do both pointwise defined chain homotopy equivalences and natural chain homotopy equivalences result in an equivalence relation on this category?
    \item[6.] Is it possible to alter Definition 3.5 \cite{BJORT} to use natural chain homotopy equivalence classes instead of pointwise chain homotopy equivalence classes?
\end{itemize}


\textbf{Not completed Sec 4:}
\begin{itemize}
    \item Go through section 4 and try to go through with the example of the identity in mind.
\end{itemize}


\textbf{Separate To-Do:}
\begin{itemize}
    \item Make a list/section of lemmas for $\Gamma$ and $N$ in the Dold-Kan equivalence
\end{itemize}