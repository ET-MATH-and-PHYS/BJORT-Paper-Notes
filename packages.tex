%%%%%% PACKAGES %%%%%%%



%% References
\usepackage{hyperref}
\hypersetup{
    colorlinks,
    citecolor=black,
    filecolor=black,
    linkcolor=black,
    urlcolor=black
}
\usepackage{nameref}
\usepackage{url}
%\usepackage{apacite} % helps you apa cite
\usepackage[numbers]{natbib} % implements both author-year and numbered references


%% Math packages
\usepackage{amsmath} % Math display options
\usepackage{amsthm}
\usepackage{mathtools} % General math tools
\usepackage{array, longtable} % Allows you to write arrays
\usepackage{bbold}
\usepackage{empheq} % For boxing equations
%\usepackage{physics} % useful math symbols and macros (VERY IMPORTANT)
\usepackage{cancel}
% Math fonts and symbols
\usepackage{amsfonts} 
%\usepackage{mathabx}
%\usepackage{mathrsfs}
%\usepackage{txfonts} % defines times new roman as default text font and provides supporting math symbols
\usepackage{braket} % allows the use of detailed Dirac braket notation
\usepackage{amssymb}
\usepackage{mathdots}
\usepackage{siunitx} % allows you to quickly define units within math mode (IMPORTANT)
\usepackage{latexsym}
\usepackage{bbm}
%% Figure and Table Packages
\usepackage{wrapfig}
\usepackage{enumitem}
\usepackage{float}
\usepackage{booktabs} % allows weird lines within tables
\usepackage{multirow}
\usepackage{multicol} % allows for multiple columns and rows that collapse into single columns
\usepackage{longtable} % allows a table to run over multiple pages
\usepackage{colortbl} % colourful tables!
\usepackage{mathrsfs}  

\usepackage{adjustbox}

\usepackage[normalem]{ulem}

\usepackage[toc, page]{appendix}

%% Heading packages
\usepackage{titletoc,tocloft}
\usepackage{titlesec}


%% Numbering commands
\titleformat{\section}
  {\normalfont\fontsize{25}{15}\bfseries}{\thesection}{1em}{}
\titleformat{\section}
  {\normalfont\fontsize{20}{15}\bfseries}{\thesubsection}{1em}{}
\setcounter{secnumdepth}{3}  
\newcommand\numberthis{\refstepcounter{equation}\tag{\theequation}} % For equation labelling


%% Tikz packages
\usepackage[framemethod=tikz]{mdframed}
\usepackage{tikz} % For drawing commutative diagrams
\usetikzlibrary{cd}
\usetikzlibrary{calc}
\tikzset{every picture/.style={line width=0.75pt}} %set default line width to 0.75p


%% Page formatting packages
\usepackage{datetime}
\usepackage[utf8]{inputenc} % the typesetting rules
\usepackage[margin=1in]{geometry} % flexible and easy interface to change page dimensions
\setlength{\parskip}{1em} % Sets the default paragraph skip to 1em
\setlength{\parindent}{4mm} % Sets default paragraph indent
\usepackage{graphicx} % provides additional options for figures
\usepackage{changepage} % allows to change page formatting in the middle of a document (rather than the same throughout) 
\usepackage{float}
\usepackage[labelfont=bf]{caption} % altering options for captions
\usepackage{sidecap} % allows sideways captions for figures and tables
\usepackage{soul,xcolor} % can enhance striking, underlining, and highlighting (in non-math mode)
\usepackage{fancyhdr}
\setlength{\headheight}{15pt} 
\pagestyle{fancy}
\lhead[\leftmark]{}
\rhead[]{\leftmark}
\allowdisplaybreaks
\newdateformat{monthdayyeardate}{%
    \monthname[\THEMONTH]~\THEDAY, \THEYEAR}



%% Custom 
\usepackage{color} % provides coloring capabilities to everything
\definecolor{purp}{rgb}{0.29, 0, 0.51}
\definecolor{bloo}{rgb}{0, 0.13, 0.80}



%%\newtheoremstyle{note}% hnamei
%{3pt}% hSpace above
%{3pt}% hSpace belowi
%{}% hBody fonti
%{}% hIndent amounti
%{\itshape}% hTheorem head fonti
%{:}% hPunctuation after theorem headi
%{.5em}% hSpace after theorem headi
%{}% hTheorem head spec (can be left empty, meaning ‘normal’)

%%%%%%%%%%%%% THEOREM STYLES

\newtheoremstyle{BigTheorem}
{20pt}
{20pt}
{\slshape}
{}
{\Large\bfseries}
{.}
{\newline}
{\thmname{#1}\thmnumber{ #2}\thmnote{ (#3)}}



\newtheoremstyle{TheoremClassic}
{15pt}
{15pt}
{\slshape}
{}
{\bfseries}
{.}
{.5em}
{}

\newtheoremstyle{Definitions}
{15pt}
{15pt}
{\slshape}
{}
{\bfseries}
{.}
{.5em}
{\thmname{#1}\thmnumber{ #2}\thmnote{ (#3)}}


\newtheoremstyle{Custom}
{15pt}
{15pt}
{\slshape}
{}
{\bfseries}
{.}
{.5em}
{\thmname{#1}\thmnumber{ #2}\thmnote{ #3}}

\newtheoremstyle{Remarks}
{10pt}
{10pt}
{\upshape}
{}
{\bfseries}
{.}
{.5em}
{}

\newtheoremstyle{Examples}
{10pt}
{10pt}
{\upshape}
{}
{\bfseries}
{.}
{.5em}
{}


%%%%%%%%%%%%% THEOREM DEFINITIONS

\usepackage[most]{tcolorbox}
%%%%%%%%%%%%%% ENVIRONMENTS
%%% tcolorbox environments

%%% \tcolorboxenvironment: An existing environment is redefined to be allowed to be boxed inside another tcolorbox during use
\tcolorboxenvironment{proof}{% `proof' from `amsthm'  
blanker,breakable=true,left=0.3em,
borderline west={3pt}{-3pt}{black!35}}


%%% titled environments
\newtcolorbox[auto counter, number within=section]{nthm}[2][]{
colback=red!5!white,
colframe=red!5!white,
attach title to upper={},
title={\textbf{Theorem \thetcbcounter\  (#2)\ }},
coltitle={black},
sharp corners,
breakable=true,
enhanced jigsaw,
borderline west = {3pt}{-3pt}{red!75!black},
#1
}

\newtcolorbox[use counter from=nthm, number within=section]{ndefn}[2][]{
colback=blue!5!white,
colframe=blue!5!white,
attach title to upper={},
title={\textbf{Definition \thetcbcounter\  (#2)\ }},
coltitle={black},
sharp corners,
breakable=true,
enhanced jigsaw,
borderline west = {3pt}{-3pt}{blue!75!black},
#1
}

\newtcolorbox[auto counter, number within=section]{eg}[2][]{
colback=white,
attach title to upper={},
title={\underline{Example \thetcbcounter\  (#2)\ :}\smallbreak},
coltitle={green!50!black},
sharp corners,
breakable=true,
enhanced jigsaw,
frame hidden, 
borderline west = {3pt}{-3pt}{green!50!black},
#1
}

\newtcolorbox[auto counter, number within=section]{qst}[1][]{
colback=white,
attach title to upper={},
title={\underline{Question?}\smallbreak},
coltitle={blue!50!black},
sharp corners,
breakable=true,
enhanced jigsaw,
frame hidden, 
borderline west = {3pt}{-3pt}{blue!50!black},
#1
}

\newtcolorbox[auto counter, number within=section]{rmk}[1][]{
colback=white,
attach title to upper={},
title={\underline{Remark:}\smallbreak},
coltitle={red!50!black},
sharp corners,
breakable=true,
enhanced jigsaw,
frame hidden, 
borderline west = {3pt}{-3pt}{red!50!black},
#1
}


\newtcolorbox[auto counter, number within=section]{nte}[1][]{
colback=white,
attach title to upper={},
title={\underline{Note:}\smallbreak},
coltitle={red!50!white},
sharp corners,
breakable=true,
enhanced jigsaw,
frame hidden, 
borderline west = {3pt}{-3pt}{red!50!white},
#1
}


%%% untitled environments
\newtcolorbox[use counter from=nthm, number within=section]{thm}[1][]{
colback=red!5!white,
colframe=red!5!white,
attach title to upper={},
title={\textbf{Theorem \thetcbcounter\ }},
coltitle={black},
sharp corners,
breakable=true,
enhanced jigsaw,
borderline west = {3pt}{-3pt}{red!75!black},
#1
} 

\newtcolorbox[use counter from=nthm, number within=section]{defn}[1][]{
colback=blue!5!white,
colframe=blue!5!white,
attach title to upper={},
title={\textbf{Definition \thetcbcounter\ }},
coltitle={black},
sharp corners,
breakable=true,
enhanced jigsaw,
borderline west = {3pt}{-3pt}{blue!75!black},
#1
}

\newtcolorbox[use counter from=nthm, number within=section]{cor}[1][]{
colback=red!5!white,
colframe=red!5!white,
attach title to upper={},
title={\textbf{Corollary \thetcbcounter\ }},
coltitle={black},
sharp corners,
breakable=true,
enhanced jigsaw,
borderline west = {3pt}{-3pt}{red!50!black},
#1
}

\newtcolorbox[use counter from=nthm, number within=section]{prop}[1][]{
colback=orange!5!white,
colframe=orange!5!white,
attach title to upper={},
title={\textbf{Proposition \thetcbcounter} \ },
coltitle={black},
sharp corners,
breakable=true,
enhanced jigsaw,
borderline west = {3pt}{-3pt}{orange!75!black},
#1
}

\newtcolorbox[use counter from=nthm, number within=section]{lem}[1][]{
colback=orange!5!white,
colframe=orange!5!white,
attach title to upper={},
title={\textbf{Lemma \thetcbcounter} \ },
coltitle={black},
sharp corners,
breakable=true,
enhanced jigsaw,
frame hidden, 
borderline west = {3pt}{-3pt}{orange!75!black},
#1
}

\newtcolorbox[auto counter, number within=section]{home}[1][]{
colback=white,
attach title to upper={},
title={\textbf{Exercise:}\ \ },
coltitle={black},
sharp corners,
breakable=true,
enhanced jigsaw,
frame hidden, 
borderline west = {3pt}{-3pt}{black!10!lime},
#1
}

%% For proofs
\renewcommand{\qedsymbol}{$\blacksquare$}




%%%%%% MACROS %%%%%%%%%

%% Fields
\newcommand\C{\mathbb C} %%% Complex numbers
\newcommand\R{\mathbb R} %%% Real numbers
\newcommand\Z{\mathbb Z} %%% Integers
\newcommand\Q{\mathbb Q} %%% Rationals
\newcommand\N{\mathbb N} %%% Naturals
\newcommand\F{\mathbb F} %%% An arbitrary field

%% Categories
\newcommand{\cat}[1]{\textup{\textsf{#1}}}% for categories
\newcommand{\Cat}{\mathbb{C}\cat{at}}
\newcommand{\Bicat}{\mathbb{B}\cat{icat}}
\newcommand{\sSet}{\cat{s}\mathbb{S}\cat{et}}
\newcommand{\SET}{\mathbb{S}\cat{et}}
\newcommand{\Grp}{\cat{Grp}} %%% Category group
\newcommand{\Rmod}{\cat{R-Mod}} %%% Category r-modules
\newcommand{\Mon}{\cat{Mon}} %%% Category monoid
\newcommand{\Ring}{\cat{Ring}} %%% Category ring
\newcommand{\Topp}{\cat{Top}} %%% Category Topological spaces
\newcommand{\Vect}{\cat{Vect}_{k}} %%% category vector spaces'
\newcommand\Hom{\cat{Hom}} %%% Arrows
\DeclareSymbolFont{bbold}{U}{bbold}{m}{n}
\DeclareSymbolFontAlphabet{\mathbbb}{bbold}
\newcommand{\catone}{\ensuremath{\mathbbb{1}}}
\newcommand{\cattwo}{\ensuremath{\mathbbb{2}}}
\newcommand{\catthree}{\ensuremath{\mathbbb{3}}}
\newcommand{\catn}{\ensuremath{\mathbbb{n}}}
\newcommand{\adjnt}[2]{
\begin{tikzcd}
{#1} & {#2}
\arrow[""{name=0, anchor=center, inner sep=0}, "R"', shift right=2, from=1-1, to=1-2]
\arrow[""{name=1, anchor=center, inner sep=0}, "L"', shift right=2, from=1-2, to=1-1]
\arrow["\dashv"{anchor=center, rotate=-90}, draw=none, from=1, to=0]
\end{tikzcd}
}



%% Convenient brackets
\newcommand{\IP}[1]{\left\langle#1\right\rangle} %%% Inner product
\newcommand{\ABS}[1]{\lvert#1\rvert} %%% Modulus
\newcommand{\COIP}[2]{\left\langle #1\left|#2\right.\right\rangle}


%% Linear Algebra
\newcommand\GL{\text{GL}} %%% General Linear group
\newcommand\SL{\text{SL}} %%% Special linear group

\newcommand\gl{\mathfrak{gl}} %%% General linear algebra
\newcommand\g{\mathfrak{g}} %%% Lie algebra of G

%% Aligned maps
\newcommand{\map}[2]{\begin{array}{c} #1 \\ #2 \end{array}}

%% Text emphasis
\newcommand{\Emph}[1]{\textbf{\ul{\emph{#1}}}}

%% reverse mapsto
\newcommand{\mapsfrom}{\mathrel{\reflectbox{\ensuremath{\mapsto}}}}


%% Math operators
\DeclareMathOperator{\ran}{Im} %%% image
\DeclareMathOperator{\aut}{Aut} %%% Automorphisms
\DeclareMathOperator{\spn}{span} %%% span
\DeclareMathOperator{\ann}{Ann} %%% annihilator
\DeclareMathOperator{\ch}{char} %%% characteristic
\DeclareMathOperator{\Ev}{\bf{ev}} %%% evaluation
\DeclareMathOperator{\sgn}{sign} %%% sign
\DeclareMathOperator{\id}{Id} %%% identity
\DeclareMathOperator{\inn}{Inn} %%% Inner aut
\DeclareMathOperator{\en}{End} %%% Endomorphisms
\DeclareMathOperator{\sym}{Sym} %%% Group of symmetries
\newcommand{\Sob}{^{\Delta^{op}}} %%% Simplicial object category

\newcommand{\rd}[1]{\textcolor{red}{#1}}

%% Diagram Environments
\iffalse
\begin{center}
    \begin{tikzpicture}[baseline= (a).base]
        \node[scale=1] (a) at (0,0){
          \begin{tikzcd}
           
          \end{tikzcd}
        };
    \end{tikzpicture}
\end{center}
\fi





%%%%%%%%%%%%%%%%%%%%%%%